\begin{center}
{\bf Multi-fidelity Monte Carlo for uncertainty quantification \\ in the free boundary Grad-Shafranov equation} \\[1ex]

% Author names and affiliations
Jiaxing Liang  \email{jl508@rice.edu} \\[1ex]
{\small Department of Computational Applied Mathematics and Operations Research, Rice University}
\end{center}

\noindent
We investigate recent work on uncertainty quantification for the Grad–Shafranov free boundary problem. The focus is on quantifying the variability of plasma equilibria under parametric uncertainty in coil currents using multi-fidelity Monte Carlo (MFMC) sampling strategies. We will present a reformulation of the MFMC framework as an optimization problem, which provides optimal sample allocations based on a prescribed error tolerance for variance. A simple rounding technique makes this allocation directly applicable in practice. In addition, we will introduce a dynamic strategy for estimating correlation coefficients, which accounts for confidence interval information to mitigate bias in Pearson correlation estimates. Numerical experiments demonstrate that MFMC attains the same level of accuracy as standard Monte Carlo while reducing computational cost by nearly two orders of magnitude. The results illustrate how multi-fidelity approaches can efficiently capture plasma boundary dynamics and geometric features, underscoring their promise for large-scale uncertainty quantification in fusion modeling.
