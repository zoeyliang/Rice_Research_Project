%!TEX root = main.tex
% ====================================================
\section{Introduction}\label{sec:Intro}
% ====================================================

Monte Carlo (MC) methods are widely used for estimating statistical quantities in scientific and engineering applications. However, their computational cost can become prohibitive when sampling requires the evaluation of 
high-fidelity models (HFMs). 
Multi-fidelity Monte Carlo (MFMC) methods \cite{BPeherstorfer_KWillcox_MDGunzburger_2016a} address this 
challenge by exploiting correlations between models of varying accuracy and cost. 
By combining the HFM and the low-fidelity models (LFMs) within a single estimator, MFMC computes an
unbiased estimate of the HFM output with a reduced variance compared to MC estimator, given a fixed computational budget.

This paper develops approaches for computing sample size allocations across the different models. While sample
size allocation is a crucial ingredient of MFMC, it has received limited attention in the literature.
The original MFMC paper \cite{BPeherstorfer_KWillcox_MDGunzburger_2016a} poses the sample size allocation 
problem as an optimization problem in real variables, provides an analytic solution to this optimization problem
under additional assumptions on the model correlations and costs, and proposed to round down the real-valued
solutions to obtain integer-valued sample sizes. Obtaining integer-valued sample sizes ensures that the
resulting MFMC estimator stays within the computational budget, but it can increase the estimator variance substantially.
When the computational budget is small, simply rounding down can result in zero samples for the HFM, which introduces
bias in the MFMC estimator. To ensure that the MFMC estimator is unbiased, 
\cite{AGruber_MGunzburger_LJu_ZWang_2023a} modify the approach of  \cite{BPeherstorfer_KWillcox_MDGunzburger_2016a} to ensure that the HFM is always sampled.
However, rounding down is still used to compute sample sizes of LFMs.
When rounding down is applied, some computational budget remains unused, which could be used to adjust
LFM sample sizes for additional variance reduction. Moreover, the current approaches in \cite{BPeherstorfer_KWillcox_MDGunzburger_2016a} and \cite{AGruber_MGunzburger_LJu_ZWang_2023a} 
can lead to identical sample sizes for subsequent models. In this case, one of these models is not contributing to 
MFMC variance reduction, but its cost is still allocated to the budget. 
Our approach addresses these and other shortcomings.



We provide theoretical analysis of the proposed scheme establishing bounds on variance and guarantees budget admissibility of the estimator using integer-valued sample estimation. Numerical experiments confirm that our iterative allocation consistently outperforms conventional rounding strategies, particularly in regimes with tight computational budgets or highly heterogeneous model costs.

The paper is organized as follows: Section~\ref{sec:MFMC} reviews the MFMC formulation and continuous allocation theory; Section~\ref{sec:Iterative_IntegerValued_Sample_Size} develops the iterative integer-valued allocation scheme and its theoretical foundations; Section~\ref{sec:Modified_IntegerValued_Sample_Size} addresses modifications for ill-conditioned scenarios; Section~\ref{sec:Num_Result} presents numerical results illustrating the performance of the method.









