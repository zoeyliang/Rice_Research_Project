\documentclass[final,3p,times,11pt]{elsarticle}
\usepackage[USenglish]{babel}
\usepackage{amsmath,amssymb,amsthm, mathrsfs,multirow}
\usepackage{mathtools}
\usepackage{graphicx}
\usepackage{stmaryrd}
\usepackage[dvipsnames]{xcolor}
\usepackage{cancel}
\usepackage{ulem}
\usepackage{tabularx}
\usepackage{comment}
%\usepackage{subcaption}
%\usepackage[show]{ed}
%\usepackage{showkeys}
%\usepackage{showlabels}
%\usepackage[notcite,notref]{showkeys}
%\usepackage{refcheck}
% \usepackage[ruled,vlined]{algorithm2e}
\usepackage[linesnumbered,ruled,vlined]{algorithm2e}
\definecolor{Myblue}{rgb}{.2 0.4 1}

\usepackage{hyperref}
\hypersetup{
    %bookmarks=true,         % show bookmarks bar?
    colorlinks = true,       % false: boxed links; true: colored links
}



% ==============   Macros  ====================
\newcommand{\real}{\mathbb{R}} 
\newcommand{\nat}{\mathbb{N}} 


\newcommand{\mynabla}{\widetilde{\nabla}} 
\newcommand{\jump}[1]{[\![#1]\!]}
\newcommand{\HEcolor}[1]{{\textcolor{blue}{#1}}}
\newcommand{\TSVcolor}[1]{{\textcolor{orange}{#1}}}
\newcommand{\JLcolor}[1]{{\textcolor{violet}{#1}}} %violet
\newcommand{\Grids}{\boldsymbol{\chi}}

\newtheorem{theorem}{Theorem}%[section]
\newtheorem{lemma}{Lemma}%[section]
\newtheorem{VariationalForm}[theorem]{Variational Formulation}
% =============================================

\journal{}
\makeatletter
\def\ps@pprintTitle{%
 \let\@oddhead\@empty
 \let\@evenhead\@empty
 \def\@oddfoot{}%
 \let\@evenfoot\@oddfoot}
\makeatother




\begin{document}
\begin{frontmatter}
\title{Iterative estimation of integer sample size for multi-fidelity Monte Carlo\tnoteref{t1}}
\tnotetext[t1]{This research was supported in part by AFOSR Grant FA9550-22-1-0004 and NSF Grant DMS-2231482.}



% \author[RiceCMOR]{Damon Spencer}
% \ead{heinken@rice.edu}
% \address[RiceCMOR]{Department of Computational Applied Mathematics \& Operations Research, Rice University.}
\author[MH]{Matthias Heinkenschloss}
\ead{heinken@rice.edu}
\address[MH]{Department of Computational Applied Mathematics \& Operations Research and The Ken Kennedy Institute, Rice  University.}
\author[JL]{Jiaxing Liang}
\ead{jl508@rice.edu}
\address[JL]{Department of Computational Applied Mathematics \& Operations Research, Rice University.}

\begin{abstract}
In multifidelity Monte Carlo (MFMC) methods, optimal sample allocations are typically derived by solving a continuous optimization problem, with integer solutions obtained through flooring operations. This rounding approach frequently leads to budget underutilization and suboptimal variance reduction in practical implementations. In this work, we present a novel iterative scheme that directly computes integer-valued sample allocations for MFMC estimation. The proposed method guarantees a better adherence to total budget constraints while generating integer-feasible solutions that more closely approximate the continuous optimum. Numerical results demonstrate that our approach achieves lower estimator variance compared to conventional rounding strategies. These advancements offer significant practical benefits for uncertainty quantification applications requiring optimal resource allocation across multiple model fidelities.
\end{abstract}





\begin{keyword}
Multi-fidelity Monte Carlo \sep Sample Allocation \sep Integer Optimization  \sep Uncertainty Quantification \sep Variance Reduction.
%
\MSC[2020] 65C05\sep 62K05 \sep 49M20.
\end{keyword}
\end{frontmatter}

%!TEX root = main.tex
%%%%%%%%%%%%%%%%%%%%%%%%%%%%%%%%%%%%%%%%%%%%%%%%%%%

% ========================================
\section{Introduction}\label{sec:intro}
% ========================================
The Monte Carlo (MC) method is a fundamental tool for uncertainty quantification in computational science and engineering. It estimates statistical properties of quantities influenced by stochastic inputs through repeated evaluations of a deterministic model. Its non-intrusive nature and broad applicability -- requiring no assumptions about the smoothness or structure of the input distributions -- make it widely appealing. However, a key limitation of the MC method is its slow convergence rate of $1/\sqrt{N}$, where $N$ is the number of samples, which leads to high computational costs, particularly when the evaluation of each model is expensive. Such costs are especially pronounced in the context of non-linear partial differential equations (PDEs), which typically require high-fidelity numerical discretizations -- such as fine-mesh finite element methods -- to accurately resolve the solution. Consequently, direct application of standard MC methods to these problems can become computationally prohibitive. To mitigate this burden, surrogate models have been developed as low-fidelity approximations that retain essential features of the high-fidelity model while offering substantial reductions in cost. These models accelerate sampling by approximating the underlying system at coarser resolution or through simplified representations while retaining sufficient accuracy for statistical inference. However, the simplifications inherent in surrogate construction can introduce bias and reduce accuracy, particularly in regions where the surrogate fails to capture important solution features. For instance, \cite{ElLiSa:2022} demonstrates that surrogate models based on stochastic collocation can significantly accelerate MC sampling while maintaining reliable accuracy. An alternative strategy to accelerate Monte Carlo sampling is provided by the multilevel Monte Carlo (MLMC) method \cite{BaScZo:2011,Gi:2008,Gi:2015}, which constructs a hierarchy of spatial discretizations to efficiently estimate statistical quantities. The key idea is to compute most of the variance contribution using inexpensive coarse-grid models, while reserving costly fine-grid evaluations for corrections at the highest resolution. This hierarchical approach significantly reduces the total computational cost compared to standard MC methods. For example, \cite{ElLiSa:2023} investigates the application of MLMC for uncertainty quantification in PDEs, demonstrating its effectiveness in reducing computational burden. In this paper, we address these limitations by adopting the multifidelity Monte Carlo (MFMC) framework \cite{PeGuWi:2018,PeWiGu:2016,PeGuWi:2018}, which generalizes the multilevel paradigm by combining models of varying fidelity through control variates. Unlike MLMC, MFMC does not require a nested hierarchy of discretizations, allowing for greater flexibility in the selection and integration of surrogate models. While MFMC incurs some upfront cost associated with surrogate construction and variance estimation, it offers some computational savings over both standard MC and MLMC, especially in regimes where low-fidelity models are inexpensive yet sufficiently accurate to inform high-fidelity predictions.



The paper is organized as follows. Section~\ref{sec:Grad-Shafranov} introduces the Grad-Shafranov free boundary problem under uncertainty. Section~\ref{sec:SC} reviews the sparse grid stochastic collocation technique, which serves as the foundation for constructing low-fidelity models in the MFMC framework. Sections~\ref{sec:MC} and~\ref{sec:MFMC} present the Monte Carlo finite element method and its multifidelity extension, respectively. As the discussion of MLMC closely follows \cite{ElLiSa:2025}, we do not provide a detailed exposition in this work. Finally, Section~\ref{sec:Num-Exp} reports numerical experiments evaluating the efficiency and accuracy of the proposed methods.

 
 Other literature
\cite{NAretz_MGunzburger_MMorlighem_KWillcox_2025a}
\cite{AAGorodetsky_GGeraci_MSEldred_JDJakeman_2020a}



% Finally, the paper concludes with Section \ref{sec:Conclusion}, summarizing the key findings and contributions. 
% An appendix is included, containing technical mathematical details and proofs relevant to the problem and methods discussed.


 
%!TEX root = main.tex
%%%%%%%%%%%%%%%%%%%%%%%%%%%%%%%%%%%%%%%%%%%%%%%%%%%


% ====================================================
\section{Multi-fidelity Monte Carlo}\label{sec:MFMC}
% ====================================================
%
MLMC methods \cite{MBGiles_2015a,SHeinrich_2001a} and 
MFMC methods  \cite{BPeherstorfer_KWillcox_MDGunzburger_2016a, BPeherstorfer_KWillcox_MDGunzburger_2018b} 
build on control variate techniques to reduce the statistical error below $(1-\theta)\epsilon^2$  
at a computational cost less than \eqref{eq:MC-work}.
While both MFMC and MLMC share the goal of variance reduction, they differ in structure and sampling.
In MLMC, corrections consisting of differences between successive levels are added to the coarse level, 
each difference between successive levels is computed with independent samples,
and a decreasing number of samples is used, the higher the level (i.e., fidelity) of the models.
MFMC, by contrast, adds corrections of low-fidelity models to the high-fidelity model and incorporates increasing numbers of inexpensive, low-fidelity samples. 
A key distinction is that MFMC reuses samples across fidelity levels. 

Next we briefly review the core principles underlying multi-fidelity Monte Carlo, drawing on \cite{BPeherstorfer_KWillcox_MDGunzburger_2016a}.  


%%%%%%%%%%%%%%%%%%%%%%%%%%%%%%%%%%%%%%%%%%%
\subsection{Multi-fidelity Monte Carlo revisited}
The MFMC framework combines a high-fidelity model $u_{h,1} = u_h \in L_{\mathbb{P}}^2(W, \cU)$ 
with a series of lower-fidelity models $u_{h,k} \in L_{\mathbb{P}}^2(W, \cU)$, $k =1,\dots, K$.
Here, as we will make precise below,
fidelity is not related to an almost everywhere pointwise error $\| u_{h,1}(\omega) - u_{h,k}(\omega) \|_U$,
but is related to the correlation between the $k$-th model $u_{h,k}$ and the high-fidelity model $u_{h,1}$.
We assume that as $k$ increases, the fidelity of the $k$-th model decreases and also
that the computational cost of evaluating the $k$-th model at a sample decreases.



For each  $u_{h,k}(\omega)$  and each pair of $u_{h,k}(\omega)$ and $u_{h,j}(\omega)$, 
we define the variance and the Pearson correlation coefficient as
\begin{equation*}
    \sigma_k^2 = \mathbb{V}\left[u_{h,k}(\omega)\right],\qquad \rho_{k,j} 
                       = \frac{\text{Cov}\left[ u_{h,k}(\omega), u_{h,j}(\omega)\right]}{\sigma_k\sigma_j}, \quad k,j=1,\dots, K,
\end{equation*}
where the covariance is 
$\text{Cov}[u_{h,k}, u_{h,j}] := \mathbb{E}[\langle u_{h,k} - \mathbb{E}[u_{h,k}], u_{h,j} - \mathbb{E}[u_{h,j}]\rangle_U]$.
By definition, $\rho_{k,k}=1$. 

Given $N_1 < N_2 < \ldots < N_K$ and i.i.d.\ samples $\omega^{(1)}, \ldots, \omega^{(N_K)}$,
the MFMC estimator $A^{\text{MF}}$ combines an MC estimate of the high-fidelity model with differences 
of MC estimates of the lower-fidelity level models.
The MFMC estimator is defined as
\begin{equation}\label{eq:MFMC_estimator}
    A^{\text{MF}}
     := A^{\text{MC}}_{1,N_1} + \sum_{k=2}^K \alpha_k\left(A^{\text{MC}}_{k,N_k} - A^{\text{MC}}_{k,N_{k-1}} \right),
\end{equation}
where
\begin{equation}\label{eq:MFMC_estimator_MCk}
     A^{\text{MC}}_{k,M} :=  \frac{1}{M} \sum_{i=1}^{M}   u_{h,k}(\omega^{(i)}), \quad M \in \{ N_{k-1}, N_k \},
\end{equation}
and $\alpha_k\in \mathbb{R}$ are weights that will be determined below.
Note that the $N_{k-1}$ evaluations $u_{h,k}(\omega^{(i)})$, $i = 1, \ldots,  N_{k-1}$,
used in  $A^{\text{MC}}_{k,N_{k-1}}$ are reused in  the computation of $A^{\text{MC}}_{k,N_k}$.
are reused. This reuse introduces statistical dependence between 
$A^{\text{MC}}_{k,N_{k-1}}$ and $A^{\text{MC}}_{k,N_k}$. 
If we denote the MC average over the $N_k - N_{k-1}$ samples $u_{h,k}(\omega^{(i)})$ 
not included in $A^{\text{MC}}_{k,N_{k-1}}$ by
\[
     A^{\text{MC}}_{k,N_k \backslash N_{k-1}}
      =  \frac{1}{N_k-N_{k-1}}  \sum_{i=N_{k-1}+1}^{N_k}   u_{h,k}(\omega^{(i)}), 
\]
the MFMC estimator \eqref{eq:MFMC_estimator} can be written as
\begin{equation}\label{eq:MFMC_estimator_independent}
    A^{\text{MF}} 
    = A^{\text{MC}}_{1,N_1} 
      +  \sum_{k=2}^K \alpha_k\left(1-\frac{N_{k-1}}{N_k}\right)
                               \left(A_{k,N_k\backslash N_{k-1}}^{\text{MC}}-A_{k,N_{k-1}}^{\text{MC}}\right).
\end{equation}
In this formulation, the two terms in each correction are evaluated on independent sample sets, 
which simplifies variance analysis. 
We also express the MFMC estimator in compact form
\begin{equation*}\label{eq:MFMC_estimator_Correction}
          A^{\text{MF}} = Y_1 + \sum_{k=2}^K \alpha_k Y_k,
\end{equation*}
where the correction terms $Y_k$ are defined as
\begin{equation} \label{eq:MFMC_Yk}
       Y_1 := A^{\text{MC}}_{1,N_1},\qquad 
       Y_k := A^{\text{MC}}_{k,N_k} - A^{\text{MC}}_{k,N_{k-1}}
               =\left(1-\frac{N_{k-1}}{N_k}\right)
                 \left(A_{k,N_k\backslash N_{k-1}}^{\text{MC}}-A_{k,N_{k-1}}^{\text{MC}}\right), \quad k=2\ldots, K.
\end{equation}
%
Because $\mathbb{E}\big[ A^{\text{MC}}_{k,M} \big] = \mathbb{E}\big[ u_{h,k} \big]$, $M \in \{ N_{k-1}, N_k \}$,
$\mathbb{E}[Y_1] =  \mathbb{E}[u_{h,1}]  = \mathbb{E}\big[ u_h \big]$ and  $\mathbb{E}[Y_k] = 0$ for $k\ge 2$.
Consequently, for any selection of weights $\alpha_k$, 
the MFMC estimator is  unbiased, satisfying $\mathbb{E}[A^{\text{MF}}] =  \mathbb{E}[u_h]$. 


The variances of the correction terms $Y_k$ are
\begin{equation}\label{eq:Var_Yk}
    \mathbb{V}\left[Y_1\right] = \frac{\sigma_1^2}{N_1}, \quad \mathbb{V}\left[Y_k\right] = \left(1-\frac{N_{k-1}}{N_k}\right)^2\left(\frac{\sigma_k^2}{N_{k-1}}+\frac{\sigma_k^2}{N_k-N_{k-1}}\right) = \left(\frac{1}{N_{k-1}} - \frac{1}{N_k}\right)\sigma_k^2.
\end{equation}
The corrections term $Y_k$, $k\ge 2$, are correlated with the high-fidelity estimator $Y_1$.
However, although $Y_k$ and $Y_j$, $2\le k<j \le K$, share overlapping sample sets and are therefore statistically dependent, they are uncorrelated.

\begin{lemma}\label{lemma:Y_k_Y_j}
Let $2\le k<j\le K$. Then 
  The correction terms $Y_k$ and $Y_j$, $2\le k<j \le K$, defined in \eqref{eq:MFMC_Yk} are uncorrelated,
  $\operatorname{Cov} [Y_k,Y_j ]=0$,  $2\le k<j \le K$.
\end{lemma}
The proof of Lemma~\ref{lemma:Y_k_Y_j} is given in Appendix~\ref{sec:proof_lemma:Y_k_Y_j}.

 Using the covariance identity derived from \cite[Lemma~3.2]{PeWiGu:2016}, yields
%
\begin{equation}\label{eq:Cov_Yk}
% \text{Cov}(Y_k,Y_j) =0,\quad \text{for } \;2\le k<j \le K,\qquad 
\text{Cov}[Y_1,Y_k] = - \left(\frac{1}{N_{k-1}} - \frac{1}{N_k}\right)\rho_{1,k}\sigma_1\sigma_k, \quad \text{for } \; k\ge 2.
\end{equation}
%
Combining \eqref{eq:Var_Yk} and \eqref{eq:Cov_Yk}, the total variance of the MFMC estimator is 
%
\begin{align}
    \nonumber
    \mathbb{V}\left[A^{\text{MF}}\right] &= \mathbb{V}\left[Y_1\right] + \mathbb{V}\left[\sum_{k=2}^K \alpha_kY_k\right]+2\;\text{Cov}\left[Y_1,\sum_{k=2}^K \alpha_k Y_k \right],\\
    \nonumber
    &=\mathbb{V}\left[Y_1\right] + \sum_{k=2}^K \alpha_k^2 \mathbb{V}\left[Y_k\right]+2\sum_{2\le k<j\le K} \alpha_k\alpha_j\; \text{Cov}[Y_k,Y_j] +2\sum_{k=2}^K \alpha_k\;\text{Cov}\left[Y_1, Y_k\right],\\
    % \nonumber
    % &=\mathbb{V}\left(Y_1\right) + \sum_{k=2}^K \alpha_k^2 \mathbb{V}\left(Y_k\right) +2\sum_{k=2}^K \alpha_k\;\text{Cov}\left(Y_1, Y_k\right),\\
    \label{eq:MFMC_variance}
    &=\frac{\sigma_1^2}{N_1} + \sum_{k=2}^K \left(\frac{1}{N_{k-1}} - \frac{1}{N_k}\right)\left(\alpha_k^2\sigma_k^2 - 2\alpha_k\rho_{1,k}\sigma_1\sigma_k\right).
\end{align}
%
The normalized mean square error of the multi-fidelity Monte Carlo estimator, $\mathcal{E}_{A^{\text{MF}}}^2$, quantifies its accuracy and is decomposed into two components -- the bias error $\mathcal{E}_{\text{Bias}}^2$ and the statistical error $\mathcal{E}_{\text{Stat}}^2$, the decomposition is written as 
%
\[
\mathcal{E}_{A^{\text{MF}}}^2= \frac{\left\Vert\mathbb{E}[u]-\mathbb{E}\left[A^{\text{MF}}\right] \right\Vert_{U}^2+\mathbb E\left[\left\Vert\mathbb{E}\left[A^{\text{MF}}\right]-A^{\text{MF}} \right\Vert_{U}^2\right]}{\left\Vert\mathbb{E}[u] \right\Vert_{U}^2} =\frac{\left\Vert\mathbb{E}[u]-\mathbb{E}\left[A^{\text{MF}}\right] \right\Vert_{U}^2}{\left\Vert\mathbb{E}[u] \right\Vert_{U}^2}+ \frac{\mathbb{V}\left[A^{\text{MF}}\right]}{\left\Vert\mathbb{E}[u] \right\Vert_{U}^2}=\mathcal{E}_{\text{Bias}}^2 + \mathcal{E}_{\text{Stat}}^2,
\]
%
where the variance term $\mathbb{V}[A^{\text{MF}}]$  can be explicitly expressed using \eqref{eq:MFMC_variance}. 

\MH{Need to revise the following.
Multifidelity methods do not have a level and their costs shouldn't be estimated using $M_L$.} \JLcolor{$M_L$ is the spatial grid point nodes that is required to satisfy the discretization error of high fidelity model. }
A splitting ratio $\theta$ is introduced as before to balance the contributions between these two components. The spatial resolution required to achieve the biased tolerance $\theta \epsilon^2$ is determined by estimating the number of spatial grid points $M_L$ at refinement level $L$, given by
\MH{Multifidelity methods do not have a level and their costs shouldn't be estimated using $M_L$.} \JLcolor{You are correct that multifidelity methods, in general, do not rely on a notion of levels. However, in my later cost estimate (Theorem 2), the parameter L appears because the low-fidelity models are constructed using a spatial hierarchy. Thus, the cost estimate is specifically tailored to the class of multifidelity models I consider. My goal is to clarify why, in this setting, the cost scales like $\epsilon^{-1}$}
%
\begin{equation}
    \label{eq:SLSGC_MLS_SpatialGridsNo}
    M_L = M_0s^{-L} \ge \left(\frac{\theta\epsilon}{c_u}\right)^{-\frac 1 {\alpha}} \qquad \text{ and } \qquad     L = \left\lceil \frac{1}{\alpha}\log_s \left(\frac{c_u M_0^\alpha}{\theta\epsilon}\right) \right\rceil,
\end{equation}
%
where $M_0$ is the number of grid points at the coarsest level, $s>1$ is the spatial refinement factor, $\alpha$ represents the convergence rate of the spatial discretization, and $c_u$ is a constant characterizing the discretization scheme. To determine the optimal sample sizes $N_k$ and control variate weights $\alpha_k$ in the MFMC estimator \eqref{eq:MFMC_estimator_independent}, we express the total computational cost for the MFMC estimator
%
\[
\mathcal{W}^{\text{MF}} = \sum_{k=1}^K C_kN_k,
\]
%
where $C_k$ is the cost of generating a single sample of model $u_{h,k}$, and $N_k$ is the corresponding sample count. Unlike previous formulations \cite{PeWiGu:2016} that derive sample sizes based on a fixed computational budget, our approach directly expresses the sample sizes and computational resources in terms of the desired accuracy $\epsilon$. This formulation offers greater flexibility in applications where accuracy targets are more relevant than rigid cost constraints. We formulate an optimization problem to determine the optimal sample sizes $N_k$ and weights $\alpha_k$ by minimizing the total sampling cost $\mathcal{W}^{\text{MF}}$, subject to three constraints. First, the normalized statistical error $\mathcal{E}_{\text{Stat}}^2$ enforces the desired estimator accuracy $(1-\theta)\epsilon^2$. Second,  the monotonicity constraints $N_{k-1}\le N_k$ for $k=2,\ldots, K$ ensures consistent sample reuse across fidelity levels. Third, all sample sizes must be non-negative. This leads to the following constrained optimization problem
%
\begin{equation}\label{eq:Optimization_pb_sample_size}
    \begin{array}{ll}
    \min \limits_{\begin{array}{c}\scriptstyle N_1,\ldots, N_K\in \mathbb{R} \\[-4pt]
\scriptstyle \alpha_2,\ldots,\alpha_K\in \mathbb{R}
\end{array}} &\displaystyle\sum\limits_{k=1}^K C_kN_k,\\
       \;\,\text{subject to} &\mathbb{V}\left[A^{\text{MF}}\right]- \epsilon_{\text{tar}}^2 = 0,\\[2pt]
       &\displaystyle -N_1\le 0,\quad \displaystyle N_{k-1}-N_k\le 0, \;\; k=2\ldots,K.
    \end{array}
\end{equation}
%
Since the finite variance implicitly indicates $N_1 > 0$, the problem remains well-posed. The solution to this problem, which yields explicit expressions for the optimal real-valued sample sizes and weights, is presented in Theorem~\ref{thm:Sample_size_est}. The proof of Theorem~\ref{thm:Sample_size_est} is provided in the appendix.




%
\begin{theorem}[Optimal MFMC Sample Allocation]
\label{thm:Sample_size_est}
Consider an ensemble of $K$ models $\{u_{h,k}\}_{k=1}^K$ each characterized by the standard deviation $\sigma_k$ of its output, the correlation coefficient $\rho_{1,k}$ with the highest-fidelity model $u_{h,1}$, and the computational cost per sample evaluation $C_k$. Define $\Delta_k = \rho_{1,k}^2 - \rho_{1,k+1}^2$ for $k = 1, \dots, K$, with the boundary convention $\rho_{1,K+1} = 0$. Assume the following conditions hold
%
\begin{alignat*}{3}
&(i)\;\; \textit{Correlation monotonicity}: \quad && |\rho_{1,1}| > \cdots > |\rho_{1,K}|, \\ 
&(ii)\;\; \textit{Cost-correlation ratio}: \quad && \frac{\Delta_k}{C_k} > \frac{\Delta_{k-1}}{C_{k-1}}, \quad k=2,\ldots,K. 
\end{alignat*}
%
Under these assumptions, the solution to the optimization problem \eqref{eq:Optimization_pb_sample_size} yields optimal weights $\alpha_k^*$ and sample sizes $N_k^*$
%
\begin{align}
    % \label{eq:MFMC_coefficients}
    % &\alpha_k^*=\frac{\rho_{1,k}\sigma_1}{\sigma_k},\\
    \label{eq:MFMC_SampleSize}
    &\alpha_k^*=\frac{\rho_{1,k}\sigma_1}{\sigma_k},\qquad \;N_k^*=\frac{\sigma_1^2}{\epsilon_\text{tar}^2}\sqrt{\frac{\Delta_k}{C_k}}\sum_{j=1}^K\sqrt{C_j\Delta_{j}}.
\end{align}
%
The resulting MFMC estimator \eqref{eq:MFMC_variance} achieves a variance of
%
\begin{equation}
\label{eq:MFMC_variance_optimal}
\mathbb{V}\left[A^{\text{MF}}\right] =
% \frac{\sigma_1^2}{N_1^*} - \sum_{k=2}^K \left(\frac{1}{N_{k-1}^*} - \frac{1}{N_k^*}\right)\rho_{1,k}^2\sigma_1^2=
\sigma_1^2\sum_{k=1}^K\frac{\Delta_k}{N_k^*},
\end{equation}
%
with total computational cost
%
\begin{equation}\label{eq:MFMC_sampling_cost}
    \mathcal{W}^\text{MF} = \sum_{k=1}^K C_k N_k^* = \frac{\sigma_1^2}{\epsilon_{\text{tar}}^2}\left(\sum_{k=1}^K\sqrt{C_k\Delta_k}\right)^2.
\end{equation}
%
\end{theorem}

In practical implementation, the correlation coefficients $\rho_{1,k}$ and computational costs $C_k$ are typically unknown a priori and must be estimated via pilot sampling. Additionally, the theoretically optimal sample sizes $N_k^* \in \mathbb{R}$ require integer rounding for implementation. Departing from the conditional rounding strategy in \cite{GrGuJuWa:2023, PeWiGu:2016} (floor function when $N_k^* \ge 1$, ceiling otherwise), we adopt a uniform rounding scheme: all sample sizes are rounded up using the ceiling function $\lceil N_k^* \rceil$. This approach guarantees $\mathbb{V}[A^{\mathrm{MF}}] \leq \epsilon_{\mathrm{tar}}^2$ since increased sample sizes reduce estimator variance in \eqref{eq:MFMC_variance_optimal}. The computational cost after rounding satisfies
%
\begin{equation}\label{eq:sampling_cost_bound}
    \sum_{k=1}^K C_k N_k^*\le \sum_{k=1}^K C_k \left\lceil N_k^*\right\rceil<\sum_{k=1}^K C_k N_k^* + \sum_{k=1}^K C_k,
\end{equation}
%
where the additive overhead $\sum_{k=1}^K C_k$ arises from the bound $N_k^*\le \lceil N_k^*\rceil< N_k^*+1$. Under Theorem~\ref{thm:Sample_size_est}'s cost-correlation ratio assumption, optimal sample sizes exhibit strict monotonicity $N_1^* < \cdots < N_K^*$. We impose the feasibility condition $\sum_{k=1}^K C_k N_k^* \geq \sum_{k=1}^K C_k$ to exclude degenerate cases where $N_k^* < 1$ for all $k$ models; in such cases, the additive rounding overhead is asymptotically dominated by $\sum_{k=1}^K C_k N_k^*$. Consequently, integer-rounded cost preserve the asymptotic scaling of \eqref{eq:MFMC_sampling_cost}.

% define $B_k := C_k(\rho_{1,k}^2 - \rho_{1,k+1}^2)$ for $k=1,\dots,K$ with $\rho_{1,K+1} = 0$. Condition (ii) of Theorem~\ref{thm:Sample_size_est} implies
% % Substituting into the sampling cost expression,  \eqref{eq:MFMC_sampling_cost} becomes
% % %
% % \begin{equation*}\label{eq:MFMC_sampling_cost_2}
% %     \mathcal{W}^{\text{MF}} = \sum_{k=1}^K C_k N_k^* = \frac{\sigma_1^2}{\epsilon_{\text{tar}}^2}\left(\sum_{k=1}^K\sqrt{B_k} \right)^2.
% % \end{equation*}
% %
% % The quantity $B_k$ depends on the product of the cost per sample $C_k$ and the difference between two successive correlations $(\rho_{1,k}^2 - \rho_{1,k+1}^2)$. Depending on how these components interact, $B_k$ may decay, grow, or remain constant as $k$ increases.
% %
% \begin{equation}
% \label{eq:Bk_Ck_decay_rate}
%     \frac{\sqrt{B_k}}{\sqrt{B_{k-1}}}>\frac{C_k}{C_{k-1}}, \quad k=2,\ldots,K.
% \end{equation}
% %
% This inequality induces a strictly increasing sequence $\{\sqrt{B_k}/C_k\}_{k=1}^K$. Consequently, as $K \to \infty$, the sequence $\sqrt{B_k}$ decays slower (or grows faster) than $C_k$ in relative terms. Combined with the feasibility condition ($\sum_{k=1}^K C_kN_k^*\ge \sum_{k=1}^K C_k$), this ensures the overhead $\sum_{k=1}^K C_k$ is asymptotically negligible compared to $\sum_{k=1}^K C_k N_k^* \propto \left( \sum_{k=1}^K \sqrt{B_k} \right)^2$. 


% This inequality implies that, in the asymptotic regime where $K$ is large,  the sequence $\sqrt{B_k}$ decays more slowly -- or grows more rapidly -- than the cost sequence $C_k$, regardless of the specific trend of $\sqrt{B_k}$. \JLcolor{Using this fact and the assumption that $\sum_{k=1}^K C_kN_k^*\ge \sum_{k=1}^K C_k$}, the additive overhead term $\sum_{k=1}^K C_k$ in the cost bounds becomes asymptotically negligible relative to the leading-order term $\sum_{k=1}^K C_kN_k^*$. 


% Using the fact that $N_k$ increases and the value of $\alpha_k$, we observe that the MFMC estimator variance $\mathbb{V}\left(A^{\text{MFMC}}\right)$ in \eqref{eq:MFMC_variance2} always decreases as the model number $K$ increases. This reflects the fact that the low fidelity models are used as control variates to reduce the variance of the high fidelity model. However, this $K$ cannot be arbitrarily large, since the first summation term in \eqref{eq:MFMC_sampling_cost} grows, the second summation reflect the variance decay of the MFMC estimator. Thus this is a tie between these two terms. If $K$ is sufficiently large,  in order to achieve an optimal sampling cost, we need to study the decay and growth of these two terms. We will choose the $K$ such that the product of two summation terms in \eqref{eq:MFMC_sampling_cost} is minimum, i.e. If $K$ is sufficiently large, we need to find $K\in \mathbb{N}$ such that 
% \begin{equation}\label{eq:Optimal_K}
%    K = \text{argmin} \sum_{k=1}^K\sqrt{\left(\rho_{1,k}^2 - \rho_{1,k+1}^2\right)C_k}\sum_{k=1}^K\left(\sqrt{\frac{C_k}{\rho_{1,k}^2 - \rho_{1,k+1}^2}} - \sqrt{\frac{C_{k-1}}{\rho_{1,{k-1}}^2 - \rho_{1,k}^2}}\right)\rho_{1,k}^2. 
% \end{equation}
The efficiency gain relative to standard Monte Carlo is quantified through the cost ratio
%
\begin{equation}\label{eq:MFMC_sampling_cost_efficiency}
    \xi(\boldsymbol{\rho}) = \frac{\mathcal{W}^\text{MF}}{\mathcal{W}^\text{MC}} = \frac{1}{C_1} \left(\sum_{k=1}^K\sqrt{C_k\left(\rho_{1,k}^2 - \rho_{1,k+1}^2\right)}\right)^2,
\end{equation}
%
where $\boldsymbol{\rho} = (\rho_{1,1},\ldots, \rho_{1,K})$ is the correlation coefficient vector and smaller $\xi$ indicates greater efficiency gains for the MFMC estimator.


% Further more, we observe that
% \begin{align*}
%     \mathcal{W}_\text{MC}\mathbb{V}\left(A^{\text{MC}}\right) &=\frac{C_1\sigma_1^2}{\left\Vert\mathbb{E}(f_1) \right\Vert_{U}^2},\\
%  \mathcal{W}_\text{MFMC}\mathbb{V}\left(A^{\text{MFMC}}\right) &=  \frac{\sigma_1^2}{\left\Vert\mathbb{E}(f_1) \right\Vert_{U}^2}\sum_{k=1}^K\sqrt{\left(\rho_{1,k}^2 - \rho_{1,k+1}^2\right)C_k}\sum_{k=1}^K\left(\sqrt{\frac{C_k}{\rho_{1,k}^2 - \rho_{1,k+1}^2}} - \sqrt{\frac{C_{k-1}}{\rho_{1,{k-1}}^2 - \rho_{1,k}^2}}\right)\rho_{1,k}^2.
% \end{align*}
% This implies that if both Monte Carlo and multifidelity Monte Carlo have  a same sampling cost, then $\mu=  \mathbb{V}\left(A^{\text{MFMC}}\right)/\mathbb{V}\left(A^{\text{MC}}\right)$. Therefore, 

 
%!TEX root = ../main.tex
% ====================================================
\section{Optimal Sample Size Allocation}\label{sec:MFMC_Nk_optimize}
% ====================================================

% ====================================================
\subsection{Integer Programming Formulation}  \label{sec:MFMC_Nk_optimize_IP}
% ====================================================
The optimal samples $1 \le m_1 \le m_2 \le \ldots \le m_K$ are computed to minimize
the variance \eqref{eq:MFMC_variance} of the MFMC estimator
subject to (s.t.) a constraint on the cost \eqref{eq:MFMC_cost} to execute the MFMC estimator.
A first version of the optimization problem to compute $1 \le m_1 \le m_2 \le \ldots \le m_K$ is
\begin{subequations}\label{eq:Optimization_sample_size_N}
    \begin{align}
    \min \quad &\sigma_1^2  \sum_{k=1}^K \frac{ \rho_{1,k}^2 - \rho_{1,k+1}^2}{m_k},   \\
       \text{s.t.}\quad & \sum_{k=1}^K C_km_k \le p,       \label{eq:Optimization_sample_size_m_budget}  \\
                                & m_1\ge 1,\quad  m_k \ge m_{k-1}, \quad k=2\ldots,K,\\
                                &m_1,\ldots, m_K\in \nat.
    \end{align}
\end{subequations}
The issue with this formulation is that if $m_k = m_{k-1}$, the $k$-th model
does not contribute to the variance, which is easier to see from \eqref{eq:MFMC_variance_a}.
Therefore, if $m_k = m_{k-1}$, the $k$-th model should not be executed, but it still contributes to
the computing budget \eqref{eq:Optimization_sample_size_m_budget}.
To fix this issue we introduce binary variables $z_2, \ldots, z_K \in \{0,1\}$ such that
$z_k = 1$ if $m_k > m_{k-1}$ and model $k$ will be sampled $m_k$ times, and 
$z_k = 0$ if $m_k = m_{k-1}$ and model $k$ will be skippted.
The optimization formulation requires 
an upper bound $M$ for all possible differences $m_k - m_{k-1}$, $k=2\ldots,K$.
For example, $M = p/ C_K$ is such an upper bound because  $p/ C_K \ge m_K \ge m_k - m_{k-1}$, $k=2\ldots,K$.
The optimization problem formulation for the optimal sample size selection is
\begin{subequations}\label{eq:Optimization_sample_size_Nz}
    \begin{align}
    \min \quad &\sigma_1^2  \sum_{k=1}^K \frac{ \rho_{1,k}^2 - \rho_{1,k+1}^2}{m_k},   \\
       \text{s.t.}\quad &  C_1 m_1 + \sum_{k=2}^K z_k C_k m_k \le p,       \label{eq:Optimization_sample_size_Nz_budget}  \\
                                & m_1\ge 1,\quad  m_k \ge m_{k-1}, \quad k=2\ldots,K,\\
                                & m_k - m_{k-1} \ge z_k, \quad M z_k \ge m_k - m_{k-1},  \quad k=2\ldots,K,  \label{eq:Optimization_sample_size_Nz_z}  \\
                                &m_1,\ldots, m_K\in \nat, \quad z_2 ,\ldots, z_K\in \{0,1\}.
    \end{align}
\end{subequations}
The constraints \eqref{eq:Optimization_sample_size_Nz_z} ensures that $z_k = 0$ if $m_k = m_{k-1}$ and 
$z_k = 1$ if $m_k >  m_{k-1}$. If $z_k = 0$, model $k$ will not be executed and does not contribute to the 
computational cost \eqref{eq:Optimization_sample_size_Nz_budget}.




% ====================================================
\subsection{Relaxation}  \label{sec:MFMC_Nk_optimize_relax}
% ====================================================
Instead of solving the integer programming problem \eqref{eq:Optimization_sample_size_Nz} or
even \eqref{eq:Optimization_sample_size_N}, previous papers including
\cite{BPeherstorfer_KWillcox_MDGunzburger_2016a} and \cite{AGruber_MGunzburger_LJu_ZWang_2023a}
have considered a relaxation  and then used rounding to obtain integer sample sizes.
Specifically, \cite{BPeherstorfer_KWillcox_MDGunzburger_2016a}
considered the problem
\begin{subequations}\label{eq:Optimization_sample_size_m_relaxed}
    \begin{align}
    \min \quad &\sigma_1^2  \sum_{k=1}^K \frac{ \rho_{1,k}^2 - \rho_{1,k+1}^2}{m_k},   \\
       \text{s.t.}\quad & \sum_{k=1}^K C_km_k \le p,       \\
                                & m_1\ge 0,\quad  m_k \ge m_{k-1}, \quad k=2\ldots,K,\\
                                &m_1,\ldots, m_K\in \real.
    \end{align}
\end{subequations}
The formulation \eqref{eq:Optimization_sample_size_m_relaxed} potentially suffers from the same isses as 
\eqref{eq:Optimization_sample_size_N}, namely that if f $m_k = m_{k-1}$, the $k$-th model does not
provide variance reduction, but it's cost is included. 
However, under conditions specified in the following Theorem~\ref{thm:Sample_size_real}, the solution
of \eqref{eq:Optimization_sample_size_m_relaxed} can be computed analytically, and it satisfies
$0  < m_1^* < m_2^* <  \ldots < m_K^*$.


The following theorem is proven in \cite[Th.~3.4]{BPeherstorfer_KWillcox_MDGunzburger_2016a}.

\begin{theorem}[Optimal MFMC Real-Valued Sample Allocation]   \label{thm:Sample_size_real}
   Let models $u_1, \ldots, u_k \in   L_{\mathbb{P}}^2(W, {\mathcal U})$ with
   standard deviations $\sigma_k$, correlation coefficients $\rho_{1,k}$, $k = 2, \ldots, K$,
   between the HF model $u_1$  and the  LF models $u_k$, $k = 2, \ldots, K$, and 
   per-sample costs $C_1, \ldots, C_K$ be given.
   If 
   \begin{subequations}\label{eq:Sample_size_real_assumptions}
   \begin{align}
       \label{eq:Sample_size_real_assumptions_a}
        & |\rho_{1,1}| > \cdots > |\rho_{1,K}|, & \text{(monotone correlations)} \\
      \label{eq:Sample_size_real_assumptions_b}
        & \frac{ \rho_{1,k}^2 - \rho_{1,k+1}^2 }{C_k} > \frac{ \rho_{1,k-1}^2 - \rho_{1,k}^2 }{C_{k-1}}, \quad k=2,\ldots,K, 
                                                                  & \text{(cost-correlation ratio)}
    \end{align}
    \end{subequations}
    hold, where $\rho_{1,K+1} :=0$, then the solution of \eqref{eq:Optimization_sample_size_m_relaxed} is
    \begin{equation}\label{eq:MFMC_RealValued_Sample_Size}
             m_k^* = \sqrt{\frac{\rho_{1,k}^2 - \rho_{1,k+1}^2}{C_k}} \; 
                          \frac{p}{\sum_{j=1}^K \sqrt{C_j (\rho_{1,j}^2 - \rho_{1,j+1}^2)}},
     \end{equation}
    the cost constraint is active $\sum_{k=1}^K C_km_k^* = p$, 
    and the resulting minimal variance of the MFMC estimator is
     \begin{equation}\label{eq:MFMC_variance_optimal}
           \mathcal{V}^{\text{MF}}(m_1^*, \ldots, m_K^*)
           =   \sigma_1^2  \sum_{k=1}^K \frac{ \rho_{1,k}^2 - \rho_{1,k+1}^2}{m_k^*}
           =  \frac{\sigma_1^2}{p}\!\left(\sum_{k=1}^K \sqrt{C_k (\rho_{1,k}^2 - \rho_{1,k+1}^2) }\right)^{\!2}.
       \end{equation}
\end{theorem}

Note that because of the cost-correlation ratio assumption, the  solution \eqref{eq:MFMC_RealValued_Sample_Size}
satisfies $0  < m_1^* < m_2^* <  \ldots < m_K^*$.
To obtain integer samples,  \cite[p.~A3171]{BPeherstorfer_KWillcox_MDGunzburger_2016a}
round down, i.e., use $\lfloor m_1^* \rfloor, \ldots, \lfloor m_K^* \rfloor$. 
Rounding down will reduce the cost, i.e., the rounded down sample sizes are still feasible for
\eqref{eq:Optimization_sample_size_m_relaxed}. Rounding down increases the variance.
Rounding may lead to $\lfloor m_1^* \rfloor = 0$, which introduces bias,  
$\mathbb{E}[A^{\text{MF}}] \not=  \mathbb{E}[u_1]$,
or may lead to $\lfloor m_k^* \rfloor = \lfloor m_{k-1}^* \rfloor$ for some $k \in \{ 2, \ldots, K\}$, 
which means the $k$-th model does not contribute to variance reduction, but its cost is included in the
computational budget.

The case $\lfloor m_1^* \rfloor = 0$ can happen if the computational budget $p$ is small.
If $0< m_1^* < 1$, \cite{AGruber_MGunzburger_LJu_ZWang_2023a} use $\lceil m_1^* \rceil = 1$,
and iterative recompute integer sample sizes from a modified version of 
Theorem~\ref{thm:Sample_size_real}. See Algorithm~2 in \cite{AGruber_MGunzburger_LJu_ZWang_2023a}.
However, their iterative sample size computation can generate integer sample sizes
with $1 = m_1 =  m_2 = \ldots = m_k$ for some $l \in \{ 2, \ldots, K\}$. 
See Tables~1 and 2 in \cite{AGruber_MGunzburger_LJu_ZWang_2023a}.
In this case, the models $2$ to $k$ do not contribute to variance reduction, but their cost is included in the
computational budget.

The issues using the rounded-down solution of \eqref{eq:MFMC_RealValued_Sample_Size},
namely, that $\lfloor m_1^* \rfloor = 0$ or  $\lfloor m_k^* \rfloor = \lfloor m_{k-1}^* \rfloor$ for some $k \in \{ 2, \ldots, K\}$, 
is more likely to occur when the total computational budget $p$ is small relative to the cost $C_1$ of the HF.
See, e.g., the numerical examples in  \cite{BPeherstorfer_KWillcox_MDGunzburger_2016a} or \cite{AGruber_MGunzburger_LJu_ZWang_2023a}.

Generally, because $m_k^*-1 < \lfloor m_k^*\rfloor \le m_k^*$, the floor operation induces bounded 
sub-optimality, producing the bounds
%
\begin{subequations}\label{eq:bounds_for_floor}
\begin{align}
    \mathcal{V}^{\text{MF}}\left(\lfloor m_1^* \rfloor, \ldots \lfloor m_K^* \rfloor \right)
    & \in \left[\frac{\sigma_1^2}{p}\left(\sum_{k=1}^K \sqrt{C_k  (\rho_{1,k}^2 - \rho_{1,k+1}^2)}\right)^2,
                   \sum_{k=1}^K\frac{ (\rho_{1,k}^2 - \rho_{1,k+1}^2)}{m_k^*-1}\right),       \\
   \mathcal{W}^{\text{MF}}\left(\lfloor m_1^* \rfloor, \ldots \lfloor m_K^* \rfloor \right)
   &\in \left( p-\sum_{k=1}^K C_k, p\right].
\end{align}
\end{subequations}
If $\sum_{k=1}^K C_k \ll p$, the work 
$\mathcal{W}^{\text{MF}}\big(\lfloor m_1^* \rfloor, \ldots, \lfloor m_K^* \rfloor \big)
 \approx  \mathcal{W}^{\text{MF}}\big( m_1^* , \ldots,  m_K^* \big) = p$.



% ====================================================
\subsection{Model selection}
% ====================================================
\MH{may not need a subsection on this. Can simply reference}
The analytical solution of \eqref{eq:Optimization_sample_size_m_relaxed} in 
Theorem~\ref{thm:Sample_size_real} requires that the models satsfy 
\eqref{eq:Sample_size_real_assumptions}.
Thus, given models, we need to select the models from the available set such that the parameters 
associated with the selected models satisfy the two conditions in Theorem \ref{thm:Sample_size_real}, as well as the $\mathcal{V}^{\text{MF}}$ as small as possible (Note that minimize \eqref{eq:MFMC_variance_optimal} is a selection of model that associated with 
different cost and $\rho_{1,k}^2 - \rho_{1,k+1}^2$, 
and is independent of budget $p$). Let $\mathcal{S}^*=\{1, \ldots, K^*\}$ be the indices of $K^*$ available models. 
We seek a subset $\mathcal{S}=\{i_1,i_2, \ldots,i_{K}\}\subseteq \mathcal{S}^* (K\le K^*)$ of indices that minimizes the sampling cost of multifidelity Monte Carlo estimator. Note that $\mathcal{S}$ is non-empty and $i_1=1$ since the high fidelity model must be included. 
We will follow the exhaustive algorithm in \cite[Algorithm~1]{BPeherstorfer_KWillcox_MDGunzburger_2016a} 
for $2^{K^*-1}$ subsets of $\mathcal{S}^*$.  This algorithm gives the indices of the selected model.

\normalem
\begin{algorithm}[!ht]
\label{algo:MFMC_Algo_model_selection}
\DontPrintSemicolon    
   \KwIn{Models $u_1, \ldots, u_{K^*}$ ordered such that $\rho_{1,2}^2 \ge \ldots \ge \rho_{1,K^*}$,
             and corresponding sample costs  $C_1, \ldots, C_{K^*}$.}\vspace{1ex}
    
    \KwOut{ Selected index set $\mathcal{S}$.}\vspace{1ex}
    \hrule \vspace{1ex}

   % Estimate $\rho_{1,k}$ and $C_k$ for each model $f_k$ using $m_0$ samples.
   
   
   Set $\mathcal{S}=\{1,\ldots, K^*\}$. 
   
   Initialize $v_{\min}=C_1$, $\mathcal{S}=\{1\}$. Let $ \mathcal{\widehat S}$ be all $2^{K-1}$ ordered subsets of $\mathcal{S}^*$, each containing the high fidelity model with index $1$. 
   % Set $ \mathcal{\widehat S}_1=\mathcal{S}^*$.

    % $(2 \le j \le 2^{K-1})$
    \For{each subset $\mathcal{\widehat S}_j$\,}{

    {
    \If{ Condition \eqref{eq:Sample_size_real_assumptions_b} from Theorem \ref{thm:Sample_size_real} is satisfied}{
    Compute $\Delta_k$ and $v = \left(\sum_{k=1}^K \sqrt{C_k (\rho_{1,k}^2 - \rho_{1,k+1}^2) }\right)^{\!2}$.
    \MH{What is $K$?}
    
    \If{$v<v_{\min}$}{
    {
    Update $\mathcal{S} = \mathcal{\widehat S}_j$ and $v_{\min} = v$.
    }
    } 
    }
    }
    $j=j+1$.
    }
    Return  $\mathcal{S}$.
\caption{Multi-fidelity Model Selection}
\end{algorithm}
\ULforem












% ====================================================
\section{Iterative sample size estimation for MFMC}\label{sec:Iterative_IntegerValued_Sample_Size}
% ====================================================
To address this question, we propose an iterative scheme for estimating sample sizes that preserves the total computational budget while naturally extending the real-valued MFMC formulation.  
Starting from the standard MFMC sample allocation \cite{PeGuWi:2018}, we define the iterative real-valued sample size sequence as
%
\begin{equation}
    \label{eq:MFMC_New_RealValued_Sample_Size}
    H_1^* = \sqrt{\frac{\Delta_1}{C_1}}\frac{p}{\sum_{j=1}^K\sqrt{C_j\Delta_j}}, 
    \qquad 
    H_k^* = \sqrt{\frac{\Delta_k}{C_k}}\frac{p-\sum_{j=1}^{k-1}C_jH_j^*}{\sum_{j=k}^K\sqrt{C_j\Delta_j}}, 
    \quad k = 2,\ldots, K.
\end{equation}
%
The next theorem establishes that this iterative construction yields exactly the same real-valued solution as the standard MFMC sample size \eqref{eq:MFMC_RealValued_Sample_Size}.  
Consequently, the total cost and normalized variance remain unchanged.


\begin{theorem}[Iterative real-valued sample size for MFMC]\label{thm:MFMC_Iteravie_RealValued_Sample_Size}

For the iterative real-valued sample size defined in \eqref{eq:MFMC_New_RealValued_Sample_Size}, 
the resulting values coincide with the standard real-valued MFMC sample sizes in 
\eqref{eq:MFMC_RealValued_Sample_Size}, i.e.,
%
\[
H_k^* = N_k^*
    = \sqrt{\frac{\Delta_k}{C_k}}\,
      \frac{p}{\sum_{j=1}^K \sqrt{C_j\Delta_j}}.
\]
%
Consequently,
%
\[
\sum_{k=1}^K C_k H_k^* = p, 
\qquad  
f(H_k^*) = \sum_{k=1}^K \frac{\Delta_k}{H_k^*} 
= \frac{1}{p} \left(\sum_{k=1}^K \sqrt{C_k\Delta_k}\right)^2.
\]
%
\end{theorem}




\begin{proof}
Define the partial sum and remainder
\[
    T_k = \sum_{j=1}^k C_j H_j^*, 
    \qquad 
    R_k = p - T_k,
\]
For $k=0$ and $k=1$, we have
\[
    T_0 = 0, 
    \qquad 
    T_1 = C_1H_1^* 
    = p\frac{\sqrt{C_1\Delta_1}}{\sum_{j=1}^K \sqrt{C_j\Delta_j}},
    \qquad
    R_0 = p,
    \qquad
    R_1 = p-T_1
    = p\frac{\sum_{j=2}^K \sqrt{C_j\Delta_j}}{\sum_{j=1}^K \sqrt{C_j\Delta_j}}.
\]
For general $k\ge 1$, the iterative definition \eqref{eq:MFMC_New_RealValued_Sample_Size} gives
\[
    H_k^*
    = \sqrt{\frac{\Delta_k}{C_k}}\,
      \frac{R_{k-1}}{\sum_{j=k}^K \sqrt{C_j \Delta_j}},
\]
%
and therefore
%
\[
    R_k 
    = R_{k-1} - C_k H_k^*
    = \frac{\sum_{j=k+1}^K \sqrt{C_j \Delta_j}}
           {\sum_{j=k}^K \sqrt{C_j \Delta_j}} \, R_{k-1}.
\]
%
Hence $R_k$ forms a geometric sequence, from which it follows that
%
\[
R_k=\frac{p}{S}\sum_{j=k+1}^K\sqrt{C_j\Delta_j}.
\]
%
where the aggregate cost–variance weight $S$ serves as a normalization factor that balances the contributions of model cost and variance reduction across all fidelity levels and is defined as
%
\begin{equation}\label{eq:aggregate_cost–variance_weight_S}
    S = \sum_{j=1}^K \sqrt{C_j \Delta_j}.
\end{equation}
%
Substituting this relation into the expression for $H_k^*$ yields
%
\[
    H_k^*
    = \frac{p}{S}\sqrt{\frac{\Delta_k}{C_k}}.
\]
In particular, for $k = K$ we obtain $R_K = 0$, implying
\[
    T_K = p - R_K = p,
\]
which verifies that $\sum_{k=1}^K C_k H_k^* = p$. Finally,
\[
    f(H_k^*)
    = \sum_{k=1}^K \frac{\Delta_k}{H_k^*}
    = \frac{1}{p} \left(\sum_{k=1}^K \sqrt{C_k\Delta_k}\right)^2.
\]
\end{proof}
%

Now we formulate an iterative scheme for computing the \textit{integer-valued} sample sizes in MFMC. We introduce the following real-valued proxy sequence, unlike the remaining budget in \eqref{eq:MFMC_New_RealValued_Sample_Size} where the remaining budget $p$ is subtract from the total budget with real-valued sample size, here the remaining budget $p$ is subtract from the total budget with integer-valued sample size
%
\begin{equation}
    \label{eq:MFMC_New_IntegerValued_Sample_Size}
    M_1^* = \sqrt{\frac{\Delta_1}{C_1}}\frac{p}{\sum_{j=1}^K\sqrt{C_j\Delta_j}}, 
    \qquad 
    M_k^* = \sqrt{\frac{\Delta_k}{C_k}}\frac{p-\sum_{j=1}^{k-1}C_j\left\lfloor M_j^* \right\rfloor}{\sum_{j=k}^K\sqrt{C_j\Delta_j}}, 
    \quad k = 2,\ldots, K.
\end{equation}
%
The integer-valued sample sizes are then given by $\lfloor M_k^* \rfloor$ for $k = 1,\ldots, K$.  
This procedure begins with the standard real-valued MFMC solution $M_1^*$ in \eqref{eq:MFMC_RealValued_Sample_Size}, from which the integer allocation $\lfloor M_1^* \rfloor$ is obtained.  
The cost associated with $\lfloor M_1^* \rfloor$ samples is subtracted from the total budget, and the same allocation principle \eqref{eq:MFMC_RealValued_Sample_Size} is applied to the remaining budget to compute $\lfloor M_2^* \rfloor$, and so on.  
In this way, integer-valued sample sizes are determined iteratively while respecting the total budget constraint.

In particular, Theorem \ref{thm:MFMC_New_IntegerValued_Cost} shows that the iterative allocation scheme ensures that the accumulated cost does not exceed the prescribed budget, while improving upon the direct flooring approach in terms of budget utilization.


\begin{theorem}[Cost bound for the iterative integer-valued sample allocation]
\label{thm:MFMC_New_IntegerValued_Cost} 

Let $\lfloor M_k^* \rfloor$ denote the integer-valued sample sizes obtained from the iterative scheme \eqref{eq:MFMC_New_IntegerValued_Sample_Size}, and let $\lfloor N_k^* \rfloor$ denote those obtained by directly flooring the standard real-valued MFMC sample allocation \eqref{eq:MFMC_RealValued_Sample_Size}. 
Assume that the prescribed computational budget satisfies 
%
\begin{equation}\label{eq:p_bound}
     p \ge \sum_{k=1}^K C_k.
\end{equation}
%
Then, the total cost associated with the iterative integer-valued allocation is bounded by
\begin{equation}\label{eq:Iterative_integer_sample_size_cost_bound}
    \sum_{k=1}^K C_k \left\lfloor N_k^* \right\rfloor
    \;\le\;
    \sum_{k=1}^K C_k \left\lfloor M_k^* \right\rfloor
    \;\le\;
    p.
\end{equation}
\end{theorem}


\begin{proof}
We first establish that the total cost of the iterative scheme does not exceed the prescribed budget, i.e.,
\[
\sum_{k=1}^K C_k \left\lfloor M_k^* \right\rfloor \le p.
\]
Define the cumulative integer cost up to level $k$ as
\[
T_k = \sum_{j=1}^k C_j\left\lfloor M_j^* \right\rfloor.
\]
Since $\lfloor M_j^* \rfloor \le M_j^*$, we claim, and prove by induction, that for each $k = 1, \ldots, K$,
\begin{equation}\label{eq:Tk_bound}
T_k \le \frac{p}{S}\sum_{j=1}^k \sqrt{C_j \Delta_j}.
\end{equation}
where $S$ is defined in \eqref{eq:aggregate_cost–variance_weight_S}. Inequality \eqref{eq:Tk_bound} shows that the cumulative integer cost up to level $k$ is bounded by a proportional share of the total budget, scaled by $S$.




The base case $k=1$ follows immediately,
\[
T_1=C_1 \left\lfloor M_1^* \right\rfloor \le C_1M_1^* = \frac{p}{S}\sqrt{C_1\Delta_1},
\]
so \eqref{eq:Tk_bound} holds for \(k=1\). Assume \eqref{eq:Tk_bound} holds for \(k-1\). By definition of \(M_k^*\),
%
\[
M_k^* = \sqrt{\frac{\Delta_k}{C_k}}\frac{p - T_{k-1}}{\sum_{j=k}^K \sqrt{C_j\Delta_j}},
\]
%
and hence
%
\[
C_k \left\lfloor M_k^* \right\rfloor \le C_k M_k^*  = \sqrt{C_k\Delta_k}\frac{p-T_{k-1}}{\sum_{j=k}^K\sqrt{C_j\Delta_j}}.
\]
%

Using the inductive hypothesis and simplifying the resulting algebraic expression yields
\begin{align*}
    T_k &= T_{k-1}+C_k\left\lfloor M_k^* \right\rfloor \\
    &\le T_{k-1} + \sqrt{C_k\Delta_k}\frac{p-T_{k-1}}{\sum_{j=k}^K\sqrt{C_j\Delta_j}}
    =T_{k-1}\left(1-\frac{\sqrt{C_k\Delta_k}}{\sum_{j=k}^K\sqrt{C_j\Delta_j}}\right) + p\frac{\sqrt{C_k\Delta_k}}{\sum_{j=k}^K\sqrt{C_j\Delta_j}}\\
    &\le \frac{p}{S}\sum_{j=1}^{k-1} \sqrt{C_j\Delta_j}\frac{\sum_{j=k+1}^K\sqrt{C_j\Delta_j}}{\sum_{j=k}^K\sqrt{C_j\Delta_j}}+p\frac{\sqrt{C_k\Delta_k}}{\sum_{j=k}^K\sqrt{C_j\Delta_j}}=\frac{p}{S}\cdot \frac{\sum_{j=1}^{k-1} \sqrt{C_j\Delta_j}\sum_{j=k+1}^K\sqrt{C_j\Delta_j}+S\sqrt{C_k\Delta_k}}{\sum_{j=k}^K\sqrt{C_j\Delta_j}}\\
    &=\frac{p}{S}\cdot \frac{\sum_{j=1}^{k-1} \sqrt{C_j\Delta_j}\sum_{j=k+1}^K\sqrt{C_j\Delta_j}+\sqrt{C_k\Delta_k}\left(\sum_{j=1}^{k-1}\sqrt{C_j\Delta_j}+\sum_{j=k}^K\sqrt{C_j\Delta_j}\right)}{\sum_{j=k}^K\sqrt{C_j\Delta_j}}
    %=\frac{p}{S}\frac{\sum_{j=k}^K\sqrt{C_j\Delta_j}\sum_{j=1}^k\sqrt{C_j\Delta_j}}{\sum_{j=k}^K\sqrt{C_j\Delta_j}}
    =\frac{p}{S}\sum_{j=1}^k\sqrt{C_j\Delta_j},
\end{align*}
%
which completes the inductive step. Thus, inequality \eqref{eq:Tk_bound} holds for all $k$. 
In particular, when $k=K$, we obtain
\begin{equation}\label{eq:MFMC_iterative_total_cost}
T_K = \sum_{j=1}^K C_j\left\lfloor M_j^*\right\rfloor \le p,
\end{equation}
confirming that the iterative scheme never exceeds the prescribed computational budget. To establish the lower bound in \eqref{eq:Iterative_integer_sample_size_cost_bound}, we compare the auxiliary sequences $M_k^*$ and $N_k^*$. Using \eqref{eq:Tk_bound}, we have
%
\[
M_k^* = \sqrt{\frac{\Delta_k}{C_k}}\frac{p - T_{k-1}}{\sum_{j=k}^K\sqrt{C_j\Delta_j}} \ge \sqrt{\frac{\Delta_k}{C_k}}\frac{p-\frac{p}{S}\sum_{j=1}^{k-1}\sqrt{C_j\Delta_j}}{\sum_{j=k}^K\sqrt{C_j\Delta_j}} = \sqrt{\frac{\Delta_k}{C_k}}\frac{p}{S}=N_k^*, \qquad k \ge 1.
\]
% 
Monotonicity of the floor function yields \(\lfloor M_k^*\rfloor\ge\lfloor N_k^*\rfloor\) for every \(k\), and summing after multiplying by \(C_k\) gives the desired lower bound in \eqref{eq:Iterative_integer_sample_size_cost_bound}.  Combining this result with \eqref{eq:MFMC_iterative_total_cost} completes the proof.

\vspace{4mm}
\noindent{{\it When equality holds.}}
The upper bound in \eqref{eq:Iterative_integer_sample_size_cost_bound} is attained if and only if no budget remains unused, i.e.,
\[
\sum_{k=1}^K C_k \left(M_k^* - \left\lfloor M_k^*\right\rfloor\right) = 0.
\]
Because each fractional part satisfies $M_k^* - \lfloor M_k^* \rfloor \in [0,1)$, this condition holds precisely when every $M_k^*$ is an integer. Hence, equality occurs when all $M_k^* \in \mathbb{Z}$.

The lower bound is attained if and only if $\lfloor M_k^* \rfloor = \lfloor N_k^* \rfloor$ for every $k$, or equivalently, when both real-valued allocations $M_k^*$ and $N_k^*$ lie in the same integer interval:
\[
\left\lfloor N_k^*\right\rfloor\le M_k^* < \left\lfloor N_k^*\right\rfloor + 1.
\]
Since $M_k^* \ge N_k^*$, this occurs precisely when the two share the same integer part for all $k$.
\end{proof}



Theorem \ref{thm:MFMC_New_IntegerValued_Variance} shows that the iterative integer-valued allocation achieves a variance no smaller than the continuous optimum but no greater than that obtained by direct flooring, providing a better variance–cost tradeoff under a fixed budget.

\begin{theorem}[Normalized variance bound for the iterative integer-valued sample allocation]
\label{thm:MFMC_New_IntegerValued_Variance}

Let $\lfloor M_k^* \rfloor$ denote the integer-valued sample sizes obtained from the iterative allocation scheme \eqref{eq:MFMC_New_IntegerValued_Sample_Size}, and let $\lfloor N_k^* \rfloor$ denote those obtained by directly flooring the real-valued optimal allocation \eqref{eq:MFMC_RealValued_Sample_Size}. 
Assume that the computational budget $p$ satisfies \eqref{eq:p_bound}, and that the variance-related quantities $\Delta_k$ are defined as in Theorem~\ref{thm:Sample_size_est} and meet the same conditions. 
Then, the normalized variance associated with the iterative integer-valued allocation satisfies the following bounds
%
\begin{equation}\label{eq:Iterative_Integer_Variance_Bound}
\frac{1}{p}\left(\sum_{k=1}^K \sqrt{C_k \Delta_k}\right)^2
= \sum_{k=1}^K \frac{\Delta_k}{N_k^*}
\;\le\;
\sum_{k=1}^K \frac{\Delta_k}{\left\lfloor M_k^* \right\rfloor}
\;\le\;
\sum_{k=1}^K \frac{\Delta_k}{\left\lfloor N_k^* \right\rfloor}.
\end{equation}
%
\end{theorem}



\begin{proof}
We first establish the right-hand inequality in \eqref{eq:Iterative_Integer_Variance_Bound}.  
From Theorem~\ref{thm:MFMC_New_IntegerValued_Cost}, it follows that $\lfloor M_k^* \rfloor \ge \lfloor N_k^* \rfloor$ for all $k$.  
Since $x \mapsto \Delta_k/x$ is strictly decreasing for $x > 0$, we obtain
\[
\sum_{k=1}^K \frac{\Delta_k}{\left\lfloor M_k^* \right\rfloor}
\le 
\sum_{k=1}^K \frac{\Delta_k}{\left\lfloor N_k^* \right\rfloor},
\]
establishing the upper bound. To show the lower bound, we apply the Cauchy--Schwarz inequality:
\[
\left(\sum_{k=1}^K \sqrt{C_k \Delta_k}\right)^2
\le
\left(\sum_{k=1}^K C_k \left\lfloor M_k^* \right\rfloor\right)
\left(\sum_{k=1}^K \frac{\Delta_k}{\left\lfloor M_k^* \right\rfloor}\right).
\]
By the cost bound proved in \eqref{eq:MFMC_iterative_total_cost}, we have 
$\sum_{k=1}^K C_k \lfloor M_k^* \rfloor \le p$, hence
%
\[
\frac{1}{p}\left(\sum_{k=1}^K \sqrt{C_k \Delta_k}\right)^2
\le
\sum_{k=1}^K \frac{\Delta_k}{\left\lfloor M_k^* \right\rfloor}.
\]
%
This establishes the lower bound in \eqref{eq:Iterative_Integer_Variance_Bound}.

\vspace{4mm}
\noindent{{\it When equality holds.}}
Equality in the Cauchy--Schwarz step holds if and only if there exists a constant $\lambda>0$ such that 
\[
\left\lfloor M_k^* \right\rfloor = \lambda \sqrt{\frac{\Delta_k}{C_k}},
\]
which corresponds to the continuous optimal allocation $M_k^* = N_k^*$. Therefore, equality in \eqref{eq:Iterative_Integer_Variance_Bound} holds if and only if the iterative scheme reproduces the continuous solution exactly, i.e., when all $\lfloor M_k^* \rfloor = N_k^*$ and the total cost equals $p$.

\end{proof}
% ------------------
% Next consider the variance
% \begin{align*}
%     f_{\text{act}}(\overline{N_k})&=\sum_{i=1}^{K}\frac{\Delta_k}{\overline{N_k}}=\sum_{i=1}^{k-1}\frac{\Delta_i}{\overline{N_i}}+\frac{\Delta_k}{\overline{N_k}}+\sum_{i=k+1}^{K}\frac{\Delta_i}{\overline{N_i}}\\
%     &\in \left[\sum_{i=1}^{k-1}\frac{\Delta_i}{\overline{N_i}}+\frac{\Delta_k}{\overline{N_k}}+\sum_{i=k+1}^K\frac{\Delta_{i}}{N_i^*},\; \sum_{i=1}^{k-1}\frac{\Delta_i}{\overline{N_i}}+\frac{\Delta_k}{\overline{N_k}}+\sum_{i=k+1}^K\frac{\Delta_{i}}{N_i^*-1}\right)=[f_1,f_2)\\
%     f_1&=\sum_{i=1}^{k-1}\frac{\Delta_i}{\overline{N_i}}+\frac{\Delta_k}{\overline{N_k}}+\sum_{i=k+1}^K\frac{\Delta_{i}}{N_i^*}=\sum_{i=1}^{k-1}\frac{\Delta_i}{\overline{N_i}}+\frac{\Delta_k}{\overline{N_k}}+\sum_{i=k+1}^K\sqrt{C_i\Delta_i}\frac{\sum_{j=i}^K\sqrt{C_j\Delta_j}}{p-\sum_{j=1}^{i-1}C_j\overline{N_j}}\\
%     f_2&=\sum_{i=1}^{k-1}\frac{\Delta_i}{\overline{N_i}}+\frac{\Delta_k}{\overline{N_k}}+\sum_{i=k+1}^K \ldots\\
% \end{align*}
% \begin{align*}
%     \frac{d f_1}{d \overline{N_k}}&=-\frac{\Delta_k}{\overline{N_k}^2}+C_k\sum_{i=k+1}^K\sqrt{C_i\Delta_i}\frac{\sum_{j=i}^K\sqrt{C_j\Delta_j}}{\left(p-\sum_{j=1}^{i-1}C_j\overline{N_j}\right)^2}\\
%     \frac{d^2 f_1}{d^2 \overline{N_k}}&=\frac{2\Delta_k}{\overline{N_k}^3}+2C_k^2\sum_{i=k+1}^K\sqrt{C_i\Delta_i}\frac{\sum_{j=i}^K\sqrt{C_j\Delta_j}}{\left(p-\sum_{j=1}^{i-1}C_j\overline{N_j}\right)^3}
% \end{align*}
% Note that $\frac{d^2 f_1}{d^2 \overline{N_k}}>0$ whenever $p>\sum_{j=1}^{i-1}C_j\overline{N_j}$. this means $f_1$ is convex in $\overline{N_k}$.
\input{./Sections/Modified_iterative_Integer_Valued_Sample_Size}
% ====================================================
\section{Numerical results}\label{sec:Num_Result}
% ====================================================

\subsection{First example}
%
\begin{table}[ht]
\centering
\scalebox{1}{
\begin{tabular}{|c|c|c|c|c|c|c|}
\hline
Model index &1 &2 &3 &4 &5 \\
\hline
Correlation coeff $\rho_{1,k}$ &1     &9.9977e-01   &9.9925e-01  &9.9728e-01   &9.8390e-01\\
% \hline
% Standard deviation $\sigma_k$ &1.0840e-02    &1.0838e-02   &1.1001e-02  &1.1549e-02   &9.5720e-03\\
\hline
Cost &73&7.0318e-03 &1.4018e-03 &5.0613e-04 &2.6803e-04\\
\hline
\end{tabular}
}
\caption{Parameters from plasma problem.}
\label{Tab:Parameters}
\end{table}
%






% %
% \begin{table}[ht]
% \centering
% \scalebox{0.6}{
% \begin{tabular}{|c|c|c|c|c|c|c|c|c|c|}
% \hline
% Total cost $P$ &73.05 &73.051 &73.052 &73.053 &73.054 &73.055 &73.056\\
% \hline
% Sample size (real valued) &[1,   134,   588   2540,  21100] &[1,   135,   588,   2541,  21101]&same &same &[1,   135,   588,   2541,  21102] &same &same\\
% \hline
% Sample size (integer program) &[1, 2, 3, 11, 97]&[1, 2, 3, 12, 99]&-&-&-&-&-\\
% \hline
% CPU time for integer program [s] & 0.27 &0.32 &$>$ 1000 &$>$ 1000 &$>$ 1000 &$>$ 1000 &$>$ 1000\\
% \hline
% \end{tabular}
% }
% \caption{Sample size for real-valued optimization and integer optimization.}
% \label{Tab:Sample_Size}
% \end{table}
% %

%
\begin{table}[ht]
\centering
\scalebox{1}{
\begin{tabular}{|c|c|c|c|c|c|c|c|c|c|}
\hline
&Sample size &Total cost $p$ &$f$\\
\hline
Real valued &[2.4129e+00 3.6959e+02 1.6102e+03 6.9567e+03 5.7770e+04]&200&2.1645e-04\\
\hline
Integer, floor &[2, 369, 1610, 6956, 57769]&1.698561e+02&2.5580e-04\\
\hline
Integer, iterative&[2, 836, 3644, 15744, 130749]&1.999999e+02&2.4138e-04\\
% \hline
% CPU time for integer program [s] & 0.27\\
\hline
\end{tabular}
}
\caption{Sample size for real-valued optimization and integer optimization for $p=200$.}
\label{Tab:Sample_Size_1}
\end{table}
%

%
\begin{table}[ht]
\centering
\scalebox{1}{
\begin{tabular}{|c|c|c|c|c|c|c|c|c|c|}
\hline
&Sample size &Total cost $p$ &$f$\\
\hline
Real valued &[8.8106e-01,   1.3495e+02,   5.8795e+02,   2.5402e+03,  2.1095e+04]&73.03&5.9275e-04\\
\hline
Integer, floor &[0,         134,         587,        2540,  21094]&8.7045&$\infty$\\
\hline
Modified &[1,     1,     1,     7,    62]&7.302859e+01&2.4834e-02\\
% \hline
\hline
Integer, iterative &[1,     1,     1,     7,    67]&7.302993e+01&2.3668e-02\\
% \hline
% CPU time for integer program [s] & 0.27\\
\hline
\end{tabular}
}
\caption{Sample size for real-valued optimization and integer optimization for $p=73.03$.}
\label{Tab:Sample_Size_1}
\end{table}
%



\subsection{Second example}

%
\begin{table}[ht]
\centering
\scalebox{1}{
\begin{tabular}{|c|c|c|c|c|c|c|}
\hline
Model index &1 &2 &3 &4 \\
\hline
Correlation coeff $\rho_{1,k}$ &1     &9.999882e-01  &9.999743e-01 &9.958253e-01\\
% \hline
% Standard deviation $\sigma_k$ &0.03\\
\hline
Cost &44.395 &6.8409e-01 &2.9937e-01 &1.9908e-04\\
\hline
\end{tabular}
}
\caption{Parameters from Peherstorfer's paper \cite{PeWiGu:2016}.}
\label{Tab:Parameters}
\end{table}
%



%
\begin{table}[ht]
\centering
\scalebox{1}{
\begin{tabular}{|c|c|c|c|c|c|c|c|c|c|}
\hline
&Sample size &Total cost $p$ &$f$\\
\hline
Real valued &[1.449950e+00 1.267731e+01 3.307436e+02 1.403572e+05]&200&5.0571e-05\\
\hline
Integer, floor &[1, 12, 330, 140357]&1.793385e+02&5.8074e-05\\
\hline
Integer, iterative&[1, 14, 380, 162081]&1.999999e+02&5.3495e-05\\
% \hline
% CPU time for integer program [s] & 0.27\\
\hline
\end{tabular}
}
\caption{Sample size for real-valued optimization and integer optimization for $p=200$.}
\label{Tab:Sample_Size_1}
\end{table}
%



%
\begin{table}[ht]
\centering
\scalebox{1}{
\begin{tabular}{|c|c|c|c|c|c|c|c|c|c|}
\hline
&Sample size &Total cost $p$ &$f$\\
\hline
Real valued &[3.334886e-01 2.915781e+00 7.607103e+01 3.228216e+04]&46&2.1987267e-04\\
\hline
Integer, floor &[0, 2, 76, 32282]&3.054700e+01&$\infty$\\
\hline
Modified &[1, 1, 2, 1018]&4.588049e+01&5.1658e-03\\
% \hline
\hline
Integer, iterative &[1, 1, 2, 1618]&4.599994e+01&4.8046e-03\\
% \hline
% CPU time for integer program [s] & 0.27\\
\hline
\end{tabular}
}
\caption{Sample size for real-valued optimization and integer optimization for $p=46$.}
\label{Tab:Sample_Size_1}
\end{table}
%



 
% ====================================================
\section{Conclusion}
% ====================================================
This paper has introduced a novel iterative framework for integer-valued sample size allocation in multi-fidelity Monte Carlo estimation, addressing a critical implementation gap between theoretical continuous optima and practical discrete requirements. Through a formulation grounded in dynamic programming principles and Bellman's principle of optimality, we have developed a sequential allocation scheme that preserves the theoretical foundations of MFMC estimation while enforcing integer constraints and maintaining strict budget adherence. From a practical perspective, the algorithm's linear computational complexity and sequential decision-making structure make it particularly suitable for high-dimensional fidelity hierarchies and resource-constrained environments. 

The proposed method demonstrates significant advantages over conventional approaches: it reduces the budget under-utilization inherent in direct flooring strategies, and achieves superior variance characteristics compared to existing modified rounding procedures. 

 


\bibliographystyle{abbrv}
% \bibliographystyle{alphaurl}
\bibliography{references_liang}
% \bibliography{reference}
\end{document}


