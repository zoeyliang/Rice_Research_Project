\documentclass[final,3p,times,11pt]{elsarticle}
\usepackage[USenglish]{babel}
\usepackage{amsmath,amssymb,amsthm, mathrsfs,multirow}
\usepackage{mathtools}
\usepackage{graphicx}
\usepackage{stmaryrd}
\usepackage[dvipsnames]{xcolor}
\usepackage{cancel}
\usepackage{ulem}
\usepackage{tabularx}
\usepackage{comment}
%\usepackage{subcaption}
%\usepackage[show]{ed}
%\usepackage{showkeys}
%\usepackage{showlabels}
%\usepackage[notcite,notref]{showkeys}
%\usepackage{refcheck}
% \usepackage[ruled,vlined]{algorithm2e}
\usepackage[linesnumbered,ruled,vlined]{algorithm2e}
\definecolor{Myblue}{rgb}{.2 0.4 1}

\usepackage{hyperref}
\hypersetup{
    %bookmarks=true,         % show bookmarks bar?
    colorlinks = true,       % false: boxed links; true: colored links
}



% ==============   Macros  ====================
\newcommand{\real}{\mathbb{R}} 
\newcommand{\nat}{\mathbb{N}} 


\newcommand{\mynabla}{\widetilde{\nabla}} 
\newcommand{\jump}[1]{[\![#1]\!]}
\newcommand{\HEcolor}[1]{{\textcolor{blue}{#1}}}
\newcommand{\TSVcolor}[1]{{\textcolor{orange}{#1}}}
\newcommand{\JLcolor}[1]{{\textcolor{violet}{#1}}} %violet
\newcommand{\Grids}{\boldsymbol{\chi}}

\newtheorem{theorem}{Theorem}%[section]
\newtheorem{lemma}{Lemma}%[section]
\newtheorem{VariationalForm}[theorem]{Variational Formulation}
% =============================================

\journal{}
\makeatletter
\def\ps@pprintTitle{%
 \let\@oddhead\@empty
 \let\@evenhead\@empty
 \def\@oddfoot{}%
 \let\@evenfoot\@oddfoot}
\makeatother




\begin{document}
\begin{frontmatter}
\title{Iterative estimation of integer sample size for multi-fidelity Monte Carlo\tnoteref{t1}}
\tnotetext[t1]{This research was supported in part by AFOSR Grant FA9550-22-1-0004 and NSF Grant DMS-2231482.}



% \author[RiceCMOR]{Damon Spencer}
% \ead{heinken@rice.edu}
% \address[RiceCMOR]{Department of Computational Applied Mathematics \& Operations Research, Rice University.}
\author[MH]{Matthias Heinkenschloss}
\ead{heinken@rice.edu}
\address[MH]{Department of Computational Applied Mathematics \& Operations Research and The Ken Kennedy Institute, Rice  University.}
\author[JL]{Jiaxing Liang}
\ead{jl508@rice.edu}
\address[JL]{Department of Computational Applied Mathematics \& Operations Research, Rice University.}

\begin{abstract}
Optimal sample allocation in multifidelity Monte Carlo (MFMC) methods is conventionally obtained from a continuous relaxation of the allocation problem, followed by rounding to enforce integer sample sizes. However, such post hoc rounding often leads to budget under-utilization and diminished variance reduction efficiency. This paper presents a novel iterative allocation scheme that computes integer-valued sample sizes directly within a recursive optimization framework. The proposed method preserves the analytical structure of the continuous optimum while ensuring strict adherence to budget constraints and maintaining monotonicity across fidelity levels. Theoretical analysis establishes improved variance and cost bounds compared to direct rounding strategies, and numerical experiments confirm enhanced estimator accuracy and budget utilization. The results demonstrate that the proposed scheme provides a robust and practical framework for optimal sample allocation in multifidelity Monte Carlo estimation.
\end{abstract}





\begin{keyword}
Multi-fidelity Monte Carlo \sep Sample Allocation \sep Integer Optimization  \sep Uncertainty Quantification \sep Variance Reduction.
%
\MSC[2020] 65C05\sep 62K05 \sep 49M20.
\end{keyword}
\end{frontmatter}

% ========================================
\section{Introduction}\label{sec:intro}
% ========================================
The pursuit of controlled nuclear fusion as a clean and virtually limitless energy source has spurred extensive research into the physics of magnetic confinement in fusion reactors. At the core of this effort lies the Grad–Shafranov free-boundary problem, which governs the equilibrium state of plasma in axially symmetric geometries, such as those found in Tokamaks. The governing equation encapsulates the intricate interplay between magnetic fields and plasma pressure, which determines critical confinement and stability properties essential for efficient plasma performance. However, the predictive accuracy of these models is significantly challenged by uncertainties in the parameters arising from measurement limitations, model assumptions, and operational variability. Addressing these uncertainties effectively requires advanced computational frameworks capable of robust statistical analysis, enabling accurate predictions of the plasma equilibrium response under diverse scenarios and ensuring reliable assessments of reactor designs and operations.

This study focuses on estimating the expectation of the solution operator associated with plasma equilibrium with uncertainties in the parameters. The Monte Carlo (MC) method, a classical and widely used approach in uncertainty quantification, relies on repeated executions of deterministic solvers to generate ensembles of realizations for stochastic inputs. Despite its versatility, its practicality is often limited by its slow asymptotic convergence rate of $1/\sqrt{N}$, which often requires a substantial number of sample realizations, $N$, to achieve reliable accuracy. For problems involving non-linear partial differential equations (PDEs), this slow convergence translates into a significant computational burden, as each realization typically demands a high-fidelity numerical approximation, such as those obtained via finite element method, which are computationally expensive due to their fine spatial resolution. Consequently, the cost of using the MC method can quickly escalate, particularly for high-dimensional problems or those requiring precise solutions. To alleviate this challenge, low-fidelity models have been proposed as computationally efficient alternatives to high-fidelity simulations.  These models aim to approximate the underlying system with reduced computational cost while maintaining an acceptable level of accuracy. For example, \cite{ElLiSa:2022} demonstrates how low-fidelity models constructed using stochastic collocation can effectively accelerate Monte Carlo sampling by exploiting simplified representations of the system. Similarly,
\cite{ElLiSa:2023} investigates hierarchical coarse spatial grids to develop low-fidelity surrogate models for multilevel Monte Carlo (MLMC) frameworks \cite{BaScZo:2011,Gi:2008}. Building on these ideas, studies such as \cite{ElLiSa:2025, Li:2024} combine stochastic collocation techniques with multilevel approaches, constructing low-fidelity models on coarse grids for MLMC sampling, further reducing computational expenses. While these methods achieve notable reductions in computational cost, they introduce the risk of compromising accuracy due to the inherent simplifications in the surrogate models. The trade-off between computational efficiency and solution accuracy is, therefore, a critical issue that warrants careful examination to ensure the reliability of results.


In this work, we delve into the multi-fidelity Monte Carlo (MFMC) method \cite{PeWiGu:2016, PeGuWi:2018}, which uses the control variate approach to exploit correlations between a computationally expensive high-fidelity model and a series of low-fidelity models. The MFMC method distinguishes itself from the MLMC approach by adopting a different strategy to construct its estimator. In MLMC, sample corrections are accumulated starting from the coarsest grid representation, using independent samples across successive spatial grid resolutions, and the sample size decreases with increasing grid fidelity to optimize computational effort. In contrast, the MFMC estimator follows an inverted paradigm: it initiates the accumulation of corrections with the most refined model representation and progressively incorporates corrections from lower-fidelity models. As the fidelity of the model decreases, the sample size increases, allowing less accurate but computationally inexpensive models to contribute to the overall estimate. Crucially, a distinguishing feature of the MFMC method is its reuse of samples within the same model hierarchy in the correction. This reuse avoids the computational redundancy of generating new samples at each fidelity level, effectively enhancing the overall efficiency of the sampling process. In addition to its computational efficiency, the MFMC method offers notable advantages over surrogate-based MLMC methods, such as those discussed in \cite{ElLiSa:2025, Li:2024}, which rely heavily on the characterization of interpolation errors. Such reliance can be a limiting factor, particularly in scenarios where interpolation errors decay slowly or require complicated error analysis. The MFMC approach circumvents this challenge by accommodating diverse surrogate models without taking into account the explicit treatment of interpolation errors, offering greater flexibility in its application. This adaptability extends the utility of MFMC to a broader range of modeling scenarios, making it particularly valuable in contexts where achieving a balance between computational efficiency and solution accuracy is critical. 

Nevertheless, the MFMC method is not without challenges. One notable limitation lies in its reliance on sufficiently large sample sizes to accurately estimate critical statistical parameters, such as variances and correlation coefficients, between high- and low-fidelity models. These estimates are obtained during the \textit{offline computations}, a preparatcory process involving tasks such as parameter estimation and the construction of surrogate models. Once the surrogates are built and parameters are generated in the offline phase, they will then be used in the \textit{online computations}, where the MFMC estimator is assembled and used to perform uncertainty quantification. However, achieving accurate parameter approximations in the offline phase can require substantial computational effort, which, in turn, may offset some of the efficiency gains in the online phase. Despite these challenges, the MFMC method remains an attractive approach due to its potential to accelerate the sampling process. The trade-offs between offline and online cost emphasize the importance of evaluating its applicability on a case-by-case basis to fully realize its potential benefits. In this study, we extend the analysis of the MFMC method \cite{PeWiGu:2016} by explicitly deriving the required sample size and computational cost as functions of the prescribed accuracy requirements. Our primary objective is to demonstrate that the MFMC method can achieve significant acceleration in the sampling process while maintaining statistical fidelity. As such, we show that MFMC provides a practical and robust framework for addressing the complex challenges of uncertainty quantification, particularly in the context of plasma equilibrium modeling.


 
The paper is organized as follows. In Section \ref{sec:Grad-Shafranov}, we introduce the Grad-Shafranov free boundary problem under uncertainty. Section \ref{sec:SC} provides an overview of the sparse grid stochastic collocation technique, which forms the basis to construct low-fidelity models used in the multi-fidelity Monte Carlo framework. Sections \ref{sec:MC} and \ref{sec:MFMC} discuss the Monte Carlo Finite Element method and its multi-fidelity variant. Finally, Section \ref{sec:Num-Exp} presents numerical experiments that access both efficiency and accuracy of these methods.



% Finally, the paper concludes with Section \ref{sec:Conclusion}, summarizing the key findings and contributions. 
% An appendix is included, containing technical mathematical details and proofs relevant to the problem and methods discussed. 
%!TEX root = ../main.tex
% ====================================================
\section{Multi-fidelity Monte Carlo}\label{sec:MFMC}
% ====================================================
This section reviews the multi-fidelity Monte Carlo (MFMC) method, following the foundational formulation in \cite{PeWiGu:2016}. The MFMC framework uses an ensemble of models with varying computational cost and accuracy to construct a variance-reduced estimator for high-fidelity expectation. Let $u_1:\Omega \to U$ denote the high-fidelity (HF) model that provides accurate but expensive evaluations, and let $\{u_k\}_{k=2}^K$ denote low-fidelity (LF) models that offer cheaper approximations. The central goal of MFMC is to allocate a fixed computational budget across these models to minimize estimator variance while maintaining unbiasedness.

We introduce some key statistical quantities that describe the model. We represent the random output of model $u_k$ on the probability space $(\Omega,\mathcal{F},\mathbb{P})$ by $u_k(\boldsymbol{\omega})$, abbreviated as $u_k$. For each pair of models $u_k,u_j$, define the variance and correlation coefficient
%
\begin{equation*}
    \sigma_k^2 = \mathbb{V}\!\left[u_k\right],\qquad 
    \rho_{k,j} = \frac{\text{Cov}\!\left[u_k,u_j\right]}{\sigma_k\sigma_j}, 
    \quad k,j=1,\dots,K,
\end{equation*}
%
where the covariance is defined as $\text{Cov}[u_k,u_j] := \mathbb{E}[\langle u_k - \mathbb{E}[u_k], u_j - \mathbb{E}[u_j]\rangle_U]$ and $\rho_{k,k}=1$. The pairwise correlations between fidelity levels quantify the statistical dependence that drives variance reduction through effective control variates.

The MFMC estimator architecture uses a nested sampling strategy that reuses computational evaluations across fidelity levels. Let $A_{1,N_1}^{\text{MC}}$ denote the standard Monte Carlo estimator of $\mathbb{E}[u_1]$ based on $N_1$ HF samples. The MFMC estimator augments this with corrections from lower fidelities via control variates
%
\begin{equation}\label{eq:MFMC_estimator}
A^{\text{MF}} := A^{\text{MC}}_{1,N_1} + \sum_{k=2}^K \alpha_k\left(\overline{A}_{k,N_k} - \overline{A}_{k,N_{k-1}}\right),
\end{equation}
%
where $\alpha_k \in \mathbb{R}$ are control variate weights and $\overline{A}_{k,N}$ denotes the sample average of $N$ evaluations of model $u_k$. A critical aspect of this construction is the nested sampling structure: the estimator $\overline{A}_{k,N_{k}}$ reuses all $N_{k-1}$ samples from $\overline{A}_{k,N_{k-1}}$, possibly supplemented by additional $N_{k} - N_{k-1}$ samples. The reuse of LF evaluations across levels enhances efficiency but induces sample statistical dependencies that complicate variance analysis.



To facilitate analysis, we reformulate the estimator so that its constituent terms are statistically independent. Partitioning the $N_k$ LF samples into disjoint sets of sizes $N_{k-1}$ and $N_k-N_{k-1}$ yields the equivalent independent form
%
\begin{equation}\label{eq:MFMC_estimator_independent}
    A^{\text{MF}} = A^{\text{MC}}_{1,N_1} +  \sum_{k=2}^K \alpha_k\!\left(1-\frac{N_{k-1}}{N_k}\right)\left(A^{\text{MC}}_{k,N_k\backslash N_{k-1}}-A^{\text{MC}}_{k,N_{k-1}}\right),
\end{equation}
%
where $A_{k,N_k\backslash N_{k-1}}^{\text{MC}}$ is the MC average over the $N_k-N_{k-1}$ new samples (defined to be zero when $N_k=N_{k-1}$).


The statistical properties of the MFMC estimator emerge clearly from its component-wise decomposition. Define
%
\begin{equation}\label{eq:MFMC_Yk}
Y_1 := A^{\text{MC}}_{1,N_1},\quad 
Y_k := \left(1-\frac{N_{k-1}}{N_k}\right)\!\left(A^{\text{MC}}_{k,N_k\backslash N_{k-1}} - A^{\text{MC}}_{k,N_{k-1}}\right), \;\; k=2\ldots, K,
\end{equation}
%
then the MFMC estimator can be expressed into a compact form $A^{\text{MF}} = Y_1 + \sum_{k=2}^K \alpha_k Y_k$. Since each $Y_k$ for $k\ge2$ represents a difference of two independent estimators for the same $\mathbb{E}[u_k]$, we immediately obtain $\mathbb{E}[Y_k]=0$ and the MFMC estimator is unbiased: $\mathbb{E}[A^{\text{MF}}]=\mathbb{E}[u_1]$. The variances of the components are
%
\begin{equation}\label{eq:Var_Yk}
    \mathbb{V}[Y_1] = \frac{\sigma_1^2}{N_1}, \qquad 
    \mathbb{V}[Y_k] = \left(\frac{1}{N_{k-1}} - \frac{1}{N_k}\right)\sigma_k^2, \;\; k=2\ldots, K.
\end{equation}
%
A key statistical insight, formalized in Lemma~\ref{lemma:Y_k_Y_j}, establishes that the correction terms are mutually uncorrelated despite sample reuse.
%
\begin{lemma}\label{lemma:Y_k_Y_j}
For $2\le k<j\le K$, 
% the correction terms $Y_k$ and $Y_j$ defined in \eqref{eq:MFMC_Yk} are uncorrelated, i.e., 
$\operatorname{Cov} [Y_k,Y_j ]=0$.
\end{lemma}
%
The proof is provided in the Appendix.

Each correction $Y_k$($k\ge2$) is correlated with $Y_1$, with covariance
\begin{equation}\label{eq:Cov_Yk}
\operatorname{Cov}[Y_1,Y_k] = -\!\left(\frac{1}{N_{k-1}} - \frac{1}{N_k}\right)\rho_{1,k}\sigma_1\sigma_k,
\end{equation}
as shown in \cite[Lemma~3.2]{PeWiGu:2016}. Combining \eqref{eq:Var_Yk} and \eqref{eq:Cov_Yk} gives
%
\begin{equation}\label{eq:MFMC_variance}
    \mathcal{V}^{\text{MF}}
    =\frac{\sigma_1^2}{N_1} 
    + \sum_{k=2}^K \left(\frac{1}{N_{k-1}} - \frac{1}{N_k}\right)\!\left(\alpha_k^2\sigma_k^2 - 2\alpha_k\rho_{1,k}\sigma_1\sigma_k\right).
\end{equation}
%

In order to determine optimal sample sizes $N_k$ and weights $\alpha_k$ in the MFMC estimator \eqref{eq:MFMC_estimator_independent}, an optimization problem is formulated \cite{PeWiGu:2016} by minimizing the estimator variance \eqref{eq:MFMC_variance} subject to a fixed budget $p$. Let $C_k$ denote the per-sample cost of model $u_k$, the total computational cost is 
%
\[
\mathcal{W}^{\text{MF}} = \sum_{k=1}^K C_k N_k,
\]
%
and the constrained optimization problem becomes
%
\begin{equation}\label{eq:Optimization_pb_sample_size}
    \begin{array}{ll}
    \min &\mathcal{V}^{\text{MF}}\left(\alpha_k,N_k\right),\\
       \text{subject to} &\displaystyle\sum\limits_{k=1}^K C_kN_k=p,\\[2pt]
       &\displaystyle N_1\ge 0,\quad \displaystyle N_{k-1}\le N_k, \;\; k=2\ldots,K,\\
       &N_1,\ldots, N_K\in \mathbb{R},\\
       &\alpha_2,\ldots,\alpha_K\in \mathbb{R}.
    \end{array}
\end{equation}
%
Note that for each level $k\ge 2$, $\alpha_k$ enters only through a quadratic expression independent of $N_k$ in the variance term. This separable structure allows a fundamental simplification of the variance functional, which allows hierarchical minimization
%
\begin{equation*}
    \min_{\alpha_k,\, N_k} \mathcal{V}^{\text{MF}}\left(\alpha_k, N_k\right)
    = \min_{N_k}\Big(\min_{\alpha_k} \mathcal{V}^{\text{MF}}(\alpha_k, N_k)\Big).
\end{equation*}
%
The hierarchical minimization admits a closed-form solution for optimal weights by solving the inner optimization $\partial \mathcal{V}^{\text{MF}}/\partial \alpha_k = 0$, yielding 
%
\begin{equation}\label{eq:MFMC_weights}
    \alpha_k^* = \frac{\rho_{1,k}\sigma_1}{\sigma_k}.
\end{equation}
%
Substituting $\alpha_k^*$ into \eqref{eq:MFMC_variance} simplifies the variance to 
%
\begin{equation*}
    \mathcal{V}^{\text{MF}}\left(\alpha_k^*, N_k\right)
    = \sigma_1^2\sum_{k=1}^K \frac{\Delta_k}{N_k},
\end{equation*}
%
where $\Delta_k = \rho_{1,k}^2 - \rho_{1,k+1}^2$ for $k = 1, \dots, K$ with $\rho_{1,K+1}=0$. This reduces the joint optimization to a continuous resource allocation problem involving only sample allocation
%
\begin{equation}\label{eq:Optimization_pb_sample_size_reduced}
    \begin{array}{ll}
    \min &\displaystyle f(N_k) =\sum_{k=1}^K \frac{\Delta_k}{N_k},\\
       \text{subject to} &\displaystyle\sum\limits_{k=1}^K C_kN_k=p,\\[2pt]
       &\displaystyle -N_1\le 0,\quad \displaystyle N_{k-1}-N_k\le 0, \;\; k=2\ldots,K,\\
       &N_1,\ldots, N_K\in \mathbb{R},
    \end{array}
\end{equation}
%
where $f(N_k)$ is the {\it normalized variance functional}. Under suitable monotonicity and ordering assumptions, this problem admits an analytic solution that characterizes the optimal allocation of resources across fidelity levels.


%
\begin{theorem}[Optimal MFMC real-valued sample allocation]\label{thm:Sample_size_est}
Consider $K$ models $\{u_{k}\}_{k=1}^K$ with standard deviations $\sigma_k$, correlation coefficients $\rho_{1,k}$ of LF model $u_k$ with the HF model $u_1$, and per-sample costs $C_k$. Define $\Delta_k = \rho_{1,k}^2 - \rho_{1,k+1}^2$ for $k = 1, \dots, K$ with $\rho_{1,K+1}=0$. Assume the following conditions hold
%
\begin{alignat*}{3}
&(i)\;\textit{Monotone correlations:} &\quad& |\rho_{1,1}| > \cdots > |\rho_{1,K}|,\\
&(ii)\;\textit{Cost-correlation ratio:} &\quad& \frac{\Delta_{k}}{C_k} > \frac{\Delta_{k-1}}{C_{k-1}}, \quad k=2,\ldots,K.
\end{alignat*}
%
Then the optimal control weights and sample sizes for \eqref{eq:Optimization_pb_sample_size} are
%
\begin{equation}\label{eq:MFMC_RealValued_Sample_Size}
    \alpha_k^* = \frac{\rho_{1,k}\sigma_1}{\sigma_k}, \qquad
    N_k^* = \sqrt{\frac{\Delta_k}{C_k}}\,
    \frac{p}{\sum_{j=1}^K \sqrt{C_j \Delta_j}}.
\end{equation}
%
% \[
% r_k^* = \sqrt{\frac{C_1\Delta_k}{C_k\Delta_1}},\quad N_1^* = \frac{p}{\sum_{k=1}^K C_k r^*_k}, \quad N_k^*=N_1^*r_k^*.
% \] 
% %
% \JLcolor{alternatively, in my way to represent it without mentioning the vector $\boldsymbol{r}^*$, we have}
%
The resulting minimal variance of the MFMC estimator is
\begin{equation}\label{eq:MFMC_variance_optimal}
\mathcal{V}^{\text{MF}}
= \sigma_1^2\sum_{k=1}^K \frac{\Delta_k}{N_k^*}=\frac{\sigma_1^2}{p}\!\left(\sum_{k=1}^K \sqrt{C_k \Delta_k}\right)^{\!2}.
\end{equation}
\end{theorem}
%


Differentiating the normalized variance and cost with respect to the sample sizes gives
%
\[
\frac{\partial f}{\partial N_k} = -\frac{\Delta_k}{N_k^2},
\qquad 
\frac{\partial \mathcal{W}^{\text{MF}}}{\partial N_k} = C_k.
\]
%
These relations quantify the variance–cost trade-off: increasing samples at any level reduces variance at the expense of computational resources. At the continuous optimum \eqref{eq:MFMC_RealValued_Sample_Size}, the marginal variance reduction per unit cost $\Delta_k/(C_k N_k^2)$ is identical across all active models, establishing a balanced resource allocation that characterizes the optimal allocation.

While Theorem~\ref{thm:Sample_size_est} provides real-valued optimal allocations $N_k^*$, practical implementation requires integer sample sizes. The standard approach \cite{PeWiGu:2016} applies the floor function $\lfloor N_k^* \rfloor$ to ensure budget feasibility. The realized variance and cost are
%
\[
f\left(\left\lfloor N_k^* \right\rfloor\right) = \sum_{k=1}^K\frac{\Delta_{k}}{\left\lfloor N_k^* \right\rfloor}, \qquad \mathcal{W}^{\text{MF}}\left(\left\lfloor N_k^* \right\rfloor\right) = \sum_{k=1}^K C_k\left\lfloor N_k^* \right\rfloor.
\]
%
Since $N_k^*-1 < \lfloor N_k^*\rfloor \le N_k^*$, the floor operation induces bounded sub-optimality, producing the bounds
%
\begin{equation}\label{eq:bounds_for_floor}
\begin{aligned}
    % f\left(\left\lfloor N_k^* \right\rfloor\right)&\in \left[\sum_{k=1}^K\frac{\Delta_{k}}{N_k^*},\; \sum_{k=1}^K\frac{\Delta_{k}}{N_k^*-1}\right) = \left[\frac{1}{p}\left(\sum_{k=1}^K \sqrt{C_k\Delta_k}\right)^2, \sum_{k=1}^K\frac{\Delta_{k}}{\frac{p}{\sum_{j=1}^K \sqrt{C_j\Delta_j}}\sqrt{\frac{\Delta_k}{C_k}}-1}\right)\\
    % &=\left[\frac{1}{p}\left(\sum_{k=1}^K \sqrt{C_k\Delta_k}\right)^2, \sum_{k=1}^K \sqrt{C_k\Delta_k}\sum_{k=1}^K\frac{\sqrt{C_k\Delta_{k}}}{p-\sqrt{\frac{C_k}{\Delta_k}}\sum_{j=1}^K \sqrt{C_j\Delta_j}}\right)\\
    % &=\sum_{k=1}^K \sqrt{C_k\Delta_k}\left[\frac{\sum_{k=1}^K \sqrt{C_k\Delta_k}}{p},\sum_{k=1}^K\frac{\sqrt{C_k\Delta_{k}}}{p-\sqrt{\frac{C_k}{\Delta_k}}\sum_{j=1}^K \sqrt{C_j\Delta_j}}\right)\\
    % \mathcal{W}^{\text{MF}}\left(\left\lfloor N_k^* \right\rfloor\right) &\in \left(\sum_{k=1}^KC_kN_k^*-\sum_{k=1}^K C_k, \sum_{k=1}^KC_kN_k^*\right]=\left( p-\sum_{k=1}^K C_k,p\right].
    f\left(\left\lfloor N_k^* \right\rfloor\right) \in \left[\frac{1}{p}\left(\sum_{k=1}^K \sqrt{C_k\Delta_k}\right)^2, \sum_{k=1}^K\frac{\Delta_{k}}{N_k^*-1}\right), \qquad
\mathcal{W}^{\text{MF}}\left(\left\lfloor N_k^* \right\rfloor\right)\in \left( p-\sum_{k=1}^K C_k, p\right].
\end{aligned}
\end{equation}
%
The term $\sum_{k=1}^K C_k$ represents the rounding-induced slack in the budget, which becomes negligible asymptotically as $p \to \infty$. However, in the pre-asymptotic regime -- where the total budget $p$ is moderate -- this  slack can lead to significant under-utilization of the computational resources. This observation naturally motivates \textit{the development of  alternative integer-valued allocation strategies that reduce slack and achieve tighter budget utilization.}





% This quantity is the marginal variance reduction rate — how much the total variance decreases when you spend more samples at level by taking one more sample cost $C_k$. So the marginal variance reduction per unit cost
% \[
% \frac{-\frac{\partial f}{\partial N_k}}{C_k} = \frac{\Delta_k}{C_kN_k^2}
% \]
% It quantifies that How much variance reduction we get per unit cost at level $k$.
% At the optimum, the system reaches equilibrium where every active model yields the same return per cost unit,
% \[
% \frac{\Delta_k}{C_kN_k^2} = \text{Constant}=\frac{1}{p^2}\left(\sum_{k=1}^K \sqrt{C_k \Delta_k}\right)^{\!2}, \quad \text{for all active}\;\; k.
% \]
















 
%!TEX root = ../main.tex
% ====================================================
\section{Optimal Sample Size Allocation}\label{sec:MFMC_Nk_optimize}
% ====================================================

% ====================================================
\subsection{Integer Programming Formulation}  \label{sec:MFMC_Nk_optimize_IP}
% ====================================================
The optimal samples $1 \le m_1 \le m_2 \le \ldots \le m_K$ are computed to minimize
the variance \eqref{eq:MFMC_variance} of the MFMC estimator
subject to (s.t.) a constraint on the cost \eqref{eq:MFMC_cost} to execute the MFMC estimator.
A first version of the optimization problem to compute $1 \le m_1 \le m_2 \le \ldots \le m_K$ is
\begin{subequations}\label{eq:Optimization_sample_size_N}
    \begin{align}
    \min \quad &\sigma_1^2  \sum_{k=1}^K \frac{ \rho_{1,k}^2 - \rho_{1,k+1}^2}{m_k},   \\
       \text{s.t.}\quad & \sum_{k=1}^K C_km_k \le b,       \label{eq:Optimization_sample_size_m_budget}  \\
                                & m_1\ge 1,\quad  m_k \ge m_{k-1}, \quad k=2\ldots,K,\\
                                &m_1,\ldots, m_K\in \nat.
    \end{align}
\end{subequations}
The issue with this formulation is that if $m_k = m_{k-1}$, the $k$-th model
does not contribute to the variance, which is easier to see from \eqref{eq:MFMC_variance_a}.
Therefore, if $m_k = m_{k-1}$, the $k$-th model should not be executed, but it still contributes to
the computing budget \eqref{eq:Optimization_sample_size_m_budget}.
To fix this issue we introduce binary variables $z_2, \ldots, z_K \in \{0,1\}$ such that
$z_k = 1$ if $m_k > m_{k-1}$ and model $k$ will be sampled $m_k$ times, and 
$z_k = 0$ if $m_k = m_{k-1}$ and model $k$ will be skippted.
The optimization formulation requires 
an upper bound $M$ for all possible differences $m_k - m_{k-1}$, $k=2\ldots,K$.
For example, $M = p/ C_K$ is such an upper bound because  $p/ C_K \ge m_K \ge m_k - m_{k-1}$, $k=2\ldots,K$.
The optimization problem formulation for the optimal sample size selection is
\begin{subequations}\label{eq:Optimization_sample_size_Nz}
    \begin{align}
    \min \quad &\sigma_1^2  \sum_{k=1}^K \frac{ \rho_{1,k}^2 - \rho_{1,k+1}^2}{m_k},   \\
       \text{s.t.}\quad &  C_1 m_1 + \sum_{k=2}^K z_k C_k m_k \le b,       \label{eq:Optimization_sample_size_Nz_budget}  \\
                                & m_1\ge 1,\quad  m_k \ge m_{k-1}, \quad k=2\ldots,K,\\
                                & m_k - m_{k-1} \ge z_k, \quad M z_k \ge m_k - m_{k-1},  \quad k=2\ldots,K,  \label{eq:Optimization_sample_size_Nz_z}  \\
                                &m_1,\ldots, m_K\in \nat, \quad z_2 ,\ldots, z_K\in \{0,1\}.
    \end{align}
\end{subequations}
The constraints \eqref{eq:Optimization_sample_size_Nz_z} ensures that $z_k = 0$ if $m_k = m_{k-1}$ and 
$z_k = 1$ if $m_k >  m_{k-1}$. If $z_k = 0$, model $k$ will not be executed and does not contribute to the 
computational cost \eqref{eq:Optimization_sample_size_Nz_budget}.




% ====================================================
\subsection{Relaxation}  \label{sec:MFMC_Nk_optimize_relax}
% ====================================================
Instead of solving the integer programming problem \eqref{eq:Optimization_sample_size_Nz} or
even \eqref{eq:Optimization_sample_size_N}, previous papers including
\cite{BPeherstorfer_KWillcox_MDGunzburger_2016a} and \cite{AGruber_MGunzburger_LJu_ZWang_2023a}
have considered a relaxation  and then used rounding to obtain integer sample sizes.
Specifically, \cite{BPeherstorfer_KWillcox_MDGunzburger_2016a}
considered the problem
\begin{subequations}\label{eq:Optimization_sample_size_m_relaxed}
    \begin{align}
    \label{eq:Optimization_sample_size_m_relaxed_obj}
    \min \quad &\sigma_1^2  \sum_{k=1}^K \frac{ \rho_{1,k}^2 - \rho_{1,k+1}^2}{m_k},   \\
       \text{s.t.}\quad & \sum_{k=1}^K C_km_k \le b,       \\
                                & m_1\ge 0,\quad  m_k \ge m_{k-1}, \quad k=2\ldots,K,\\
                                &m_1,\ldots, m_K\in \real.
    \end{align}
\end{subequations}
The formulation \eqref{eq:Optimization_sample_size_m_relaxed} potentially suffers from the same isses as 
\eqref{eq:Optimization_sample_size_N}, namely that if f $m_k = m_{k-1}$, the $k$-th model does not
provide variance reduction, but it's cost is included. 
However, under conditions specified in the following Theorem~\ref{thm:Sample_size_real}, the 
real valued solution
of \eqref{eq:Optimization_sample_size_m_relaxed} can be computed analytically, and it satisfies
$0  < m_1^* < m_2^* <  \ldots < m_K^*$.
The following theorem is proven in \cite[Th.~3.4]{BPeherstorfer_KWillcox_MDGunzburger_2016a}.

\begin{theorem}[Optimal MFMC Real-Valued Sample Allocation]   \label{thm:Sample_size_real}
   Let models $u_1, \ldots, u_k \in   L_{\mathbb{b}}^2(W, {\mathcal U})$ with
   standard deviations $\sigma_k$, correlation coefficients $\rho_{1,k}$, $k = 2, \ldots, K$,
   between the HF model $u_1$  and the  LF models $u_k$, $k = 2, \ldots, K$, and 
   per-sample costs $C_1, \ldots, C_K$ be given.
   If 
   \begin{subequations}\label{eq:Sample_size_real_assumptions}
   \begin{align}
       \label{eq:Sample_size_real_assumptions_a}
        & |\rho_{1,1}| > \cdots > |\rho_{1,K}|, & \text{(monotone correlations)} \\
      \label{eq:Sample_size_real_assumptions_b}
        & \frac{ \rho_{1,k}^2 - \rho_{1,k+1}^2 }{C_k} > \frac{ \rho_{1,k-1}^2 - \rho_{1,k}^2 }{C_{k-1}}, \quad k=2,\ldots,K, 
                                                                  & \text{(cost-correlation ratio)}
    \end{align}
    \end{subequations}
    hold, where $\rho_{1,K+1} :=0$, then the unique solution of \eqref{eq:Optimization_sample_size_m_relaxed} is
    \begin{equation}\label{eq:MFMC_RealValued_Sample_Size}
             m_k^* = \sqrt{\frac{\rho_{1,k}^2 - \rho_{1,k+1}^2}{C_k}} \; 
                          \frac{b}{\sum_{l=1}^K \sqrt{C_l (\rho_{1,l}^2 - \rho_{1,l+1}^2)}},
             \quad k = 1, \ldots, K, 
     \end{equation}
    the cost constraint is active $\sum_{k=1}^K C_km_k^* = p$, 
    and the resulting minimal variance of the MFMC estimator is
     \begin{equation}\label{eq:MFMC_variance_optimal}
           \mathcal{V}^{\text{MF}}(m_1^*, \ldots, m_K^*)
           =   \sigma_1^2  \sum_{k=1}^K \frac{ \rho_{1,k}^2 - \rho_{1,k+1}^2}{m_k^*}
           =  \frac{\sigma_1^2}{b}\!\left(\sum_{k=1}^K \sqrt{C_k (\rho_{1,k}^2 - \rho_{1,k+1}^2) }\right)^{\!2}.
       \end{equation}
\end{theorem}

Note that because of the cost-correlation ratio assumption \eqref{eq:Sample_size_real_assumptions_b}, 
the  solution \eqref{eq:MFMC_RealValued_Sample_Size}
satisfies $0  < m_1^* < m_2^* <  \ldots < m_K^*$.

To obtain integer samples,  \cite[p.~A3171]{BPeherstorfer_KWillcox_MDGunzburger_2016a}
round down, i.e., use $\lfloor m_1^* \rfloor, \ldots, \lfloor m_K^* \rfloor$. 
Rounding down will reduce the cost, i.e., the rounded down sample sizes are still feasible for
\eqref{eq:Optimization_sample_size_m_relaxed}.  Rounding down increases the variance.
Thus, rounding down the sample sizes of more costly, higher fidelity samples, frees up computational 
budget that could be used for additional  samples of less costly, lower fidelity samples to reduce the variance.

Rounding down may lead to $\lfloor m_1^* \rfloor = 0$, which introduces bias,  
$\mathbb{E}[A^{\text{MF}}] \not=  \mathbb{E}[u_1]$,
or may lead to $\lfloor m_k^* \rfloor = \lfloor m_{k-1}^* \rfloor$ for some $k \in \{ 2, \ldots, K\}$, 
which means the $k$-th model does not contribute to variance reduction, but its cost is included in the
computational budget.
The case $\lfloor m_1^* \rfloor = 0$ can happen if the computational budget $p$ is small.
If $0< m_1^* < 1$, \cite{AGruber_MGunzburger_LJu_ZWang_2023a} use $\lceil m_1^* \rceil = 1$,
and iterative recompute integer sample sizes from a modified version of 
Theorem~\ref{thm:Sample_size_real}. See Algorithm~2 in \cite{AGruber_MGunzburger_LJu_ZWang_2023a}.
However, their iterative sample size computation can generate integer sample sizes
with $1 = m_1 =  m_2 = \ldots = m_k$ for some $l \in \{ 2, \ldots, K\}$. 
See Tables~1 and 2 in \cite{AGruber_MGunzburger_LJu_ZWang_2023a}.
In this case, the models $2$ to $k$ do not contribute to variance reduction, but their cost is included in the
computational budget.

Finally, while the  solution \eqref{eq:MFMC_RealValued_Sample_Size} satisfies $0  < m_1^* < m_2^* <  \ldots < m_K^*$,
some rounded down sample sizes may be identical,  $\lfloor m_k^* \rfloor = \lfloor m_{k-1}^* \rfloor$ for some 
$k \in \{ 2, \ldots, K\}$, in which case the samples of the $k$-th model do not contribute to variance
reduction, but their costs are charged to the budget. 
The issues $\lfloor m_1^* \rfloor = 0$ or $\lfloor m_k^* \rfloor = \lfloor m_{k-1}^* \rfloor$ for some 
$k \in \{ 2, \ldots, K\}$, are more likely to occur when the total computational budget $b$ is small relative
 to the cost $C_1$ of the HF.
See, e.g., the numerical examples in  \cite{BPeherstorfer_KWillcox_MDGunzburger_2016a} or \cite{AGruber_MGunzburger_LJu_ZWang_2023a}.

To avoid issues with simple rounding, we propose to compute sample sizes recursively.
The general idea is similar to the one in \cite{AGruber_MGunzburger_LJu_ZWang_2023a},
but \cite{AGruber_MGunzburger_LJu_ZWang_2023a} use a recursion only to avoid
$\lfloor m_1^* \rfloor = \lfloor m_k^* \rfloor = 0$ for some $k \in \{ 2, \ldots, K\}$,
which can happen when the budget $b$ is small.

%Generally, because $m_k^*-1 < \lfloor m_k^*\rfloor \le m_k^*$, the floor operation induces bounded 
%sub-optimality, producing the bounds
%%
%\begin{subequations}\label{eq:bounds_for_floor}
%\begin{align}
%    \mathcal{V}^{\text{MF}}\left(\lfloor m_1^* \rfloor, \ldots \lfloor m_K^* \rfloor \right)
%    & \in \left[\frac{\sigma_1^2}{b}\left(\sum_{k=1}^K \sqrt{C_k  (\rho_{1,k}^2 - \rho_{1,k+1}^2)}\right)^2,
%                   \sum_{k=1}^K\frac{ (\rho_{1,k}^2 - \rho_{1,k+1}^2)}{m_k^*-1}\right),       \\
%   \mathcal{W}^{\text{MF}}\left(\lfloor m_1^* \rfloor, \ldots \lfloor m_K^* \rfloor \right)
%   &\in \left( p-\sum_{k=1}^K C_k, p\right].
%\end{align}
%\end{subequations}
%If $\sum_{k=1}^K C_k \ll p$, the work 
%$\mathcal{W}^{\text{MF}}\big(\lfloor m_1^* \rfloor, \ldots, \lfloor m_K^* \rfloor \big)
% \approx  \mathcal{W}^{\text{MF}}\big( m_1^* , \ldots,  m_K^* \big) = p$.



% ====================================================
\subsection{Model selection}
% ====================================================
\MH{may not need a subsection on this. Can simply reference}
The analytical solution of \eqref{eq:Optimization_sample_size_m_relaxed} in 
Theorem~\ref{thm:Sample_size_real} requires that the models satsfy 
\eqref{eq:Sample_size_real_assumptions}.
Thus, given models, we need to select the models from the available set such that the parameters 
associated with the selected models satisfy the two conditions in Theorem \ref{thm:Sample_size_real}, as well as the $\mathcal{V}^{\text{MF}}$ as small as possible (Note that minimize \eqref{eq:MFMC_variance_optimal} is a selection of model that associated with 
different cost and $\rho_{1,k}^2 - \rho_{1,k+1}^2$, 
and is independent of budget $p$). Let $\mathcal{S}^*=\{1, \ldots, K^*\}$ be the indices of $K^*$ available models. 
We seek a subset $\mathcal{S}=\{i_1,i_2, \ldots,i_{K}\}\subseteq \mathcal{S}^* (K\le K^*)$ of indices that minimizes the sampling cost of multifidelity Monte Carlo estimator. Note that $\mathcal{S}$ is non-empty and $i_1=1$ since the high fidelity model must be included. 
We will follow the exhaustive algorithm in \cite[Algorithm~1]{BPeherstorfer_KWillcox_MDGunzburger_2016a} 
for $2^{K^*-1}$ subsets of $\mathcal{S}^*$.  This algorithm gives the indices of the selected model.

\normalem
\begin{algorithm}[!ht]
\label{algo:MFMC_Algo_model_selection}
\DontPrintSemicolon    
   \KwIn{Models $u_1, \ldots, u_{K^*}$ ordered such that $|\rho_{1,2}| \ge \ldots \ge |\rho_{1,K^*}|$,
             and corresponding sample costs  $C_1, \ldots, C_{K^*}$.}\vspace{1ex}
    
    \KwOut{ Selected index set $\mathcal{S}$.}\vspace{1ex}
    \hrule \vspace{1ex}

   % Estimate $\rho_{1,k}$ and $C_k$ for each model $f_k$ using $m_0$ samples.
   
   
   Set $\mathcal{S}=\{1,\ldots, K^*\}$. 
   
   Initialize $v_{\min}=C_1$, $\mathcal{S}=\{1\}$. Let $ \mathcal{\widehat S}$ be all $2^{K-1}$ ordered subsets of $\mathcal{S}^*$, each containing the high fidelity model with index $1$. 
   % Set $ \mathcal{\widehat S}_1=\mathcal{S}^*$.

    % $(2 \le j \le 2^{K-1})$
    \For{each subset $\mathcal{\widehat S}_j$\,}{

    {
    \If{ Condition \eqref{eq:Sample_size_real_assumptions_b} from Theorem \ref{thm:Sample_size_real} is satisfied}{
    Compute $\Delta_k$ and $v = \left(\sum_{k=1}^K \sqrt{C_k (\rho_{1,k}^2 - \rho_{1,k+1}^2) }\right)^{\!2}$.
    \MH{What is $K$?}
    
    \If{$v<v_{\min}$}{
    {
    Update $\mathcal{S} = \mathcal{\widehat S}_j$ and $v_{\min} = v$.
    }
    } 
    }
    }
    $j=j+1$.
    }
    Return  $\mathcal{S}$.
\caption{Multi-fidelity Model Selection}
\end{algorithm}
\ULforem












% ====================================================
\section{Iterative sample size estimation for MFMC}\label{sec:Iterative_IntegerValued_Sample_Size}
% ====================================================

To address this challenge, we develop an iterative scheme for MFMC sample size estimation grounded in dynamic programming principles, specifically using Bellman's principle of optimality \cite{Be:1957}. This approach ensures consistency with the continuous optimal allocation while enabling sequential decision-making that naturally accommodates integer constraints in practical implementations. 

\subsection{Theoretical Justification for Dynamic Programming Decomposition}

The decomposition of the original optimization problem \eqref{eq:Optimization_pb_sample_size_reduced} into sequential subproblems is justified by the problem's mathematical structure and fundamental optimization principles:


\begin{enumerate}

    \item \textbf{Optimal Substructure and Separability}: The objective function $\sum_{k=1}^K \frac{\Delta_k}{N_k}$ exhibits complete separability across fidelity levels, while the budget constraint $\sum_{k=1}^K C_kN_k = p$ admits additive decomposition. This separable structure ensures that the optimal solution possesses the key property that any subsequence of decisions must be optimal for the corresponding subproblem with the remaining budget.

    \item \textbf{Bellman's Principle of Optimality}: If $\{N_1^*, N_2^*, \ldots, N_K^*\}$ constitutes an optimal solution to the global problem, then for any $k \in \{1,\ldots,K\}$, the truncated sequence $\{N_k^*, N_{k+1}^*, \ldots, N_K^*\}$ must be an optimal solution to the subproblem defined on levels $k$ through $K$ with residual budget $R_k = p - \sum_{j=1}^{k-1} C_j N_j^*$. This principle follows directly from the contradiction that would arise if a superior partial solution existed for any subproblem.

    \item \textbf{State Variable Sufficiency}: The recursive definition of the state variable $R_k$ 
    \[
    R_1 := p, \qquad R_k := p - \sum_{j=1}^{k-1} C_j H_j, \quad k=2,\ldots,K.
    \]
    as the remaining computational budget captures all essential information from previous decisions. The Markovian property holds because future optimization depends only on the current remaining budget $R_k$ and not on the specific history of allocations that led to this state.


    \item \textbf{Sequential Decision Framework}: At each stage $k$, given the fixed decisions $\{H_1, \ldots, H_{k-1}\}$ and current state $R_k$, the optimization over remaining levels $\{H_k, \ldots, H_K\}$ constitutes a well-defined subproblem that preserves the original problem's structure:
    \begin{equation}\label{eq:Sequential_Optimization}
        \begin{array}{ll}
        \displaystyle 
        \min_{H_k,\ldots,H_K\in \mathbb{R}} & 
            \displaystyle 
            \sum_{j=k}^K \frac{\Delta_j}{H_{j}}, \\
        \text{subject to} &
            \displaystyle \sum_{j=k}^K C_j H_j = R_k,\\
            &H_k\ge 0,\quad H_{j-1}-H_j\le 0,\;\; j=k+1,\ldots,K.
        \end{array}
    \end{equation}

    \item \textbf{Constraint Preservation}: The sequential formulation inherently maintains all original constraints. The budget constraint is preserved through the recursive budget update, while the monotonicity constraints $H_{j-1} \leq H_j$ are enforced locally at each stage, ensuring global consistency with the MFMC hierarchy requirements.
\end{enumerate}

\subsection{Convexity and Global Optimality}

The convexity of each subproblem in \eqref{eq:Sequential_Optimization} is fundamental to guaranteeing global optimality. Since $\Delta_j > 0$ and $H_j\ge 0$, the objective function $\sum_{j=k}^K \frac{\Delta_j}{H_j}$ is strictly convex on the positive orthant, while the budget constraint is linear and the monotonicity constraints form a convex polyhedron. This convexity ensures that:
\begin{itemize}
    \item Local optimal solutions are globally optimal
    \item The sequential dynamic programming approach cannot be trapped in suboptimal local minima
    \item The KKT conditions provide both necessary and sufficient conditions for optimality
\end{itemize}


Applying the optimality conditions derived in Theorem~\ref{thm:Sample_size_est} to each subproblem yields the recursive solution:
\begin{equation*}
    H_k^* = \sqrt{\frac{\Delta_k}{C_k}} \frac{R_k}{\sum_{j=k}^K\sqrt{C_j\Delta_j}},
    \qquad 
    R_{k+1} = R_k - C_k H_k^*.
\end{equation*}
Unfolding this recursion provides the explicit forward iterative process:
\begin{equation}\label{eq:MFMC_New_RealValued_Sample_Size}
    H_1^* = \sqrt{\frac{\Delta_1}{C_1}} \frac{p}{\sum_{j=1}^K\sqrt{C_j\Delta_j}}, 
    \qquad 
    H_k^* = \sqrt{\frac{\Delta_k}{C_k}} \frac{p-\sum_{j=1}^{k-1}C_jH_j^*}{\sum_{j=k}^K\sqrt{C_j\Delta_j}}, 
    \quad k = 2,\ldots, K.
\end{equation}


This iterative scheme possesses several desirable properties: it maintains feasibility at every step, enables progressive budget allocation, provides computational advantages by decomposing a high-dimensional constrained optimization into a sequence of lower-dimensional problems, and maintains guaranties of global optimality through the dynamic programming framework.

%
\begin{theorem}[Monotonicity of the iterative formulation]\label{thm:Monotonicity_H_k}
Under the assumptions of Theorem \ref{thm:Sample_size_est}, the iteratively defined sample sizes $H_k^*$ in \eqref{eq:MFMC_New_RealValued_Sample_Size} is monotonically decreasing for all $K$.
\end{theorem}
%
\begin{proof}
Let $S_k = \sum_{j=k}^K \sqrt{C_j\Delta_j}$. Then
\[
S_{k-1} = \sqrt{C_{k-1}\Delta_{k-1}}+S_k, \quad H_k^* = \sqrt{\frac{\Delta_k}{C_k}}\frac{R_k}{S_k}
\]
\[
R_k = R_{k-1}-C_{k-1}H_{k-1}^* = R_{k-1}\left(1-\frac{\sqrt{C_{k-1}\Delta_{k-1}}}{S_{k-1}}\right) = R_{k-1}\frac{S_k}{S_{k-1}}
\]
Therefore, 
\[
\frac{R_k}{S_k} = \frac{R_{k-1}}{S_{k-1}}
\]
Therefore,
\[
H_k^* = \sqrt{\frac{\Delta_k}{C_k}}\frac{R_k}{S_k}>\sqrt{\frac{\Delta_{k-1}}{C_{k-1}}}\frac{R_{k-1}}{S_{k-1}} = H_{k-1}^*, \quad k=2,\ldots,K
\]

\end{proof}


\subsection{Equivalence with Continuous MFMC Solution}

The following theorem establishes the fundamental equivalence between the iterative formulation and the standard MFMC allocation, ensuring that the sequential approach preserves all optimality properties of the original continuous solution.

%
\begin{theorem}[Equivalence of iterative and standard MFMC formulations]\label{thm:MFMC_Iterative_RealValued_Sample_Size}
\JLcolor{Under the assumptions of Theorem \ref{thm:Sample_size_est}},
The iteratively defined sample sizes $H_k^*$ in \eqref{eq:MFMC_New_RealValued_Sample_Size} are identical to the standard real-valued MFMC sample sizes $N_k^*$ in \eqref{eq:MFMC_RealValued_Sample_Size}, i.e.,
\[
H_k^* = N_k^*
    = \sqrt{\frac{\Delta_k}{C_k}}\,
      \frac{p}{\sum_{j=1}^K \sqrt{C_j\Delta_j}}.
\]
Moreover, this iterative scheme preserves both the total computational budget and the optimal variance reduction:
\[
\sum_{k=1}^K C_k H_k^* = p, 
\qquad  
f(H_k^*) = \sum_{k=1}^K \frac{\Delta_k}{H_k^*} = \frac{1}{p} \left(\sum_{k=1}^K \sqrt{C_k\Delta_k}\right)^2.
\]
\end{theorem}
%


\begin{proof}
Define the cumulative cost and remaining budget by
\[
    T_k = \sum_{j=1}^k C_j H_j^*, 
    \qquad 
    R_k = p - T_k,
\]
so that $R_0 = p$ and $T_0 = 0$. For $k=1$, from \eqref{eq:MFMC_New_RealValued_Sample_Size} we obtain
\[
    T_1 = C_1H_1^* 
    = p\,\frac{\sqrt{C_1\Delta_1}}{\sum_{j=1}^K \sqrt{C_j\Delta_j}},
    \qquad
    R_1 = p - T_1
    = p\,\frac{\sum_{j=2}^K \sqrt{C_j\Delta_j}}{\sum_{j=1}^K \sqrt{C_j\Delta_j}}.
\]

For general $k\ge 1$, the iterative definition \eqref{eq:MFMC_New_RealValued_Sample_Size} gives
\[
    H_k^*
    = \sqrt{\frac{\Delta_k}{C_k}}\,
      \frac{R_{k-1}}{\sum_{j=k}^K \sqrt{C_j \Delta_j}},
\]
which implies that the remaining budget satisfies
\[
    R_k 
    = R_{k-1} - C_k H_k^*
    = \frac{\sum_{j=k+1}^K \sqrt{C_j \Delta_j}}
           {\sum_{j=k}^K \sqrt{C_j \Delta_j}} \, R_{k-1}.
\]
Hence, $\{R_k\}_{k=1}^K$ forms a geometric sequence. Introducing the normalization factor
\begin{equation}\label{eq:aggregate_cost_variance_weight_S}
    S := \sum_{j=1}^K \sqrt{C_j \Delta_j},
\end{equation}
we obtain by recursion that
\[
    R_k = \frac{p}{S}\sum_{j=k+1}^K\sqrt{C_j\Delta_j}.
\]
Substituting this expression into the formula for $H_k^*$ yields
\[
    H_k^*
    = \frac{p}{S}\sqrt{\frac{\Delta_k}{C_k}},
\]
which coincides with the closed-form MFMC sample size in \eqref{eq:MFMC_RealValued_Sample_Size}.  
In particular, for $k=K$, we have $R_K=0$, and thus $T_K = p - R_K = p$, verifying that the total cost constraint $\sum_{k=1}^K C_k H_k^* = p$ is satisfied.  
Finally,
\[
    f(H_k^*)
    = \sum_{k=1}^K \frac{\Delta_k}{H_k^*}
    = \frac{1}{p} \left(\sum_{k=1}^K \sqrt{C_k\Delta_k}\right)^2,
\]
which completes the proof.
\end{proof}

Theorem~\ref{thm:MFMC_Iterative_RealValued_Sample_Size} establishes that the iterative construction preserves the optimality properties of the continuous MFMC allocation. This sequential framework naturally extends to {\it integer-valued} sample sizes, which are essential for practical implementations. We now develop an iterative scheme for computing such integer-valued allocations, using the continuous formulation as a foundation.

The key idea is to maintain the analytical structure of the continuous optimizer while enforcing integer feasibility through a forward recursive process. Unlike the direct flooring of the continuous solution -- which may lead to suboptimal budget utilization due to independent rounding -- the iterative approach updates the residual budget using the accumulated integer costs from preceding levels. This ensures that the allocation at each step adapts to the actual remaining resources. The recursive definition is given by
%
\begin{equation}\label{eq:MFMC_New_IntegerValued_Sample_Size}
    M_1^* = \sqrt{\frac{\Delta_1}{C_1}}\frac{p}{\sum_{j=1}^K\sqrt{C_j\Delta_j}}, 
    \qquad 
    M_k^* = \sqrt{\frac{\Delta_k}{C_k}}\frac{p-\sum_{j=1}^{k-1}C_j\left\lfloor M_j^* \right\rfloor}{\sum_{j=k}^K\sqrt{C_j\Delta_j}}, 
    \quad k = 2,\ldots, K,
\end{equation}
%
with the integer-valued sample sizes taken as $\lfloor M_k^* \rfloor$ for $k=1,\ldots,K$.

The procedure begins with the continuous allocation $M_1^*$, from which $\lfloor M_1^* \rfloor$ is derived. The cost of this integer allocation is deducted from the total budget, and the process repeats for subsequent levels using the updated residual budget. By construction, this sequential method guarantees that the total cost constraint is satisfied exactly, as the residual budget at each step reflects the actual integer expenditures. Moreover, it preserves the relative variance–cost balancing of the continuous optimizer, as the allocation weights remain proportional to $\sqrt{\Delta_k/C_k}$.


To ensure that each fidelity level contributes at least one sample, the total budget must satisfy
%
\begin{equation}\label{eq:p_bound}
    p \ge \sum_{k=1}^K C_k.
\end{equation}
%
This condition is necessary and sufficient for feasibility, as it prevents over-allocation that would violate the monotonicity constraints or lead to insufficient samples at higher fidelity levels.

The following theorem demonstrates that the iterative integer-valued scheme achieves superior budget utilization compared to direct flooring, formalizing the advantage of adaptive resource allocation.
%
\begin{theorem}[Cost bound for the iterative integer-valued sample allocation]
\label{thm:MFMC_New_IntegerValued_Cost} 
Let $\lfloor M_k^* \rfloor$ denote the integer-valued sample sizes from the iterative scheme \eqref{eq:MFMC_New_IntegerValued_Sample_Size}, and let $\lfloor N_k^* \rfloor$ denote those from direct flooring of the continuous solution \eqref{eq:MFMC_RealValued_Sample_Size}. Under the feasibility condition \eqref{eq:p_bound}, the total costs satisfy
\begin{equation}\label{eq:Iterative_integer_sample_size_cost_bound}
    \sum_{k=1}^K C_k \left\lfloor N_k^* \right\rfloor
    \;\le\;
    \sum_{k=1}^K C_k \left\lfloor M_k^* \right\rfloor
    \;\le\;
    p.
\end{equation}
\end{theorem}
%












\begin{proof}
We first show that the total cost of the iterative scheme does not exceed the prescribed budget,
\[
\sum_{k=1}^K C_k \left\lfloor M_k^* \right\rfloor \le p.
\]
Define the cumulative integer cost up to level $k$ as
\[
T_k = \sum_{j=1}^k C_j\left\lfloor M_j^* \right\rfloor.
\]
Since $\lfloor M_j^* \rfloor \le M_j^*$, we claim, and prove by induction, that for each $k = 1, \ldots, K$,
\begin{equation}\label{eq:Tk_bound}
T_k \le \frac{p}{S}\sum_{j=1}^k \sqrt{C_j \Delta_j}.
\end{equation}
where $S$ is the aggregate cost–variance weight defined in \eqref{eq:aggregate_cost_variance_weight_S}. 
Inequality \eqref{eq:Tk_bound} bounds the cumulative integer cost at level $k$ by a proportional share of the total budget, scaled by $S$.







The base case $k=1$ follows immediately,
\[
T_1=C_1 \left\lfloor M_1^* \right\rfloor \le C_1M_1^* = \frac{p}{S}\sqrt{C_1\Delta_1},
\]
so \eqref{eq:Tk_bound} holds for \(k=1\). Assume \eqref{eq:Tk_bound} holds for \(k-1\). By definition of \(M_k^*\),
%
\[
M_k^* = \sqrt{\frac{\Delta_k}{C_k}}\frac{p - T_{k-1}}{\sum_{j=k}^K \sqrt{C_j\Delta_j}},
\]
%
and hence
%
\[
C_k \left\lfloor M_k^* \right\rfloor \le C_k M_k^*  = \sqrt{C_k\Delta_k}\frac{p-T_{k-1}}{\sum_{j=k}^K\sqrt{C_j\Delta_j}}.
\]
%
Using the inductive hypothesis and simplifying yields
\begin{align*}
    T_k &= T_{k-1}+C_k\left\lfloor M_k^* \right\rfloor \\
    &\le T_{k-1} + \sqrt{C_k\Delta_k}\frac{p-T_{k-1}}{\sum_{j=k}^K\sqrt{C_j\Delta_j}}
    =T_{k-1}\left(1-\frac{\sqrt{C_k\Delta_k}}{\sum_{j=k}^K\sqrt{C_j\Delta_j}}\right) + p\frac{\sqrt{C_k\Delta_k}}{\sum_{j=k}^K\sqrt{C_j\Delta_j}}\\
    &\le \frac{p}{S}\sum_{j=1}^{k-1} \sqrt{C_j\Delta_j}\frac{\sum_{j=k+1}^K\sqrt{C_j\Delta_j}}{\sum_{j=k}^K\sqrt{C_j\Delta_j}}+p\frac{\sqrt{C_k\Delta_k}}{\sum_{j=k}^K\sqrt{C_j\Delta_j}}=\frac{p}{S}\cdot \frac{\sum_{j=1}^{k-1} \sqrt{C_j\Delta_j}\sum_{j=k+1}^K\sqrt{C_j\Delta_j}+S\sqrt{C_k\Delta_k}}{\sum_{j=k}^K\sqrt{C_j\Delta_j}}\\
    &=\frac{p}{S}\cdot \frac{\sum_{j=1}^{k-1} \sqrt{C_j\Delta_j}\sum_{j=k+1}^K\sqrt{C_j\Delta_j}+\sqrt{C_k\Delta_k}\left(\sum_{j=1}^{k-1}\sqrt{C_j\Delta_j}+\sum_{j=k}^K\sqrt{C_j\Delta_j}\right)}{\sum_{j=k}^K\sqrt{C_j\Delta_j}}
    %=\frac{p}{S}\frac{\sum_{j=k}^K\sqrt{C_j\Delta_j}\sum_{j=1}^k\sqrt{C_j\Delta_j}}{\sum_{j=k}^K\sqrt{C_j\Delta_j}}
    =\frac{p}{S}\sum_{j=1}^k\sqrt{C_j\Delta_j}.
\end{align*}
%
Thus, inequality \eqref{eq:Tk_bound} holds for all $k$. 

In particular, when $k=K$, we obtain
\begin{equation}\label{eq:MFMC_iterative_total_cost}
T_K = \sum_{j=1}^K C_j\left\lfloor M_j^*\right\rfloor \le p,
\end{equation}
confirming that the iterative scheme never exceeds the prescribed computational budget. To establish the lower bound in \eqref{eq:Iterative_integer_sample_size_cost_bound}, we compare the auxiliary sequences $M_k^*$ and $N_k^*$. From \eqref{eq:Tk_bound},
%
\[
M_k^* = \sqrt{\frac{\Delta_k}{C_k}}\frac{p - T_{k-1}}{\sum_{j=k}^K\sqrt{C_j\Delta_j}} \ge \sqrt{\frac{\Delta_k}{C_k}}\frac{p-\frac{p}{S}\sum_{j=1}^{k-1}\sqrt{C_j\Delta_j}}{\sum_{j=k}^K\sqrt{C_j\Delta_j}} = \sqrt{\frac{\Delta_k}{C_k}}\frac{p}{S}=N_k^*, \qquad k \ge 1.
\]
% 
Monotonicity of the floor function implies \(\lfloor M_k^*\rfloor\ge\lfloor N_k^*\rfloor\) for every \(k\). Multiplying by \(C_k\) and summing yields the desired lower bound in 
\eqref{eq:Iterative_integer_sample_size_cost_bound}.  
Combining this with \eqref{eq:MFMC_iterative_total_cost} completes the proof of the cost bound.


\medskip
\noindent
\textit{ Equality conditions.}
The upper bound in \eqref{eq:Iterative_integer_sample_size_cost_bound} is attained if and only if no budget remains unused, i.e.,
\[
\sum_{k=1}^K C_k \left(M_k^* - \left\lfloor M_k^*\right\rfloor\right) = 0.
\]
Because each fractional part satisfies $M_k^* - \lfloor M_k^* \rfloor \in [0,1)$, this condition holds precisely when every $M_k^*$ is an integer. Similarly, the lower bound is attained if and only if \(\lfloor M_k^* \rfloor = \lfloor N_k^* \rfloor\) for all \(k\), equivalently when both real-valued allocations \(M_k^*\) and \(N_k^*\) lie in the same integer interval
%
\[
\left\lfloor N_k^*\right\rfloor\le M_k^* < \left\lfloor N_k^*\right\rfloor + 1.
\]
Since \(M_k^* \ge N_k^*\), this occurs precisely when they share the same integer part for every \(k\).

\end{proof}


Theorem \ref{thm:MFMC_New_IntegerValued_Variance} establishes that the iterative integer-valued allocation achieves superior variance performance compared to direct flooring while maintaining proximity to the continuous optimum, thus providing an improved variance--cost tradeoff under fixed budget constraints.
%
\begin{theorem}[Normalized variance bound for the iterative integer-valued sample allocation]
\label{thm:MFMC_New_IntegerValued_Variance}
Let $\lfloor M_k^* \rfloor$ denote the integer-valued sample sizes obtained from the iterative allocation scheme \eqref{eq:MFMC_New_IntegerValued_Sample_Size}, and let $\lfloor N_k^* \rfloor$ denote those obtained by directly flooring the real-valued optimal allocation \eqref{eq:MFMC_RealValued_Sample_Size}. Under the computational budget constraint \eqref{eq:p_bound} and the standard variance conditions on $\Delta_k$ from Theorem~\ref{thm:Sample_size_est}, the normalized variance satisfies the following bounds
%
\begin{equation}\label{eq:Iterative_Integer_Variance_Bound}
\frac{1}{p}\left(\sum_{k=1}^K \sqrt{C_k \Delta_k}\right)^2
= \sum_{k=1}^K \frac{\Delta_k}{N_k^*}
\;\le\;
\sum_{k=1}^K \frac{\Delta_k}{\left\lfloor M_k^* \right\rfloor}
\;\le\;
\sum_{k=1}^K \frac{\Delta_k}{\left\lfloor N_k^* \right\rfloor}.
\end{equation}
%
\end{theorem}
%




\begin{proof}
We first prove the upper bound in \eqref{eq:Iterative_Integer_Variance_Bound}.  
From Theorem~\ref{thm:MFMC_New_IntegerValued_Cost}, it follows that $\lfloor M_k^* \rfloor \ge \lfloor N_k^* \rfloor$ for all $k$.  
Since $x \mapsto \Delta_k/x$ is strictly decreasing for $x > 0$, it immediately follows that
\[
\sum_{k=1}^K \frac{\Delta_k}{\left\lfloor M_k^* \right\rfloor} 
\le \sum_{k=1}^K \frac{\Delta_k}{\left\lfloor N_k^* \right\rfloor}.
\]

\medskip
\noindent
To establish the lower bound, we apply the Cauchy--Schwarz inequality:
%
\[
\left(\sum_{k=1}^K \sqrt{C_k \Delta_k}\right)^2
\le
\left(\sum_{k=1}^K C_k \left\lfloor M_k^* \right\rfloor\right)
\left(\sum_{k=1}^K \frac{\Delta_k}{\left\lfloor M_k^* \right\rfloor}\right).
\]
%
From the cost bound \eqref{eq:MFMC_iterative_total_cost}, it follows that 
$\sum_{k=1}^K C_k \lfloor M_k^* \rfloor \le p$, and hence
%
\[
\frac{1}{p}\left(\sum_{k=1}^K \sqrt{C_k \Delta_k}\right)^2
\le
\sum_{k=1}^K \frac{\Delta_k}{\left\lfloor M_k^* \right\rfloor}.
\]
%
which proves the lower bound.

\medskip
\noindent
\textit{Equality conditions.} 
Equality in the Cauchy--Schwarz step occurs if and only if there exists a constant $\lambda>0$ such that
\[
\left\lfloor M_k^* \right\rfloor = \lambda \sqrt{\frac{\Delta_k}{C_k}}, \qquad k=1,\ldots,K,
\]
which corresponds to the continuous optimal allocation $M_k^* = N_k^*$. Therefore, equality in \eqref{eq:Iterative_Integer_Variance_Bound} holds if and only if the iterative scheme reproduces the continuous solution exactly, i.e., when all $\lfloor M_k^* \rfloor = N_k^*$ and the total cost equals $p$.
\end{proof}



% ------------------
% Next consider the variance
% \begin{align*}
%     f_{\text{act}}(\overline{N_k})&=\sum_{i=1}^{K}\frac{\Delta_k}{\overline{N_k}}=\sum_{i=1}^{k-1}\frac{\Delta_i}{\overline{N_i}}+\frac{\Delta_k}{\overline{N_k}}+\sum_{i=k+1}^{K}\frac{\Delta_i}{\overline{N_i}}\\
%     &\in \left[\sum_{i=1}^{k-1}\frac{\Delta_i}{\overline{N_i}}+\frac{\Delta_k}{\overline{N_k}}+\sum_{i=k+1}^K\frac{\Delta_{i}}{N_i^*},\; \sum_{i=1}^{k-1}\frac{\Delta_i}{\overline{N_i}}+\frac{\Delta_k}{\overline{N_k}}+\sum_{i=k+1}^K\frac{\Delta_{i}}{N_i^*-1}\right)=[f_1,f_2)\\
%     f_1&=\sum_{i=1}^{k-1}\frac{\Delta_i}{\overline{N_i}}+\frac{\Delta_k}{\overline{N_k}}+\sum_{i=k+1}^K\frac{\Delta_{i}}{N_i^*}=\sum_{i=1}^{k-1}\frac{\Delta_i}{\overline{N_i}}+\frac{\Delta_k}{\overline{N_k}}+\sum_{i=k+1}^K\sqrt{C_i\Delta_i}\frac{\sum_{j=i}^K\sqrt{C_j\Delta_j}}{p-\sum_{j=1}^{i-1}C_j\overline{N_j}}\\
%     f_2&=\sum_{i=1}^{k-1}\frac{\Delta_i}{\overline{N_i}}+\frac{\Delta_k}{\overline{N_k}}+\sum_{i=k+1}^K \ldots\\
% \end{align*}
% \begin{align*}
%     \frac{d f_1}{d \overline{N_k}}&=-\frac{\Delta_k}{\overline{N_k}^2}+C_k\sum_{i=k+1}^K\sqrt{C_i\Delta_i}\frac{\sum_{j=i}^K\sqrt{C_j\Delta_j}}{\left(p-\sum_{j=1}^{i-1}C_j\overline{N_j}\right)^2}\\
%     \frac{d^2 f_1}{d^2 \overline{N_k}}&=\frac{2\Delta_k}{\overline{N_k}^3}+2C_k^2\sum_{i=k+1}^K\sqrt{C_i\Delta_i}\frac{\sum_{j=i}^K\sqrt{C_j\Delta_j}}{\left(p-\sum_{j=1}^{i-1}C_j\overline{N_j}\right)^3}
% \end{align*}
% Note that $\frac{d^2 f_1}{d^2 \overline{N_k}}>0$ whenever $p>\sum_{j=1}^{i-1}C_j\overline{N_j}$. this means $f_1$ is convex in $\overline{N_k}$.
\input{./Sections/Modified_iterative_Integer_Valued_Sample_Size}
% ====================================================
\section{Numerical results}\label{sec:Num_Result}
% ====================================================

\subsection{First example}
%
\begin{table}[ht]
\centering
\scalebox{1}{
\begin{tabular}{|c|c|c|c|c|c|c|}
\hline
Model index &1 &2 &3 &4 &5 \\
\hline
Correlation coeff $\rho_{1,k}$ &1     &9.9977e-01   &9.9925e-01  &9.9728e-01   &9.8390e-01\\
% \hline
% Standard deviation $\sigma_k$ &1.0840e-02    &1.0838e-02   &1.1001e-02  &1.1549e-02   &9.5720e-03\\
\hline
Cost &73&7.0318e-03 &1.4018e-03 &5.0613e-04 &2.6803e-04\\
\hline
\end{tabular}
}
\caption{Parameters from plasma problem.}
\label{Tab:Parameters}
\end{table}
%






% %
% \begin{table}[ht]
% \centering
% \scalebox{0.6}{
% \begin{tabular}{|c|c|c|c|c|c|c|c|c|c|}
% \hline
% Total cost $P$ &73.05 &73.051 &73.052 &73.053 &73.054 &73.055 &73.056\\
% \hline
% Sample size (real valued) &[1,   134,   588   2540,  21100] &[1,   135,   588,   2541,  21101]&same &same &[1,   135,   588,   2541,  21102] &same &same\\
% \hline
% Sample size (integer program) &[1, 2, 3, 11, 97]&[1, 2, 3, 12, 99]&-&-&-&-&-\\
% \hline
% CPU time for integer program [s] & 0.27 &0.32 &$>$ 1000 &$>$ 1000 &$>$ 1000 &$>$ 1000 &$>$ 1000\\
% \hline
% \end{tabular}
% }
% \caption{Sample size for real-valued optimization and integer optimization.}
% \label{Tab:Sample_Size}
% \end{table}
% %

%
\begin{table}[ht]
\centering
\scalebox{1}{
\begin{tabular}{|c|c|c|c|c|c|c|c|c|c|}
\hline
&Sample size &Total cost $p$ &$f$\\
\hline
Real valued &[2.4129e+00 3.6959e+02 1.6102e+03 6.9567e+03 5.7770e+04]&200&2.1645e-04\\
\hline
Integer, floor &[2, 369, 1610, 6956, 57769]&1.698561e+02&2.5580e-04\\
\hline
Integer, iterative&[2, 836, 3644, 15744, 130749]&1.999999e+02&2.4138e-04\\
% \hline
% CPU time for integer program [s] & 0.27\\
\hline
\end{tabular}
}
\caption{Sample size for real-valued optimization and integer optimization for $p=200$.}
\label{Tab:Sample_Size_1}
\end{table}
%

%
\begin{table}[ht]
\centering
\scalebox{1}{
\begin{tabular}{|c|c|c|c|c|c|c|c|c|c|}
\hline
&Sample size &Total cost $p$ &$f$\\
\hline
Real valued &[8.8106e-01,   1.3495e+02,   5.8795e+02,   2.5402e+03,  2.1095e+04]&73.03&5.9275e-04\\
\hline
Integer, floor &[0,         134,         587,        2540,  21094]&8.7045&$\infty$\\
\hline
Modified &[1,     1,     1,     7,    62]&7.302859e+01&2.4834e-02\\
% \hline
\hline
Integer, iterative &[1,     1,     1,     7,    67]&7.302993e+01&2.3668e-02\\
% \hline
% CPU time for integer program [s] & 0.27\\
\hline
\end{tabular}
}
\caption{Sample size for real-valued optimization and integer optimization for $p=73.03$.}
\label{Tab:Sample_Size_1}
\end{table}
%



\subsection{Second example}

%
\begin{table}[ht]
\centering
\scalebox{1}{
\begin{tabular}{|c|c|c|c|c|c|c|}
\hline
Model index &1 &2 &3 &4 \\
\hline
Correlation coeff $\rho_{1,k}$ &1     &9.999882e-01  &9.999743e-01 &9.958253e-01\\
% \hline
% Standard deviation $\sigma_k$ &0.03\\
\hline
Cost &44.395 &6.8409e-01 &2.9937e-01 &1.9908e-04\\
\hline
\end{tabular}
}
\caption{Parameters from Peherstorfer's paper \cite{PeWiGu:2016}.}
\label{Tab:Parameters}
\end{table}
%

%
\begin{table}[ht]
\centering
\scalebox{1}{
\begin{tabular}{|c|c|c|c|c|c|c|c|c|c|}
\hline
&Sample size &Total cost $p$ &$f$\\
\hline
Real valued &[3.334886e-01 2.915781e+00 7.607103e+01 3.228216e+04]&46&2.1987267e-04\\
\hline
Integer, floor &[0, 2, 76, 32282]&3.054700e+01&$\infty$\\
\hline
Modified &[1, 1, 2, 1018]&4.588049e+01&5.1658e-03\\
% \hline
\hline
Integer, iterative &[1, 1, 2, 1618]&4.599994e+01&4.8046e-03\\
% \hline
% CPU time for integer program [s] & 0.27\\
\hline
\end{tabular}
}
\caption{Sample size for real-valued optimization and integer optimization for $p=46$.}
\label{Tab:Sample_Size_1}
\end{table}
%



 
% ====================================================
\section{Conclusion}
% ====================================================
This paper has introduced a novel iterative framework for integer-valued sample size allocation in multi-fidelity Monte Carlo estimation, addressing a critical implementation gap between theoretical continuous optima and practical discrete requirements. Through a formulation grounded in dynamic programming principles and Bellman's principle of optimality, we have developed a sequential allocation scheme that preserves the theoretical foundations of MFMC estimation while enforcing integer constraints and maintaining strict budget adherence. From a practical perspective, the algorithm's linear computational complexity and sequential decision-making structure make it particularly suitable for high-dimensional fidelity hierarchies and resource-constrained environments. 

The proposed method demonstrates significant advantages over conventional approaches: it reduces the budget under-utilization inherent in direct flooring strategies, and achieves superior variance characteristics compared to existing modified rounding procedures. 

 
% \input{./Sections/Illustration}


\bibliographystyle{abbrv}
% \bibliographystyle{alphaurl}
\bibliography{references_liang}
% \bibliography{reference}
\end{document}


