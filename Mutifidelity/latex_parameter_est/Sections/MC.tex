% ====================================================
\section{Monte Carlo estimator}\label{sec:MC}
% ====================================================
The objective of this paper is to investigate the propagation of uncertainty and to efficiently approximate the parametric expectation 
%
 \begin{equation}
 \label{eq:QoI}
      \mathbb{E}\left(u(\cdot,\boldsymbol \omega)\right)=\int_W u(\cdot,\boldsymbol{\omega})\pi(\boldsymbol\omega)d\boldsymbol{\omega},
 \end{equation}
%
A standard approach for approximating this expectation is the Monte Carlo (MC) method. Let $u_h (\cdot, \boldsymbol{\omega})$ denote a discrete approximation of $u(\cdot, \boldsymbol{\omega})$, obtained via spatial discretization with mesh parameter $h$.  For convenience, we use $u$ and $u_h$ to refer to the exact ($u(\cdot,\boldsymbol \omega)$) and discrete ($u_h(\cdot,\boldsymbol \omega)$) solutions evaluated at a realization $\boldsymbol \omega$.


The Monte Carlo Finite-Element estimator $A^{\text{MC}}_{N}$ is defined as the sample mean over $N$ independent and identically distributed (i.i.d.) samples $\boldsymbol{\omega}^{(1)},\ldots,\boldsymbol{\omega}^{(N)}$
%
\begin{equation}\label{eq:MC_estimator}
    A^{\text{MC}}_{N} := \frac{1}{N}\sum_{i=1}^{N} u_{h}\left(\cdot, \boldsymbol{\omega}^{(i)}\right).
\end{equation}
%
This estimator is unbiased, satisfying $\mathbb{E}(A^{\text{MC}}_{N}) = \mathbb{E}(u_{h})$, and has variance $\mathbb{V}(A^{\text{MC}}_{N}) = \mathbb{V}( u_{h})/{N}$, where the variance of a random field is $\mathbb{V}(u) := \mathbb{E}(\left\Vert u - \mathbb{E}(u)\right\Vert_Z^2)$. By the central limit theorem, the estimator \eqref{eq:MC_estimator} converges in distribution to $\mathbb{E}(u)$ as $N\rightarrow \infty$. To quantify the accuracy of the estimator, we use the  {\it normalized mean squared error (nMSE)}, defined as
%
 \[
\mathcal{E}_{A^{\text{MC}}_{N}}^2:=\frac{\mathbb E\left[\left\Vert\mathbb{E}(u)-A^{\text{MC}}_{N} \right\Vert_{Z}^2\right]}{\left\Vert\mathbb{E}(u) \right\Vert_{Z}^2}.
\] 
%
The nMSE decomposes into two terms: the {\it bias error}, due to spatial discretization, and the {\it statistical error}, result from finite sampling
%
\[
\mathcal{E}_{A^{\text{MC}}_{N}}^2 = \frac{\left\Vert\mathbb{E}(u)-\mathbb{E}(u_{h}) \right\Vert_{Z}^2+\mathbb E\left[\left\Vert \mathbb{E}(u_{h}) -A^{\text{MC}}_{N} \right\Vert_{Z}^2\right]}{\left\Vert\mathbb{E}(u) \right\Vert_{Z}^2} = \frac{\left\Vert\mathbb{E}(u)-\mathbb{E}(u_{h}) \right\Vert_{Z}^2}{\left\Vert\mathbb{E}(u) \right\Vert_{Z}^2}+\frac{\mathbb{V}\left( u_{h}\right)}{N\left\Vert\mathbb{E}(u) \right\Vert_{Z}^2}=\mathcal{E}_{\text{Bias}}^2 + \mathcal{E}_{\text{Stat}}^2.
\]
%
Suppose the sample-wise discretization error satisfies
%
\begin{equation*} \label{eq:Assumption_uhA}
\left\|u\left(\cdot, \boldsymbol\omega^{(i)}\right)-u_h\left(\cdot,\boldsymbol\omega^{(i)}\right)\right\|_Z\leq C_m\left(\boldsymbol\omega^{(i)}\right)M^{-\alpha}\,,
\end{equation*}
%
where $C_m(\boldsymbol\omega^{(i)})$ is a constant depending only on the geometry of the spatial domain and the particular realization $\boldsymbol\omega^{(i)}$, $\alpha>0$ is the convergence rate of spatial discretization, and $M$ denotes the number of spatial degrees of freedom.

Given a user-specified threshold $\epsilon^2$  for the nMSE, we introduce a {\it splitting ratio} $\theta \in (0,1)$ to allocate the total error budget between bias and statistical components
%
\begin{equation} \label{eq:error-budget}
%\textcolor{red}{\|u-u_h\|_{L^2(\boldsymbol W,Z)}\le C_mM^{-\alpha}\le \theta_1\epsilon},\qquad\text{ and }\qquad \|u_h-\widehat u_{h}\|_{L^2(\boldsymbol W,Z)} \le C_{p} P^{-\nu}\le \theta_2\epsilon\,.  
\mathcal{E}_{\text{Bias}}^2=\|u-u_h\|_{L^2(\boldsymbol W,Z)}\le C_mM^{-\alpha}= \theta\epsilon^2, \quad\quad \mathcal{E}_{\text{Stat}}^2 = \frac{\sigma_1^2}{N\left\Vert\mathbb{E}(u) \right\Vert_{Z}^2}=(1-\theta)\epsilon^2,
\end{equation}
where $C_m$ is independent of the sample and $\sigma_1^2 = \mathbb{V}\left( u_{h}\right)$. To meet these error constraints, the number of spatial nodes $M$ and sample size $N$ must obey
%
\begin{equation}
\label{eq:SLSGC_SL_SpatialGridsNo_n_SparseGridsNo}
M\ge \left(\frac{\theta\epsilon^2}{C_m}\right)^{-\frac 1 {\alpha}},\quad\quad  N \ge  \frac{\sigma_1^2}{\epsilon_{\text{tar}}^2},
\end{equation}
%
where $\epsilon_{\text{tar}}^2 = \epsilon^2(1-\theta)\left\Vert\mathbb{E}(u) \right\Vert_{Z}^2$.
Assuming each evaluation of $u_{h}$ incurs an average cost of $C$, the total cost to compute $A^{\text{MC}}_{N}$ is
%
\[
\mathcal{W}^\text{MC}  = CN=\frac{C\sigma_1^2}{\epsilon_{\text{tar}}^2}.
\]
%
In practice, both $M$ and $N$ are rounded up to the smallest integers satisfying \eqref{eq:SLSGC_SL_SpatialGridsNo_n_SparseGridsNo}.