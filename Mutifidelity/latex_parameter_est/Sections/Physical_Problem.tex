%!TEX root = ../main.tex
% ========================================
\section{Problem setting}\label{sec:Problem_setting}
% ========================================
We are interested in solving a parametrized partial differential equation under uncertainty quantification. Let $D\subset \mathbb{R}^n$ be a bounded, Lipschitz domain, and let $\mathcal{L}$ denote a (possibly nonlinear) differential operator posed on $D$. Uncertainty is introduced through randomness in the coefficients of $\mathcal{L}$ and/or the source term. This randomness is modeled on a complete probability space $(\Omega, \mathcal{F}, P)$, where $\Omega$ denotes the set of outcomes, $\mathcal{F}\subset 2^{\Omega}$ is a $\sigma-$algebra of events, and $P:\mathcal{F}\rightarrow [0,1]$ is a probability measure. The input data -- such as the coefficient field $a=a(\boldsymbol x, \boldsymbol \omega)$ and source term $f = f(\boldsymbol x, \boldsymbol\omega)$ -- are modeled as random fields, i.e., measurable functions defined on $D\times \Omega$. The solution map $u(\cdot, \boldsymbol{\omega}): \Omega \to U$, where $U$ is an appropriate spatial function space (e.g. $H_0^1(D)$), is defined such that for almost every $\boldsymbol\omega\in \Omega$, the function $u(\cdot, \boldsymbol{\omega})$ satisfies the stochastic boundary value problem
%
\begin{equation}\label{eq:Problem}
    \mathcal{L}(a)(u) = f \quad \text{in} \;\; D
\end{equation}
%
with appropriate boundary conditions on $\partial D$. 

% Moreover, we make the assumption that $a(\cdot, \omega)$ is bounded from below (either uniform or a random variable) on $D$ and $f(\cdot, \omega)$ is square integrable with respect to $P$. Then by the Lax-Milgram theorem, the problem admits a unique solution \cite{BaNoTe:2007}.

To characterize the dependence of the solution on both spatial and probabilistic variables, we adopt a Bochner space framework. For $q\in [1,\infty]$, the Bochner space $L^q(\Omega,U)$ consists of all strongly measurable mappings $u:\Omega\rightarrow U$ with finite norm
%
\[
L^q(\Omega,U) = \left\{u:\Omega\rightarrow U\; \bigg\vert \;\left\|u\right\|_{L^q(\Omega,U)}<\infty\right\},
\]
%
where the norm $\|u\|_{L^q(\Omega,U)}$ is given by
%
\[
\left\Vert u \right\Vert_{L^q(\Omega,U)} =\left\{\begin{array}{lll}
     \left(\int_{\Omega} \left\Vert u(\cdot,\boldsymbol{\omega})  \right\Vert_{U}^q \pi(\boldsymbol{\omega})d\boldsymbol{\omega} \right)^{1/q} = \left(\mathbb{E}\left[\left\Vert u(\cdot,\boldsymbol{\omega})  \right\Vert_{U}^q\right]\right)^{1/q}, & 0<q<\infty, \\
     \text{ess} \sup_{\boldsymbol{\omega}\in \Omega}\left\Vert u(\cdot,\boldsymbol{\omega})  \right\Vert_{U}, & q=\infty.
\end{array}
\right.
\]
and $\pi(\boldsymbol{\omega})$ denotes the joint probability density of $\boldsymbol{\omega}$.

The primary goal of this work is to analyze how uncertainties in the input data propagate through the partial differential equation to affect the solution $u$, and to develop efficient numerical methods for computing statistical quantities of interest. A fundamental example is the parametric expectation of the solution:
%
 \begin{equation}
 \label{eq:QoI}
      \mathbb{E}\left[u(\cdot,\boldsymbol \omega)\right]=\int_\Omega u(\cdot,\boldsymbol{\omega})\pi(\boldsymbol\omega)d\boldsymbol{\omega},
 \end{equation}
%

% Incorporating the uncertainty and 
% %
% \begin{equation}\label{eq:FreeBoundarya}
%  -\nabla\,\cdot\,\left(\frac{1}{\mu(u(\cdot, \boldsymbol{\omega})) r}\nabla u(\cdot, \boldsymbol{\omega})\right) = \left\{ \begin{array}{ll}
% \frac{d}{du} p(u(\cdot, \boldsymbol{\omega})) + \frac{1}{2\,\mu r} \frac{d}{du} g^2(u(\cdot, \boldsymbol{\omega})) & \text{ in } \Omega_p(u(\cdot, \boldsymbol{\omega})) \\
% I_k(\boldsymbol\omega)/S_k & \text{ in } \Omega_{C_k} \\
% 0 & \text{ elsewhere, } 
% \end{array}\right.
% \end{equation}