\documentclass[final,3p,times,11pt]{elsarticle}
\usepackage[USenglish]{babel}
\usepackage{amsmath,amssymb,amsthm, mathrsfs,multirow}
\usepackage{mathtools}
\usepackage{graphicx}
\usepackage{stmaryrd}
\usepackage[dvipsnames]{xcolor}
\usepackage{cancel}
\usepackage{ulem}
\usepackage{tabularx}
\usepackage{comment}
%\usepackage{subcaption}
%\usepackage[show]{ed}
%\usepackage{showkeys}
%\usepackage{showlabels}
%\usepackage[notcite,notref]{showkeys}
%\usepackage{refcheck}
% \usepackage[ruled,vlined]{algorithm2e}
\usepackage[linesnumbered,ruled,vlined]{algorithm2e}
\definecolor{Myblue}{rgb}{.2 0.4 1}

\usepackage{hyperref}
\hypersetup{
    %bookmarks=true,         % show bookmarks bar?
    colorlinks = true,       % false: boxed links; true: colored links
    % linkcolor=green
     %linkcolor=red,          % color of internal links (change box color with linkbordercolor)
     %citecolor=green,        % color of links to bibliography
    %filecolor=magenta,      % color of file links
    %urlcolor=cyan           % color of external links
}
%\usepackage{wrapfig}
%% The lineno packages adds line numbers. Start line numbering with
%% \begin{linenumbers}, end it with \end{linenumbers}. Or switch it on
%% for the whole article with \linenumbers after \end{frontmatter}.
%\usepackage{lineno}


% ==============   Macros  ====================
\newcommand{\mynabla}{\widetilde{\nabla}} 
\newcommand{\jump}[1]{[\![#1]\!]}
\newcommand{\HEcolor}[1]{{\textcolor{blue}{#1}}}
\newcommand{\TSVcolor}[1]{{\textcolor{orange}{#1}}}
\newcommand{\JLcolor}[1]{{\textcolor{violet}{#1}}} %violet
\newcommand{\Grids}{\boldsymbol{\chi}}

\newtheorem{theorem}{Theorem}%[section]
\newtheorem{VariationalForm}[theorem]{Variational Formulation}
% =============================================

\journal{}
\makeatletter
\def\ps@pprintTitle{%
 \let\@oddhead\@empty
 \let\@evenhead\@empty
 \def\@oddfoot{}%
 \let\@evenfoot\@oddfoot}
\makeatother




\begin{document}
\begin{frontmatter}
\title{Efficient parameter estimation for Multi-fidelity Monte Carlo for uncertainty quantification}


\author[umdcs]{Matthias Heinkenschloss}
\ead{heinken@rice.edu}
\address[umdcs]{Department of Computational Applied Mathematics \& Operations Research, Rice University.}
\author[umdm]{Jiaxing Liang}
\ead{jl508@rice.edu}
\address[umdm]{Department of Computational Applied Mathematics \& Operations Research, Rice University.}
% \author[UA]{Tonatiuh S\'anchez-Vizuet}
% \ead{tonatiuh@arizona.edu}
% \address[UA]{Department of Mathematics, The University of Arizona.}
\begin{abstract}
% We investigate the Grad-Shafranov free boundary problem in Tokamak fusion reactors under the influence of parameter uncertainties. Using both traditional Monte Carlo and multi-fidelity Monte Carlo sampling approaches, we quantify the impact of these uncertainties on model predictions, emphasizing the statistical characterization of solution variability across diverse parameter regimes. Our numerical results reveals that the multi-fidelity Monte Carlo estimator achieves statistical accuracy comparable to the Monte Carlo approach. However, the multi-fidelity method demonstrates superior computational efficiency, achieving a cost reduction by a factor of ..., while preserving fidelity in representing plasma boundary dynamics and geometric parameters. This work underscores the efficiency of multi-fidelity frameworks in addressing the computational demands of uncertainty quantification in complex fusion reactor models, offering a robust pathway for enhancing predictive capabilities in plasma physics. 
\end{abstract}

\begin{keyword}
Multi-fidelity Monte Carlo Finite-Element \sep Parameter estimation for correlation coefficient\sep Parametric expectation \sep Sparse Grid Stochastic Collocation \sep Uncertainty Quantification
%
\MSC[2020] 
\end{keyword}
\end{frontmatter}

% ========================================
\section{Introduction}\label{sec:intro}
% ========================================
The pursuit of controlled nuclear fusion as a clean and virtually limitless energy source has spurred extensive research into the physics of magnetic confinement in fusion reactors. At the core of this effort lies the Grad–Shafranov free-boundary problem, which governs the equilibrium state of plasma in axially symmetric geometries, such as those found in Tokamaks. The governing equation encapsulates the intricate interplay between magnetic fields and plasma pressure, which determines critical confinement and stability properties essential for efficient plasma performance. However, the predictive accuracy of these models is significantly challenged by uncertainties in the parameters arising from measurement limitations, model assumptions, and operational variability. Addressing these uncertainties effectively requires advanced computational frameworks capable of robust statistical analysis, enabling accurate predictions of the plasma equilibrium response under diverse scenarios and ensuring reliable assessments of reactor designs and operations.

This study focuses on estimating the expectation of the solution operator associated with plasma equilibrium with uncertainties in the parameters. The Monte Carlo (MC) method, a classical and widely used approach in uncertainty quantification, relies on repeated executions of deterministic solvers to generate ensembles of realizations for stochastic inputs. Despite its versatility, its practicality is often limited by its slow asymptotic convergence rate of $1/\sqrt{N}$, which often requires a substantial number of sample realizations, $N$, to achieve reliable accuracy. For problems involving non-linear partial differential equations (PDEs), this slow convergence translates into a significant computational burden, as each realization typically demands a high-fidelity numerical approximation, such as those obtained via finite element method, which are computationally expensive due to their fine spatial resolution. Consequently, the cost of using the MC method can quickly escalate, particularly for high-dimensional problems or those requiring precise solutions. To alleviate this challenge, low-fidelity models have been proposed as computationally efficient alternatives to high-fidelity simulations.  These models aim to approximate the underlying system with reduced computational cost while maintaining an acceptable level of accuracy. For example, \cite{ElLiSa:2022} demonstrates how low-fidelity models constructed using stochastic collocation can effectively accelerate Monte Carlo sampling by exploiting simplified representations of the system. Similarly,
\cite{ElLiSa:2023} investigates hierarchical coarse spatial grids to develop low-fidelity surrogate models for multilevel Monte Carlo (MLMC) frameworks \cite{BaScZo:2011,Gi:2008}. Building on these ideas, studies such as \cite{ElLiSa:2025, Li:2024} combine stochastic collocation techniques with multilevel approaches, constructing low-fidelity models on coarse grids for MLMC sampling, further reducing computational expenses. While these methods achieve notable reductions in computational cost, they introduce the risk of compromising accuracy due to the inherent simplifications in the surrogate models. The trade-off between computational efficiency and solution accuracy is, therefore, a critical issue that warrants careful examination to ensure the reliability of results.


In this work, we delve into the multi-fidelity Monte Carlo (MFMC) method \cite{PeWiGu:2016, PeGuWi:2018}, which uses the control variate approach to exploit correlations between a computationally expensive high-fidelity model and a series of low-fidelity models. The MFMC method distinguishes itself from the MLMC approach by adopting a different strategy to construct its estimator. In MLMC, sample corrections are accumulated starting from the coarsest grid representation, using independent samples across successive spatial grid resolutions, and the sample size decreases with increasing grid fidelity to optimize computational effort. In contrast, the MFMC estimator follows an inverted paradigm: it initiates the accumulation of corrections with the most refined model representation and progressively incorporates corrections from lower-fidelity models. As the fidelity of the model decreases, the sample size increases, allowing less accurate but computationally inexpensive models to contribute to the overall estimate. Crucially, a distinguishing feature of the MFMC method is its reuse of samples within the same model hierarchy in the correction. This reuse avoids the computational redundancy of generating new samples at each fidelity level, effectively enhancing the overall efficiency of the sampling process. In addition to its computational efficiency, the MFMC method offers notable advantages over surrogate-based MLMC methods, such as those discussed in \cite{ElLiSa:2025, Li:2024}, which rely heavily on the characterization of interpolation errors. Such reliance can be a limiting factor, particularly in scenarios where interpolation errors decay slowly or require complicated error analysis. The MFMC approach circumvents this challenge by accommodating diverse surrogate models without taking into account the explicit treatment of interpolation errors, offering greater flexibility in its application. This adaptability extends the utility of MFMC to a broader range of modeling scenarios, making it particularly valuable in contexts where achieving a balance between computational efficiency and solution accuracy is critical. 

Nevertheless, the MFMC method is not without challenges. One notable limitation lies in its reliance on sufficiently large sample sizes to accurately estimate critical statistical parameters, such as variances and correlation coefficients, between high- and low-fidelity models. These estimates are obtained during the \textit{offline computations}, a preparatcory process involving tasks such as parameter estimation and the construction of surrogate models. Once the surrogates are built and parameters are generated in the offline phase, they will then be used in the \textit{online computations}, where the MFMC estimator is assembled and used to perform uncertainty quantification. However, achieving accurate parameter approximations in the offline phase can require substantial computational effort, which, in turn, may offset some of the efficiency gains in the online phase. Despite these challenges, the MFMC method remains an attractive approach due to its potential to accelerate the sampling process. The trade-offs between offline and online cost emphasize the importance of evaluating its applicability on a case-by-case basis to fully realize its potential benefits. In this study, we extend the analysis of the MFMC method \cite{PeWiGu:2016} by explicitly deriving the required sample size and computational cost as functions of the prescribed accuracy requirements. Our primary objective is to demonstrate that the MFMC method can achieve significant acceleration in the sampling process while maintaining statistical fidelity. As such, we show that MFMC provides a practical and robust framework for addressing the complex challenges of uncertainty quantification, particularly in the context of plasma equilibrium modeling.


 
The paper is organized as follows. In Section \ref{sec:Grad-Shafranov}, we introduce the Grad-Shafranov free boundary problem under uncertainty. Section \ref{sec:SC} provides an overview of the sparse grid stochastic collocation technique, which forms the basis to construct low-fidelity models used in the multi-fidelity Monte Carlo framework. Sections \ref{sec:MC} and \ref{sec:MFMC} discuss the Monte Carlo Finite Element method and its multi-fidelity variant. Finally, Section \ref{sec:Num-Exp} presents numerical experiments that access both efficiency and accuracy of these methods.



% Finally, the paper concludes with Section \ref{sec:Conclusion}, summarizing the key findings and contributions. 
% An appendix is included, containing technical mathematical details and proofs relevant to the problem and methods discussed. 
%!TEX root = ../main.tex
% ========================================
\section{Problem setting}\label{sec:Problem_setting}
% ========================================
We are interested in solving a parametrized partial differential equation under uncertainty quantification. Let $D\subset \mathbb{R}^n$ be a bounded, Lipschitz domain, and let $\mathcal{L}$ denote a (possibly nonlinear) differential operator posed on $D$. Uncertainty is introduced through randomness in the coefficients of $\mathcal{L}$ and/or the source term. This randomness is modeled on a complete probability space $(\Omega, \mathcal{F}, P)$, where $\Omega$ denotes the set of outcomes, $\mathcal{F}\subset 2^{\Omega}$ is a $\sigma-$algebra of events, and $P:\mathcal{F}\rightarrow [0,1]$ is a probability measure. The input data -- such as the coefficient field $a=a(\boldsymbol x, \boldsymbol \omega)$ and source term $f = f(\boldsymbol x, \boldsymbol\omega)$ -- are modeled as random fields, i.e., measurable functions defined on $D\times \Omega$. The solution map $u(\cdot, \boldsymbol{\omega}): \Omega \to U$, where $U$ is an appropriate spatial function space (e.g. $H_0^1(D)$), is defined such that for almost every $\boldsymbol\omega\in \Omega$, the function $u(\cdot, \boldsymbol{\omega})$ satisfies the stochastic boundary value problem
%
\begin{equation}\label{eq:Problem}
    \mathcal{L}(a)(u) = f \quad \text{in} \;\; D
\end{equation}
%
with appropriate boundary conditions on $\partial D$. 

% Moreover, we make the assumption that $a(\cdot, \omega)$ is bounded from below (either uniform or a random variable) on $D$ and $f(\cdot, \omega)$ is square integrable with respect to $P$. Then by the Lax-Milgram theorem, the problem admits a unique solution \cite{BaNoTe:2007}.

To characterize the dependence of the solution on both spatial and probabilistic variables, we adopt a Bochner space framework. For $q\in [1,\infty]$, the Bochner space $L^q(\Omega,U)$ consists of all strongly measurable mappings $u:\Omega\rightarrow U$ with finite norm
%
\[
L^q(\Omega,U) = \left\{u:\Omega\rightarrow U\; \bigg\vert \;\left\|u\right\|_{L^q(\Omega,U)}<\infty\right\},
\]
%
where the norm $\|u\|_{L^q(\Omega,U)}$ is given by
%
\[
\left\Vert u \right\Vert_{L^q(\Omega,U)} =\left\{\begin{array}{lll}
     \left(\int_{\Omega} \left\Vert u(\cdot,\boldsymbol{\omega})  \right\Vert_{U}^q \pi(\boldsymbol{\omega})d\boldsymbol{\omega} \right)^{1/q} = \left(\mathbb{E}\left[\left\Vert u(\cdot,\boldsymbol{\omega})  \right\Vert_{U}^q\right]\right)^{1/q}, & 0<q<\infty, \\
     \text{ess} \sup_{\boldsymbol{\omega}\in \Omega}\left\Vert u(\cdot,\boldsymbol{\omega})  \right\Vert_{U}, & q=\infty.
\end{array}
\right.
\]
and $\pi(\boldsymbol{\omega})$ denotes the joint probability density of $\boldsymbol{\omega}$.

The primary goal of this work is to analyze how uncertainties in the input data propagate through the partial differential equation to affect the solution $u$, and to develop efficient numerical methods for computing statistical quantities of interest. A fundamental example is the parametric expectation of the solution:
%
 \begin{equation}
 \label{eq:QoI}
      \mathbb{E}\left[u(\cdot,\boldsymbol \omega)\right]=\int_\Omega u(\cdot,\boldsymbol{\omega})\pi(\boldsymbol\omega)d\boldsymbol{\omega},
 \end{equation}
%

% Incorporating the uncertainty and 
% %
% \begin{equation}\label{eq:FreeBoundarya}
%  -\nabla\,\cdot\,\left(\frac{1}{\mu(u(\cdot, \boldsymbol{\omega})) r}\nabla u(\cdot, \boldsymbol{\omega})\right) = \left\{ \begin{array}{ll}
% \frac{d}{du} p(u(\cdot, \boldsymbol{\omega})) + \frac{1}{2\,\mu r} \frac{d}{du} g^2(u(\cdot, \boldsymbol{\omega})) & \text{ in } \Omega_p(u(\cdot, \boldsymbol{\omega})) \\
% I_k(\boldsymbol\omega)/S_k & \text{ in } \Omega_{C_k} \\
% 0 & \text{ elsewhere, } 
% \end{array}\right.
% \end{equation} 
% % ============================================================
\section{Sparse grid stochastic collocation}\label{sec:SC}
% ============================================================
We briefly outline the sparse grid stochastic collocation method \cite{BaNoRi:2000, KlBa:2005, MaNi:2009, Sm:1963} using a generic solution $u$ for illustration. Starting from a univariate set of $m_i$ collocation nodes $X^i = \left\{x_1^i,\ldots, x_{m_i}^i\right\}$ over $[-1,1]$, the univariate interpolation operator is
%
\[
I_{X^{i}}[u]:=\sum_{j=1}^{m_{i}} u(\cdot, x_j^i)\phi_j,
\]
%
where $\phi_j$ are Lagrange basis functions satisfying $\phi_j(x_k^i) = \delta_{jk}$. To extend this construction to a $d$-dimensional parameter space,  tensor products of univariate operators are constructed. Rather than using a full tensor grid, which suffers from exponential growth in $d$, the sparse grid approach selects a reduced set of nodes per dimension to build a sparse approximation. At {\it level} $q\; (\text{where }q\ge d)$, the sparse grid nodes are
%
\begin{equation*}
H(q,d) = \bigcup_{q-d+1\le|\boldsymbol{i}|\le q} \left(X^{i_1}\times \cdots\times X^{i_d}\right)\in [-1,1]^d, 
\end{equation*}
%
where $|\boldsymbol{i}| = i_1+\ldots+i_d$ specifies the refinement rule. These nodes yield a sparse but sufficiently rich representation of the domain, capturing key features of $u$ with far fewer evaluations than a full grid.

We use Chebyshev extrema as collocation nodes \cite{BaNoRi:2000, ClCu:1960}, defined by $x_j^i=-\cos(\frac{ \pi(j-1)}{m_i-1})$ for $j=1, \ldots, m_i$ with $m_1 =1$ and $m_i = 2^{i-1}+1$ for $i\ge 2$, ensuring nestedness $X^i\subset X^{i+1}$. The corresponding multidimensional sparse grid nodes satisfy 
%
\begin{equation}
\label{eq:NestedColPts}
H(q,d)\subset H(q+1,d),\quad \text{and}\quad H(q,d) = \bigcup_{|\boldsymbol{i}|=q} \left(X^{i_1}\times \cdots\times X^{i_d}\right).
\end{equation}
%
Interpolation over $H(q,d)$  is performed using the {\it Smolyak quadrature formula}, 
%
\begin{equation}
\label{eq: Smolyak_Quad_formula}
\mathcal{S}_{q, d}[u] = \sum_{q+1\le |\boldsymbol{i}|\le q+d} (-1)^{q+d-|\boldsymbol{i}|} \binom{d-1}{q+d-|\boldsymbol{i}|}\cdot \left(\mathrm I_{X^{i_1}}\otimes\cdots\otimes \mathrm I_{X^{i_d}}\right) [u].
\end{equation} 
%
which combines univariate interpolants into a high-dimensional approximation. This hierarchical and sparse structure allows for efficient reuse of evaluations and makes the method well-suited for high-dimensional stochastic problems.


% Let $N$ denote the number of sparse grid nodes. The sparse grid stochastic collocation method is equivalent to solving $N$ deterministic parametrized problems \eqref{eq:FreeBoundary} at each nodal point in $H(q,d)$.

% For our model problem, the sparse grid stochastic collocation method constructs the surrogate function $ \mathcal{S}_{q,d}(u)$ as per \eqref{eq: Smolyak_Quad_formula} by computing the direct solution of the discrete version of \eqref{eq:FreeBoundary} at isotropic sparse grid nodes \eqref{eq:NestedColPts} with the Clenshaw-Curtis quadrature abscissa  \cite{BaNoRi:2000,ClCu:1960}. 

% As discussed in \cite{NoTeWe:2008,TeJaWe:2015}, consider the function $u \in C^0(W,Z)$, where the parameter space $W$ and the solution space $Z$ are defined in \eqref{eq:ParameterSpace} and \eqref{eq:Soln_space} respectively. Let the interval in the $k$-th dimension be defined as $W_k = \left[\widetilde{\omega}_k-\tau \left\vert \widetilde{\omega}_k\right\vert, \widetilde{\omega}_k+\tau \left\vert \widetilde{\omega}_k\right\vert\right]$. The complementary multi-dimensional parameter space that excludes the $k$-th dimension is
% %
% \[
% W_k^c = \prod_{i=1, i\neq k}^d W_i.
% \]
% %
% Now, for any fixed element $\omega_k^c \in W_k^c$, and for each $\omega_k\in W_k$, we assume the function $u(\cdot,\omega_k,\omega_k^c): W_k \rightarrow C^0(W_k^c;Z)$ admits an analytic extension  $u(\cdot, z,\omega_k^c)$ in the complex plane, specifically in the region 
% %
% \[
% W_k^{*}:=\{z\in \mathbb{C}: \text{dist} (z,W_k)\le \iota_k \;\text{ for some } \iota_k>0\},
% \]
% %
% where $\iota_k$ denotes the proximity of the analytic extension to the real interval $W_k$. Under these assumptions, the interpolation error associated with the sparse grid method demonstrates an algebraic convergence rate
% %
% \begin{equation} \label{eq:coll-error-bound_2}
%   \big\|u-\mathcal{S}_{q, d} (u)\big\|_\infty = C P^{-\mu},
% \end{equation}
% %
% where $P$ denotes the sparse grid node count, $C$ is a constant dependent on dimension $d$ and analytic extension proximity to the interval $W_k$, and $\displaystyle \mu$ is related to the dimension of parameter space and function's analytic extension in the complex plane.




% Compared to the regularity assumption of $u$ in \cite{ElLiSa:2022}, the assumption for \eqref{eq:coll-error-bound_2} is stronger in the sense that the solution $u$ with respect to the random variable $\boldsymbol{\omega}$ can be analytically extended into the complex plane region by varying with one dimension of the random variable while keeping the other dimensions fixed.  This enhancement allows for a tighter interpolation error bound compared to the regularity assumption in \cite{ElLiSa:2022}.

%!TEX root = main.tex
%%%%%%%%%%%%%%%%%%%%%%%%%%%%%%%%%%%%%%%%%%%%%%%%%%%

% ====================================================
\section{Monte Carlo Method}\label{sec:MC}
% ====================================================

A standard approach to approximate the expectation in \eqref{eq:expectation_of_u} is the Monte Carlo (MC) method; 
see, e.g.,  \cite{MBGiles_2015a,MDGunzburger_CGWebster_GZhang_2014a}.
Because $u$ cannot be evaluated, but only a approximation $u_h \in   L_{\mathbb{P}}^2(W, \cU)$ can be 
computed, we estimate $\mathbb{E}[u_{h}]$, 
The MC estimator $A^{\text{MC}}_{N}$ of $\mathbb{E}[u_{h}]$ 
is the sample mean over $N$ independent and identically distributed (i.i.d.) realizations 
\begin{equation}\label{eq:MC_estimator}
    A^{\text{MC}}_{N} := \frac{1}{N}\sum_{i=1}^{N} u_{h}\big(\omega^{(i)} \big).
\end{equation}
%
This estimator is unbiased,  $\mathbb{E}[A^{\text{MC}}_{N}] = \mathbb{E}[u_{h}]$, 
and has variance $\mathbb{V}[A^{\text{MC}}_{N}] = N^{-1} \mathbb{V}[u_{h}]$, 
where the variance is defined as in \eqref{eq:variance_of_u}.
By the central limit theorem, the MC estimator $A^{\text{MC}}_{N}$ converges in distribution to $\mathbb{E}[u_h]$ as $N$ approaches infinity. 

To quantify the total approximation error of the estimator, we consider the  {\it normalized mean squared error (nMSE)}, defined as
%
 \[
\mathcal{E}_{A^{\text{MC}}_{N}}^2:= \mathbb E\left[ \big\| \mathbb{E}[u]-A^{\text{MC}}_{N}  \big\| _{U}^2\right]  \big/ \, \big\| \mathbb{E}[u]  \big\| _{U}^2.
\] 
%
The nMSE decomposes into two contributions: a {\it bias error} from spatial discretization, and a {\it statistical error} due to finite sampling
%
\[
\mathcal{E}_{A^{\text{MC}}_{N}}^2 
= \frac{ \big\| \mathbb{E}[u]-\mathbb{E}[u_{h}]  \big\| _{U}^2+\mathbb E\left[ \big\|  \mathbb{E}[u_{h}] -A^{\text{MC}}_{N}  \big\| _{U}^2\right]}{ \big\| \mathbb{E}[u]  \big\| _{U}^2} 
= \frac{ \big\| \mathbb{E}[u]-\mathbb{E}[u_{h}]  \big\| _{U}^2}{ \big\| \mathbb{E}[u]  \big\| _{U}^2}+\frac{\mathbb{V}\left[u_{h}\right]}{N \big\| \mathbb{E}[u]  \big\| _{U}^2}
=\mathcal{E}_{\text{Bias}}^2 + \mathcal{E}_{\text{Stat}}^2.
\]
%
Suppose the sample-wise discretization error satisfies
%
\begin{equation*} \label{eq:Assumption_uhA}
       \left\|u\left(\omega^{(i)}\right)-u_h\left(\omega^{(i)}\right)\right\|_U
       \leq C_m\left(\omega^{(i)}\right)M^{-\alpha}\,,
\end{equation*}
%
where $C_m(\omega^{(i)})$ is a constant depending only on the geometry of the spatial domain and the particular realization $\omega^{(i)}$, $\alpha>0$ is the convergence rate of spatial discretization, and $M$ denotes the number of spatial degrees of freedom. For simplicity and analytical tractability, we assume this constant is uniformly bounded across all realizations, i.e. $C_m(\omega^{(i)})\le C_m$ for some $C_m=\sup_{\omega \in \Omega} C_m(\omega)>0$ independent of the sample realization $\omega^{(i)}$ \cite{BaNoTe:2007,BaScZo:2011}.


Given a user-specified threshold $\epsilon^2$  for the nMSE, we introduce a {\it splitting ratio} $\theta \in (0,1)$ to allocate the total error budget between bias and statistical components
%
\begin{equation} \label{eq:error-budget}
%\textcolor{red}{\|u-u_h\|_{L^2(\boldsymbol W,U)}\le C_mM^{-\alpha}\le \theta_1\epsilon},\qquad\text{ and }\qquad \|u_h-\widehat u_{h}\|_{L^2(\boldsymbol W,U)} \le C_{p} P^{-\nu}\le \theta_2\epsilon\,.  
\mathcal{E}_{\text{Bias}}^2=\|u-u_h\|_{L^2(\boldsymbol \Omega,U)}\le C_mM^{-\alpha}= \theta\epsilon^2, \quad\quad \mathcal{E}_{\text{Stat}}^2 = \frac{\sigma_1^2}{N \big\| \mathbb{E}(u)  \big\| _{U}^2}=(1-\theta)\epsilon^2,
\end{equation}
where $C_m$ is independent of the sample and $\sigma_1^2 = \mathbb{V}\left( u_{h}\right)$. To meet these error constraints, the number of spatial nodes $M$ and sample size $N$ must obey
%
\begin{equation}
\label{eq:SLSGC_SL_SpatialGridsNo_n_SparseGridsNo}
M\ge \left(\frac{\theta\epsilon^2}{C_m}\right)^{-\frac 1 {\alpha}},\quad\quad  N \ge  \frac{\sigma_1^2}{\epsilon_{\text{tar}}^2},
\end{equation}
%
where $\epsilon_{\text{tar}}^2 = \epsilon^2(1-\theta) \big\| \mathbb{E}(u)  \big\| _{U}^2$.
Assuming each evaluation of $u_{h}$ incurs an average cost of $C$, the total cost to compute $A^{\text{MC}}_{N}$ is
%
\[
\mathcal{W}^\text{MC}  = CN=\frac{C\sigma_1^2}{\epsilon_{\text{tar}}^2}.
\]
%
In practice, both $M$ and $N$ are rounded up to the smallest integers satisfying \eqref{eq:SLSGC_SL_SpatialGridsNo_n_SparseGridsNo}.
%!TEX root = ../main.tex
% ====================================================
\section{Multi-fidelity Monte Carlo}\label{sec:MFMC}
% ====================================================
This section reviews the multi-fidelity Monte Carlo (MFMC) method, following the foundational formulation in \cite{PeWiGu:2016}. The MFMC framework uses an ensemble of models with varying computational cost and accuracy to construct a variance-reduced estimator for high-fidelity expectation. Let $u_1:\Omega \to U$ denote the high-fidelity (HF) model that provides accurate but expensive evaluations, and let $\{u_k\}_{k=2}^K$ denote low-fidelity (LF) models that offer cheaper approximations. The central goal of MFMC is to allocate a fixed computational budget across these models to minimize estimator variance while maintaining unbiasedness.

We introduce some key statistical quantities that describe the model. We represent the random output of model $u_k$ on the probability space $(\Omega,\mathcal{F},\mathbb{P})$ by $u_k(\boldsymbol{\omega})$, abbreviated as $u_k$. For each pair of models $u_k,u_j$, define the variance and correlation coefficient
%
\begin{equation*}
    \sigma_k^2 = \mathbb{V}\!\left[u_k\right],\qquad 
    \rho_{k,j} = \frac{\text{Cov}\!\left[u_k,u_j\right]}{\sigma_k\sigma_j}, 
    \quad k,j=1,\dots,K,
\end{equation*}
%
where the covariance is defined as $\text{Cov}[u_k,u_j] := \mathbb{E}[\langle u_k - \mathbb{E}[u_k], u_j - \mathbb{E}[u_j]\rangle_U]$ and $\rho_{k,k}=1$. The pairwise correlations between fidelity levels quantify the statistical dependence that drives variance reduction through effective control variates.

The MFMC estimator architecture uses a nested sampling strategy that reuses computational evaluations across fidelity levels. Let $A_{1,N_1}^{\text{MC}}$ denote the standard Monte Carlo estimator of $\mathbb{E}[u_1]$ based on $N_1$ HF samples. The MFMC estimator augments this with corrections from lower fidelities via control variates
%
\begin{equation}\label{eq:MFMC_estimator}
A^{\text{MF}} := A^{\text{MC}}_{1,N_1} + \sum_{k=2}^K \alpha_k\left(\overline{A}_{k,N_k} - \overline{A}_{k,N_{k-1}}\right),
\end{equation}
%
where $\alpha_k \in \mathbb{R}$ are control variate weights and $\overline{A}_{k,N}$ denotes the sample average of $N$ evaluations of model $u_k$. A critical aspect of this construction is the nested sampling structure: the estimator $\overline{A}_{k,N_{k}}$ reuses all $N_{k-1}$ samples from $\overline{A}_{k,N_{k-1}}$, possibly supplemented by additional $N_{k} - N_{k-1}$ samples. The reuse of LF evaluations across levels enhances efficiency but induces sample statistical dependencies that complicate variance analysis.



To facilitate analysis, we reformulate the estimator so that its constituent terms are statistically independent. Partitioning the $N_k$ LF samples into disjoint sets of sizes $N_{k-1}$ and $N_k-N_{k-1}$ yields the equivalent independent form
%
\begin{equation}\label{eq:MFMC_estimator_independent}
    A^{\text{MF}} = A^{\text{MC}}_{1,N_1} +  \sum_{k=2}^K \alpha_k\!\left(1-\frac{N_{k-1}}{N_k}\right)\left(A^{\text{MC}}_{k,N_k\backslash N_{k-1}}-A^{\text{MC}}_{k,N_{k-1}}\right),
\end{equation}
%
where $A_{k,N_k\backslash N_{k-1}}^{\text{MC}}$ is the MC average over the $N_k-N_{k-1}$ new samples (defined to be zero when $N_k=N_{k-1}$).


The statistical properties of the MFMC estimator emerge clearly from its component-wise decomposition. Define
%
\begin{equation}\label{eq:MFMC_Yk}
Y_1 := A^{\text{MC}}_{1,N_1},\quad 
Y_k := \left(1-\frac{N_{k-1}}{N_k}\right)\!\left(A^{\text{MC}}_{k,N_k\backslash N_{k-1}} - A^{\text{MC}}_{k,N_{k-1}}\right), \;\; k=2\ldots, K,
\end{equation}
%
then the MFMC estimator can be expressed into a compact form $A^{\text{MF}} = Y_1 + \sum_{k=2}^K \alpha_k Y_k$. Since each $Y_k$ for $k\ge2$ represents a difference of two independent estimators for the same $\mathbb{E}[u_k]$, we immediately obtain $\mathbb{E}[Y_k]=0$ and the MFMC estimator is unbiased: $\mathbb{E}[A^{\text{MF}}]=\mathbb{E}[u_1]$. The variances of the components are
%
\begin{equation}\label{eq:Var_Yk}
    \mathbb{V}[Y_1] = \frac{\sigma_1^2}{N_1}, \qquad 
    \mathbb{V}[Y_k] = \left(\frac{1}{N_{k-1}} - \frac{1}{N_k}\right)\sigma_k^2, \;\; k=2\ldots, K.
\end{equation}
%
A key statistical insight, formalized in Lemma~\ref{lemma:Y_k_Y_j}, establishes that the correction terms are mutually uncorrelated despite sample reuse.
%
\begin{lemma}\label{lemma:Y_k_Y_j}
For $2\le k<j\le K$, 
% the correction terms $Y_k$ and $Y_j$ defined in \eqref{eq:MFMC_Yk} are uncorrelated, i.e., 
$\operatorname{Cov} [Y_k,Y_j ]=0$.
\end{lemma}
%
The proof is provided in the Appendix.

Each correction $Y_k$($k\ge2$) is correlated with $Y_1$, with covariance
\begin{equation}\label{eq:Cov_Yk}
\operatorname{Cov}[Y_1,Y_k] = -\!\left(\frac{1}{N_{k-1}} - \frac{1}{N_k}\right)\rho_{1,k}\sigma_1\sigma_k,
\end{equation}
as shown in \cite[Lemma~3.2]{PeWiGu:2016}. Combining \eqref{eq:Var_Yk} and \eqref{eq:Cov_Yk} gives
%
\begin{equation}\label{eq:MFMC_variance}
    \mathcal{V}^{\text{MF}}
    =\frac{\sigma_1^2}{N_1} 
    + \sum_{k=2}^K \left(\frac{1}{N_{k-1}} - \frac{1}{N_k}\right)\!\left(\alpha_k^2\sigma_k^2 - 2\alpha_k\rho_{1,k}\sigma_1\sigma_k\right).
\end{equation}
%

In order to determine optimal sample sizes $N_k$ and weights $\alpha_k$ in the MFMC estimator \eqref{eq:MFMC_estimator_independent}, an optimization problem is formulated \cite{PeWiGu:2016} by minimizing the estimator variance \eqref{eq:MFMC_variance} subject to a fixed budget $p$. Let $C_k$ denote the per-sample cost of model $u_k$, the total computational cost is 
%
\[
\mathcal{W}^{\text{MF}} = \sum_{k=1}^K C_k N_k,
\]
%
and the constrained optimization problem becomes
%
\begin{equation}\label{eq:Optimization_pb_sample_size}
    \begin{array}{ll}
    \min &\mathcal{V}^{\text{MF}}\left(\alpha_k,N_k\right),\\
       \text{subject to} &\displaystyle\sum\limits_{k=1}^K C_kN_k=p,\\[2pt]
       &\displaystyle N_1\ge 0,\quad \displaystyle N_{k-1}\le N_k, \;\; k=2\ldots,K,\\
       &N_1,\ldots, N_K\in \mathbb{R},\\
       &\alpha_2,\ldots,\alpha_K\in \mathbb{R}.
    \end{array}
\end{equation}
%
Note that for each level $k\ge 2$, $\alpha_k$ enters only through a quadratic expression independent of $N_k$ in the variance term. This separable structure allows a fundamental simplification of the variance functional, which allows hierarchical minimization
%
\begin{equation*}
    \min_{\alpha_k,\, N_k} \mathcal{V}^{\text{MF}}\left(\alpha_k, N_k\right)
    = \min_{N_k}\Big(\min_{\alpha_k} \mathcal{V}^{\text{MF}}(\alpha_k, N_k)\Big).
\end{equation*}
%
The hierarchical minimization admits a closed-form solution for optimal weights by solving the inner optimization $\partial \mathcal{V}^{\text{MF}}/\partial \alpha_k = 0$, yielding 
%
\begin{equation}\label{eq:MFMC_weights}
    \alpha_k^* = \frac{\rho_{1,k}\sigma_1}{\sigma_k}.
\end{equation}
%
Substituting $\alpha_k^*$ into \eqref{eq:MFMC_variance} simplifies the variance to 
%
\begin{equation*}
    \mathcal{V}^{\text{MF}}\left(\alpha_k^*, N_k\right)
    = \sigma_1^2\sum_{k=1}^K \frac{\Delta_k}{N_k},
\end{equation*}
%
where $\Delta_k = \rho_{1,k}^2 - \rho_{1,k+1}^2$ for $k = 1, \dots, K$ with $\rho_{1,K+1}=0$. This reduces the joint optimization to a continuous resource allocation problem involving only sample allocation
%
\begin{equation}\label{eq:Optimization_pb_sample_size_reduced}
    \begin{array}{ll}
    \min &\displaystyle f(N_k) =\sum_{k=1}^K \frac{\Delta_k}{N_k},\\
       \text{subject to} &\displaystyle\sum\limits_{k=1}^K C_kN_k=p,\\[2pt]
       &\displaystyle -N_1\le 0,\quad \displaystyle N_{k-1}-N_k\le 0, \;\; k=2\ldots,K,\\
       &N_1,\ldots, N_K\in \mathbb{R},
    \end{array}
\end{equation}
%
where $f(N_k)$ is the {\it normalized variance functional}. Under suitable monotonicity and ordering assumptions, this problem admits an analytic solution that characterizes the optimal allocation of resources across fidelity levels.


%
\begin{theorem}[Optimal MFMC real-valued sample allocation]\label{thm:Sample_size_est}
Consider $K$ models $\{u_{k}\}_{k=1}^K$ with standard deviations $\sigma_k$, correlation coefficients $\rho_{1,k}$ of LF model $u_k$ with the HF model $u_1$, and per-sample costs $C_k$. Define $\Delta_k = \rho_{1,k}^2 - \rho_{1,k+1}^2$ for $k = 1, \dots, K$ with $\rho_{1,K+1}=0$. Assume the following conditions hold
%
\begin{alignat*}{3}
&(i)\;\textit{Monotone correlations:} &\quad& |\rho_{1,1}| > \cdots > |\rho_{1,K}|,\\
&(ii)\;\textit{Cost-correlation ratio:} &\quad& \frac{\Delta_{k}}{C_k} > \frac{\Delta_{k-1}}{C_{k-1}}, \quad k=2,\ldots,K.
\end{alignat*}
%
Then the optimal control weights and sample sizes for \eqref{eq:Optimization_pb_sample_size} are
%
\begin{equation}\label{eq:MFMC_RealValued_Sample_Size}
    \alpha_k^* = \frac{\rho_{1,k}\sigma_1}{\sigma_k}, \qquad
    N_k^* = \sqrt{\frac{\Delta_k}{C_k}}\,
    \frac{p}{\sum_{j=1}^K \sqrt{C_j \Delta_j}}.
\end{equation}
%
% \[
% r_k^* = \sqrt{\frac{C_1\Delta_k}{C_k\Delta_1}},\quad N_1^* = \frac{p}{\sum_{k=1}^K C_k r^*_k}, \quad N_k^*=N_1^*r_k^*.
% \] 
% %
% \JLcolor{alternatively, in my way to represent it without mentioning the vector $\boldsymbol{r}^*$, we have}
%
The resulting minimal variance of the MFMC estimator is
\begin{equation}\label{eq:MFMC_variance_optimal}
\mathcal{V}^{\text{MF}}
= \sigma_1^2\sum_{k=1}^K \frac{\Delta_k}{N_k^*}=\frac{\sigma_1^2}{p}\!\left(\sum_{k=1}^K \sqrt{C_k \Delta_k}\right)^{\!2}.
\end{equation}
\end{theorem}
%


Differentiating the normalized variance and cost with respect to the sample sizes gives
%
\[
\frac{\partial f}{\partial N_k} = -\frac{\Delta_k}{N_k^2},
\qquad 
\frac{\partial \mathcal{W}^{\text{MF}}}{\partial N_k} = C_k.
\]
%
These relations quantify the variance–cost trade-off: increasing samples at any level reduces variance at the expense of computational resources. At the continuous optimum \eqref{eq:MFMC_RealValued_Sample_Size}, the marginal variance reduction per unit cost $\Delta_k/(C_k N_k^2)$ is identical across all active models, establishing a balanced resource allocation that characterizes the optimal allocation.

While Theorem~\ref{thm:Sample_size_est} provides real-valued optimal allocations $N_k^*$, practical implementation requires integer sample sizes. The standard approach \cite{PeWiGu:2016} applies the floor function $\lfloor N_k^* \rfloor$ to ensure budget feasibility. The realized variance and cost are
%
\[
f\left(\left\lfloor N_k^* \right\rfloor\right) = \sum_{k=1}^K\frac{\Delta_{k}}{\left\lfloor N_k^* \right\rfloor}, \qquad \mathcal{W}^{\text{MF}}\left(\left\lfloor N_k^* \right\rfloor\right) = \sum_{k=1}^K C_k\left\lfloor N_k^* \right\rfloor.
\]
%
Since $N_k^*-1 < \lfloor N_k^*\rfloor \le N_k^*$, the floor operation induces bounded sub-optimality, producing the bounds
%
\begin{equation}\label{eq:bounds_for_floor}
\begin{aligned}
    % f\left(\left\lfloor N_k^* \right\rfloor\right)&\in \left[\sum_{k=1}^K\frac{\Delta_{k}}{N_k^*},\; \sum_{k=1}^K\frac{\Delta_{k}}{N_k^*-1}\right) = \left[\frac{1}{p}\left(\sum_{k=1}^K \sqrt{C_k\Delta_k}\right)^2, \sum_{k=1}^K\frac{\Delta_{k}}{\frac{p}{\sum_{j=1}^K \sqrt{C_j\Delta_j}}\sqrt{\frac{\Delta_k}{C_k}}-1}\right)\\
    % &=\left[\frac{1}{p}\left(\sum_{k=1}^K \sqrt{C_k\Delta_k}\right)^2, \sum_{k=1}^K \sqrt{C_k\Delta_k}\sum_{k=1}^K\frac{\sqrt{C_k\Delta_{k}}}{p-\sqrt{\frac{C_k}{\Delta_k}}\sum_{j=1}^K \sqrt{C_j\Delta_j}}\right)\\
    % &=\sum_{k=1}^K \sqrt{C_k\Delta_k}\left[\frac{\sum_{k=1}^K \sqrt{C_k\Delta_k}}{p},\sum_{k=1}^K\frac{\sqrt{C_k\Delta_{k}}}{p-\sqrt{\frac{C_k}{\Delta_k}}\sum_{j=1}^K \sqrt{C_j\Delta_j}}\right)\\
    % \mathcal{W}^{\text{MF}}\left(\left\lfloor N_k^* \right\rfloor\right) &\in \left(\sum_{k=1}^KC_kN_k^*-\sum_{k=1}^K C_k, \sum_{k=1}^KC_kN_k^*\right]=\left( p-\sum_{k=1}^K C_k,p\right].
    f\left(\left\lfloor N_k^* \right\rfloor\right) \in \left[\frac{1}{p}\left(\sum_{k=1}^K \sqrt{C_k\Delta_k}\right)^2, \sum_{k=1}^K\frac{\Delta_{k}}{N_k^*-1}\right), \qquad
\mathcal{W}^{\text{MF}}\left(\left\lfloor N_k^* \right\rfloor\right)\in \left( p-\sum_{k=1}^K C_k, p\right].
\end{aligned}
\end{equation}
%
The term $\sum_{k=1}^K C_k$ represents the rounding-induced slack in the budget, which becomes negligible asymptotically as $p \to \infty$. However, in the pre-asymptotic regime -- where the total budget $p$ is moderate -- this  slack can lead to significant under-utilization of the computational resources. This observation naturally motivates \textit{the development of  alternative integer-valued allocation strategies that reduce slack and achieve tighter budget utilization.}





% This quantity is the marginal variance reduction rate — how much the total variance decreases when you spend more samples at level by taking one more sample cost $C_k$. So the marginal variance reduction per unit cost
% \[
% \frac{-\frac{\partial f}{\partial N_k}}{C_k} = \frac{\Delta_k}{C_kN_k^2}
% \]
% It quantifies that How much variance reduction we get per unit cost at level $k$.
% At the optimum, the system reaches equilibrium where every active model yields the same return per cost unit,
% \[
% \frac{\Delta_k}{C_kN_k^2} = \text{Constant}=\frac{1}{p^2}\left(\sum_{k=1}^K \sqrt{C_k \Delta_k}\right)^{\!2}, \quad \text{for all active}\;\; k.
% \]

















% % ========================================
\section{Numerical experiments}\label{sec:Num-Exp}
% ========================================
%
We present numerical results for Monte Carlo and multi-fidelity Monte Carlo sampling methods to estimate $\mathbb{E}(u)$. 



Consider the stochastic elliptic PDE on a 2D domain $D=[0,1]^2$
\[
-\nabla\cdot \left(a(x,\omega)\nabla u(x,\omega)\right) =f(x), \quad x\in D,
\]

with homogeneous Dirichlet boundary conditions, $f(x)=1$ and a random diffusion coefficient
\[
a(x,\omega) = \exp \left(Y(x,\omega)\right),
\]
where $Y(x,\omega)$ us a mean-zero Gaussian random field with a squared exponential covariance kernel.







 














   


















\section{Acknowledgment}\label{sec:Acknowledgment}



This work was supported in part by the Big-Data Private-Cloud Research Cyberinfrastructure MRI-award funded by NSF under grant CNS-1338099 and by Rice University's Center for Research Computing.

Jiaxing Liang was partially supported by AFOSR grant FA9550-22-1-0004. 



\bibliographystyle{abbrv}
% \bibliographystyle{alphaurl}
\bibliography{references_liang}
\end{document}


