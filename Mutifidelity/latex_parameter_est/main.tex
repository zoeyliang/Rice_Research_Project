\documentclass[final,3p,times,11pt]{elsarticle}
\usepackage[USenglish]{babel}
\usepackage{amsmath,amssymb,amsthm, mathrsfs,multirow}
\usepackage{mathtools}
\usepackage{graphicx}
\usepackage{stmaryrd}
\usepackage[dvipsnames]{xcolor}
\usepackage{cancel}
\usepackage{ulem}
\usepackage{tabularx}
\usepackage{comment}
%\usepackage{subcaption}
%\usepackage[show]{ed}
%\usepackage{showkeys}
%\usepackage{showlabels}
%\usepackage[notcite,notref]{showkeys}
%\usepackage{refcheck}
% \usepackage[ruled,vlined]{algorithm2e}
\usepackage[linesnumbered,ruled,vlined]{algorithm2e}
\definecolor{Myblue}{rgb}{.2 0.4 1}

\usepackage{hyperref}
\hypersetup{
    %bookmarks=true,         % show bookmarks bar?
    colorlinks = true,       % false: boxed links; true: colored links
    % linkcolor=green
     %linkcolor=red,          % color of internal links (change box color with linkbordercolor)
     %citecolor=green,        % color of links to bibliography
    %filecolor=magenta,      % color of file links
    %urlcolor=cyan           % color of external links
}
%\usepackage{wrapfig}
%% The lineno packages adds line numbers. Start line numbering with
%% \begin{linenumbers}, end it with \end{linenumbers}. Or switch it on
%% for the whole article with \linenumbers after \end{frontmatter}.
%\usepackage{lineno}


% ==============   Macros  ====================
\newcommand{\mynabla}{\widetilde{\nabla}} 
\newcommand{\jump}[1]{[\![#1]\!]}
\newcommand{\HEcolor}[1]{{\textcolor{blue}{#1}}}
\newcommand{\TSVcolor}[1]{{\textcolor{orange}{#1}}}
\newcommand{\JLcolor}[1]{{\textcolor{violet}{#1}}} %violet
\newcommand{\Grids}{\boldsymbol{\chi}}

\newtheorem{theorem}{Theorem}%[section]
\newtheorem{lemma}{Lemma}%[section]
\newtheorem{VariationalForm}[theorem]{Variational Formulation}
% =============================================

\journal{}
\makeatletter
\def\ps@pprintTitle{%
 \let\@oddhead\@empty
 \let\@evenhead\@empty
 \def\@oddfoot{}%
 \let\@evenfoot\@oddfoot}
\makeatother




\begin{document}
\begin{frontmatter}
\title{Estimation of model correlations for multi-fidelity Monte Carlo for uncertainty quantification}


\author[umdcs]{Matthias Heinkenschloss}
\ead{heinken@rice.edu}
\address[umdcs]{Department of Computational Applied Mathematics and Operations Research and the Ken Kennedy Institute, Rice University.}
\author[umdm]{Jiaxing Liang}
\ead{jl508@rice.edu}
\address[umdm]{Department of Computational Applied Mathematics and Operations Research, Rice University.}
% \author[UA]{Tonatiuh S\'anchez-Vizuet}
% \ead{tonatiuh@arizona.edu}
% \address[UA]{Department of Mathematics, The University of Arizona.}
\begin{abstract}
We present an adaptive parameter estimation methodology to address the challenge of unknown model correlation coefficients in the multi-fidelity Monte Carlo framework for efficient uncertainty quantification in the presence of parametric variability. In practical applications for multi-fidelity Monte Carlo, key parameters such as model correlation coefficients are typically unknown and must be estimated from sample statistics. We address this challenge by deriving sensitivity bounds and confidence intervals for correlation coefficient estimators, which inform a sequential pilot sampling strategy with rigorous stopping criteria. This enables dynamic calibration of the multi-fidelity Monte Carlo estimators while ensuring controlled error propagation from parameter estimation to overall accuracy and cost. The resulting algorithm adaptively identifies the minimal pilot sample size required for reliable performance, avoiding unnecessary computational expense. Numerical experiments confirm the effectiveness of the proposed approach in delivering accurate statistical estimates with significantly fewer high-fidelity evaluations than standard Monte Carlo. This framework substantially improves the practicality and efficiency of MFMC in realistic scenarios.
\end{abstract}

\begin{keyword}
Multi-fidelity Monte Carlo Finite-Element \sep Parameter estimation for correlation coefficient\sep Parametric expectation \sep Uncertainty quantification
%
\MSC[2020] 
\end{keyword}
\end{frontmatter}

% ========================================
\section{Introduction}\label{sec:intro}
% ========================================
The pursuit of controlled nuclear fusion as a clean and virtually limitless energy source has spurred extensive research into the physics of magnetic confinement in fusion reactors. At the core of this effort lies the Grad–Shafranov free-boundary problem, which governs the equilibrium state of plasma in axially symmetric geometries, such as those found in Tokamaks. The governing equation encapsulates the intricate interplay between magnetic fields and plasma pressure, which determines critical confinement and stability properties essential for efficient plasma performance. However, the predictive accuracy of these models is significantly challenged by uncertainties in the parameters arising from measurement limitations, model assumptions, and operational variability. Addressing these uncertainties effectively requires advanced computational frameworks capable of robust statistical analysis, enabling accurate predictions of the plasma equilibrium response under diverse scenarios and ensuring reliable assessments of reactor designs and operations.

This study focuses on estimating the expectation of the solution operator associated with plasma equilibrium with uncertainties in the parameters. The Monte Carlo (MC) method, a classical and widely used approach in uncertainty quantification, relies on repeated executions of deterministic solvers to generate ensembles of realizations for stochastic inputs. Despite its versatility, its practicality is often limited by its slow asymptotic convergence rate of $1/\sqrt{N}$, which often requires a substantial number of sample realizations, $N$, to achieve reliable accuracy. For problems involving non-linear partial differential equations (PDEs), this slow convergence translates into a significant computational burden, as each realization typically demands a high-fidelity numerical approximation, such as those obtained via finite element method, which are computationally expensive due to their fine spatial resolution. Consequently, the cost of using the MC method can quickly escalate, particularly for high-dimensional problems or those requiring precise solutions. To alleviate this challenge, low-fidelity models have been proposed as computationally efficient alternatives to high-fidelity simulations.  These models aim to approximate the underlying system with reduced computational cost while maintaining an acceptable level of accuracy. For example, \cite{ElLiSa:2022} demonstrates how low-fidelity models constructed using stochastic collocation can effectively accelerate Monte Carlo sampling by exploiting simplified representations of the system. Similarly,
\cite{ElLiSa:2023} investigates hierarchical coarse spatial grids to develop low-fidelity surrogate models for multilevel Monte Carlo (MLMC) frameworks \cite{BaScZo:2011,Gi:2008}. Building on these ideas, studies such as \cite{ElLiSa:2025, Li:2024} combine stochastic collocation techniques with multilevel approaches, constructing low-fidelity models on coarse grids for MLMC sampling, further reducing computational expenses. While these methods achieve notable reductions in computational cost, they introduce the risk of compromising accuracy due to the inherent simplifications in the surrogate models. The trade-off between computational efficiency and solution accuracy is, therefore, a critical issue that warrants careful examination to ensure the reliability of results.


In this work, we delve into the multi-fidelity Monte Carlo (MFMC) method \cite{PeWiGu:2016, PeGuWi:2018}, which uses the control variate approach to exploit correlations between a computationally expensive high-fidelity model and a series of low-fidelity models. The MFMC method distinguishes itself from the MLMC approach by adopting a different strategy to construct its estimator. In MLMC, sample corrections are accumulated starting from the coarsest grid representation, using independent samples across successive spatial grid resolutions, and the sample size decreases with increasing grid fidelity to optimize computational effort. In contrast, the MFMC estimator follows an inverted paradigm: it initiates the accumulation of corrections with the most refined model representation and progressively incorporates corrections from lower-fidelity models. As the fidelity of the model decreases, the sample size increases, allowing less accurate but computationally inexpensive models to contribute to the overall estimate. Crucially, a distinguishing feature of the MFMC method is its reuse of samples within the same model hierarchy in the correction. This reuse avoids the computational redundancy of generating new samples at each fidelity level, effectively enhancing the overall efficiency of the sampling process. In addition to its computational efficiency, the MFMC method offers notable advantages over surrogate-based MLMC methods, such as those discussed in \cite{ElLiSa:2025, Li:2024}, which rely heavily on the characterization of interpolation errors. Such reliance can be a limiting factor, particularly in scenarios where interpolation errors decay slowly or require complicated error analysis. The MFMC approach circumvents this challenge by accommodating diverse surrogate models without taking into account the explicit treatment of interpolation errors, offering greater flexibility in its application. This adaptability extends the utility of MFMC to a broader range of modeling scenarios, making it particularly valuable in contexts where achieving a balance between computational efficiency and solution accuracy is critical. 

Nevertheless, the MFMC method is not without challenges. One notable limitation lies in its reliance on sufficiently large sample sizes to accurately estimate critical statistical parameters, such as variances and correlation coefficients, between high- and low-fidelity models. These estimates are obtained during the \textit{offline computations}, a preparatcory process involving tasks such as parameter estimation and the construction of surrogate models. Once the surrogates are built and parameters are generated in the offline phase, they will then be used in the \textit{online computations}, where the MFMC estimator is assembled and used to perform uncertainty quantification. However, achieving accurate parameter approximations in the offline phase can require substantial computational effort, which, in turn, may offset some of the efficiency gains in the online phase. Despite these challenges, the MFMC method remains an attractive approach due to its potential to accelerate the sampling process. The trade-offs between offline and online cost emphasize the importance of evaluating its applicability on a case-by-case basis to fully realize its potential benefits. In this study, we extend the analysis of the MFMC method \cite{PeWiGu:2016} by explicitly deriving the required sample size and computational cost as functions of the prescribed accuracy requirements. Our primary objective is to demonstrate that the MFMC method can achieve significant acceleration in the sampling process while maintaining statistical fidelity. As such, we show that MFMC provides a practical and robust framework for addressing the complex challenges of uncertainty quantification, particularly in the context of plasma equilibrium modeling.


 
The paper is organized as follows. In Section \ref{sec:Grad-Shafranov}, we introduce the Grad-Shafranov free boundary problem under uncertainty. Section \ref{sec:SC} provides an overview of the sparse grid stochastic collocation technique, which forms the basis to construct low-fidelity models used in the multi-fidelity Monte Carlo framework. Sections \ref{sec:MC} and \ref{sec:MFMC} discuss the Monte Carlo Finite Element method and its multi-fidelity variant. Finally, Section \ref{sec:Num-Exp} presents numerical experiments that access both efficiency and accuracy of these methods.



% Finally, the paper concludes with Section \ref{sec:Conclusion}, summarizing the key findings and contributions. 
% An appendix is included, containing technical mathematical details and proofs relevant to the problem and methods discussed. 
%!TEX root = ../main.tex
% ========================================
\section{Problem setting}\label{sec:Problem_setting}
% ========================================
We are interested in solving a parametrized partial differential equation under uncertainty quantification. Let $D\subset \mathbb{R}^n$ be a bounded, Lipschitz domain, and let $\mathcal{L}$ denote a (possibly nonlinear) differential operator posed on $D$. Uncertainty is introduced through randomness in the coefficients of $\mathcal{L}$ and/or the source term. This randomness is modeled on a complete probability space $(\Omega, \mathcal{F}, P)$, where $\Omega$ denotes the set of outcomes, $\mathcal{F}\subset 2^{\Omega}$ is a $\sigma-$algebra of events, and $P:\mathcal{F}\rightarrow [0,1]$ is a probability measure. The input data -- such as the coefficient field $a=a(\boldsymbol x, \boldsymbol \omega)$ and source term $f = f(\boldsymbol x, \boldsymbol\omega)$ -- are modeled as random fields, i.e., measurable functions defined on $D\times \Omega$. The solution map $u(\cdot, \boldsymbol{\omega}): \Omega \to U$, where $U$ is an appropriate spatial function space (e.g. $H_0^1(D)$), is defined such that for almost every $\boldsymbol\omega\in \Omega$, the function $u(\cdot, \boldsymbol{\omega})$ satisfies the stochastic boundary value problem
%
\begin{equation}\label{eq:Problem}
    \mathcal{L}(a)(u) = f \quad \text{in} \;\; D
\end{equation}
%
with appropriate boundary conditions on $\partial D$. 

% Moreover, we make the assumption that $a(\cdot, \omega)$ is bounded from below (either uniform or a random variable) on $D$ and $f(\cdot, \omega)$ is square integrable with respect to $P$. Then by the Lax-Milgram theorem, the problem admits a unique solution \cite{BaNoTe:2007}.

To characterize the dependence of the solution on both spatial and probabilistic variables, we adopt a Bochner space framework. For $q\in [1,\infty]$, the Bochner space $L^q(\Omega,U)$ consists of all strongly measurable mappings $u:\Omega\rightarrow U$ with finite norm
%
\[
L^q(\Omega,U) = \left\{u:\Omega\rightarrow U\; \bigg\vert \;\left\|u\right\|_{L^q(\Omega,U)}<\infty\right\},
\]
%
where the norm $\|u\|_{L^q(\Omega,U)}$ is given by
%
\[
\left\Vert u \right\Vert_{L^q(\Omega,U)} =\left\{\begin{array}{lll}
     \left(\int_{\Omega} \left\Vert u(\cdot,\boldsymbol{\omega})  \right\Vert_{U}^q \pi(\boldsymbol{\omega})d\boldsymbol{\omega} \right)^{1/q} = \left(\mathbb{E}\left[\left\Vert u(\cdot,\boldsymbol{\omega})  \right\Vert_{U}^q\right]\right)^{1/q}, & 0<q<\infty, \\
     \text{ess} \sup_{\boldsymbol{\omega}\in \Omega}\left\Vert u(\cdot,\boldsymbol{\omega})  \right\Vert_{U}, & q=\infty.
\end{array}
\right.
\]
and $\pi(\boldsymbol{\omega})$ denotes the joint probability density of $\boldsymbol{\omega}$.

The primary goal of this work is to analyze how uncertainties in the input data propagate through the partial differential equation to affect the solution $u$, and to develop efficient numerical methods for computing statistical quantities of interest. A fundamental example is the parametric expectation of the solution:
%
 \begin{equation}
 \label{eq:QoI}
      \mathbb{E}\left[u(\cdot,\boldsymbol \omega)\right]=\int_\Omega u(\cdot,\boldsymbol{\omega})\pi(\boldsymbol\omega)d\boldsymbol{\omega},
 \end{equation}
%

% Incorporating the uncertainty and 
% %
% \begin{equation}\label{eq:FreeBoundarya}
%  -\nabla\,\cdot\,\left(\frac{1}{\mu(u(\cdot, \boldsymbol{\omega})) r}\nabla u(\cdot, \boldsymbol{\omega})\right) = \left\{ \begin{array}{ll}
% \frac{d}{du} p(u(\cdot, \boldsymbol{\omega})) + \frac{1}{2\,\mu r} \frac{d}{du} g^2(u(\cdot, \boldsymbol{\omega})) & \text{ in } \Omega_p(u(\cdot, \boldsymbol{\omega})) \\
% I_k(\boldsymbol\omega)/S_k & \text{ in } \Omega_{C_k} \\
% 0 & \text{ elsewhere, } 
% \end{array}\right.
% \end{equation} 
% % ============================================================
\section{Sparse grid stochastic collocation}\label{sec:SC}
% ============================================================
We briefly outline the sparse grid stochastic collocation method \cite{BaNoRi:2000, KlBa:2005, MaNi:2009, Sm:1963} using a generic solution $u$ for illustration. Starting from a univariate set of $m_i$ collocation nodes $X^i = \left\{x_1^i,\ldots, x_{m_i}^i\right\}$ over $[-1,1]$, the univariate interpolation operator is
%
\[
I_{X^{i}}[u]:=\sum_{j=1}^{m_{i}} u(\cdot, x_j^i)\phi_j,
\]
%
where $\phi_j$ are Lagrange basis functions satisfying $\phi_j(x_k^i) = \delta_{jk}$. To extend this construction to a $d$-dimensional parameter space,  tensor products of univariate operators are constructed. Rather than using a full tensor grid, which suffers from exponential growth in $d$, the sparse grid approach selects a reduced set of nodes per dimension to build a sparse approximation. At {\it level} $q\; (\text{where }q\ge d)$, the sparse grid nodes are
%
\begin{equation*}
H(q,d) = \bigcup_{q-d+1\le|\boldsymbol{i}|\le q} \left(X^{i_1}\times \cdots\times X^{i_d}\right)\in [-1,1]^d, 
\end{equation*}
%
where $|\boldsymbol{i}| = i_1+\ldots+i_d$ specifies the refinement rule. These nodes yield a sparse but sufficiently rich representation of the domain, capturing key features of $u$ with far fewer evaluations than a full grid.

We use Chebyshev extrema as collocation nodes \cite{BaNoRi:2000, ClCu:1960}, defined by $x_j^i=-\cos(\frac{ \pi(j-1)}{m_i-1})$ for $j=1, \ldots, m_i$ with $m_1 =1$ and $m_i = 2^{i-1}+1$ for $i\ge 2$, ensuring nestedness $X^i\subset X^{i+1}$. The corresponding multidimensional sparse grid nodes satisfy 
%
\begin{equation}
\label{eq:NestedColPts}
H(q,d)\subset H(q+1,d),\quad \text{and}\quad H(q,d) = \bigcup_{|\boldsymbol{i}|=q} \left(X^{i_1}\times \cdots\times X^{i_d}\right).
\end{equation}
%
Interpolation over $H(q,d)$  is performed using the {\it Smolyak quadrature formula}, 
%
\begin{equation}
\label{eq: Smolyak_Quad_formula}
\mathcal{S}_{q, d}[u] = \sum_{q+1\le |\boldsymbol{i}|\le q+d} (-1)^{q+d-|\boldsymbol{i}|} \binom{d-1}{q+d-|\boldsymbol{i}|}\cdot \left(\mathrm I_{X^{i_1}}\otimes\cdots\otimes \mathrm I_{X^{i_d}}\right) [u].
\end{equation} 
%
which combines univariate interpolants into a high-dimensional approximation. This hierarchical and sparse structure allows for efficient reuse of evaluations and makes the method well-suited for high-dimensional stochastic problems.


% Let $N$ denote the number of sparse grid nodes. The sparse grid stochastic collocation method is equivalent to solving $N$ deterministic parametrized problems \eqref{eq:FreeBoundary} at each nodal point in $H(q,d)$.

% For our model problem, the sparse grid stochastic collocation method constructs the surrogate function $ \mathcal{S}_{q,d}(u)$ as per \eqref{eq: Smolyak_Quad_formula} by computing the direct solution of the discrete version of \eqref{eq:FreeBoundary} at isotropic sparse grid nodes \eqref{eq:NestedColPts} with the Clenshaw-Curtis quadrature abscissa  \cite{BaNoRi:2000,ClCu:1960}. 

% As discussed in \cite{NoTeWe:2008,TeJaWe:2015}, consider the function $u \in C^0(W,Z)$, where the parameter space $W$ and the solution space $Z$ are defined in \eqref{eq:ParameterSpace} and \eqref{eq:Soln_space} respectively. Let the interval in the $k$-th dimension be defined as $W_k = \left[\widetilde{\omega}_k-\tau \left\vert \widetilde{\omega}_k\right\vert, \widetilde{\omega}_k+\tau \left\vert \widetilde{\omega}_k\right\vert\right]$. The complementary multi-dimensional parameter space that excludes the $k$-th dimension is
% %
% \[
% W_k^c = \prod_{i=1, i\neq k}^d W_i.
% \]
% %
% Now, for any fixed element $\omega_k^c \in W_k^c$, and for each $\omega_k\in W_k$, we assume the function $u(\cdot,\omega_k,\omega_k^c): W_k \rightarrow C^0(W_k^c;Z)$ admits an analytic extension  $u(\cdot, z,\omega_k^c)$ in the complex plane, specifically in the region 
% %
% \[
% W_k^{*}:=\{z\in \mathbb{C}: \text{dist} (z,W_k)\le \iota_k \;\text{ for some } \iota_k>0\},
% \]
% %
% where $\iota_k$ denotes the proximity of the analytic extension to the real interval $W_k$. Under these assumptions, the interpolation error associated with the sparse grid method demonstrates an algebraic convergence rate
% %
% \begin{equation} \label{eq:coll-error-bound_2}
%   \big\|u-\mathcal{S}_{q, d} (u)\big\|_\infty = C P^{-\mu},
% \end{equation}
% %
% where $P$ denotes the sparse grid node count, $C$ is a constant dependent on dimension $d$ and analytic extension proximity to the interval $W_k$, and $\displaystyle \mu$ is related to the dimension of parameter space and function's analytic extension in the complex plane.




% Compared to the regularity assumption of $u$ in \cite{ElLiSa:2022}, the assumption for \eqref{eq:coll-error-bound_2} is stronger in the sense that the solution $u$ with respect to the random variable $\boldsymbol{\omega}$ can be analytically extended into the complex plane region by varying with one dimension of the random variable while keeping the other dimensions fixed.  This enhancement allows for a tighter interpolation error bound compared to the regularity assumption in \cite{ElLiSa:2022}.

%!TEX root = main.tex
%%%%%%%%%%%%%%%%%%%%%%%%%%%%%%%%%%%%%%%%%%%%%%%%%%%

% ====================================================
\section{Monte Carlo Method}\label{sec:MC}
% ====================================================

A standard approach to approximate the expectation in \eqref{eq:expectation_of_u} is the Monte Carlo (MC) method; 
see, e.g.,  \cite{MBGiles_2015a,MDGunzburger_CGWebster_GZhang_2014a}.
Because $u$ cannot be evaluated, but only a approximation $u_h \in   L_{\mathbb{P}}^2(W, \cU)$ can be 
computed, we estimate $\mathbb{E}[u_{h}]$, 
The MC estimator $A^{\text{MC}}_{N}$ of $\mathbb{E}[u_{h}]$ 
is the sample mean over $N$ independent and identically distributed (i.i.d.) realizations 
\begin{equation}\label{eq:MC_estimator}
    A^{\text{MC}}_{N} := \frac{1}{N}\sum_{i=1}^{N} u_{h}\big(\omega^{(i)} \big).
\end{equation}
%
This estimator is unbiased,  $\mathbb{E}[A^{\text{MC}}_{N}] = \mathbb{E}[u_{h}]$, 
and has variance $\mathbb{V}[A^{\text{MC}}_{N}] = N^{-1} \mathbb{V}[u_{h}]$, 
where the variance is defined as in \eqref{eq:variance_of_u}.
By the central limit theorem, the MC estimator $A^{\text{MC}}_{N}$ converges in distribution to $\mathbb{E}[u_h]$ as $N$ approaches infinity. 

To quantify the total approximation error of the estimator, we consider the  {\it normalized mean squared error (nMSE)}, defined as
%
 \[
\mathcal{E}_{A^{\text{MC}}_{N}}^2:= \mathbb E\left[ \big\| \mathbb{E}[u]-A^{\text{MC}}_{N}  \big\| _{U}^2\right]  \big/ \, \big\| \mathbb{E}[u]  \big\| _{U}^2.
\] 
%
The nMSE decomposes into two contributions: a {\it bias error} from spatial discretization, and a {\it statistical error} due to finite sampling
%
\[
\mathcal{E}_{A^{\text{MC}}_{N}}^2 
= \frac{ \big\| \mathbb{E}[u]-\mathbb{E}[u_{h}]  \big\| _{U}^2+\mathbb E\left[ \big\|  \mathbb{E}[u_{h}] -A^{\text{MC}}_{N}  \big\| _{U}^2\right]}{ \big\| \mathbb{E}[u]  \big\| _{U}^2} 
= \frac{ \big\| \mathbb{E}[u]-\mathbb{E}[u_{h}]  \big\| _{U}^2}{ \big\| \mathbb{E}[u]  \big\| _{U}^2}+\frac{\mathbb{V}\left[u_{h}\right]}{N \big\| \mathbb{E}[u]  \big\| _{U}^2}
=\mathcal{E}_{\text{Bias}}^2 + \mathcal{E}_{\text{Stat}}^2.
\]
%
Suppose the sample-wise discretization error satisfies
%
\begin{equation*} \label{eq:Assumption_uhA}
       \left\|u\left(\omega^{(i)}\right)-u_h\left(\omega^{(i)}\right)\right\|_U
       \leq C_m\left(\omega^{(i)}\right)M^{-\alpha}\,,
\end{equation*}
%
where $C_m(\omega^{(i)})$ is a constant depending only on the geometry of the spatial domain and the particular realization $\omega^{(i)}$, $\alpha>0$ is the convergence rate of spatial discretization, and $M$ denotes the number of spatial degrees of freedom. For simplicity and analytical tractability, we assume this constant is uniformly bounded across all realizations, i.e. $C_m(\omega^{(i)})\le C_m$ for some $C_m=\sup_{\omega \in \Omega} C_m(\omega)>0$ independent of the sample realization $\omega^{(i)}$ \cite{BaNoTe:2007,BaScZo:2011}.


Given a user-specified threshold $\epsilon^2$  for the nMSE, we introduce a {\it splitting ratio} $\theta \in (0,1)$ to allocate the total error budget between bias and statistical components
%
\begin{equation} \label{eq:error-budget}
%\textcolor{red}{\|u-u_h\|_{L^2(\boldsymbol W,U)}\le C_mM^{-\alpha}\le \theta_1\epsilon},\qquad\text{ and }\qquad \|u_h-\widehat u_{h}\|_{L^2(\boldsymbol W,U)} \le C_{p} P^{-\nu}\le \theta_2\epsilon\,.  
\mathcal{E}_{\text{Bias}}^2=\|u-u_h\|_{L^2(\boldsymbol \Omega,U)}\le C_mM^{-\alpha}= \theta\epsilon^2, \quad\quad \mathcal{E}_{\text{Stat}}^2 = \frac{\sigma_1^2}{N \big\| \mathbb{E}(u)  \big\| _{U}^2}=(1-\theta)\epsilon^2,
\end{equation}
where $C_m$ is independent of the sample and $\sigma_1^2 = \mathbb{V}\left( u_{h}\right)$. To meet these error constraints, the number of spatial nodes $M$ and sample size $N$ must obey
%
\begin{equation}
\label{eq:SLSGC_SL_SpatialGridsNo_n_SparseGridsNo}
M\ge \left(\frac{\theta\epsilon^2}{C_m}\right)^{-\frac 1 {\alpha}},\quad\quad  N \ge  \frac{\sigma_1^2}{\epsilon_{\text{tar}}^2},
\end{equation}
%
where $\epsilon_{\text{tar}}^2 = \epsilon^2(1-\theta) \big\| \mathbb{E}(u)  \big\| _{U}^2$.
Assuming each evaluation of $u_{h}$ incurs an average cost of $C$, the total cost to compute $A^{\text{MC}}_{N}$ is
%
\[
\mathcal{W}^\text{MC}  = CN=\frac{C\sigma_1^2}{\epsilon_{\text{tar}}^2}.
\]
%
In practice, both $M$ and $N$ are rounded up to the smallest integers satisfying \eqref{eq:SLSGC_SL_SpatialGridsNo_n_SparseGridsNo}.
%!TEX root = ../main.tex
% ====================================================
\section{Multi-fidelity Monte Carlo}\label{sec:MFMC}
% ====================================================
This section reviews the multi-fidelity Monte Carlo (MFMC) method, following the foundational formulation in \cite{PeWiGu:2016}. The MFMC framework uses an ensemble of models with varying computational cost and accuracy to construct a variance-reduced estimator for high-fidelity expectation. Let $u_1:\Omega \to U$ denote the high-fidelity (HF) model that provides accurate but expensive evaluations, and let $\{u_k\}_{k=2}^K$ denote low-fidelity (LF) models that offer cheaper approximations. The central goal of MFMC is to allocate a fixed computational budget across these models to minimize estimator variance while maintaining unbiasedness.

We introduce some key statistical quantities that describe the model. We represent the random output of model $u_k$ on the probability space $(\Omega,\mathcal{F},\mathbb{P})$ by $u_k(\boldsymbol{\omega})$, abbreviated as $u_k$. For each pair of models $u_k,u_j$, define the variance and correlation coefficient
%
\begin{equation*}
    \sigma_k^2 = \mathbb{V}\!\left[u_k\right],\qquad 
    \rho_{k,j} = \frac{\text{Cov}\!\left[u_k,u_j\right]}{\sigma_k\sigma_j}, 
    \quad k,j=1,\dots,K,
\end{equation*}
%
where the covariance is defined as $\text{Cov}[u_k,u_j] := \mathbb{E}[\langle u_k - \mathbb{E}[u_k], u_j - \mathbb{E}[u_j]\rangle_U]$ and $\rho_{k,k}=1$. The pairwise correlations between fidelity levels quantify the statistical dependence that drives variance reduction through effective control variates.

The MFMC estimator architecture uses a nested sampling strategy that reuses computational evaluations across fidelity levels. Let $A_{1,N_1}^{\text{MC}}$ denote the standard Monte Carlo estimator of $\mathbb{E}[u_1]$ based on $N_1$ HF samples. The MFMC estimator augments this with corrections from lower fidelities via control variates
%
\begin{equation}\label{eq:MFMC_estimator}
A^{\text{MF}} := A^{\text{MC}}_{1,N_1} + \sum_{k=2}^K \alpha_k\left(\overline{A}_{k,N_k} - \overline{A}_{k,N_{k-1}}\right),
\end{equation}
%
where $\alpha_k \in \mathbb{R}$ are control variate weights and $\overline{A}_{k,N}$ denotes the sample average of $N$ evaluations of model $u_k$. A critical aspect of this construction is the nested sampling structure: the estimator $\overline{A}_{k,N_{k}}$ reuses all $N_{k-1}$ samples from $\overline{A}_{k,N_{k-1}}$, possibly supplemented by additional $N_{k} - N_{k-1}$ samples. The reuse of LF evaluations across levels enhances efficiency but induces sample statistical dependencies that complicate variance analysis.



To facilitate analysis, we reformulate the estimator so that its constituent terms are statistically independent. Partitioning the $N_k$ LF samples into disjoint sets of sizes $N_{k-1}$ and $N_k-N_{k-1}$ yields the equivalent independent form
%
\begin{equation}\label{eq:MFMC_estimator_independent}
    A^{\text{MF}} = A^{\text{MC}}_{1,N_1} +  \sum_{k=2}^K \alpha_k\!\left(1-\frac{N_{k-1}}{N_k}\right)\left(A^{\text{MC}}_{k,N_k\backslash N_{k-1}}-A^{\text{MC}}_{k,N_{k-1}}\right),
\end{equation}
%
where $A_{k,N_k\backslash N_{k-1}}^{\text{MC}}$ is the MC average over the $N_k-N_{k-1}$ new samples (defined to be zero when $N_k=N_{k-1}$).


The statistical properties of the MFMC estimator emerge clearly from its component-wise decomposition. Define
%
\begin{equation}\label{eq:MFMC_Yk}
Y_1 := A^{\text{MC}}_{1,N_1},\quad 
Y_k := \left(1-\frac{N_{k-1}}{N_k}\right)\!\left(A^{\text{MC}}_{k,N_k\backslash N_{k-1}} - A^{\text{MC}}_{k,N_{k-1}}\right), \;\; k=2\ldots, K,
\end{equation}
%
then the MFMC estimator can be expressed into a compact form $A^{\text{MF}} = Y_1 + \sum_{k=2}^K \alpha_k Y_k$. Since each $Y_k$ for $k\ge2$ represents a difference of two independent estimators for the same $\mathbb{E}[u_k]$, we immediately obtain $\mathbb{E}[Y_k]=0$ and the MFMC estimator is unbiased: $\mathbb{E}[A^{\text{MF}}]=\mathbb{E}[u_1]$. The variances of the components are
%
\begin{equation}\label{eq:Var_Yk}
    \mathbb{V}[Y_1] = \frac{\sigma_1^2}{N_1}, \qquad 
    \mathbb{V}[Y_k] = \left(\frac{1}{N_{k-1}} - \frac{1}{N_k}\right)\sigma_k^2, \;\; k=2\ldots, K.
\end{equation}
%
A key statistical insight, formalized in Lemma~\ref{lemma:Y_k_Y_j}, establishes that the correction terms are mutually uncorrelated despite sample reuse.
%
\begin{lemma}\label{lemma:Y_k_Y_j}
For $2\le k<j\le K$, 
% the correction terms $Y_k$ and $Y_j$ defined in \eqref{eq:MFMC_Yk} are uncorrelated, i.e., 
$\operatorname{Cov} [Y_k,Y_j ]=0$.
\end{lemma}
%
The proof is provided in the Appendix.

Each correction $Y_k$($k\ge2$) is correlated with $Y_1$, with covariance
\begin{equation}\label{eq:Cov_Yk}
\operatorname{Cov}[Y_1,Y_k] = -\!\left(\frac{1}{N_{k-1}} - \frac{1}{N_k}\right)\rho_{1,k}\sigma_1\sigma_k,
\end{equation}
as shown in \cite[Lemma~3.2]{PeWiGu:2016}. Combining \eqref{eq:Var_Yk} and \eqref{eq:Cov_Yk} gives
%
\begin{equation}\label{eq:MFMC_variance}
    \mathcal{V}^{\text{MF}}
    =\frac{\sigma_1^2}{N_1} 
    + \sum_{k=2}^K \left(\frac{1}{N_{k-1}} - \frac{1}{N_k}\right)\!\left(\alpha_k^2\sigma_k^2 - 2\alpha_k\rho_{1,k}\sigma_1\sigma_k\right).
\end{equation}
%

In order to determine optimal sample sizes $N_k$ and weights $\alpha_k$ in the MFMC estimator \eqref{eq:MFMC_estimator_independent}, an optimization problem is formulated \cite{PeWiGu:2016} by minimizing the estimator variance \eqref{eq:MFMC_variance} subject to a fixed budget $p$. Let $C_k$ denote the per-sample cost of model $u_k$, the total computational cost is 
%
\[
\mathcal{W}^{\text{MF}} = \sum_{k=1}^K C_k N_k,
\]
%
and the constrained optimization problem becomes
%
\begin{equation}\label{eq:Optimization_pb_sample_size}
    \begin{array}{ll}
    \min &\mathcal{V}^{\text{MF}}\left(\alpha_k,N_k\right),\\
       \text{subject to} &\displaystyle\sum\limits_{k=1}^K C_kN_k=p,\\[2pt]
       &\displaystyle N_1\ge 0,\quad \displaystyle N_{k-1}\le N_k, \;\; k=2\ldots,K,\\
       &N_1,\ldots, N_K\in \mathbb{R},\\
       &\alpha_2,\ldots,\alpha_K\in \mathbb{R}.
    \end{array}
\end{equation}
%
Note that for each level $k\ge 2$, $\alpha_k$ enters only through a quadratic expression independent of $N_k$ in the variance term. This separable structure allows a fundamental simplification of the variance functional, which allows hierarchical minimization
%
\begin{equation*}
    \min_{\alpha_k,\, N_k} \mathcal{V}^{\text{MF}}\left(\alpha_k, N_k\right)
    = \min_{N_k}\Big(\min_{\alpha_k} \mathcal{V}^{\text{MF}}(\alpha_k, N_k)\Big).
\end{equation*}
%
The hierarchical minimization admits a closed-form solution for optimal weights by solving the inner optimization $\partial \mathcal{V}^{\text{MF}}/\partial \alpha_k = 0$, yielding 
%
\begin{equation}\label{eq:MFMC_weights}
    \alpha_k^* = \frac{\rho_{1,k}\sigma_1}{\sigma_k}.
\end{equation}
%
Substituting $\alpha_k^*$ into \eqref{eq:MFMC_variance} simplifies the variance to 
%
\begin{equation*}
    \mathcal{V}^{\text{MF}}\left(\alpha_k^*, N_k\right)
    = \sigma_1^2\sum_{k=1}^K \frac{\Delta_k}{N_k},
\end{equation*}
%
where $\Delta_k = \rho_{1,k}^2 - \rho_{1,k+1}^2$ for $k = 1, \dots, K$ with $\rho_{1,K+1}=0$. This reduces the joint optimization to a continuous resource allocation problem involving only sample allocation
%
\begin{equation}\label{eq:Optimization_pb_sample_size_reduced}
    \begin{array}{ll}
    \min &\displaystyle f(N_k) =\sum_{k=1}^K \frac{\Delta_k}{N_k},\\
       \text{subject to} &\displaystyle\sum\limits_{k=1}^K C_kN_k=p,\\[2pt]
       &\displaystyle -N_1\le 0,\quad \displaystyle N_{k-1}-N_k\le 0, \;\; k=2\ldots,K,\\
       &N_1,\ldots, N_K\in \mathbb{R},
    \end{array}
\end{equation}
%
where $f(N_k)$ is the {\it normalized variance functional}. Under suitable monotonicity and ordering assumptions, this problem admits an analytic solution that characterizes the optimal allocation of resources across fidelity levels.


%
\begin{theorem}[Optimal MFMC real-valued sample allocation]\label{thm:Sample_size_est}
Consider $K$ models $\{u_{k}\}_{k=1}^K$ with standard deviations $\sigma_k$, correlation coefficients $\rho_{1,k}$ of LF model $u_k$ with the HF model $u_1$, and per-sample costs $C_k$. Define $\Delta_k = \rho_{1,k}^2 - \rho_{1,k+1}^2$ for $k = 1, \dots, K$ with $\rho_{1,K+1}=0$. Assume the following conditions hold
%
\begin{alignat*}{3}
&(i)\;\textit{Monotone correlations:} &\quad& |\rho_{1,1}| > \cdots > |\rho_{1,K}|,\\
&(ii)\;\textit{Cost-correlation ratio:} &\quad& \frac{\Delta_{k}}{C_k} > \frac{\Delta_{k-1}}{C_{k-1}}, \quad k=2,\ldots,K.
\end{alignat*}
%
Then the optimal control weights and sample sizes for \eqref{eq:Optimization_pb_sample_size} are
%
\begin{equation}\label{eq:MFMC_RealValued_Sample_Size}
    \alpha_k^* = \frac{\rho_{1,k}\sigma_1}{\sigma_k}, \qquad
    N_k^* = \sqrt{\frac{\Delta_k}{C_k}}\,
    \frac{p}{\sum_{j=1}^K \sqrt{C_j \Delta_j}}.
\end{equation}
%
% \[
% r_k^* = \sqrt{\frac{C_1\Delta_k}{C_k\Delta_1}},\quad N_1^* = \frac{p}{\sum_{k=1}^K C_k r^*_k}, \quad N_k^*=N_1^*r_k^*.
% \] 
% %
% \JLcolor{alternatively, in my way to represent it without mentioning the vector $\boldsymbol{r}^*$, we have}
%
The resulting minimal variance of the MFMC estimator is
\begin{equation}\label{eq:MFMC_variance_optimal}
\mathcal{V}^{\text{MF}}
= \sigma_1^2\sum_{k=1}^K \frac{\Delta_k}{N_k^*}=\frac{\sigma_1^2}{p}\!\left(\sum_{k=1}^K \sqrt{C_k \Delta_k}\right)^{\!2}.
\end{equation}
\end{theorem}
%


Differentiating the normalized variance and cost with respect to the sample sizes gives
%
\[
\frac{\partial f}{\partial N_k} = -\frac{\Delta_k}{N_k^2},
\qquad 
\frac{\partial \mathcal{W}^{\text{MF}}}{\partial N_k} = C_k.
\]
%
These relations quantify the variance–cost trade-off: increasing samples at any level reduces variance at the expense of computational resources. At the continuous optimum \eqref{eq:MFMC_RealValued_Sample_Size}, the marginal variance reduction per unit cost $\Delta_k/(C_k N_k^2)$ is identical across all active models, establishing a balanced resource allocation that characterizes the optimal allocation.

While Theorem~\ref{thm:Sample_size_est} provides real-valued optimal allocations $N_k^*$, practical implementation requires integer sample sizes. The standard approach \cite{PeWiGu:2016} applies the floor function $\lfloor N_k^* \rfloor$ to ensure budget feasibility. The realized variance and cost are
%
\[
f\left(\left\lfloor N_k^* \right\rfloor\right) = \sum_{k=1}^K\frac{\Delta_{k}}{\left\lfloor N_k^* \right\rfloor}, \qquad \mathcal{W}^{\text{MF}}\left(\left\lfloor N_k^* \right\rfloor\right) = \sum_{k=1}^K C_k\left\lfloor N_k^* \right\rfloor.
\]
%
Since $N_k^*-1 < \lfloor N_k^*\rfloor \le N_k^*$, the floor operation induces bounded sub-optimality, producing the bounds
%
\begin{equation}\label{eq:bounds_for_floor}
\begin{aligned}
    % f\left(\left\lfloor N_k^* \right\rfloor\right)&\in \left[\sum_{k=1}^K\frac{\Delta_{k}}{N_k^*},\; \sum_{k=1}^K\frac{\Delta_{k}}{N_k^*-1}\right) = \left[\frac{1}{p}\left(\sum_{k=1}^K \sqrt{C_k\Delta_k}\right)^2, \sum_{k=1}^K\frac{\Delta_{k}}{\frac{p}{\sum_{j=1}^K \sqrt{C_j\Delta_j}}\sqrt{\frac{\Delta_k}{C_k}}-1}\right)\\
    % &=\left[\frac{1}{p}\left(\sum_{k=1}^K \sqrt{C_k\Delta_k}\right)^2, \sum_{k=1}^K \sqrt{C_k\Delta_k}\sum_{k=1}^K\frac{\sqrt{C_k\Delta_{k}}}{p-\sqrt{\frac{C_k}{\Delta_k}}\sum_{j=1}^K \sqrt{C_j\Delta_j}}\right)\\
    % &=\sum_{k=1}^K \sqrt{C_k\Delta_k}\left[\frac{\sum_{k=1}^K \sqrt{C_k\Delta_k}}{p},\sum_{k=1}^K\frac{\sqrt{C_k\Delta_{k}}}{p-\sqrt{\frac{C_k}{\Delta_k}}\sum_{j=1}^K \sqrt{C_j\Delta_j}}\right)\\
    % \mathcal{W}^{\text{MF}}\left(\left\lfloor N_k^* \right\rfloor\right) &\in \left(\sum_{k=1}^KC_kN_k^*-\sum_{k=1}^K C_k, \sum_{k=1}^KC_kN_k^*\right]=\left( p-\sum_{k=1}^K C_k,p\right].
    f\left(\left\lfloor N_k^* \right\rfloor\right) \in \left[\frac{1}{p}\left(\sum_{k=1}^K \sqrt{C_k\Delta_k}\right)^2, \sum_{k=1}^K\frac{\Delta_{k}}{N_k^*-1}\right), \qquad
\mathcal{W}^{\text{MF}}\left(\left\lfloor N_k^* \right\rfloor\right)\in \left( p-\sum_{k=1}^K C_k, p\right].
\end{aligned}
\end{equation}
%
The term $\sum_{k=1}^K C_k$ represents the rounding-induced slack in the budget, which becomes negligible asymptotically as $p \to \infty$. However, in the pre-asymptotic regime -- where the total budget $p$ is moderate -- this  slack can lead to significant under-utilization of the computational resources. This observation naturally motivates \textit{the development of  alternative integer-valued allocation strategies that reduce slack and achieve tighter budget utilization.}





% This quantity is the marginal variance reduction rate — how much the total variance decreases when you spend more samples at level by taking one more sample cost $C_k$. So the marginal variance reduction per unit cost
% \[
% \frac{-\frac{\partial f}{\partial N_k}}{C_k} = \frac{\Delta_k}{C_kN_k^2}
% \]
% It quantifies that How much variance reduction we get per unit cost at level $k$.
% At the optimum, the system reaches equilibrium where every active model yields the same return per cost unit,
% \[
% \frac{\Delta_k}{C_kN_k^2} = \text{Constant}=\frac{1}{p^2}\left(\sum_{k=1}^K \sqrt{C_k \Delta_k}\right)^{\!2}, \quad \text{for all active}\;\; k.
% \]

















\section{Parameter estimation for multi-fidelity Monte Carlo}\label{sec:Parameter_Estimation}

To estimate the correlation coefficients from a pilot sample of size $Q$, we use unbiased Monte Carlo estimators for the sample covariance and standard deviations of the high- and low-fidelity models. Let $\widehat{\text{Cov}}$ denote the sample covariance between the high- and low-fidelity outputs, and let $\widehat\sigma_1$ and $\widehat\sigma_k$ denote their respective sample standard deviations. The resulting estimator for the correlation coefficient is given by
%
\[
\widehat\rho_{1,k} = \frac{\widehat{\text{Cov}}}{\widehat\sigma_1 \widehat\sigma_k} = \frac{\sum_{i=1}^Q\left\langle u_{h,1}^{(i)} - \overline{u}_{h,1},  u_{h,k}^{(i)} - \overline{u}_{h,k} \right\rangle}{\sqrt{\sum_{i=1}^Q \left\langle u_{h,1}^{(i)} - \overline{u}_{h,1}, u_{h,1}^{(i)} - \overline{u}_{h,1} \right\rangle} \sqrt{\sum_{i=1}^Q \left\langle u_{h,k}^{(i)} - \overline{ u}_{h,k}, u_{h,k}^{(i)} - \overline{u}_{h,k} \right\rangle}},
\]
%
where the sample means are defined as $\overline{u}_{h,1} = Q^{-1}\sum_{i=1}^Q u_{h,1}^{(i)}$ and $\overline{  u}_{h,k} = Q^{-1}\sum_{i=1}^Q u_{h,k}^{(i)}$. Since this estimator involves a non-linear ratio of random variables, it is biased in finite samples. To characterize its behavior, we analyze the mean squared error between the true correlation $\rho_{1,k}$ and its sample estimate, decomposed into squared bias and variance
%
\begin{equation}
\label{eq:MSE_rho}
    \mathbb{E}\left[\left(\rho_{1,k} - \widehat\rho_{1,k}\right)^2\right]= \underbrace{\left(\rho_{1,k} - \mathbb{E}\left[\widehat\rho_{1,k}\right]\right)^2}_{\text{Bias}}+\underbrace{\mathbb{E}\left[\left( \mathbb{E}\left[\widehat\rho_{1,k}\right]-\widehat\rho_{1,k}\right)^2\right]}_{\text{Variance}}=\left(\rho_{1,k} - \mathbb{E}\left[\widehat\rho_{1,k}\right]\right)^2+\mathbb{V}\left[\widehat\rho_{1,k}\right].
\end{equation}
%
To derive asymptotic approximations for the bias and variance, we apply the multivariate delta method \cite{Cr:1946,Oe:1992}, which linearizes a function of random variables via Taylor expansion around its mean. Let the parameter vector be $s = (\rho_{1,k}\sigma_1\sigma_k, \sigma_1, \sigma_k)^T$, and let the sample estimate be $\widehat s = (\widehat{\text{Cov}}, \widehat\sigma_1, \widehat\sigma_k)^T$. Under the central limit theorem, $\widehat s$ converges in distribution to $s$ with $\sqrt{Q}(\widehat s-s)\sim \mathcal{N}(0,\Sigma)$, where $\Sigma$ is the asymptotic covariance matrix of the estimators. Defining the correlation coefficient function $f(s) = s_1 / (s_2 s_3)$, and assuming that the gradient of $f$ exists and is non-zero, we expand $f(\widehat s)$ about $s$ to obtain
%
\begin{equation}
\label{eq:Correlated_Coeff_approx}
  \widehat\rho_{1,k} \approx \rho_{1,k} + \nabla f |_{s}^T \left(\widehat s-s\right), 
  % + \left(s^{(Q)}-s\right)^T H\left(s^{(Q)}-s\right),
\end{equation}
%
where the gradient is $\nabla f|_{s} = (\frac{1}{\sigma_1\sigma_k},-\frac{\rho_{1,k}}{\sigma_1},-\frac{\rho_{1,k}}{\sigma_k} )^T$ and $H$ is the Hessian matrix of the second derivatives. Provided that the components of $\widehat s$ have sufficiently many bounded moments, this expansion gives a valid leading-order approximation of the bias and variance of $\widehat \rho_{1,k}$. In general, for a large sample size $Q (\ge 500)$,  the bias and variance of the sample estimate admit asymptotic expansions of the form
%
\begin{equation*}
\label{eq:Expectation_var_rho}
    \mathbb{E}\left(\widehat \rho_{1,k}\right) =\rho_{1,k}+\frac{a_1}{Q} + \mathcal{O}\left(\frac 1 {Q^2}\right),\qquad \text{Var}\left(\widehat \rho_{1,k}\right)= \frac{a_2}{Q} + \mathcal{O}\left(\frac{1}{Q^2}\right).
\end{equation*}
%
where the constants $a_1$ and $a_2$ depend on the distribution of the underlying random variables. Under the classical assumption of bivariate normality, explicit expressions for these constants are available \cite{Fi:1915, Ha:2007, Ri:1932, So:1913}: $a_1 = -(\rho_{1,k} - \rho_{1,k}^3)/2$ and $a_2 = (1 - \rho_{1,k}^2)^2$. The variance is `instable' since it depends on $\rho_{1,k}$. Using it to construct confidence interval will suffers from the issue that the coverage is not accurate enough. Moreover, the sampling distribution of Pearson's correlation coefficient is not normally distributed. It can be highly skewed, especially when the sample size is small or when the population correlation is near $\pm 1$. This skewed distribution makes it difficult to calculate confidence intervals and conduct statistical tests. In such cases, Fisher's $z$-transformation \cite{Fi:1915, Fi:1921} for $\widehat \rho_{1,k}$ provides an effective means to construct confidence intervals for $\rho_{1,k}$. It shifts $\widehat\rho_{1,k}$ to $z_k$ via an inverse hyperbolic tangent function
%
\begin{equation}
\label{eq:Fisher_z}
    z_k  = \text{tanh}^{-1}\left(\widehat\rho_{1,k}\right) = \frac 1 2\ln \left(\frac{1+\widehat\rho_{1,k}}{1-\widehat\rho_{1,k}}\right).
\end{equation}
%
The transformed distribution for $\widehat \rho_{1,k}$ transform correlation coefficients $\widehat \rho_k$ into a variable $z_k$ that is approximately normally distributed, and the variance ($\text{Var}[z_k] = 1/(Q - 3)$) is stable in the sense that it is independent of $\rho_{1,k}$. This indicates $z_k$ is approximately normally distributed even for moderate $Q$. Although its derivation assumes bivariate normality, the transformation remains effective in practice when the data exhibit moderate deviations from normality and are not contaminated by extreme outliers. 

In contrast, our setting does not assume a specific distribution for the model outputs. The bivariate normality assumption is often inappropriate for multifidelity models, where nonlinear mappings or discretization artifacts may induce non-Gaussian dependencies. Instead, a nonparametric asymptotic framework \cite{Og:2006, Pi:1937} is adopted to provide consistent estimators in the large-$Q$ limit ($\ge 500$) without requiring distributional assumptions. In this regime, the leading-order terms simplify to $a_1 = 0$ and $a_2 = 1$. While this asymptotic characterization offers valuable insight, our goal is to design procedures that remain effective for small pilot sample sizes, where these approximations may no longer hold. In particular, the asymptotic expressions for bias and variance cannot reliably inform the choice of $Q$ in practical settings. To address this, we adopt a sequential analysis framework \cite{La:2001,Wa:1947}, which enables dynamic adjustment of the sample size during data collection. Rather than fixing large $Q$ in advance, this approach allows sampling to terminate early once predefined accuracy criteria are satisfied -- offering substantial computational savings over static designs. To develop effective stopping rules for this adaptive scheme, we begin by analyzing the sensitivity of key performance metrics -- specifically, the MFMC estimator’s variance and cost-efficiency -- with respect to perturbations in the estimated correlation coefficient. However, sensitivity analysis alone cannot ensure that $\widehat \rho_{1,k}$ achieves the required level of statistical accuracy.  To accommodate such scenarios, we additionally construct confidence intervals for $\rho_{1,k}$ that remain valid under non-Gaussian output distributions. By combining sensitivity-based diagnostics with confidence interval-based uncertainty quantification, we formulate a robust, real-time adaptive sampling strategy that guarantees accurate correlation estimation and improves the overall efficiency of the MFMC procedure.





\subsection{Sensitivity analysis of cost efficiency and variance}
To analyze the sensitivity of cost efficiency and variance with respect to the correlation coefficients, let $\boldsymbol{\rho}$ denote the true correlation vector, and consider a perturbation $\boldsymbol{\rho} + \Delta \boldsymbol{\rho}$. A first-order Taylor expansion yields the following approximations for the corresponding changes in cost efficiency $\xi$ and estimator variance $\mathbb{V}(A^{\text{MF}})$


% Using these two estimates, we determine the optimal choice of $Q$ by ensuring that the mean square error does not exceed a prescribed threshold $\delta$, we allocate a fraction $\theta_1$ to bias and $1-\theta_1$ to variance. Using the error splitting in \eqref{eq:MSE_rho}, we obtain the required pilot sample size
% applying Chebyshev’s inequality $P(|\mathbb{E}(\rho_{1,k}^{(Q)})-\rho_{1,k}^{(Q)}|\ge \nu)\le \text{Var}(\rho_{1,k}^{(Q)})/\nu^2$ with $\nu = (1-\theta_1)\delta_1$ gives
% %
% \[
% P\left(\left|\mathbb{E}\left(\rho_{1,k}^{(Q)}\right)-\rho_{1,k}^{(Q)}\right|\ge \nu\right)\le \frac{\text{Var}\left(\rho_{1,k}^{(Q)}\right)}{\nu^2}
% \]
% %
% %
% \[
% \frac{(1-\rho_{1,k}^2)^2}{Q\nu^2} = \frac{(1-\rho_{1,k}^2)^2}{(1-\theta_1)^2Q\delta_1^2}\le 1\rightarrow Q\ge \frac{(1-\rho_{1,k}^2)^2}{(1-\theta_1)^2\delta_1^2}.
% \]
% %
% Combining these results, a lower bound on $Q$ can be determined as
%
% \begin{equation}
% \label{eq:Offline_Sample_Size}
%     Q\ge \max_{k} \left(\frac{\left|a_1\right|}{\sqrt{\theta_1\delta} }, \frac{a_2}{(1-\theta_1)\delta}\right).
% \end{equation}
% %
% Note in \eqref{eq:Offline_Sample_Size}, we still need to estimate the true correlation coefficients in order to estimate the lower bound of pilot sample size $Q$. However, the sample statistics also depends on $Q$,  we thus  iteratively update $Q$ until convergence is reached. % However, when sampling with a small sample size that does not rely on assumptions about the underlying data distribution, non-parametric method like  bootstrapping \cite{Wa:2006} and sequential analysis \cite{Wa:1947} provide alternative strategies for estimating $Q$. 



%
\[
\Delta\xi=\xi(\boldsymbol{\rho}+\Delta \boldsymbol{\rho}) - \xi(\boldsymbol{\rho}) \approx \sum_{k=2}^K \frac{\partial \xi}{\partial \rho_{1,k}} \Delta\rho_{1,k},\quad \quad \Delta \mathbb{V}\left[A^{\text{MF}}\right]\approx \sum_{k=2}^K \frac{\partial  \mathbb{V}\left[A^{\text{MF}}\right]}{\partial  \rho_{1,k}}  \Delta\rho_{1,k},
\]
%
where the partial derivatives quantify the sensitivities of $\xi$ and $\mathbb{V}(A^{\text{MF}})$ to perturbations in the correlation coefficients. These are given by
%
\begin{align}
% \frac{\partial  \xi}{\partial  \rho_{1,1}} &=\frac{2\sum_{j=1}^K\sqrt{C_j\left(\rho_{1,j}^2 - \rho_{1,j+1}^2\right)}}{C_1}\frac{C_1\rho_{1,1}}{\sqrt{C_1(\rho_{1,1}^2-\rho_{1,2}^2)  }}\\
\label{eq:partial_xi_rho}
\frac{\partial  \xi}{\partial  \rho_{1,k}} 
&=\frac{2SS^\prime}{C_1}, \quad \forall\; k=2,\ldots, K,\\
\label{eq:partial_var_rho}
\frac{\partial  \mathbb{V}\left[A^{\text{MF}}\right]}{\partial  \rho_{1,k}} 
&=\sigma_1^2\left[2\rho_{1,k}\left(\frac{1}{N_{k}} - \frac{1}{N_{k-1}}\right)-\left( \frac{\rho_{1,k-1}^2 -\rho_{1,k}^2 }{N_{k-1}^2}\frac{\partial N_{k-1}}{\partial  \rho_{1,k}}+\frac{\rho_{1,k}^2 -\rho_{1,k+1}^2 }{N_k^2}\frac{\partial N_k}{\partial  \rho_{1,k}}\right)\right]=\epsilon_{\text{tar}}^2\frac{S^\prime \left(S-T\right)}{S^2},
\end{align}
%
with
%
\begin{align}
\label{eq:S_n_S_prime}
S& = \sum_{j=1}^K\sqrt{C_j\left(\rho_{1,j}^2-\rho_{1,j+1}^2\right)},\quad
S^\prime = \frac{\partial  S}{\partial  \rho_{1,k}} = \rho_{1,k}\left(\sqrt{\frac{C_k}{\rho_{1,k}^2-\rho_{1,k+1}^2}} - \sqrt{\frac{C_{k-1}}{\rho_{1,k-1}^2-\rho_{1,k}^2}}\right),\\
\nonumber
T &=  \sqrt{C_{k-1}\left(\rho_{1,k-1}^2 - \rho_{1,k}^2\right)} + \sqrt{C_{k}\left(\rho_{1,k}^2 - \rho_{1,k+1}^2\right)},\\
\nonumber
\frac{\partial N_{k}}{\partial  \rho_{1,k}}&=\frac{\sigma_1^2}{\epsilon_{\text{tar}}^2}\left[\frac{\rho_{1,k}}{\sqrt{C_{k}\left(\rho_{1,k}^2 - \rho_{1,k+1}^2\right)}}S+S^\prime \sqrt{\frac{\rho_{1,k}^2 - \rho_{1,k+1}^2}{C_{k}}}\right],\\
\nonumber
\frac{\partial N_{k-1}}{\partial  \rho_{1,k}}&=\frac{\sigma_1^2}{\epsilon_{\text{tar}}^2}\left[\frac{-\rho_{1,k}}{\sqrt{C_{k-1}\left(\rho_{1,k-1}^2 - \rho_{1,k}^2\right)}}S+S^\prime \sqrt{\frac{\rho_{1,k-1}^2 - \rho_{1,k}^2}{C_{k-1}}}\right].
% S^{\prime\prime}&= \frac{\partial^2  S}{\partial^2  \rho_{1,k}} = \frac{S^\prime}{\rho_{1,k}} - \rho_{1,k}^2\left(\frac{\sqrt{C_k}}{(\rho_{1,k}^2-\rho_{1,k+1}^2)^{3/2}}+\frac{\sqrt{C_{k-1}}}{(\rho_{1,k-1}^2-\rho_{1,k}^2)^{3/2}}\right).
\end{align}
%
Under standard MFMC assumptions, the derivative $S^\prime$ is negative, while both $S$ and $S - T$ are positive. Consequently, the partial derivatives $\partial \xi / \partial \rho_{1,k}$ and $\partial \mathbb{V}[A^{\text{MF}}] / \partial \rho_{1,k}$ are negative. This implies that increasing $\rho_{1,k}$ improves cost efficiency and reduces the variance of the estimator. To quantify the impact of correlation errors, we apply the Cauchy–Schwarz inequality to obtain bounds on the relative errors
%
\begin{align}
\label{eq:delta_xi_bound}
    \frac{\left|\Delta \xi\right|}{\xi}&\le \underbrace{\frac{1}{\xi}\sqrt{\sum_{k=2}^K \left(\frac{\partial \xi}{\partial \rho_{1,k}}\right)^2}}_{A_0} \cdot \sqrt{\sum_{k=2}^K\left(\Delta\rho_{1,k}\right)^2}=\frac{2}{S}\sqrt{\sum_{k=2}^K(S^\prime)^2} \cdot \sqrt{\sum_{k=2}^K\left(\Delta\rho_{1,k}\right)^2},\\
    \label{eq:delta_var_bound}
    \frac{\left|\Delta \mathbb{V}\left[A^{\text{MF}}\right]\right|}{\mathbb{V}\left[A^{\text{MF}}\right]}&\le
    % \le \underbrace{\frac{1}{\mathbb{V}\left(A^{\text{MF}}\right)}\sqrt{\sum_{k=2}^K \left(\frac{\partial \mathbb{V}\left(A^{\text{MF}}\right)}{\partial \rho_{1,k}}\right)^2}}_{A_1}\cdot \sqrt{\sum_{k=2}^K\left(\Delta\rho_{1,k}\right)^2}=
    \underbrace{\frac{1}{S^2}\sqrt{\sum_{k=2}^K\left(S^\prime \left(S-T\right)\right)^2}}_{A_1}\cdot \sqrt{\sum_{k=2}^K\left(\Delta\rho_{1,k}\right)^2}.
\end{align}
%
To enforce accuracy in the correlation estimates, we introduce a user-defined relative tolerance $\delta$ and define $A = \max(A_0, A_1)$. Imposing the bound $|\Delta \rho_{1,k}| \le \delta / (A \sqrt{K - 1})$ ensures that the total relative errors in both cost efficiency and variance remain below $\delta$.




% \JLcolor{Given $\delta$, we first estimate $C^\prime$, then choose $\delta_2$ as $\delta/C^\prime$, $\delta_1=\delta_2/\sqrt{K-1}$, and select $N$ by \eqref{eq:Offline_Sample_Size} for all $k$.}


% The term $\left(1-\sqrt{\frac{C_{k-1}(\rho_{1,k}^2-\rho_{1,k+1}^2)}{C_k(\rho_{1,k-1}^2-\rho_{1,k}^2)}}\right)$ in $\partial \xi/\partial \rho_{1,k}$ encodes MFMC’s selection criteria, ensuring the derivative’s negativity when models are optimally ordered. This indicates that higher $\rho_{1,k}$ improves low-fidelity models’ variance reduction efficiency, reducing reliance on costly high-fidelity evaluations. This reinforces the idea that high-quality low-fidelity models—those that are more aligned with the high-fidelity results—can significantly lower the reliance on expensive high-fidelity evaluations, making the entire multi-fidelity approach more cost-effective.


\subsection{Confidence interval for correlation coefficient}
While the dynamic strategy based on cost efficiency and variance reduction is effective, it may terminate prematurely -- before the correlation coefficients are estimated with sufficient accuracy. To mitigate this risk, we introduce an additional stopping criterion based on confidence intervals. Since we do not assume the underlying random variables are normally distributed and aim to work with small pilot sample sizes, nonparametric methods provide a natural solution. In particular, bootstrap techniques \cite{Ef:1979, EfTi:1993} and their extensions \cite{BeDeToMeBaRo:2007} enable the construction of distribution-free confidence intervals through repeated resampling with replacement. These methods are especially effective for small sample sizes (e.g., $Q \leq 30$), requiring minimal assumptions and accommodating non-Gaussian behavior.

When the pilot sample size $Q$ is small and the true correlation is near $\pm 1$, the sampling distribution of the Pearson correlation coefficient becomes notably skewed, the Fisher $z$-transformation to the sample correlation $\widehat \rho_{1,k}$ also handles this issue. To further accommodate non-Gaussian variables, we pre-process the data by ranking the values of $u_{h,1}^{(i)}$ and ${u}_{h,k}^{(i)}$ in {\it ascending order}. We then compute the Spearman rank-order correlation coefficient as the Pearson correlation between the ranks. Applying the Fisher transformation to this rank-based statistic yields a transformed variable $z_k$ with standard error $\sigma_{z_k} = 1.03/\sqrt{Q - 3}$ \cite{BiHi:2017, FiHaPe:1957}. A $1-\alpha$ confidence interval for $z_{k}$ is given by $z_k \pm z_{\alpha/2}\sigma_{z_k}$, where the z-score $z_{\alpha/2}$ is the $\alpha/2$-th quantile for the normal distribution $N(0,1)$. For instance, an $95\%$ confidence interval corresponds to $z_{\alpha/2} = 1.96$. Once confidence interval for $z_k$ is obtained, we invert it via the hyperbolic tangent to obtain the confidence interval for $\rho_{1,k}$
%
\begin{align}
    \label{eq:Confidence_Interval_rho}
    \text{CI}_{\rho_{1,k}} &:= \text{tanh}\left(z_k \pm  z_{\alpha/2}\sigma_{z_k}\right)
    =\left[1-\frac{2}{\left(\frac{1+\widehat\rho_{1,k}}{1-\widehat\rho_{1,k}}\right)e^{-2z_{\alpha/2}\sigma_{z_k}}+1}, 1-\frac{2}{\left(\frac{1+\widehat\rho_{1,k}}{1-\widehat\rho_{1,k}}\right)e^{2z_{\alpha/2}\sigma_{z_k}}+1}\right]
    % = \left[\frac{e^{2(z_k - 1.96\sigma_{z_k})}-1}{e^{2(z_k - 1.96\sigma_{z_k})}+1},\; \frac{e^{2(z_k + 1.96\sigma_{z_k})}-1}{e^{2(z_k + 1.96\sigma_{z_k})}+1}\right].
\end{align}
%
We can calculate the center and the length of the confidence interval explicitly
\[
\text{length} = \frac{2}{\left(\frac{1+\widehat\rho_{1,k}}{1-\widehat\rho_{1,k}}\right)e^{-2z_\alpha\sigma_{z_k}}+1}-\frac{2}{\left(\frac{1+\widehat\rho_{1,k}}{1-\widehat\rho_{1,k}}\right)e^{2z_\alpha\sigma_{z_k}}+1}
\]
Note that as $\widehat\rho_{1,k}$ increases from -1 to 1, $(1+\widehat\rho_{1,k})/(1-\widehat\rho_{1,k})$ increases from 0 to infinity, the center of the confidence interval increases and the confidence interval is always within [-1,1]. If $\widehat\rho_{1,k}\in[0,1]$ increases, $(1+\widehat\rho_{1,k})/(1-\widehat\rho_{1,k})\in [1,\infty)$ increases, interval length decreases, the center of the confidence interval is greater than $\widehat \rho_{1,k}$. When $\widehat\rho_{1,k}\in[-1,0]$ increases, $(1+\widehat\rho_{1,k})/(1-\widehat\rho_{1,k})\in [0,1)$, interval length increases, the center of the confidence interval is smaller than $\widehat \rho_{1,k}$. The largest length of confidence interval occurs when $\widehat \rho_{1,k}=0$, the largest length is $2(e^{2z_{\alpha/2}\sigma_{z_k}}-1)/(e^{2z_{\alpha/2}\sigma_{z_k}}+1)$. Moreover, we can also show that the length of the confidence interval is an even function of $\widehat \rho_{1,k}$, i.e. length$(\widehat \rho_{1,k})=$ length$(-\widehat \rho_{1,k})$. See Figure \ref{fig:CI_plot}.

%
\begin{figure}[!b]\centering
\begin{tabular}{ccc}
\includegraphics[width=0.3\linewidth]{./figures/CI_Q_5.pdf}&
\includegraphics[width=0.3\linewidth]{./figures/CI_Q_30.pdf}&
\includegraphics[width=0.3\linewidth]{./figures/CI_distance.pdf}
\end{tabular}
\caption{Left: confidence interval for $\rho_{1,k}$ with $Q=5$. Middle: confidence interval for $\rho_{1,k}$ with $Q=30$. Right: length of confidence interval for $Q=30$. It is an even function in terms of $\widehat \rho_{1,k}$. All plots are generated with $\widehat \rho_{1,k}=-1+0.05t$ with $t=0,\ldots,40$. 95\% confidence interval with $z_{\alpha/2}=1.96$.} 
\label{fig:CI_plot} 
\end{figure}
%

This confidence interval, together with the sensitivity analysis, forms the basis of a robust stopping rule for adaptive sampling. Specifically, we determine the minimal pilot sample size $Q$ required for reliable parameter estimation by sequentially updating sample statistics as $Q$ increases \cite{La:2001,Wa:1947}. Sampling continues until both accuracy and confidence criteria are satisfied. This adaptive strategy is summarized in Algorithm~\ref{algo:Parameter_Estimation} and is integrated into the enhanced model selection procedure described in Section~\ref{sec:Model_Selection} (Algorithm~\ref{algo:enhanced_mfmc_selection}).
%
\begin{algorithm}[!ht]
\DontPrintSemicolon

    \KwIn{Tolerance $\delta$, splitting ratio $\theta_1 = 0.5$, number of low-fidelity model $K$, initial sample size $Q_0$, sample size correction $dQ = Q_0$. Initializations for Welford's algorithm: proxies of mean, variance and covariance of high- and low-fidelity models $\widehat m_1^{(0)} = 0, \widehat m_k^{(0)} = 0$, $\widehat v_1^{(0)}=0, \widehat v_k^{(0)}=0$, $\widehat r_k^{(0)}=0$.}
    \KwOut{Sample size $Q$ for dynamic sampling, estimated parameters $\sigma_1,\alpha_k, \boldsymbol{\rho}$, cost efficiency $\xi$.}

    
    $\text{AddSample = True}.$

    $p=0, \xi^{(p)} = 0.$
    
    \While{AddSample = True}{
    
    \For{$k=2,\ldots, K$}{
    
        \For{$i = 1,\cdots, dQ $}
    {
    $j=p+i$.
    
    Estimate sample means $\widehat m_1^{(j)}, \widehat m_k^{(j)}$, standard deviations $\widehat\sigma_1^{(j)}, \widehat\sigma_k^{(j)}$, covariances $\widehat{\text{Cov}}^{(j)}$ and correlated coefficients $\widehat\rho_{1,k}^{(j)}$ by Welford's algorithm.
    }
    }
    % [$\text{index},\xi^{(p)}$] = Multi-fidelity Model Selection ($\boldsymbol{\rho}^{(p)},\boldsymbol{C}$).
    
    
    [$\text{index},\xi^{(p+dQ)}$] = Multi-fidelity Model Selection ($\widehat{\boldsymbol{\rho}}^{(p+dQ)},\boldsymbol{C}$).

    

    % Using selected models, estimate pilot sample size $Q_t$ via \eqref{eq:Offline_Sample_Size}.
    
    % Compute  $\xi^{(j)}$ by \eqref{eq:MFMC_sampling_cost_efficiency} with the selected $K^*$ models.
    
    
    \If{$\max\left\{\left|\frac{\xi^{(p+dQ)}-\xi^{(p)}}{\xi^{(p+dQ)}}\right|, \left|\frac{\mathbb{V}\left[A^{\text{MF}}\right]^{(p+dQ)}-\mathbb{V}\left[A^{\text{MF}}\right]^{(p)}}{\mathbb{V}\left[A^{\text{MF}}\right]^{(p+dQ)}}\right|\right\}<\delta$}
    % $\&$ $\left|\frac{\sigma_{k}^{(j)}-\sigma_{k}^{(j-1)}}{\sigma_{k}^{(j)}}\right|<\delta$ $\&$ $\left|\frac{\widehat \sigma_{k}^{(j)}-\widehat \sigma_{k}^{(j-1)}}{\widehat \sigma_{k}^{(j)}}\right|<\delta$ for all $k=2,\ldots, K$}
    {

    Using \eqref{eq:delta_xi_bound} and \eqref{eq:delta_var_bound} to compute threshold $\delta_1 = 2\delta/(A \sqrt{K^* - 1})$ for $\boldsymbol{\rho}$.
    
    Compute $z_k$ and the confidence interval $\text{CI}_{\rho_{1,k}}$ as in \eqref{eq:Fisher_z} and \eqref{eq:Confidence_Interval_rho}.
    

    
    \If{ the length of confidence interval is bigger than $\delta_1$}
    {
    \text{AddSample = False.}
    }
    \Else {
    \text{AddSample = True.}
    }
    }
    \Else {
    \text{AddSample = True.}
    
    $\xi^{(p)} = \xi^{(p+dQ)}$.
    
     $p=p+dQ$.}
    
    % \If{$j<Q_t$}
    % {
    % AddSample = True
    % }
    
    }    
    
    $Q=j$, $\sigma_1 = \widehat\sigma_1^{(j)}(\text{index})$, $\sigma_k = \widehat\sigma_k^{(j)}(\text{index})$, $\boldsymbol{\rho} = \widehat{\boldsymbol{\rho}}^{(j)}(\text{index})$.
\caption{Dynamic strategy for parameter estimation}\label{algo:Parameter_Estimation}
\end{algorithm}


\subsection{Model selection}\label{sec:Model_Selection}
After obtaining the confidence interval \eqref{eq:Confidence_Interval_rho} for correlation coefficients $\{\rho_{1,k}\}_{k=1}^{K}$, we want to select the correlation coefficient to meet the criterion of the Theorem \ref{thm:Sample_size_est}. The issue occurs when the confidence interval of correlation coefficients may overlap, as shown in Figure \ref{fig:CI_plot}. Overlapping confidence intervals does not directly mean that there is no significant difference in the parameters, and requires hypothesis testing or more precise confidence interval analysis. The following two methods select the model with correlations satisfy the two conditions in Theorem \ref{thm:Sample_size_est}. 
\newline 

\noindent {\bf Bootstrap method}
This method does not consider the distribution and is thus flexible, but may suffer from extra sampling cost. Repeatedly sample for $B$ times, create a sample statistics with Bootstrap method. For condition (i) of Theorem \ref{thm:Sample_size_est}, let $H_0:$ there was no significant difference in adjacent correlation coefficients (i.e. $|\rho_{1,k}|\le |\rho_{1,k+1}|$). $H_1:$ correlation coefficients decrease significant (i.e. $|\rho_{1,k}|> |\rho_{1,k+1}|$).  Estimate $|\widehat \rho_{k}^{(b)}|$ and $|\widehat \rho_{k+1}^{(b)}|$. Then estimate the $1-\alpha$ confidence interval for $|\widehat \rho_{k}^{(b)}| - |\widehat \rho_{k+1}^{(b)}|$. If the lower bound of the confidence interval is positive, then reject $H_0$ and accept $H_1$, and vice versa. If $H_1$ is accepted, then remove the model with $\widehat \rho_{1,k}$, since it has a larger cost. For condition (ii) of Theorem \ref{thm:Sample_size_est}, estimate $
D_k^{(b)} = C_{k-1}((\widehat\rho_{1,k}^{(b)})^2-(\widehat \rho_{1,k+1}^{(b)})^2)-C_k((\widehat\rho_{1,k-1}^{(b)})^2-(\widehat \rho_{1,k}^{(b)})^2)$, and compute its corresponding $1-\alpha$ confidence intervals. If the lower bound of the confidence interval is positive, then the second condition (ii) holds. Finally, accept the confidence interval with $\widehat \rho_{1,k+1}$ for $k\ge 1$.
\newline 

\noindent {\bf Transformation based method}
For condition (i) of Theorem \ref{thm:Sample_size_est}, Under normality of the transformed variable $z_k$ and $z_{k+1}$ as in \eqref{eq:Fisher_z}, we first observe that tanh function is increasing with argument range from 0 to 1, therefore consider 
\[
\left|\widehat \rho_{1,k}\right| = \text{tanh}(|z_k|), \quad D = \left|\widehat \rho_{1,k}\right|-\left|\widehat \rho_{1,k+1}\right|=\text{tanh}(|z_k|)-\text{tanh}(|z_{k+1}|)
\]
Consider Delta approximation (first order Taylor approximation) to D, we yield
\[
D\approx\frac{\partial D}{\partial Z_k}(z_k-\mu_{z_k})+\frac{\partial D}{\partial Z_{k+1}}(z_{k+1}-\mu_{z_{k+1}}),
\]
where $\frac{\partial D}{\partial Z_k} = \text{sign}(z_k)4e^{2|z_k|}/(e^{2|z_k|}+1)^2$ and $\mu_{z_k}$ is the expected value.  And the variance of $D$ is
\[
\text{Var}(D) \approx \left(\frac{\partial D}{\partial z_k}\right)^2 \text{Var}(z_k)+\left(\frac{\partial D}{\partial z_{k+1}}\right)^2 \text{Var}(z_{k+1})+2 \frac{\partial D}{\partial z_k} \frac{\partial D}{\partial z_{k+1}}\text{Cov}(z_k,z_{k+1})=\frac{1.03^2}{Q-3}\left[\frac{16e^{4|z_k|}}{(e^{2|z_k|}+1)^4} +\frac{16e^{4|z_{k+1}|}}{(e^{2|z_{k+1}|}+1)^4}\right]
\]
Note that since $z_k$ and $z_{k+1}$ are independent, then the covariance term becomes 0 and after the Fisher transformation and its exact variance is known as $\sigma_{z_k} = 1.03/\sqrt{Q - 3}$.

We will consider t test (Student t-test) to determine whether there is a significant difference between the means of two groups of samples using small sample size. If the sample size is lagre ($>30$), then t distribution degenerates to the normal distribution. Especially for our case, to determine if  $| \rho_{1,k}|$ and $|\rho_{1,k+1}|$ have a significant difference  with overlapping confidence intervals. The assumption for t test requires at least approximately normal distribution for sample means, this does not hold since the transformed statistic $D$ is in general not approximately normal, unless $Q>30$. Consider independent two-sample t-test statistic
% We consider independent two-sample t-test. 



\[
t = \frac{D}{\sqrt{\text{Var}(D)}} \sim t_\nu, \quad \nu = Q-3
\]
where $t_\nu$ is a t distribution  with degrees of freedom of $\nu = Q-3$. If $t>t_{\alpha,\nu}$, where $t_{\alpha,\nu}$ is a value in the t distribution table,  we reject $H_0$, we think $\widehat \rho_{1,k}$ is obvious larger than $\widehat \rho_{1,k+1}$. Otherwise, $|\widehat \rho_{1,k}|$ is very close to $|\widehat \rho_{1,k+1}|$, in this scenario, we will discard the model with $|\widehat \rho_{1,k}|$ and keep the model with $|\widehat \rho_{1,k+1}|$.

% If sample variances are similar in the sense that $0.5<\sigma_{z_k}^2/\sigma_{z_{k+1}}^2<2$, the corresponding $t$ statistic is
% \[
% t_k=\frac{D}{s_p\sqrt{\frac{1}{Q_1}+\frac{1}{Q_2}}}, \quad s_p = \sqrt{\frac{(Q_1 - 1)\sigma_{z_k}^2 + (Q_2 - 1)\sigma_{z_{k+1}}^2}{Q_1+Q_2-2}}
% \]
% Otherwise,

% \[
% t_k = \frac{\Delta z_k}{\sqrt{\frac{\sigma_{z_k}^2}{Q_1} + \frac{\sigma_{z_{k+1}}^2}{Q_2}}}
% \]







For condition (ii) of Theorem \ref{thm:Sample_size_est}, consider
\[
E = -C_{k}\widehat \rho_{1,k-1}^2+(C_{k-1} + C_k)\widehat\rho_{1,k}^2  - C_{k-1}\widehat\rho_{1,k+1}^2 =-C_{k}\text{tanh}^2(z_{k-1}) + (C_{k-1} + C_k)\text{tanh}^2(z_{k})-C_{k-1}\text{tanh}^2(z_{k+1})
\]
we want to check if $E>0$ holds or not. 
%
\begin{align*}
    E &\approx \sum_{i=k-1}^{k+1}\frac{\partial E}{\partial z_i}(z_i-\mu_{z_i})
\end{align*}
%
where
\begin{equation*}
    \frac{\partial E}{\partial z_i}= \left\{\begin{array}{ll}
-2C_k\text{tanh}(z_{k-1})\text{sech}^2(z_{k-1}), & i=k-1,\\
2(C_{k-1} + C_k)\text{tanh}(z_{k})\text{sech}^2(z_{k}), & i=k,\\
- 2C_{k-1}\text{tanh}(z_{k+1})\text{sech}^2(z_{k+1}), & i=k+1.
\end{array}
\right.
\end{equation*}


with variance
\[
\text{Var}(E)\approx \sum_{i=k-1}^{k+1}\left(\frac{\partial E}{\partial z_i}\right)^2 \text{Var}(z_i) + 2\sum_{i<j}\frac{\partial E}{\partial z_i}\frac{\partial E}{\partial z_j}\text{Cov}(z_i,z_j)
\]
Note that since $z_{k-1}$, $z_k$ and $z_{k+1}$ are independent, then the covariance term becomes 0. Consider
\[
t=\frac{E}{\sqrt{\text{Var}(E)}} \sim t_\nu, \quad \nu = Q-3
\]
If $t>t_{\alpha,\nu}$, we accept accept $E>0$, namely the second condition (ii) holds.







% %
% \begin{equation*}\label{eq:Optimization_pb_model_selection}
%     \begin{array}{lll}
%     \displaystyle\min_{S^*} &\displaystyle \xi,\\
%        \text{s.t.} &\displaystyle |\rho_{1,1}|>\ldots>|\rho_{1,K^*}|,\\
%        &\displaystyle \frac{C_{i-1}}{C_i}>\frac{\rho_{1,i-1}^2-\rho_{1,i}^2}{\rho_{1,i}^2-\rho_{1,i+1}^2}, \quad i=1,\ldots,{K^*}, \quad \rho_{1,K^*+1}=0,\\
%     \end{array}
% \end{equation*}
% %

% \normalem
% \begin{algorithm}[!ht]
% \label{algo:MFMC_Algo_model_selection}
% \DontPrintSemicolon    
%    \KwIn{$K$ candidate models $\widehat  u_{h, k}$ with coefficients $\rho_{1,k}$, $\sigma_1$, $\sigma_k$ and cost per sample $C_k$.}\vspace{1ex}
    
%     \KwOut{Selected $K^*$ models $\widehat u_{h, i}$ in $\mathcal{S}^*$, with coefficients $\rho_{1,i}$, $\alpha_i$ and $C_i$ for each model $\widehat u_{h, i}$.}\vspace{1ex}
%     \hrule \vspace{1ex}

%    % Estimate $\rho_{1,k}$ and $C_k$ for each model $u_{h, k}$ using $N_0$ samples.
   
   
%    Sort $u_{h, k}$ by decreasing $\rho_{1,k}$ to create $\mathcal{S}=\{\widehat u_{h, k}\}_{k=1}^K$. 
   
%    Initialize $w^*=C_1$, $\mathcal{S}^*=\{\widehat u_{h, 1}\}$. Let $ \mathcal{\widehat S}$ be all $2^{K-1}$ ordered subsets of $\mathcal{S}$, each containing $\widehat u_{h, 1}$. 
%    % Set $ \mathcal{\widehat S}_1=\mathcal{S}^*$.

%     % $(2 \le j \le 2^{K-1})$
%     \For{each subset $\mathcal{\widehat S}_j$\,}{

%     {
%     \If{ condition $(ii)$ from Theorem \ref{thm:Sample_size_est} is satisfied}{
%     Compute the objective function value $w$ using \eqref{eq:MFMC_sampling_cost_efficiency}.
    
%     \If{$w<w^*$}{
%     {
%     Update $\mathcal{S}^* = \mathcal{\widehat S}_j$ and $w^* = w$.
%     }
%     } 
%     }
%     }
%     $j=j+1$.
%     }
%     Compute $\alpha_i$ for $\mathcal{S}^*$, $i=2,\dots, K^*$ by \eqref{eq:MFMC_coefficients}.
% \caption{Multi-fidelity Model Selection--\JLcolor{\cite[Algorithm~1]{PeWiGu:2016}}}
% \end{algorithm}
% \ULforem


\normalem
\begin{algorithm}[!ht]
\label{algo:enhanced_mfmc_selection}
\DontPrintSemicolon
\SetAlgoVlined
\SetKwProg{Fn}{Function}{}{}
\SetKwInOut{Input}{Input}
\SetKwInOut{Output}{Output}

\Input{%
Vectors of correlation coefficients $\boldsymbol{\rho}$, costs per sample $\boldsymbol{C}$, sample deviations $\boldsymbol{\sigma}$. 
}
\Output{%
  Set of selected models $\mathcal{S}^*=\{u_{h,i}\}_{i\in I^*}$, correlations of selected models $\boldsymbol{\rho}^*$, costs of selected models $\boldsymbol{C}^*$, minimal cost efficiency ratio $\xi_{\text{min}}$, weights $\alpha_i^*$.
}
\hrule
 
\Fn{[idx\textunderscore for\textunderscore model, $\xi_{\text{min}}$] = Model\textunderscore Selection\textunderscore Backtrack ($\boldsymbol{\rho}, \boldsymbol{C}$)}{
Sort the correlation coefficients by non-increasing $|\rho_{1,k}|$ with order $r$. Relabel $\rho_{1,k}, C_k$ for all $k$ as $\boldsymbol{\rho}, \boldsymbol{C}$.


Initialization: 
current$\_$idx = 1, $\xi_{\text{min}}=1$, global$\_$idx = []. %$\boldsymbol{\rho}=[1]$, $\boldsymbol{C}=[C_1]$,


\vspace{3mm}
\textbf{Backtrack} $(\text{current}\_\text{idx},\, \xi_{\text{min}},\, 2)$. 

idx\textunderscore for\textunderscore model = r(global$\_$idx).
\vspace{3mm}

\Fn{ $[\mathcal{S}^*,\, \boldsymbol{\rho}^*, \,\boldsymbol{C}^*, \xi_{\text{min}}]$ = \textbf{Backtrack} $\left(\text{current}\_\text{idx}, \, \xi, \,k_{\text{next}}\right)$}{


  \If{$\xi \leq \xi_{\text{min}}\,$ }{
    $\xi_{\text{min}}=\xi$.

    global$\_$idx = current$\_$idx.
  }
  % \Else {
  %   $\mathcal{S}^* = \mathcal{S}$, $\boldsymbol{\rho}^* = \boldsymbol{\rho}$, $\boldsymbol{C}^* = \boldsymbol{C}$, $\xi_{\text{min}}=\xi$.
  % }
  
  \If{$k_{\text{next}} > K$}{ 
    \Return
  }
  
  \For{$k = k_{\text{next}}$ \textbf{to} $K$}{ 
     % $\rho_{1,\text{last}} = \boldsymbol{\rho}_{\text{end}}$, $C_{\text{last}} = \boldsymbol{C}_{\text{end}}$.
     previous\textunderscore idx = current$\_$idx (end).

     % $\rho_k = \boldsymbol{\rho}(k), C_k = \boldsymbol{C}(k)$

     % $\rho_{\text{next}}=0$

     
    \If{% 
      $\frac{\boldsymbol{C}({\text{previous}\_\text{idx}})}{\boldsymbol{C}(k)} > \frac{\boldsymbol{\rho}({\text{previous}\_\text{idx}})^2 - \boldsymbol{\rho}(k)^2}{\boldsymbol{\rho}(k)^2}$ 
    }{
        Continue to next iteration.
    }

        % $\rho_k\_$vec = [$\boldsymbol{\rho}$(cur$\_$ind), $\rho_k$]
        
        Compute $\xi$ via \eqref{eq:MFMC_sampling_cost_efficiency} for indices  
      [current\textunderscore idx, $k$].

      \If{$\xi\ge \xi_{\text{min}}$ or $\,\xi>1$}{ 
    Continue to next iteration.
    }
      
      \textbf{Backtrack} $(\, [\text{current}\_\text{idx},k],\xi, k+1)$.
  }
}
}
\vspace{3mm} 


 
$I^* = \text{idx}\_\text{for}\_\text{model}, K^* = |I^*|, \boldsymbol{\rho}^* = \boldsymbol{\rho} (I^*)$, $\boldsymbol{C}^* = \boldsymbol{C} (I^*)$, $\boldsymbol{\sigma}^* = \boldsymbol{\sigma} (I^*)$.

Selected models $\mathcal{S}^* = \{u_{h,k}\}_{k\in \mathcal{I^*}}$ with weights $\alpha_i^*$ for $i=2,...,K^*$ via \eqref{eq:MFMC_SampleSize}.




\caption{Multi-fidelity Model Selection with Backtracking Pruning}
\end{algorithm}
\ULforem




\normalem
\begin{algorithm}[!ht]
\label{algo:MFMC_Algo}
\DontPrintSemicolon

    
   \KwIn{Selected $K^*$ models $u_{h, k}$ in $\mathcal{S}^*$, parameters $\rho_{1,k}$, $\alpha_k$ and $C_k$ for each $u_{h, k}$,  tolerance $\epsilon$. }\vspace{1ex}
    
    \KwOut{Sample sizes $N_k$ for $K^*$ models, expectation 
    estimate $A^{\text{MF}}$.}\vspace{1ex}
    \hrule \vspace{1ex}
    

    Compute the sample size $N_k$ for $1\leq k\leq K^*$ by \eqref{eq:MFMC_SampleSize} and generate i.i.d. $N_1$ and $N_k-N_{k-1}$ samples for $k=2,\ldots, K^*$.

    Evaluate $u_{h, 1}$ to obtain $u_{h, 1}(\boldsymbol{\omega}^i)$ for $i = 1,\ldots,N_1$ and compute $A_{1,N_1}^{\text{MC}}$ by \eqref{eq:MC_estimator}.
    
    \For{$k = 2,\ldots,K^* $\,}{

    Evaluate $u_{h, k}$ to obtain $u_{h, k}(\boldsymbol{\omega}^i)$ for $i = 1,\ldots,N_{k-1}$ and compute $A_{k,N_{k-1}}^{\text{MC}}$ by \eqref{eq:MC_estimator}.

    Evaluate $u_{h, k}$ to obtain $u_{h, k}(\boldsymbol{\omega}^i)$ for $i = 1,\ldots,N_k-N_{k-1}$ and compute $A_{k,N_k\backslash N_{k-1}}^{\text{MC}}$ by \eqref{eq:MC_estimator}.

    % Store $N_{k-1}$ and $N_{k}-N_{k-1}$ samples as $N_k$ samples.
    }

    Compute $A^{\text{MF}}$ by \eqref{eq:MFMC_estimator_independent}.
    
\caption{Multifidelity Monte Carlo}
\end{algorithm}
\ULforem

% \normalem
% \begin{algorithm}[!ht]
% \label{algo:MFMC_Algo}
% \DontPrintSemicolon

    
%    \KwIn{Models $f_k$ in $\mathcal{S}^*$, parameters $\rho_k$, $\alpha_k$ and $C_k$ for each $f_k$ in $\mathcal{S}^*$,  tolerance $\epsilon$. }\vspace{1ex}
    
%     \KwOut{Sample sizes $N_k$ for $K^*$ models, expectation 
%     estimate $A^{\text{MFMC}}$.}\vspace{1ex}
%     \hrule \vspace{1ex}
%     Compute initial sample sizes $\boldsymbol{N}=[N_1,\ldots, N_{K^*}]$ using \eqref{eq:MFMC_SampleSize}. Set $\boldsymbol{N}_{\text{old}} = \boldsymbol{0}$ and $\boldsymbol{dN} = \boldsymbol{N}$. 
    
%     Initialize sample means $A_{1,N_1}^{\text{MC}}, A_{k,N_{k-1}}^{\text{MC}}, A_{k,N_k\backslash N_{k-1}}^{\text{MC}}=0. $
    
%     \While{$\sum_k dN_k>0$\,}{

%     Evaluate $dN_{1}$ samples for $f_1$ to obtain $f_1(\boldsymbol{\omega}^i)$. Update $A_{1,N_1}^{\text{MC}} = \frac{\boldsymbol{N}_{\text{old}}^1 A_{1,N_1}^{\text{MC}}+\sum_i f_1(\boldsymbol{\omega}^i)}{\boldsymbol{N}_{\text{old}}^1+dN_1}$ and $\sigma_1$.

%     Store $dN_1$ samples.
    
%     \For{$2\le k\le K^*$\,}{
    
%         % \For{$i = 1,\ldots,dN_k $\,}
%     % {
%     Evaluate previously stored $dN_{k-1}$ samples for $f_k$ to obtain $f_k(\boldsymbol{\omega}^i)$. Update $A_{k,N_{k-1}}^{\text{MC}} = \frac{\boldsymbol{N}_{\text{old}}^k A_{k,N_{k-1}}^{\text{MC}}+\sum_i f_k(\boldsymbol{\omega}^i)}{\boldsymbol{N}_{\text{old}}^k+dN_{k-1}}$. 
    
%     Collect new $dN_{k}-dN_{k-1}$ samples. Evaluate $f_k$ to obtain $f_k(\boldsymbol{\omega}^i)$. Update $A_{k,N_k\backslash N_{k-1}}^{\text{MC}} = \frac{\boldsymbol{N}_{\text{old}}^k A_{k,N_k\backslash N_{k-1}}^{\text{MC}}+\sum_i f_k(\boldsymbol{\omega}^i)}{\boldsymbol{N}_{\text{old}}^k+dN_{k}-dN_{k-1}}$. 

    
%     Compute $\sigma_k, \rho_{1,k}$.

%     Store $dN_{k-1}$ and $dN_{k}-dN_{k-1}$ samples as $dN_k$ samples.
    
%     \If{Condition (i) \& (ii) in Theorem \ref{thm:Sample_size_est} is not satisfied \,}{
%     Reselect models via Algorithm \ref{algo:MFMC_Algo_model_selection} with a larger sample size and restart.

%     Break. 
%     }

%     }
    
    
    
%     \vspace{4mm}
%     $\boldsymbol{N}_{\text{old}} \leftarrow \boldsymbol{N}$
    
%     Update $\alpha_k$ and the sample size $\boldsymbol{N}$ by \eqref{eq:MFMC_coefficients} 
%  and \eqref{eq:MFMC_SampleSize}.

%     $\boldsymbol{dN} \leftarrow \max \left\{\boldsymbol N-\boldsymbol N_{\text{old}}, \boldsymbol{0}\right\}.$

    
%     }
%     Compute $A^{\text{MFMC}}$ using $A_{1,N_1}^{\text{MC}}, A_{k,N_{k-1}}^{\text{MC}}, A_{k,N_k\backslash N_{k-1}}^{\text{MC}}$ and $\alpha_k$ from step 4, 7, 8, 15, by \eqref{eq:MFMC_estimator_independent}.
% \caption{Multi-fidelity Monte Carlo}
% \end{algorithm}
% \ULforem


% \begin{theorem}
% \label{thm:Sampling_cost_est}
% Let $f_k$ be $K$ models that satisfy the following conditions
% %
% \begin{alignat*}{8}
%     &(i)\;\; |\rho_{1,1}|>\ldots>|\rho_{1,K}|& \qquad \qquad
%     &(ii)\;\; \frac{C_{k-1}}{C_k}>\frac{\rho_{1,k-1}^2-\rho_{1,k}^2}{\rho_{1,k}^2-\rho_{1,k+1}^2},\;\;k=2,\ldots,K.
% \end{alignat*}
% %
% Suppose there exists $0<s<q<1$ such that 
% $C_k = c_s s^{k}$, $\rho_{1,k}^2 = q^{ k-1}$, then 
% \begin{equation*}
%     \mathcal{W}_\text{MFMC} = 
% \end{equation*}

% \end{theorem}
% \begin{proof}
% Since $q>s$, condition (ii) is satisfied.
% \begin{align*}
% \rho_{1,k}^2 - \rho_{1,k+1}^2&=q^k\left(\frac1 q-1\right),\quad \rho_{1,k-1}^2 - \rho_{1,k}^2=q^k\frac 1 q\left(\frac1 q-1\right)\\
%     \mathcal{W}_\text{MFMC} &= \frac{\sigma_1^2}{\left\Vert\mathbb{E}(f_1) \right\Vert_{Z}^2\epsilon^2}\sum_{k=1}^K\sqrt{\left(\rho_{1,k}^2 - \rho_{1,k+1}^2\right)C_k}\sum_{k=1}^K\left(\sqrt{\frac{C_k}{\rho_{1,k}^2 - \rho_{1,k+1}^2}} - \sqrt{\frac{C_{k-1}}{\rho_{1,{k-1}}^2 - \rho_{1,k}^2}}\right)\rho_{1,k}^2,\\
%     &=\frac{\sigma_1^2}{\left\Vert\mathbb{E}(f_1) \right\Vert_{Z}^2\epsilon^2} \sum_{k=1}^K\sqrt{q^{k}s^{ k}}\left(\sqrt{\frac{s(1-q)}{1-\rho_{1,2}^2}} + \sum_{k=2}^K\left(\sqrt{\frac{s^{k}}{q^{ k}}} - \sqrt{\frac{q s^{ k}}{s q^{ k}}}\right)q^{k} + \left(\sqrt{\frac{s^{ K}(1-q)}{q^{K}}}-\sqrt{\frac{q s^{ K}}{s q^{K}}}\right)q^{K}\right)\\
%     &\propto \frac{1}{\epsilon^2} \sum_{k=1}^K\left(q^{\frac{1}{2}}s^{\frac{1}{2}}\right)^k
% \end{align*}
    
% \end{proof}


%
\subsection{Confidence interval for sample size and total cost}
Given the confidence intervals $\text{CI}_{\rho_{1,k}}$ for the correlation coefficients as defined in \eqref{eq:Confidence_Interval_rho}, we aim to derive the corresponding confidence intervals $\text{CI}_{N_k}$ for the required sample sizes in the multi-fidelity Monte Carlo estimator. 

% This involves solving an optimization problem that minimizes the total computational cost while satisfying an upper bound on the estimator variance and enforcing monotonicity constraints on the sample sizes. 
%
\begin{theorem}
\label{thm:Sample_size_est_conf_interval} 
Let $\text{CI}_{\rho_{1,k}}$ be the confidence interval for the estimated correlation coefficient as defined in \eqref{eq:Confidence_Interval_rho} with pilot sample size $Q$, and let $N_k$ denote the optimal sample size from \eqref{thm:Sample_size_est}. Let
$\Delta_k = \widehat \rho_{1,k}^2 - \widehat\rho_{1,k+1}^2.$ Let the confidence interval for $\Delta_k$ be $\left[\Delta_k^{\text{lower}},\Delta_k^{\text{upper}}\right]$

Define the lower and upper bounds for the sample size
%
\begin{equation}\label{eq:upper_lower_N_k}
    N_k^{\text{lower}} = \frac{\sigma_1^2}{\epsilon_{\text{tar}}^2}\sqrt{\frac{\Delta_k^{\text{lower}}}{C_k}}\sum_{j=1}^K\sqrt{C_j\Delta_j^{\text{lower}}},\quad N_k^{\text{upper}} = \frac{\sigma_1^2}{\epsilon_{\text{tar}}^2}\sqrt{\frac{\Delta_k^{\text{upper}}}{C_k}}\sum_{j=1}^K\sqrt{C_j\Delta_j^{\text{upper}}}
\end{equation}


%
Then the confidence interval for $N_k$ is
%
\[
\text{CI}_{N_k} = N_k\left(\widehat\rho_{1,k}\right)\pm t_{\alpha/2,Q-3}\sqrt{\text{Var}\left[N_k\left(\widehat\rho_{1,k}\right)\right]}\cap \left[N_k^{\text{lower}}, N_k^{\text{upper}}\right].
\]
%
\end{theorem}



Finally, this range induces a confidence interval for the cost-efficiency metric $\xi$ introduced in \eqref{eq:MFMC_sampling_cost}, with bounds given by TOBEFINISHED.
%
\begin{equation}\label{eq:MFMC_sampling_cost_efficiency_CI}
    \xi \in  \left[\sum_{k=1}^K C_k N_k^{\text{low}},\;\; \right]\left[\frac{1}{C_1} \left(\sum_{j=1}^K\sqrt{C_j\Delta_j^{\text{lower}}}\right)^2,\frac{1}{C_1} \left(\sum_{j=1}^K\sqrt{C_j\Delta_j^{\text{upper}}}\right)^2\right].
\end{equation}
%


% ========================================
\section{Fix interval parameter estimation}\label{sec:Fix_interval_parameter_est}
% ========================================
Theorem \eqref{thm:Sample_size_est} provide an estimate for the weights and sample size assuming we know the parameters exactly. However, in reality, we dont know the exact real value and what we know is an interval for which the parameters lie in. We consider a fixed interval for each parameter $\rho_{1,k}\in \text{CI}_{\rho_{1,k}}=[b_k,d_k]$. We want to derive similar results compared to Theorem \eqref{thm:Sample_size_est} for interval-based parameters.

%
\begin{equation}\label{eq:Optimization_pb_sample_size2}
    \begin{array}{ll}
    \min \limits_{\begin{array}{c}\scriptstyle N_1,\ldots, N_K\in \mathbb{R} \\[-4pt]
\scriptstyle \alpha_2,\ldots,\alpha_K\in \mathbb{R}
\end{array}} &\displaystyle\sum\limits_{k=1}^K C_kN_k,\\
       \;\,\text{subject to} &\mathbb{V}\left[A^{\text{MF}}\right]\le \epsilon_{\text{tar}}^2, \quad \forall \rho_{1,k} \in [b_k,d_k]\\[2pt]
       &\displaystyle -N_1\le 0,\quad \displaystyle N_{k-1}-N_k\le 0, \;\; k=2\ldots,K.
    \end{array}
\end{equation}
%

% From \eqref{eq:partial_L_alpha_k}, the interval of $\alpha_k$ is 
% \[
% \alpha_k\in \left[\frac{a_k\sigma_1}{\sigma_k}, \frac{b_k\sigma_1}{\sigma_k}\right].
% \]
% Let $G_k =\alpha_k^2\sigma_k^2-2\alpha_k\rho_{1,k}\sigma_1\sigma_k$. It is a quadratic function in terms of $\alpha_k$. $\partial G_k/\partial \alpha_k = 2\alpha_k\sigma_k^2 - 2\rho_{1,k}\sigma_1\sigma_k$.


% \[
% G_k^{\text{min}}(\alpha_k) = -\rho_{1,k}^2\sigma_1^2,\quad
% G_k^{\text{max}}(\alpha_k) =\left\{\begin{array}{ll}
% \left(b_k^2-2a_kb_k\right)\sigma_1^2, & 0\le a_k\le b_k,\;\; \text{ or } \;\;a_k\le 0, b_k\ge 0, a_k+b_k\le 0,\\
% \left(a_k^2-2a_kb_k\right)\sigma_1^2, &a_k\le b_k\le 0,\;\; \text{ or } \;\;a_k\le 0, b_k\ge 0, a_k+b_k\ge 0.
% \end{array}
% \right.
% \]

% Define $\Delta_k = G_{k+1} - G_k$, and estimate the following upper and lower bounds for $\Delta_k$,
% \begin{align*}
% \Delta_k^{\text{lower}}&:=G^{\text{min}}_{k+1}(\alpha_{k+1}) - G^{\text{max}}_{k}(\alpha_k)\\
% &=\left\{\begin{array}{ll}
% \left[-\rho_{1,k+1}^2-\left(b_k^2-2a_kb_k\right)\right]\sigma_1^2, & 0\le a_k\le b_k,\;\; \text{ or } \;\;a_k\le 0, b_k\ge 0, a_k+b_k\le 0,\\
% \left[-\rho_{1,k+1}^2-\left(a_k^2-2a_kb_k\right)\right]\sigma_1^2, &a_k\le b_k\le 0,\;\; \text{ or } \;\;a_k\le 0, b_k\ge 0, a_k+b_k\ge 0.
% \end{array}
% \right.\\
% &=\left\{\begin{array}{ll}
% \left[-\max\left\{a_{k+1}^2,b_{k+1}^2\right\}-\left(b_k^2-2a_kb_k\right)\right]\sigma_1^2, & 0\le a_k\le b_k,\;\; \text{ or } \;\;a_k\le 0, b_k\ge 0, a_k+b_k\le 0,\\
% \left[-\max\left\{a_{k+1}^2,b_{k+1}^2\right\}-\left(a_k^2-2a_kb_k\right)\right]\sigma_1^2, &a_k\le b_k\le 0,\;\; \text{ or } \;\;a_k\le 0, b_k\ge 0, a_k+b_k\ge 0.
% \end{array}
% \right.
% \end{align*}



% \begin{align*}
%  \Delta_k^{\text{upper}}&:= G^{\text{max}}_{k+1}(\alpha_{k+1}) - G^{\text{min}}_{k}(\alpha_k)\\
%  &=\left\{\begin{array}{ll}
% \left[b_{k+1}^2-2a_{k+1}b_{k+1}+\rho_{1,k}^2\right]\sigma_1^2, & 0\le a_{k+1}\le b_{k+1},\;\; \text{ or } \;\;a_{k+1}\le 0, b_{k+1}\ge 0, a_{k+1}+b_{k+1}\le 0,\\
% \left[a_{k+1}^2-2a_{k+1}b_{k+1}+\rho_{1,k}^2\right]\sigma_1^2, &a_{k+1}\le b_{k+1}\le 0,\;\; \text{ or } \;\;a_{k+1}\le 0, b_{k+1}\ge 0, a_{k+1}+b_{k+1}\ge 0.
% \end{array}
% \right.\\
% &=\left\{\begin{array}{ll}
% \left[b_{k+1}^2-2a_{k+1}b_{k+1}+\max\left\{a_{k}^2,b_{k}^2\right\}\right]\sigma_1^2, & 0\le a_{k+1}\le b_{k+1},\;\; \text{ or } \;\;a_{k+1}\le 0, b_{k+1}\ge 0, a_{k+1}+b_{k+1}\le 0,\\
% \left[a_{k+1}^2-2a_{k+1}b_{k+1}+\max\left\{a_{k}^2,b_{k}^2\right\}\right]\sigma_1^2, &a_{k+1}\le b_{k+1}\le 0,\;\; \text{ or } \;\;a_{k+1}\le 0, b_{k+1}\ge 0, a_{k+1}+b_{k+1}\ge 0.
% \end{array}
% \right.
% \end{align*}

% Note that $\widetilde{\Delta}_k^{\text{lower}}$ can be positive, negative or equal to zero, due to the overlapping of the interval of the parameters, but $\Delta_k^{\text{upper}}$ is always positive. Define $\widetilde{\Delta}_k^{\text{lower}}\le \Delta_k \le \Delta_k^{\text{upper}}$, where
% \[
% \widetilde{\Delta}_k^{\text{lower}}:=\Delta_i^{\text{lower}}, \quad i=\max\left\{i\le k\;\;\vert\;\; \Delta_i^{\text{lower}}>0\right\}.
% \]
% Note that $\max\{i\le k\vert \Delta_i^{\text{lower}}>0\}$ cannot be empty since $N_1$ is not equal to zero.

% From $\partial L/\partial N_k=0$, 
% \[
% N_k = \sqrt{\lambda_0}\sqrt{\frac{\Delta_k}{C_k}}
% \]

% The variance 
% \[
% \mathbb{V}\left[A^{\text{MF}}\right] = \sum_{k=1}^K\frac{G_{k+1} - G_k}{N_k} = \sum_{k=1}^K\frac{\Delta_k}{N_k} =\frac{1}{\sqrt{\lambda_0}}\sum_{k=1}^K \sqrt{C_k\Delta_k}= \epsilon_{\text{tar}}^2
% \]
% with $G_{K+1} = 0$.
% \[
% \sqrt{\lambda_0} = \frac{1}{\epsilon_{\text{tar}}^2}\sum_{k=1}^K \sqrt{C_k\Delta_k}
% \]

% \[
% N_k = \frac{1}{\epsilon_{\text{tar}}^2}\sqrt{\frac{\Delta_k}{C_k}}\sum_{j=1}^K \sqrt{C_j\Delta_j}.
% \]

% \[
% N_k \in \left[\frac{1}{\epsilon_{\text{tar}}^2}\sqrt{\frac{\widetilde{\Delta}_k^{\text{lower}}}{C_k}}\sum_{j=1}^K \sqrt{C_j\widetilde{\Delta}_k^{\text{lower}}},\frac{1}{\epsilon_{\text{tar}}^2}\sqrt{\frac{\Delta_k^{\text{upper}}}{C_k}}\sum_{j=1}^K \sqrt{C_j\Delta_j^{\text{upper}}}\right]
% \]



As shown in \eqref{eq:partial_var_rho}, since $S^\prime<0$ in \eqref{eq:S_n_S_prime}, the variance $\mathbb{V}(A^{\text{MF}})$ is monotonically decreasing functions of the correlation coefficients $\rho_{1,k}$. To ensure the variance always falls below the tolerance, we therefore require $\mathbb{V}\left[A^{\text{MF}}\right]$ evaluate at the lower bound $b_k$ of the intervals be equal to $\epsilon_{\text{tar}}^2$, namely
%
\[
\frac{\sigma_1^2}{N_1} + \sum_{k=2}^K \left(\frac{1}{N_{k-1}} - \frac{1}{N_k}\right)\left(\alpha_k^2\sigma_k^2 - 2\alpha_kb_{k}\sigma_1\sigma_k\right) = \epsilon_{\text{tar}}^2
\]
%
The sample sizes that holds for all $\rho_{1,k}\in [b_k,d_k]$ is then
%
\[
N_k = \frac{\sigma_1^2}{\epsilon_{\text{tar}}^2}\sqrt{\frac{b_k^2-b_{k+1}^2}{C_k}}\sum_{j=1}^K\sqrt{C_j\left(b_j^2-b_{j+1}^2\right)},\quad \text{for}\quad  k=1,\ldots,K.
\]
% %
% At the upper endpoint $\text{CI}_{\rho_{1,k}}^{\text{U}}$, the variance constraint is strictly satisfied, and the corresponding Lagrange multiplier $\lambda_0$ vanishes due to complementary slackness. The optimization problem then reduces to minimizing the cost subject only to the non-decreasing sample sizes
% %
% \[
% \min \limits_{\begin{array}{c}\scriptstyle N_1,\ldots, N_K\in \mathbb{R}\end{array}}\sum\limits_{k=1}^K C_kN_k, \quad \text{subject to} \quad  0\le N_1\le \cdots \le N_K.
% \]
% %
% While the trivial solution $N_k^{\text{min}} = 0$ satisfies these constraints, it does not produce a meaningful lower bound. To obtain a more informative characterization of the uncertainty in $N_k$, we linearize $N_k$ with respect to the estimated correlation $\rho_{1,k}^{(Q)}$ using a first-order Taylor expansion
% %
% \[
%  N_k\left(\rho_{1,k}^{(Q)}\right)\approx N_k^*+ \frac{\partial N_k}{\partial \rho_{1,k}} \left( \rho_{1,k}^{(Q)}-\rho_{1,k}\right).
% \]
% %
% The variance of this linearized estimate is approximated by
% %
% \begin{align*}
%     \text{Var}\left(N_k\left(\rho_{1,k}^{(Q)}\right)\right) &\approx \left(\frac{\partial N_k}{\partial \rho_{1,k}}\Bigg |_{\rho_{1,k} = \rho_{1,k}^{(Q)}} \right)^2 \cdot \text{Var}\left(\rho_{1,k}^{(Q)}\right) \approx \left(\frac{\partial N_k}{\partial \rho_{1,k}}\Bigg |_{\rho_{1,k} = \rho_{1,k}^{(Q)}} \right)^2 \cdot \left(\frac{\partial \text{tanh}(z)}{\partial z}\right)^2\text{Var}(z),\\
%     &= \left(\frac{\partial N_k}{\partial \rho_{1,k}}\Bigg |_{\rho_{1,k} = \rho_{1,k}^{(Q)}} \right)^2 \cdot \left(1-\left(\rho_{1,k}^{(Q)}\right)^2\right)^2\frac{1.03^2}{Q-3}.
% \end{align*}
% %
% This leads to a confidence interval for the linearized sample size estimate
% %
% \[
% \text{CI}_{N_k} := \left[\text{CI}_{N_k}^{\text{L}},\text{CI}_{N_k}^{\text{U}}\right]=\left[N_k^*-1.96\sqrt{\text{Var}\left(N_k\left(\rho_{1,k}^{(Q)}\right)\right)}, N_k^*+1.96\sqrt{\text{Var}\left(N_k\left(\rho_{1,k}^{(Q)}\right)\right)}\right].
% \]
% %
% which we then intersect with the upper bound $N_k^{\text{max}}$ to obtain a truncated and feasible confidence interval
% %
% \[
% \text{CI}_{N_k} = \left[\text{CI}_{N_k}^{\text{L}},\text{CI}_{N_k}^{\text{U}}\right]\cap \left[0, N_k^{\text{max}}\right].
% \]
% %


% Finally, this range induces a confidence interval for the cost-efficiency metric $\xi$ introduced in \eqref{eq:MFMC_sampling_cost}, with bounds given by
% %
% \begin{equation}\label{eq:MFMC_sampling_cost_efficiency_CI}
%     \xi \in  \left[\sum_{k=1}^K C_k \text{CI}_{N_k}^{\text{L}},\;\;\min \left\{\sum_{k=1}^K C_k \text{CI}_{N_k}^{\text{U}},\frac{1}{C_1} \left(\sum_{k=1}^K\sqrt{C_k\left(a_{k}^2 - a_{k+1}^2\right)}\right)^2\right\}\right].
% \end{equation}
% %

% ========================================
\section{Numerical experiments}\label{sec:Num-Exp}
% ========================================
%
We present numerical results for Monte Carlo and multi-fidelity Monte Carlo sampling methods to estimate $\mathbb{E}(u)$. 



Consider the stochastic elliptic PDE on a 2D domain $D=[0,1]^2$
\[
-\nabla\cdot \left(a(x,\omega)\nabla u(x,\omega)\right) =f(x), \quad x\in D,
\]

with homogeneous Dirichlet boundary conditions, $f(x)=1$ and a random diffusion coefficient
\[
a(x,\omega) = \exp \left(Y(x,\omega)\right),
\]
where $Y(x,\omega)$ us a mean-zero Gaussian random field with a squared exponential covariance kernel.







 














   


















\section{Acknowledgment}\label{sec:Acknowledgment}



This work was supported in part by the Big-Data Private-Cloud Research Cyberinfrastructure MRI-award funded by NSF under grant CNS-1338099 and by Rice University's Center for Research Computing.

Jiaxing Liang was partially supported by AFOSR grant FA9550-22-1-0004. 
% ====================================================
\section{Appendix}\label{sec:Appendix}
% ====================================================
\subsection{More lemmas and theorems}
\begin{lemma}\label{lemma:Y_k_Y_j}
Let $2\le k<j\le K$. Then the correction terms $Y_k$ and $Y_j$ defined in \eqref{eq:MFMC_Yk} are uncorrelated; that is,
\begin{equation*}
    \operatorname{Cov} \left[Y_k,Y_j\right]=0.
\end{equation*}
\end{lemma}

\begin{proof}
Fix indices $2\le k<j\le K$. Since $N_{k-1}\le N_k\le N_{j-1}$, we can partition the $N_{j-1}$ samples into three mutually disjoint subsets with sizes $N_{k-1}$, $N_{k}-N_{k-1}$, and $N_{j-1} - N_{k}$. The samples in these three subsets are mutually independent. From the definition of $Y_k$ in \eqref{eq:MFMC_Yk},  the covariance between $Y_k$ and $Y_j$ is then given by
\begin{align*}
    \operatorname{Cov}\left[Y_k,Y_j\right] &= \left(\frac{N_{k-1}}{N_k}-1\right) \left(\frac{N_{j-1}}{N_j}-1\right)\operatorname{Cov}\left[A_{k, N_{k-1}}^{\text{MC}} - A_{k,N_{k}\backslash N_{k-1}}^{\text{MC}}, A_{j,N_{j-1}}^{\text{MC}} - A_{j,N_{j}\backslash N_{j-1}}^{\text{MC}}\right] \\
    & = M \left(\operatorname{Cov}\left[A_{k,N_{k-1}}^{\text{MC}} - A_{k,N_{k}\backslash N_{k-1}}^{\text{MC}}, A_{j,N_{j-1}}^{\text{MC}}\right] - \operatorname{Cov}\left[A_{k,N_{k-1}}^{\text{MC}} - A_{k,N_{k}\backslash N_{k-1}}^{\text{MC}}, A_{j,N_{j}\backslash N_{j-1}}^{\text{MC}}\right] \right)\\
    & = M \operatorname{Cov}\left[A_{k,N_{k-1}}^{\text{MC}} - A_{k,N_{k}\backslash N_{k-1}}^{\text{MC}}, A_{j,N_{j-1}}^{\text{MC}}\right]
\end{align*}
where we define $M = (N_{k-1}/N_k-1) (N_{j-1}/N_j-1)$. The second term in the covariance vanishes due to independence between the samples used in $A_{j,N_{j}\backslash N_{j-1}}^{\text{MC}}$ and those in $Y_k$. Next, we express $A_{j,N_{j-1}}^{\text{MC}}$ as a weighted Monte Carlo estimator over the three disjoint sample subsets
%
\begin{equation*}
    A_{j,N_{j-1}}^{\text{MC}} = \frac{N_{k-1}}{N_{j-1}}A_{j,N_{k-1}}^{\text{MC}} + \frac{N_k - N_{k-1}}{N_{j-1}} A_{j,N_{k}\backslash N_{k-1}}^{\text{MC}} + \frac{N_{j-1} - N_k}{N_{j-1}} A_{j,N_{j-1}\backslash N_{k}}^{\text{MC}}.
\end{equation*}
%
Substituting this expansion into the covariance expression yields
%
\begin{align*}
    % \frac{\operatorname{Cov}\left[Y_k,Y_j\right]}{M} &= 
    &\operatorname{Cov}\left[A_{k,N_{k-1}}^{\text{MC}} - A_{k,N_{k}\backslash N_{k-1}}^{\text{MC}}, A_{j,N_{j-1}}^{\text{MC}}\right]
    % &= \operatorname{Cov}\left[A_{N_{k-1}}^k, \frac{N_{k-1}}{N_{j-1}}A_{N_{k-1}}^j + \frac{N_k - N_{k-1}}{N_{j-1}} A_{N_{k}\backslash N_{k-1}}^j + \frac{N_{j-1} - N_k}{N_{j-1}} A_{N_{j-1}\backslash N_{k}}^j\right] \\
    % &- \operatorname{Cov}\left[ A_{N_{k}\backslash N_{k-1}}^k, \frac{N_{k-1}}{N_{j-1}}A_{N_{k-1}}^j + \frac{N_k - N_{k-1}}{N_{j-1}} A_{N_{k}\backslash N_{k-1}}^j + \frac{N_{j-1} - N_k}{N_{j-1}} A_{N_{j-1}\backslash N_{k}}^j\right]\\
    =\operatorname{Cov}\left[A_{k,N_{k-1}}^{\text{MC}}, \frac{N_{k-1}}{N_{j-1}}A_{j,N_{k-1}}^{\text{MC}}\right]-\operatorname{Cov}\left[ A_{k,N_{k}\backslash N_{k-1}}^{\text{MC}}, \frac{N_k - N_{k-1}}{N_{j-1}} A_{j,N_{k}\backslash N_{k-1}}^{\text{MC}} \right]\\
    &=\frac{\operatorname{Cov}\left[N_{k-1}A_{k,N_{k-1}}^{\text{MC}}, N_{k-1} A_{j,N_{k-1}}^{\text{MC}}\right]}{N_{j-1}N_{k-1}}-\frac{\operatorname{Cov}\left[(N_k-N_{k-1}) A_{k,N_{k}\backslash N_{k-1}}^{\text{MC}}, (N_k - N_{k-1}) A_{j,N_{k}\backslash N_{k-1}}^{\text{MC}} \right]}{N_{j-1}(N_k-N_{k-1})}\\
    &=\frac{\operatorname{Cov}\left[\sum_{i=1}^{N_{k-1}}u_{k}^{(i)},\sum_{i=1}^{N_{k-1}}u_{j}^{(i)}\right]}{N_{j-1}N_{k-1}}
    -\frac{\operatorname{Cov}\left[\sum_{i=1}^{N_k-N_{k-1}}u_{k}^{(i)}, \sum_{i=1}^{N_k-N_{k-1}}u_{j}^{(i)}\right]}{N_{j-1}(N_k-N_{k-1})}\\
    &=\frac{\sum_{i=1}^{N_{k-1}}\operatorname{Cov}\left[u_{k}^{(i)},u_{j}^{(i)}\right]}{N_{j-1}N_{k-1}} -\frac{\sum_{i=1}^{N_k-N_{k-1}}\operatorname{Cov}\left[u_{k}^{(i)}, u_{j}^{(i)}\right]}{N_{j-1}(N_k-N_{k-1})}=\frac{N_{k-1}\rho_{k,j}\sigma_k\sigma_j}{N_{j-1}N_{k-1}}-\frac{(N_k-N_{k-1})\rho_{k,j}\sigma_k\sigma_j}{N_{j-1}(N_k-N_{k-1})}=0
\end{align*}
\end{proof}

\subsection{Proof of Theorem \ref{thm:Sample_size_est}}

\begin{proof}
The proof consists of three main parts: (A) verification of feasibility for the closed-form solution, (B) derivation of optimality through KKT conditions, and (C) proof of global optimality via convexity and cost comparison.

\noindent {\bf  Part A: Feasibility Verification}

First, we verify that the proposed solution $(\alpha_k^*, N_k^*)$ satisfies all constraints:
\begin{enumerate}
    \item \textit{Variance constraint}: $N_k^*$ satisfies the variance constraint ($\mathbb{V}[A^{\text{MF}}] = \epsilon_{\text{tar}}^2$).
    
    \item \textit{Monotonicity constraint}: Using assumption (ii), we show strict increase in sample sizes:
    \begin{align*}
    \frac{N_k^*}{N_{k-1}^*} &= \sqrt{ \frac{(\rho_{1,k}^2 - \rho_{1,k+1}^2)/C_k}{(\rho_{1,k-1}^2 - \rho_{1,k}^2)/C_{k-1}} } = \sqrt{ \frac{C_{k-1}}{C_k} \cdot \frac{\rho_{1,k}^2 - \rho_{1,k+1}^2}{\rho_{1,k-1}^2 - \rho_{1,k}^2} } > 1,
    \end{align*}
    where the inequality follows from assumption (ii). Thus $N_k^* > N_{k-1}^*$ for $k=2,\ldots,K$.
    
    \item \textit{Non-negativity}: $N_k^* > 0$ since $\sigma_1 > 0$, $\epsilon_{\text{tar}} > 0$, and $\rho_{1,k}^2 - \rho_{1,k+1}^2 > 0$ by assumption (i).
\end{enumerate}

\noindent {\bf Part B: Optimality via KKT Conditions (inner-block cost derivation)}

Consider feasible solutions in which the sample sizes $N_k^*$ are non-decreasing but may include segments where they are constant. Let $\{\ell_1, \ldots, \ell_q\}\subseteq \{1,\ldots, K\}$ be the indices marking the start of each constant block, with $\ell_1=1$ and $\ell_{q+1} = K+1$, such that
%
\[
N_{\ell_1}<N_{\ell_2}<\ldots < N_{\ell_{q}},\quad\quad  N_{\ell_i}=N_{\ell_i+1}=\ldots = N_{\ell_{i+1}-1} <N_{\ell_{i+1}}, \qquad \text{for}\;\;  i=1,\ldots,q-1.
\]
%
This partitions the indices ${1,\dots,K}$ into $q$ contiguous blocks of constant sample sizes, increasing from one block to the next. 

Three special cases illustrate the structure of such feasible solutions. When $q=1$, all sample sizes are equal, i.e., $N_1=\ldots=N_K$. From \eqref{eq:MFMC_variance}, we then have $N_k=\sigma_1^2/\epsilon_{\text{tar}}^2$ for $k=1,\ldots, K$, $\alpha_k\in \mathbb{R}$, and the total cost is $\mathcal{W}^\text{MF} = \sigma_1^2/\epsilon_{\text{tar}}^2 \sum_{k=1}^K C_k$. When $q=K$, the sample sizes are strictly increasing, i.e., $N_1<\ldots<N_K$,  and we will show that this configuration yields the globally optimal solution. For intermediate values $1<q<K$, the sample sizes exhibit piecewise-constant blocks. In the following, to analyze this general regime, we formulate the corresponding Lagrangian and derive the associated Karush–Kuhn–Tucker (KKT) conditions.



The Lagrangian for the constrained problem, enforcing the variance constraint with multiplier $\lambda_0$ and monotonicity with $\lambda_1,\ldots, \lambda_k$, is
%
\begin{equation*}
L =\sum_{k=1}^K C_kN_k +\lambda_0 \left(\frac{\sigma_1^2}{N_1} + \sum_{k=2}^K \left(\frac{1}{N_{k-1}} - \frac{1}{N_k}\right)G_k- \epsilon_{\text{tar}}^2\right)-\lambda_1 N_1+\sum_{k=2}^K\lambda_k(N_{k-1} - N_k),
\end{equation*}
%
with $\alpha_1 = 1, \alpha_{K+1} = 0$, $\lambda_{K+1} = 0$ and $G_k = \alpha_k^2\sigma_k^2 - 2\alpha_k\rho_{1,k}\sigma_1\sigma_k$.  The KKT conditions includes
%
\[
\begin{array}{ll}
\left[\text{Stationarity}\right]&\frac{\partial L}{\partial \alpha_j}=0,\quad \frac{\partial L}{\partial N_k}=0,\quad j=2\ldots,K, \quad k=1\ldots,K,\\
\left[\text{Primal feasibility}\right]&\mathbb{V}\left[A^{\text{MF}}\right]- \epsilon_{\text{tar}}^2 = 0, \\ 
\left[\text{Primal feasibility}\right] &-N_1\le 0,\qquad N_{k-1}-N_k \le 0, \quad k=2\ldots,K,\\ 
\left[\text{Dual feasibility}\right]  &\lambda_k \ge 0,\quad k=1\ldots,K, \\ 
\left[\text{Complementary slackness}\right]  &\lambda_1 N_1=0,\qquad\lambda_k(N_{k-1}-N_k)=0,\quad k=2\ldots,K.
\end{array}
\]
%


Within each constant block, complementary slackness implies $\lambda_{\ell_i}= 0$ all $i\ge 1$, since the sample sizes within each block are equal. Under the blockwise structure $N_{\ell_i}=N_{\ell_i+1}=\ldots=N_{\ell_{i+1}-1}< N_{\ell_{i+1}}$ for $i=1,\ldots,q$, the Lagrangian simplifies to
%
\begin{equation*}
L= \sum_{i=1}^q N_{\ell_i}\sum_{k=\ell_i}^{\ell_{i+1}-1} C_k +\lambda_0 \left(\frac{\sigma_1^2}{N_{\ell_1}} + \sum_{i=2}^q \left(\frac{1}{N_{\ell_i-1}} - \frac{1}{N_{\ell_i}}\right)G_{\ell_i}- \epsilon_{\text{tar}}^2\right)-\lambda_{\ell_1} N_{\ell_1}+\sum_{i=2}^q\lambda_{\ell_{i}}(N_{\ell_{i-1}} - N_{\ell_{i}}),
\end{equation*}
%
Stationarity of the Lagrangian with respect to $\alpha_{\ell_i}$ yields
%
\begin{align}
\label{eq:partial_L_alpha_k}
    \frac{\partial L}{\partial \alpha_{\ell_i}}&=\lambda_0\left(\frac{1}{N_{\ell_i-1}} - \frac{1}{N_{\ell_i}}\right)\left(2\alpha_{\ell_i}\sigma_{\ell_i}^2 - 2\rho_{1,\ell_i}\sigma_1\sigma_{\ell_i}\right),\quad i=1,\dots,q-1,
    % \frac{\partial L}{\partial N_1}&=C_1 + \lambda_0\left(-\frac{\sigma_1^2}{N_1^2} - \frac{\alpha_2^2\sigma_2^2-2\alpha2\rho_{1,2}\sigma_1\sigma_2}{N_1^2}\right)-\lambda_1+\lambda_2,\\
    % \label{eq:partial_L_N_k}
    % \frac{\partial L}{\partial N_k}&=C_k+\lambda_0\left(\frac{G_k}{N_k^2}-\frac{G_{k+1}}{N_k^2}\right)-\lambda_k+\lambda_{k+1}, \quad k=1,\dots,K,
    % \frac{\partial L}{\partial N_K}&=C_K + \lambda_0\left(\frac{\alpha_K^2\sigma_K^2 - 2\alpha_K\rho_{1,K}\sigma_1\sigma_K}{N_K^2}\right)-\lambda_K.
\end{align}
%
Solving $\partial L/\partial \alpha_{\ell_i} = 0$ gives the optimal weights
%
\[
\alpha_{k}^* = \frac{\rho_{1,k}\sigma_1}{\sigma_{k}}, \text{ if } k=\ell_i, \text{ for }i= 1,\ldots,q.
\]
%
Note only indices $k=\ell_i$ contribute non-trivially to the estimator weights in \eqref{eq:MFMC_Yk}. For indices $k = \ell_i+1,\ldots, \ell_{i+1}-1$, the corresponding Monte Carlo estimators cancel due to shared sample sets, and thus do not appear explicitly in the final estimator. Using the optimal coefficients $\alpha_k^*$, and observing that $N_{\ell_{i}-1} = N_{\ell_{i-1}}$ for $i \ge 2$, we define $\Delta_k = (\rho_{1,k}^2 - \rho_{1,k+1}^2)\sigma_1^2$, where $\rho_{1,K+1} = 0$. Substituting into the variance expression yields the simplified form
%
\begin{equation}\label{eq:MFMC_var_convex}
    \mathbb{V}\left[A^{\text{MF}}\right] = \frac{\sigma_1^2}{N_1}+\sum_{i=2}^q
\left(\frac{1}{N_{\ell_{i}}}-\frac{1}{N_{\ell_{i}-1}}\right)\rho_{1,\ell_i}^2\sigma_1^2=\sum_{i=1}^{q} \frac{ \left(\rho_{1,\ell_i}^2-\rho_{1,\ell_{i+1}}^2\right)\sigma_1^2}{N_{\ell_i}}=\sum_{i=1}^{q} \frac{\Delta_{\ell_i}}{N_{\ell_i}}.
\end{equation}
%

Substituting the optimal coefficients into the Lagrangian and using the block-wise sample size notation, we obtain
%
\begin{equation*}
L= \sum_{i=1}^q N_{\ell_i}\sum_{k=\ell_i}^{\ell_{i+1}-1} C_k +\lambda_0 \left( \sum_{i=1}^{q} \frac{\Delta_{\ell_i}}{N_{\ell_i}}- \epsilon_{\text{tar}}^2\right)-\lambda_1 N_1+\sum_{i=2}^q\lambda_{\ell_{i}}(N_{\ell_{i-1}} - N_{\ell_{i}}).
\end{equation*}
%
The stationarity condition for $N_{\ell_i}$ yields
%
\[
\frac{\partial L}{\partial N_{\ell_i}} =\sum_{k=\ell_i}^{\ell_{i+1}-1} C_{k} -  \frac{\lambda_0\Delta_{\ell_i}}{N_{\ell_i}^2}-\lambda_{\ell_{i}}+\lambda_{\ell_{i+1}}=0,\quad i = 1, \ldots,q,
\]
%
Since the sample sizes increase strictly ($N_{\ell_{i-1}} < N_{\ell_i}$ for $i = 2, \ldots, q$), all inequality constraints are inactive, implying $\lambda_{\ell_i} = 0$ by complementary slackness. The optimal sample sizes are then
%
\begin{equation}\label{eq:sample_size_1}
    N_{\ell_i} = \sqrt{\lambda_0} \sqrt{\frac{\Delta_{\ell_i}}{\sum_{k=\ell_i}^{\ell_{i+1}-1} C_{k}}}, \;\text{ for }\; i=1,\ldots,q,
\end{equation}
%
Substituting $\alpha_k^*$ and $N_{\ell_i}$ in \eqref{eq:sample_size_1} into the variance expression \eqref{eq:MFMC_var_convex} yields
%
\begin{equation*} \label{eq:MFMC_variance2}
    \mathbb{V}\left[A^{\text{MF}}\right] = \frac{1}{\sqrt{\lambda_0}}\sum_{i=1}^q\sqrt{\Delta_{\ell_i}\sum_{k=\ell_i}^{\ell_{i+1}-1} C_k},
\end{equation*}
%
Enforcing the variance constraint $\mathbb{V}[A^{\text{MF}}] = \epsilon_{\text{tar}}^2$ leads to
%
\[
\sqrt{\lambda_0}=\frac{1}{\epsilon_{\text{tar}}^2} \sum_{i=1}^{q} \sqrt{\Delta_{\ell_i}\sum_{k=\ell_i}^{\ell_{i+1}-1} C_{k}},
\]
%
Substituting $\sqrt{\lambda_0}$ into \eqref{eq:sample_size_1} gives the explicit optimal sample sizes
%
\[
N_{\ell_i}^* = \frac{1}{\epsilon_{\text{tar}}^2}\sqrt{\frac{\Delta_{\ell_i}}{\sum_{k=\ell_i}^{\ell_{i+1}-1} C_{k}}}  \sum_{j=1}^{q} \sqrt{\Delta_{\ell_j}\sum_{k=\ell_j}^{\ell_{j+1}-1} C_{k}} \;\text{ for }\; i=1,\ldots,q.
\]
%

Note that for a fixed block partition (fixed $\ell_i$) of sample size $N_{\ell_i}^*$ with $\alpha_{\ell_i}^*$, then 
we introduce the change of variables $y_{\ell_i} = 1/N_{\ell_i}^*$, we reformulate the optimization problem \eqref{eq:Optimization_pb_sample_size} with block structure  as
%
\begin{equation}\label{eq:Optimization_pb_sample_size3}
    \begin{array}{ll}
    \min \limits_{\begin{array}{c}\scriptstyle y_{\ell_1},\ldots, y_{\ell_q}\in \mathbb{R}
\end{array}} &\displaystyle \sum_{i=1}^q C_{\ell_i}y_{\ell_i}^{-1},\\
       \;\,\text{subject to} &\displaystyle \sum_{i=1}^q \Delta_{\ell_i} y_{\ell_i}= \epsilon_{\text{tar}}^2,\\[2pt]
       &\displaystyle -y_{\ell_1}\le 0,\quad \displaystyle y_{\ell_i}-y_{\ell_{i-1}}\le 0, \;\; k=2\ldots,K.
    \end{array}
\end{equation}
%
Since $N_1 > 0$ is required for finite variance \eqref{eq:MFMC_var_convex}, the problem remains well-posed. The transformed problem in is convex as established: the objective function is convex in $y_{\ell_i}$, the equality constraint is affine, and the feasible set defined by monotonicity constraints is convex. Since any local minimum of a convex optimization problem is globally optimal, and we have found a KKT point $y_{\ell_i}^* = 1/N_{\ell_i}^*$ satisfying all optimality conditions of \eqref{eq:Optimization_pb_sample_size3}, this indicates that $N_k^*$ is the global minimizer within the block structure, but not the global minimizer of the original optimization problem \eqref{eq:Optimization_pb_sample_size}.


% Note that by ensuring the condition $(ii)$ is satisfied, we can guarantee that $N_k^*$ increases strictly as $k$ grows. 
The total cost $\mathcal{W}_{\text{block}}^{\text{MF}}$ associated with these optimal sample sizes in this block partition is
%
\begin{equation*}
\mathcal{W}_{\text{block}}^{\text{MF}} = \sum_{i=1}^q \sum_{k=\ell_i}^{\ell_{i+1}-1} C_k N_k = \sum_{i=1}^q N_{\ell_i}\sum_{k=\ell_i}^{\ell_{i+1}-1} C_k =\frac{1}{\epsilon_{\text{tar}}^2}\left(\sum_{i=1}^{q} \sqrt{\Delta_{\ell_i}\sum_{k=\ell_i}^{\ell_{i+1}-1} C_{k}}\right)^2.
\end{equation*}
%

\noindent {\bf Part C: Global optimality (inter-block cost comparison)}

To identify the global minimizer of the optimization problem \eqref{eq:Optimization_pb_sample_size}, we evaluate all minimizers among all possible block partitions
and select the one with the lowest cost. By the Cauchy-Schwarz inequality, the following inequality holds
%
\[
\sum_{i=1}^q \sqrt{ \Delta_{\ell_i} \sum_{k=\ell_i}^{\ell_{i+1}-1} C_k }  = \sum_{i=1}^q \sqrt{ \sum_{k=\ell_i}^{\ell_{i+1}-1}\Delta_k \sum_{k=\ell_i}^{\ell_{i+1}-1} C_k } \ge \sum_{i=1}^q \sum_{k=\ell_i}^{\ell_{i+1}-1} \sqrt{\Delta_k C_k} = \sum_{k=1}^K \sqrt{\Delta_k C_k}
\]
%
Equality holds if and only if each block contains exactly one model (i.e., $q=K$), strictly increasing sample size structure. However, we note that equality could also hold if multiple models in a block happen to have identical $\Delta_k/C_k$ ratios. But under general problem conditions, we cannot guarantee this coincidence. Therefore, the optimal block structure must have $q=K$. Let $\mathcal{W}^\text{MF}$ be the total cost when sample sizes increase strictly with model index $k$, i.e., without block repetitions. Thus $\mathcal{W}_{\text{block}}^{\text{MF}} \geq \mathcal{W}^{\text{MF}}$, confirming the singleton block structure ($q=K$) is globally optimal. 


In this case, the optimal coefficients and sample sizes reduce to
%
\begin{align*}
    % \label{eq:MFMC_coefficients}
    % &\alpha_k^*=\frac{\rho_{1,k}\sigma_1}{\sigma_k},\\
    \label{eq:MFMC_SampleSize}
    &\alpha_k^*=\frac{\rho_{1,k}\sigma_1}{\sigma_k},\;\; \;N_k^*=\frac{1}{\epsilon_\text{tar}^2}\sqrt{\frac{\Delta_k}{C_k}}\sum_{j=1}^K\sqrt{C_j\Delta_j},\;\; \mathcal{W}^\text{MF} = \sum_{k=1}^K C_k N_k^* = \frac{1}{\epsilon_{\text{tar}}^2}\left(\sum_{k=1}^K\sqrt{C_k\Delta_k}\right)^2\quad \text{with}\;\;\rho_{K+1}=0.
\end{align*}
%








% The total cost expression follows directly by substituting $N_k^*$ into $\sum C_k N_k^*$.


\end{proof}



\subsection{Proof of Theorem \ref{thm:Sample_cost_est}}
\begin{proof}\label{eq:Sample_cost_est}
To derive the total sampling cost of the multi-fidelity Monte Carlo estimator, we first express the cost per sample for high- and low-fidelity models using conditions (ii) and (v) from Theorem \ref{thm:Sample_cost_est}, along with the mesh scaling relation \eqref{eq:MeshGrowth}
%
\[
C_1\simeq M_L^\gamma \simeq s^{L\gamma},
\]
%
Substituting these expressions into the MFMC sampling cost formula \eqref{eq:MFMC_sampling_cost}, we obtain
%
\begin{align*}
    \mathcal{W}^\text{MF} &\simeq \epsilon^{-2}\left(\sqrt{C_1\left(1 - \rho_{1,i_2}^2\right)}+\sum_{k=2}^{K^*} \sqrt{C_{i_k}\left(\rho_{1,{i_k}}^2 - \rho_{1,i_{k+1}}^2\right)} \right)^2 \simeq \epsilon^{-2}\left(s^{\frac{(\gamma-\beta)}{2}L}\right)^2.
\end{align*}

% %
% \begin{align*}
%     \mathcal{W}^\text{MF} &\simeq \epsilon^{-2}\left(\sqrt{C_1\left(\rho_{1,1}^2 - \rho_{1,2}^2\right)}+\sum_{k=2}^{L+1} \sqrt{C_k\left(\rho_{1,k}^2 - \rho_{1,k+1}^2\right)} \right)^2 \simeq \epsilon^{-2} \left(c_1 M_L^{\frac{\gamma-\beta}{2}}+c_2\sum_{k=2}^{L+1}M_{L-k+1}^\frac{\gamma_1-\beta_1}{2}\right)^2,\\
%     &=\epsilon^{-2} \left(c_1 M_L^{\frac{\gamma-\beta}{2}}+c_2\sum_{p=0}^{L-1}M_{p}^\frac{\gamma_1-\beta_1}{2}\right)^2\simeq \epsilon^{-2}\left(c_1s^{\frac{(\gamma-\beta)}{2}L}+c_2\sum_{p=0}^{L-1}s^{\frac{(\gamma_1-\beta_1)}{2}p}\right)^2,
% \end{align*}
% %
Since $K^*\simeq L$, we use the geometric sum approximation
%
\begin{equation}
\label{eq:Geo_sum_for_s}
\sum_{p=0}^L s^{\eta p}\simeq\left\{\begin{array}{ll}
\frac{1}{1-s^{\eta}}, & \eta<0,\\
|\log \epsilon|, & \eta = 0,\\
\epsilon^{-\frac{\eta}{\alpha}}, & \eta>0,
\end{array}
\right.
\end{equation}
%
Applying \eqref{eq:SLSGC_MLS_SpatialGridsNo} and condition (i), we substitute $s^{\frac{(\gamma-\beta)}{2}L}\simeq \epsilon^{\frac{\beta-\gamma}{2\alpha}}$, leading to the asymptotic result
\[
\mathcal{W}^\text{MF} \simeq \epsilon^{-2+\frac{\beta-\gamma}{\alpha}}.
\]
\end{proof}



% \begin{proof}\label{eq:Sample_cost_est}
% To derive the total sampling cost of the multi-fidelity Monte Carlo estimator, we first express the cost per sample for high- and low-fidelity models using conditions (ii) and (v) from Theorem \ref{thm:Sample_cost_est}, along with the mesh scaling relation \eqref{eq:MeshGrowth}
% %
% \[
% C_1\simeq M_L^\gamma \simeq s^{L\gamma},\qquad  C_k \simeq M_{L-i_k+1}^{\gamma_1}\simeq s^{(L-i_k+1)\gamma_1},
% \]
% %
% Substituting these expressions into the MFMC sampling cost formula \eqref{eq:MFMC_sampling_cost} and using conditions (iii) and (iv), we obtain
% %
% \begin{align*}
%     \mathcal{W}^\text{MF} &\simeq \epsilon^{-2}\left(\sqrt{C_1\left(1 - \rho_{1,i_2}^2\right)}+\sum_{k=2}^{K^*} \sqrt{C_{i_k}\left(\rho_{1,{i_k}}^2 - \rho_{1,i_{k+1}}^2\right)} \right)^2 \simeq \epsilon^{-2}\left(c_1s^{\frac{(\gamma-\beta)}{2}L}+c_2\sum_{p=0}^{K^*-2}s^{\frac{(\gamma_1-\beta_1)}{2}p}\right)^2.
% \end{align*}

% % %
% % \begin{align*}
% %     \mathcal{W}^\text{MF} &\simeq \epsilon^{-2}\left(\sqrt{C_1\left(\rho_{1,1}^2 - \rho_{1,2}^2\right)}+\sum_{k=2}^{L+1} \sqrt{C_k\left(\rho_{1,k}^2 - \rho_{1,k+1}^2\right)} \right)^2 \simeq \epsilon^{-2} \left(c_1 M_L^{\frac{\gamma-\beta}{2}}+c_2\sum_{k=2}^{L+1}M_{L-k+1}^\frac{\gamma_1-\beta_1}{2}\right)^2,\\
% %     &=\epsilon^{-2} \left(c_1 M_L^{\frac{\gamma-\beta}{2}}+c_2\sum_{p=0}^{L-1}M_{p}^\frac{\gamma_1-\beta_1}{2}\right)^2\simeq \epsilon^{-2}\left(c_1s^{\frac{(\gamma-\beta)}{2}L}+c_2\sum_{p=0}^{L-1}s^{\frac{(\gamma_1-\beta_1)}{2}p}\right)^2,
% % \end{align*}
% % %
% Since $K^*\simeq L$, we use the geometric sum approximation
% %
% \begin{equation}
% \label{eq:Geo_sum_for_s}
% \sum_{p=0}^L s^{\eta p}\simeq\left\{\begin{array}{ll}
% \frac{1}{1-s^{\eta}}, & \eta<0,\\
% |\log \epsilon|, & \eta = 0,\\
% \epsilon^{-\frac{\eta}{\alpha}}, & \eta>0,
% \end{array}
% \right.
% \end{equation}
% %
% Applying \eqref{eq:SLSGC_MLS_SpatialGridsNo} and condition (i), we substitute $s^{\frac{(\gamma-\beta)}{2}L}\simeq \epsilon^{\frac{\beta-\gamma}{2\alpha}}$. The remaining summation term $\sum_{p=0}^{K^*-2}s^{\frac{(\gamma_1-\beta_1)}{2}p}$ follows from \eqref{eq:Geo_sum_for_s}, yielding
% %
% \[
% \mathcal{W}^\text{MF} \simeq \left\{\begin{array}{ll}
% \epsilon^{-2}\left(c_1\epsilon^{\frac{\beta-\gamma}{2\alpha}}+c_2\right)^2, & \beta_1>\gamma_1,\\
% \epsilon^{-2}\left(c_1\epsilon^{\frac{\beta-\gamma}{2\alpha}}+c_2|\log\epsilon|\right)^2, & \beta_1=\gamma_1,\\
% \epsilon^{-2}\left(c_1\epsilon^{\frac{\beta-\gamma}{2\alpha}}+c_2\epsilon^{-\frac{\gamma_1-\beta_1}{2\alpha}}\right)^2, & \beta_1<\gamma_1.
% \end{array}
% \right.
% \]
% %
% When $c_1\gg c_2$ (or equivalently, $C_1$ is much larger than $C_i$ for $i\ge 2$), the dominant term in the expression is associated with $c_1$, allowing us to neglect the contributions from $c_2$, leading to the asymptotic result
% \[
% \mathcal{W}^\text{MF} \simeq \epsilon^{-2+\frac{\beta-\gamma}{\alpha}}.
% \]
% \end{proof}
% %
% \begin{theorem}
% \label{thm:Sample_cost_est}
%  Suppose there exist positive constants $\alpha, \gamma$ such that for high-fidelity models $ u_{h,1}$ at spatial grid levels $L=1,\ldots,L_m$
% %
% \begin{alignat*}{8}
%     % &(i)\;\; |\rho_{1,1}|>\ldots>|\rho_{1,K}|,& \quad \quad
%     % &(ii)\;\; \frac{C_{k-1}}{C_k}>\frac{\rho_{1,k-1}^2-\rho_{1,k}^2}{\rho_{1,k}^2-\rho_{1,k+1}^2},\;\;k=2,\ldots,K, \quad \rho_{1,K+1}=0,\\
%     &(i)\;\; \left\Vert\mathbb{E}\left(u- u_{h,1}\right)\right\Vert_Z\simeq M_{L}^{-\alpha},\qquad
%     % &(ii)\;\; \left(\rho_{1,L}^{H}\right)^2-\left(\rho_{1,L+1}^H\right)^2 \simeq M_{L}^{-\beta},
%     % \qquad
%     &(ii)\;\; C_1 \simeq M_{L}^{\gamma},
% \end{alignat*}
% %
% where $C_1$ is the cost per sample of the high-fidelity model at level $L$. Moreover, for the low-fidelity models with index $\{i_k | \; i_k\in \mathcal{I}^*,\;k=2, \ldots, K^*\}$, suppose there exist positive constants $\beta, \beta_1, \gamma_1$ such that 
% %
% \begin{alignat*}{8}
%     % &(i)\;\; |\rho_{1,1}|>\ldots>|\rho_{1,K}|,& \quad \quad
%     % &(ii)\;\; \frac{C_{k-1}}{C_k}>\frac{\rho_{1,k-1}^2-\rho_{1,k}^2}{\rho_{1,k}^2-\rho_{1,k+1}^2},\;\;k=2,\ldots,K, \quad \rho_{1,K+1}=0,\\
%     &(iii)\;\; 1-\rho_{1,i_2}^2 \simeq M_{L}^{-\beta},
%     \qquad
%     &(iv)\;\; \rho_{1,i_k}^2-\rho_{1,i_{k+1}}^2 \simeq M_{L-i_k+1}^{-\beta_1},
%     \qquad
% &(v)\;\; C_{i_k} \simeq M_{L-i_k+1}^{\gamma_1},
% \end{alignat*}
% %
% where $\rho_{1,i_k}$ is the correlated coefficient between the high-fidelity model $ u_{L,1}$ and low fidelity model $ u_{h,i_k}$, and $C_{i_k}$ is the cost per sample for $u_{h,i_k}$. Then for any positive $\epsilon<e^{-1}$, there exists level $L$ and sample size $N_{i_k}$ for $ i_k\in \mathcal{I}^*$, as given in \eqref{eq:MFMC_SampleSize}, such that the multi-fidelity estimator $A^{\text{MF}}$ has an nMSE with
% \[
% \frac{\left\Vert\mathbb{E}(u)-A^{\text{MF}} \right\Vert_{L^2(\boldsymbol W,Z)}}{\left\Vert\mathbb{E}(u) \right\Vert_{L^2( \boldsymbol W,Z)}}<\epsilon,
% \]
% with total sampling cost
% %
% \begin{equation*}
%     \mathcal{W}^{\text{MF}} \simeq \left\{\begin{array}{ll}
% \epsilon^{-2}\left(c_1\epsilon^{\frac{\beta-\gamma}{2\alpha}}+c_2\right)^2, & \beta_1>\gamma_1,\\
% \epsilon^{-2}\left(c_1\epsilon^{\frac{\beta-\gamma}{2\alpha}}+c_2|\log\epsilon|\right)^2, & \beta_1=\gamma_1,\\
% \epsilon^{-2}\left(c_1\epsilon^{\frac{\beta-\gamma}{2\alpha}}+c_2\epsilon^{-\frac{\gamma_1-\beta_1}{2\alpha}}\right)^2, & \beta_1<\gamma_1.
% \end{array}
% \right.
% \end{equation*}
% %
% Moreover, if $c_1\gg c_2$, then even if conditions (iv) and (v) do not hold for low-fidelity models, the dominant cost term simplifies to 
% \[
% \mathcal{W}^\text{MF} \simeq \epsilon^{-2+\frac{\beta-\gamma}{\alpha}}.
% \]
% \end{theorem}
% %


% \begin{proof}\label{eq:Sample_cost_est}
% To derive the total sampling cost of the multi-fidelity Monte Carlo estimator, we first express the cost per sample for high- and low-fidelity models using conditions (ii) and (v) from Theorem \ref{thm:Sample_cost_est}, along with the mesh scaling relation \eqref{eq:MeshGrowth}
% %
% \[
% C_1\simeq M_L^\gamma \simeq s^{L\gamma},\qquad  C_k \simeq M_{L-i_k+1}^{\gamma_1}\simeq s^{(L-i_k+1)\gamma_1},
% \]
% %
% Substituting these expressions into the MFMC sampling cost formula \eqref{eq:MFMC_sampling_cost} and using conditions (iii) and (iv), we obtain
% %
% \begin{align*}
%     \mathcal{W}^\text{MF} &\simeq \epsilon^{-2}\left(\sqrt{C_1\left(1 - \rho_{1,i_2}^2\right)}+\sum_{k=2}^{K^*} \sqrt{C_{i_k}\left(\rho_{1,{i_k}}^2 - \rho_{1,i_{k+1}}^2\right)} \right)^2 \simeq \epsilon^{-2}\left(c_1s^{\frac{(\gamma-\beta)}{2}L}+c_2\sum_{p=0}^{K^*-2}s^{\frac{(\gamma_1-\beta_1)}{2}p}\right)^2.
% \end{align*}

% % %
% % \begin{align*}
% %     \mathcal{W}^\text{MF} &\simeq \epsilon^{-2}\left(\sqrt{C_1\left(\rho_{1,1}^2 - \rho_{1,2}^2\right)}+\sum_{k=2}^{L+1} \sqrt{C_k\left(\rho_{1,k}^2 - \rho_{1,k+1}^2\right)} \right)^2 \simeq \epsilon^{-2} \left(c_1 M_L^{\frac{\gamma-\beta}{2}}+c_2\sum_{k=2}^{L+1}M_{L-k+1}^\frac{\gamma_1-\beta_1}{2}\right)^2,\\
% %     &=\epsilon^{-2} \left(c_1 M_L^{\frac{\gamma-\beta}{2}}+c_2\sum_{p=0}^{L-1}M_{p}^\frac{\gamma_1-\beta_1}{2}\right)^2\simeq \epsilon^{-2}\left(c_1s^{\frac{(\gamma-\beta)}{2}L}+c_2\sum_{p=0}^{L-1}s^{\frac{(\gamma_1-\beta_1)}{2}p}\right)^2,
% % \end{align*}
% % %
% Since $K^*\simeq L$, we use the geometric sum approximation
% %
% \begin{equation}
% \label{eq:Geo_sum_for_s}
% \sum_{p=0}^L s^{\eta p}\simeq\left\{\begin{array}{ll}
% \frac{1}{1-s^{\eta}}, & \eta<0,\\
% |\log \epsilon|, & \eta = 0,\\
% \epsilon^{-\frac{\eta}{\alpha}}, & \eta>0,
% \end{array}
% \right.
% \end{equation}
% %
% Applying \eqref{eq:SLSGC_MLS_SpatialGridsNo} and condition (i), we substitute $s^{\frac{(\gamma-\beta)}{2}L}\simeq \epsilon^{\frac{\beta-\gamma}{2\alpha}}$. The remaining summation term $\sum_{p=0}^{K^*-2}s^{\frac{(\gamma_1-\beta_1)}{2}p}$ follows from \eqref{eq:Geo_sum_for_s}, yielding
% %
% \[
% \mathcal{W}^\text{MF} \simeq \left\{\begin{array}{ll}
% \epsilon^{-2}\left(c_1\epsilon^{\frac{\beta-\gamma}{2\alpha}}+c_2\right)^2, & \beta_1>\gamma_1,\\
% \epsilon^{-2}\left(c_1\epsilon^{\frac{\beta-\gamma}{2\alpha}}+c_2|\log\epsilon|\right)^2, & \beta_1=\gamma_1,\\
% \epsilon^{-2}\left(c_1\epsilon^{\frac{\beta-\gamma}{2\alpha}}+c_2\epsilon^{-\frac{\gamma_1-\beta_1}{2\alpha}}\right)^2, & \beta_1<\gamma_1.
% \end{array}
% \right.
% \]
% %
% When $c_1\gg c_2$ (or equivalently, $C_1$ is much larger than $C_i$ for $i\ge 2$), the dominant term in the expression is associated with $c_1$, allowing us to neglect the contributions from $c_2$, leading to the asymptotic result
% \[
% \mathcal{W}^\text{MF} \simeq \epsilon^{-2+\frac{\beta-\gamma}{\alpha}}.
% \]
% \end{proof}


\bibliographystyle{abbrv}
% \bibliographystyle{alphaurl}
\bibliography{references_liang}
\end{document}