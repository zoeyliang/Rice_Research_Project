% ========================================
\section{The Grad-Shafranov free boundary problem with uncertainty}\label{sec:Grad-Shafranov}
% ========================================
The quest for controlled nuclear fusion as a clean and virtually limitless energy source has driven extensive research into magnetic confinement in fusion reactors. In these systems, light nuclei—typically deuterium and tritium -- are heated to form a high-temperature plasma, which is magnetically confined by fields generated by external coils surrounding the reactor. Confinement is achieved by balancing the plasma's hydrostatic pressure against the magnetic pressure produced by both external fields and self-induced plasma currents. The resulting equilibrium state, coupled with Maxwell’s equations, is modeled by the Grad–Shafranov free-boundary equation \cite{GrRu:1958, LuSc:1957, Shafranov:1958}. In axially symmetric geometries of tokamaks, this formulation reduces to a two-dimensional formulation in the $r$-$z$ plane using cylindrical coordinates $(r, z, \varphi)$. The governing equation seeks for $u(r,z)$ and takes the form
%
\begin{subequations}\label{eq:FreeBoundary}
\begin{equation}\label{eq:FreeBoundary_GS}
 -\nabla\,\cdot\,\left(\frac{1}{\mu r}\nabla u\right) = \left\{ \begin{array}{ll}
r\frac{d}{d u} p(u) + \frac{1}{2\,\mu r} \frac{d}{d u} g^2(u) & \text{ in } \Omega_p(u) \\
I_k/S_k & \text{ in } \Omega_{C_k} \\
0 & \text{ elsewhere, } 
\end{array}\right.
\end{equation}
%
where $\nabla$ and $\nabla\cdot$ denote the Cartesian gradient and divergence operators in two dimensions; $\mu$ represents the magnetic permeability (a constant $\mu_0$ in vacuum, variable $\mu = \mu(|\nabla u|^2/r^2)$ in ferromagnetic materials), while $\Omega_p$ and $\Omega_{C_k}$ correspond to the plasma confinement domain and the cross-sections of external coils carrying currents $I_k$ over areas $S_k$. The source term in $\Omega_p$ models the toroidal plasma current density, which depends nonlinearly on $u$ in terms of the hydrostatic pressure $p(u)$ and toroidal magnetic field $g(u)$. Following the formulation in \cite{LuBr:1982}, we define 
%
\begin{equation}\label{eq:source}
\frac{d}{d u}p( u) = j_0\frac{\beta}{r_0}\left(1-u_N^{\alpha_1}\right)^{\alpha_2},  \qquad \qquad
\frac{1}{2}\frac{d}{d u}g^2(u) = j_0\mu_0r_0(1-\beta)\left(1-u_N^{\alpha_1}\right)^{\alpha_2},
\end{equation}
\end{subequations}
%
where $u_N \in [0,1]$ is the normalized poloidal flux, scaled between its values on the \textit{magnetic axis} and the plasma boundary; the parameters $r_0$, $\alpha_1$, and $\alpha_2$ characterize the outer radius of the vacuum chamber and control the sharpness of the current profile, while $\beta$ (the poloidal beta) measures the ratio of plasma pressure to magnetic pressure. The plasma boundary $\partial \Omega_p$, which depends on the solution $u$ and is defined by the last closed streamline, introduces a free-boundary aspect to the problem. The mathematical complexity is compounded by several nonlinear dependencies -- on the plasma boundary, magnetic permeability, and nonlinear source terms $p(u)$ and $g(u)$.


Let $\Omega$ be a bounded Lipschitz domain enclosing the confinement region $\Omega_p$, the external coils $\Omega_{c_i}$, and surrounding structural regions.  The solution space $Z$ for \eqref{eq:FreeBoundary}, following \cite{Gr:1999}, is defined as
%
\begin{equation}\label{eq:Soln_space}
    Z:=\left\{u:\Omega\rightarrow \mathbb{R} \,\Bigg| \,\int_\Omega u^2rdrdz<\infty; \,  \int_\Omega\frac{|\nabla u|^2}{r}drdz<\infty; \, u(0,z)=0 \right\}\cap C^0(\overline{\Omega}),
\end{equation}
%
with inner product and norm
%
\[
    \langle u,v\rangle_Z := \int_{\Omega} \frac{1}{r} \nabla u\cdot\nabla v \;\;drdz,\qquad \| u \|_{Z} :=\left(\int_\Omega\frac{|\nabla u|^2}{r} drdz\right)^{1/2}.
\]
%
In practice, the equilibrium state of the plasma is sensitive to uncertainties stemming from sources such as measurement errors and operational variability. These uncertainties induce stochastic fluctuations in both the solution and its derived quantities, potentially compromising the stability and performance of plasma confinement. This paper investigates the impact of parametric uncertainty in the coil current intensities. 

To model these uncertainties, we introduce a $d$-dimensional random vector $\boldsymbol \omega :=(\omega_1,\ldots,\omega_d)$, where each $\omega_i$ is treated as an independent random variable with an associated probability density function $\pi_k$. The baseline values are denoted by $\boldsymbol{\widetilde{\omega}} = (\widetilde{\omega}_1, \ldots, \widetilde{\omega}_d)$.  Assuming that each $\omega_k$ is uniformly distributed around its baseline value $\widetilde{\omega}_k$, with relative perturbation magnitude $\tau$, the joint probability density function is $\pi \left(\boldsymbol{\omega}\right)=\prod_{k=1}^{d} 1/(2\tau |\widetilde{\omega}_k|)$, and the corresponding $d$-dimensional parameter space is $W=\prod_{k=1}^{d}\left[\widetilde{\omega}_k-\tau \left\vert \widetilde{\omega}_k\right\vert,\widetilde{\omega}_k+\tau \left\vert \widetilde{\omega}_k \right\vert\right]$.
% %
% \begin{equation}
% \label{eq:ParameterSpace}
%  \pi \left(\boldsymbol{\omega}\right)=\prod_{k=1}^{d} \pi_k\left(\omega_{k}\right)=\prod_{k=1}^{d} \frac{1}{2\tau |\widetilde{\omega}_k|}, \qquad  
%     W := \prod_{k=1}^{d}\left[\widetilde{\omega}_k-\tau \left\vert \widetilde{\omega}_k\right\vert,\widetilde{\omega}_k+\tau \left\vert \widetilde{\omega}_k \right\vert\right].
% \end{equation}
% %
Incorporating uncertainty into the coil currents requires solving a parameterized version of the free-boundary problem \eqref{eq:FreeBoundary}, represented by a solution operator $u(\cdot, \boldsymbol{\omega}): W \to Z$, which maps each realization of the random vector $\boldsymbol \omega$ to a corresponding solution in the spatial function space $Z$. To quantify the variability introduced by stochastic parameters, we adopt the {\it weighted Bochner space}, which provides a setting for analyzing functions that depend on both spatial and parametric variables. We consider the Bochner space $L^2(W,Z)$, which consists of strongly measurable functions with finite second moments defined as
%
\[
L^2(W,Z) = \left\{u:W\rightarrow Z\; \bigg\vert \;\int_{W}\left\|u(\cdot,\boldsymbol{\omega})\right\|_{Z}^2\pi(\boldsymbol{\omega})d\boldsymbol{\omega}<\infty\right\},
\]
%
with the associated norm on $L^2(W,Z)$ is given by
%
\[
\left\Vert u \right\Vert_{L^2(\boldsymbol W,Z)} =
    \left(\int_{\boldsymbol W} \left\Vert u(\cdot,\boldsymbol{\omega})  \right\Vert_{Z}^2 \pi(\boldsymbol{\omega})d\boldsymbol{\omega} \right)^{1/2} = \left(\mathbb{E}\left[\left\Vert u(\cdot,\boldsymbol{\omega})  \right\Vert_{Z}^2\right]\right)^{1/2}\,. 
\]
%

The objective of this paper is to investigate the propagation of uncertainty and to efficiently approximate the parametric expectation
%
 \begin{equation}
 \label{eq:QoI}
      \mathbb{E}\left(u(\cdot,\boldsymbol \omega)\right)=\int_W u(\cdot,\boldsymbol{\omega})\pi(\boldsymbol\omega)d\boldsymbol{\omega},
 \end{equation}
%
along with derived quantities from \eqref{eq:QoI} such as the plasma boundary and features of the solution, including the location of $x$-points.







% Incorporating the uncertainty and 
% %
% \begin{equation}\label{eq:FreeBoundarya}
%  -\nabla\,\cdot\,\left(\frac{1}{\mu(u(\cdot, \boldsymbol{\omega})) r}\nabla u(\cdot, \boldsymbol{\omega})\right) = \left\{ \begin{array}{ll}
% \frac{d}{du} p(u(\cdot, \boldsymbol{\omega})) + \frac{1}{2\,\mu r} \frac{d}{du} g^2(u(\cdot, \boldsymbol{\omega})) & \text{ in } \Omega_p(u(\cdot, \boldsymbol{\omega})) \\
% I_k(\boldsymbol\omega)/S_k & \text{ in } \Omega_{C_k} \\
% 0 & \text{ elsewhere, } 
% \end{array}\right.
% \end{equation}
