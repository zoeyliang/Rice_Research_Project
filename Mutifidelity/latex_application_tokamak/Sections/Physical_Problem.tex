% ========================================
\section{The Grad-Shafranov free boundary problem with uncertainty}\label{sec:Grad-Shafranov}
% ========================================
The pursuit of controlled nuclear fusion as a clean, abundant energy source has spurred intensive research into magnetic confinement. We consider fusion in an ITER-class tokamak -- a toroidal device that confines high-temperature plasma using magnetic fields. A deuterium–tritium gas mixture is injected into the chamber and heated above 100 million degrees Celsius, forming a fully ionized plasma in which fusion occurs as thermal energy overcomes Coulomb repulsion. To prevent energy loss and damage, contact between plasma and vessel walls must be avoided. Since the plasma consists of charged particles, its motion can be controlled by magnetic fields. Confinement is achieved by balancing the plasma’s internal pressure against magnetic pressure from external coils and self-induced plasma currents. The resulting equilibrium, governed by Maxwell’s equations and force balance, is described by the Grad–Shafranov equation \cite{GrRu:1958, LuSc:1957, Shafranov:1958}. Assuming axisymmetry (i.e., no dependence on the toroidal angle $\varphi$), the three-dimensional problem reduces to a two-dimensional problem in the $(r, z)$ plane, where the poloidal flux function $u(r,z)$ satisfies
%
\begin{subequations}\label{eq:FreeBoundary}
\begin{equation}\label{eq:FreeBoundary_GS}
 -\nabla\,\cdot\,\left(\frac{1}{\mu r}\nabla u\right) = \left\{ \begin{array}{ll}
r\frac{d}{d u} p(u) + \frac{1}{2\,\mu r} \frac{d}{d u} g^2(u) & \text{ in } \Omega_p(u) \\
I_k/S_k & \text{ in } \Omega_{C_k} \\
0 & \text{ elsewhere, } 
\end{array}\right.
\end{equation}
%
where $\nabla$ and $\nabla \cdot$ denote gradient and divergence operator in Cartesian coordinates; the magnetic permeability $\mu$ is constant (equal to $\mu_0$) in vacuum regions and may vary within ferromagnetic materials as a function $\mu = \mu(|\nabla u|^2/r^2)$; the domain $\Omega_p$ represents the plasma region, while $\Omega_{C_k}$ corresponds to the region occupied by the $k$-th poloidal field coil $k$ carrying current $I_k$ distributed over a cross-sectional area $S_k$. The source term in $\Omega_p$ models the toroidal plasma current density, which depends nonlinearly on $u$ in terms of the hydrostatic pressure $p(u)$ and toroidal magnetic field $g(u)$. Following the formulation in \cite{LuBr:1982}, we define 
%
\begin{equation}\label{eq:source}
\frac{d}{d u}p( u) = j_0\frac{\beta}{r_0}\left(1-u_N^{\alpha_1}\right)^{\alpha_2},  \qquad \qquad
\frac{1}{2}\frac{d}{d u}g^2(u) = j_0\mu_0r_0(1-\beta)\left(1-u_N^{\alpha_1}\right)^{\alpha_2},
\end{equation}
\end{subequations}
%
where $u_N \in [0,1]$ is the normalized poloidal flux, scaled between its values on the \textit{magnetic axis} and the plasma boundary; the parameters $r_0$, $\alpha_1$, and $\alpha_2$ characterize the outer radius of the vacuum chamber and control the sharpness of the current profile, while $\beta$ (the poloidal beta) measures the ratio of plasma pressure to magnetic pressure. The plasma boundary $\partial \Omega_p$, which depends on the solution $u$ and is defined by the last closed streamline, introduces a free-boundary aspect to the problem. The problem is further complicated by nonlinear dependencies in the boundary location, source terms, and possibly spatially varying permeability.


Let $\Omega \subset \mathbb{R}^2$ be a bounded Lipschitz domain enclosing the confinement region $\Omega_p$, the external coils $\Omega_{c_i}$, and surrounding structural components.  Following \cite{Gr:1999}, the solution space $Z$ for \eqref{eq:FreeBoundary} is defined as
%
\begin{equation}\label{eq:Soln_space}
    Z:=\left\{u:\Omega\rightarrow \mathbb{R} \,\Bigg| \,\int_\Omega u^2rdrdz<\infty; \,  \int_\Omega\frac{|\nabla u|^2}{r}drdz<\infty; \, u(0,z)=0 \right\}\cap C^0(\overline{\Omega}),
\end{equation}
%
equipped with the inner product and norm
%
\[
    \langle u,v\rangle_Z := \int_{\Omega} \frac{1}{r} \nabla u\cdot\nabla v \;\;drdz,\qquad \| u \|_{Z} :=\left(\int_\Omega\frac{|\nabla u|^2}{r} drdz\right)^{1/2}.
\]
%

In practice, The equilibrium configuration is sensitive to uncertainties from measurement error, modeling assumptions, and operational variability, which propagate through the system and impact both the solution $u$ and derived quantities such as the plasma boundary and $x$-point locations. In this work, we focus on uncertainties in the coil current intensities. We model the uncertainty using a $d$-dimensional random vector $\boldsymbol{\omega} = (\omega_1, \ldots, \omega_d)$, where each $\omega_k$ is an independent random variable uniformly distributed around the baseline value $\widetilde{\omega}_k$ for some relative perturbation magnitude $\tau > 0$. The corresponding joint density is
%
\[
\pi \left(\boldsymbol{\omega}\right)=\prod_{k=1}^{d} \frac{1}{2\tau |\widetilde{\omega}_k|},\quad W=\prod_{k=1}^{d}\left[\widetilde{\omega}_k-\tau \left\vert \widetilde{\omega}_k\right\vert,\widetilde{\omega}_k+\tau \left\vert \widetilde{\omega}_k \right\vert\right].
\]
%
Incorporating uncertainty into the coil currents requires solving a parameterized version of the free-boundary problem \eqref{eq:FreeBoundary}, with solution operator $u(\cdot,\boldsymbol{\omega}) : W \to Z$ mapping each realization of $\boldsymbol{\omega}$ to a solution in $Z$. To quantify the variability introduced by stochastic parameters, we adopt the {\it weighted Bochner space}. We consider the Bochner space $L^2(W,Z)$, which consists of strongly measurable functions with finite second moments defined as
% %
% \begin{equation}
% \label{eq:ParameterSpace}
%  \pi \left(\boldsymbol{\omega}\right)=\prod_{k=1}^{d} \pi_k\left(\omega_{k}\right)=\prod_{k=1}^{d} \frac{1}{2\tau |\widetilde{\omega}_k|}, \qquad  
%     W := \prod_{k=1}^{d}\left[\widetilde{\omega}_k-\tau \left\vert \widetilde{\omega}_k\right\vert,\widetilde{\omega}_k+\tau \left\vert \widetilde{\omega}_k \right\vert\right].
% \end{equation}
% %
%
\[
L^2(W,Z) = \left\{u:W\rightarrow Z\; \bigg\vert \;\int_{W}\left\|u(\cdot,\boldsymbol{\omega})\right\|_{Z}^2\pi(\boldsymbol{\omega})d\boldsymbol{\omega}<\infty\right\},
\]
%
with the associated norm on $L^2(W,Z)$ is given by
%
\[
\left\Vert u \right\Vert_{L^2(\boldsymbol W,Z)} =
    \left(\int_{\boldsymbol W} \left\Vert u(\cdot,\boldsymbol{\omega})  \right\Vert_{Z}^2 \pi(\boldsymbol{\omega})d\boldsymbol{\omega} \right)^{1/2} = \left(\mathbb{E}\left[\left\Vert u(\cdot,\boldsymbol{\omega})  \right\Vert_{Z}^2\right]\right)^{1/2}\,. 
\]
%

The objective of this paper is to investigate the propagation of uncertainty and to efficiently approximate the parametric expectation
%
 \begin{equation}
 \label{eq:QoI}
      \mathbb{E}\left[u(\cdot,\boldsymbol \omega)\right]=\int_W u(\cdot,\boldsymbol{\omega})\pi(\boldsymbol\omega)d\boldsymbol{\omega},
 \end{equation}
%
along with derived quantities from \eqref{eq:QoI} such as the plasma boundary and features of the solution, including the location of $x$-points.







% Incorporating the uncertainty and 
% %
% \begin{equation}\label{eq:FreeBoundarya}
%  -\nabla\,\cdot\,\left(\frac{1}{\mu(u(\cdot, \boldsymbol{\omega})) r}\nabla u(\cdot, \boldsymbol{\omega})\right) = \left\{ \begin{array}{ll}
% \frac{d}{du} p(u(\cdot, \boldsymbol{\omega})) + \frac{1}{2\,\mu r} \frac{d}{du} g^2(u(\cdot, \boldsymbol{\omega})) & \text{ in } \Omega_p(u(\cdot, \boldsymbol{\omega})) \\
% I_k(\boldsymbol\omega)/S_k & \text{ in } \Omega_{C_k} \\
% 0 & \text{ elsewhere, } 
% \end{array}\right.
% \end{equation}
