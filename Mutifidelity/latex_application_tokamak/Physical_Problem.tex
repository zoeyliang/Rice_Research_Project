%!TEX root = main.tex
%%%%%%%%%%%%%%%%%%%%%%%%%%%%%%%%%%%%%%%%%%%%%%%%%%%

% ========================================
\section{Problem Definition}\label{sec:problem}
% ========================================

% ========================================
\subsection{Abstract Problem} \label{sec:abstract_problem}
% ========================================
Let $(W, \mathcal{F}, \mathbb{P})$ be a complete probability space, where
$W \subset \real^d$ is the set of outcomes, $\mathcal{F} \subset 2^{W}$ is a
$\sigma$-algebra of events, and $\mathbb{P}: \mathcal{F} \to [0,1]$ is a
probability measure. Given a Hilbert space $(\cU, \langle \cdot, \cdot \rangle_{\cU})$, we define the Bochner space
\begin{align}     \label{eq:BochnerL2}
  L_{\mathbb{P}}^2(W, \cU) := \Big\{ u: W \rightarrow \cU \;\big|\; & u \mbox{ is strongly measurable and } 
                                                                  \int_W \| u(\omega) \|_{\cU}^2 \; \textup{d}\mathbb{P}(\omega) < \infty \; \Big\}.
\end{align}
The space $ L_{\mathbb{P}}^2(W, \cU)$ is a Hilbert space with inner product 
$ \langle u, v \rangle =  \int_W  \langle  u(\omega) , v(\omega)  \rangle_{\cU} \; \textup{d}\mathbb{P}(\omega)$,

Given a function $u \in   L_{\mathbb{P}}^2(W, \cU)$, the objective of this paper is to  efficiently approximate the expectation
 \begin{equation}   \label{eq:expectation_of_u}
      \mathbb{E}[ u ] = \int_W u(\omega)  \textup{d}\mathbb{P}(\omega)
 \end{equation}
 as well as the expectation of derived quantities from $u$. We define
 \begin{equation}   \label{eq:variance_of_u}
           \mathbb{V}[u] = \mathbb{E}\Big[  \big\| u - \mathbb{E}[u]  \big\|_U^2 \Big].
  \end{equation}

In our applications, the evaluation of $u(\omega) $ given a sample $\omega \in W$ requires the solution of a partial differential equation (PDE)
In practice, this solution is not available, but an approximation computed as the solution of a discretization of the PDE is used. 
Thus, in addition to the solution  $u \in   L_{\mathbb{P}}^2(W, \cU)$ of a PDE with random coefficients, we also have a 
the solution $u_h \in   L_{\mathbb{P}}^2(W, \cU)$ of a discretization of the PDE.


% ========================================
\subsection{The Grad-Shafranov free boundary problem with uncertainty}\label{sec:Grad-Shafranov}
% ========================================
The specific problem we consider is the so-called Grad-Shafranov free boundary problem, which arises as a model in nuclear fusion, 
and which was also considered in, e.g., \cite{HCElman_JLiang_TSanchez-Vizuet_2022a}.
We begin with a review of the deterministic version of this problem. Additional details, incl.\ illustrations of the geometry, can be found, e.g., in \cite{HCElman_JLiang_TSanchez-Vizuet_2022a}.
We consider fusion in an ITER-class tokamak -- a toroidal device that confines high-temperature plasma using magnetic fields. 
A deuterium–tritium gas mixture is injected into the chamber and heated above 100 million degrees Celsius, forming a fully ionized plasma in which fusion occurs as thermal energy overcomes Coulomb repulsion. To prevent energy loss and damage, contact between plasma and vessel walls must be avoided. Since the plasma consists of charged particles, its motion can be controlled by magnetic fields. Confinement is achieved by balancing the plasma’s internal pressure against magnetic pressure from external coils and self-induced plasma currents. The resulting equilibrium, governed by Maxwell’s equations and force balance, is described by the Grad–Shafranov equation \cite{GrRu:1958, LuSc:1957, Shafranov:1958}. Assuming axisymmetry (i.e., no dependence on the toroidal angle $\varphi$), the three-dimensional problem reduces to a two-dimensional problem in the $(r, z)$ plane, where the poloidal flux function $u(r,z)$ satisfies
%
\begin{subequations}\label{eq:FreeBoundary}
\begin{equation}\label{eq:FreeBoundary_GS}
 -\nabla\,\cdot\,\left(\frac{1}{\mu r}\nabla u\right) = \left\{ \begin{array}{ll}
r\frac{d}{d u} p(u) + \frac{1}{2\,\mu r} \frac{d}{d u} g^2(u) & \text{ in } \Omega_p(u), \\
I_k/S_k & \text{ in } \Omega_{C_k} ,\\
0 & \text{ elsewhere, } 
\end{array}\right.
\end{equation}
%
where $\nabla$ and $\nabla \cdot$ denote gradient and divergence operator in Cartesian coordinates.
The domain $\Omega_p(u)$ represents the plasma region, while $\Omega_{C_k}$ corresponds to the region occupied by the $k$-th poloidal field coil $k$ carrying current $I_k$ distributed over a cross-sectional area $S_k$.
\MH{How many coils? $d$?}
In \eqref{eq:FreeBoundary}, the magnetic permeability $\mu$ is constant (equal to $\mu_0$) in vacuum regions and may vary within ferromagnetic materials 
as a function $\mu = \mu(|\nabla u|^2/r^2)$ \MH{what does this mean? What is the description of $\mu$ in the domains $\Omega$?}
The source term in $\Omega_p$ models the toroidal plasma current density, which depends nonlinearly on $u$ in terms of the hydrostatic pressure $p(u)$ and toroidal magnetic field $g(u)$. Following the formulation in \cite{LuBr:1982}, we define 
%
\begin{equation}\label{eq:source}
\frac{d}{d u}p( u) = j_0\frac{\beta}{r_0}\left(1-u_N^{\alpha_1}\right)^{\alpha_2},  \qquad \qquad
\frac{1}{2}\frac{d}{d u}g^2(u) = j_0\mu_0r_0(1-\beta)\left(1-u_N^{\alpha_1}\right)^{\alpha_2},
\end{equation}
\end{subequations}
%
where $u_N \in [0,1]$ is the normalized poloidal flux, scaled between its values on the \textit{magnetic axis} and the plasma boundary; the parameters $r_0$, $\alpha_1$, and $\alpha_2$ characterize the outer radius of the vacuum chamber and control the sharpness of the current profile, while $\beta$ (the poloidal beta) measures the ratio of plasma pressure to magnetic pressure. The plasma boundary $\partial \Omega_p$, which depends on the solution $u$ and is defined by the last closed streamline, introduces a free-boundary aspect to the problem. The problem is further complicated by nonlinear dependencies in the boundary location, source terms, and possibly spatially varying permeability.


Let $\Omega \subset \mathbb{R}^2$ be a bounded Lipschitz domain enclosing the confinement region $\Omega_p$, the external coils $\Omega_{c_i}$, and surrounding structural components.  
Following \cite{Gr:1999}, the solution space $\cU$ for \eqref{eq:FreeBoundary} is defined as

\MH{The $\cU$ below is not a Hilbert space because $C^0(\overline{\Omega})$ is not a Hilbert space!}
\begin{equation}\label{eq:Soln_space}
    \cU:=\left\{u:\Omega\rightarrow \mathbb{R} \,\Bigg| \,\int_\Omega u^2rdrdz<\infty; \,  \int_\Omega\frac{|\nabla u|^2}{r}drdz<\infty; \, u(0,z)=0 \right\}\cap C^0(\overline{\Omega}),
\end{equation}
%
equipped with the inner product and norm
%
\begin{equation}\label{eq:inner_prod_norm}
        \langle u,v\rangle_{\cU} := \int_{\Omega} \frac{1}{r} \nabla u\cdot\nabla v \;\;drdz,\qquad \| u \|_{\cU} :=\left(\int_\Omega\frac{|\nabla u|^2}{r} drdz\right)^{1/2}.
\end{equation}
%

In practice, the equilibrium configuration is sensitive to operational variability, which propagates through the system and 
impacts both the solution $u$ and derived quantities such as the plasma boundary and $x$-point locations. 
In this work, we focus on uncertainties in the coil currents $I_k$ in \eqref{eq:FreeBoundary_GS}.
We model the uncertainty using a $d$-dimensional random vector $\omega= (\omega_1, \ldots, \omega_d)$, 
where each $\omega_k$ is an independent random variable uniformly distributed around the baseline value 
$\widetilde{\omega}_k$ for some relative perturbation magnitude $\tau > 0$. The corresponding joint density is
%
\[
\pi \left(\omega\right)=\prod_{k=1}^{d} \frac{1}{2\tau |\widetilde{\omega}_k|},\quad 
W=\prod_{k=1}^{d}\Big[ \widetilde{\omega}_k-\tau  | \widetilde{\omega}_k | ,\widetilde{\omega}_k+\tau  | \widetilde{\omega}_k  | \Big].
\]
%
Incorporating uncertainty into the coil currents requires solving a parameterized version of the free-boundary problem \eqref{eq:FreeBoundary}, 
with solution operator $u(\cdot,\omega) : W \to \cU$ mapping each realization of $\omega$ to a solution in $\cU$. 
Since a density is available, in this case the Bochner space  \eqref{eq:BochnerL2} becomes
\begin{align*}     
  L^2(W, \cU) := \Big\{ u: W \rightarrow \cU \;\big|\; & 
                                    u \mbox{ is strongly measurable and } 
                                       \int_W \| u(\cdot,\omega) \|_{\cU}^2 \, \pi(\omega) \, d\omega < \infty \; \Big\}.
\end{align*}
with inner product $ \langle u, v \rangle =  \int_W  \langle  u(\omega) , v(\omega)  \rangle_{\cU} \, \pi(\omega) \, d\omega$.
The expectation   \eqref{eq:expectation_of_u} becomes
 \begin{equation}  \label{eq:QoI}
      \mathbb{E}\left[u(\cdot,\boldsymbol \omega)\right]=\int_W u(\cdot,\omega) \, \pi(\boldsymbol\omega) \,d\omega.
 \end{equation}
 In addition to estimating  \eqref{eq:QoI} , we are also interested in estimating derived quantities from $u$, such as the plasma boundary 
 and features of the solution, including the location of so-called  $x$-points.



