\documentclass[final,3p,times,11pt]{article}
\usepackage{titling}
\setlength{\droptitle}{-10em}
\usepackage[utf8]{inputenc}
% ==============   Packages  ====================
\usepackage{amsmath,amssymb,amsthm, mathrsfs}
\usepackage{mathtools}
\usepackage{graphicx}
\usepackage[shortlabels]{enumitem}
% \usepackage[top=2cm,bottom=2cm,left=2cm,right=2cm]{geometry}
\usepackage{stmaryrd}
\usepackage{soul}
\usepackage{multirow}
\usepackage[dvipsnames]{xcolor}
\usepackage{cancel}
\usepackage{ulem}
%\usepackage{subcaption}
%\usepackage[show]{ed}
%\usepackage{showkeys}
\usepackage{showlabels}
%\usepackage[notcite,notref]{showkeys}
\usepackage[ruled,vlined]{algorithm2e}
\usepackage {hyperref}
%\usepackage{natbib}
\hypersetup{
    %bookmarks=true,         % show bookmarks bar?
    colorlinks=true,       % false: boxed links; true: colored links
    %linkcolor=red,          % color of internal links (change box color with linkbordercolor)
    %citecolor=green,        % color of links to bibliography
    %filecolor=magenta,      % color of file links
    %urlcolor=cyan           % color of external links
}
%\usepackage{wrapfig}
%% The lineno packages adds line numbers. Start line numbering with
%% \begin{linenumbers}, end it with \end{linenumbers}. Or switch it on
%% for the whole article with \linenumbers after \end{frontmatter}.
%\usepackage{lineno}


% ==============   Macros  ====================
\newcommand{\mynabla}{\widetilde{\nabla}} 
\newcommand{\jump}[1]{[\![#1]\!]}
\newcommand{\HEcolor}[1]{{\textcolor{blue}{#1}}}
\newcommand{\TSVcolor}[1]{{\textcolor{orange}{#1}}}
\newcommand{\JLcolor}[1]{{\textcolor{black}{#1}}} %violet
\newcommand{\Grids}{\boldsymbol{\chi}}

\newtheorem{theorem}{Theorem}%[section]
\newtheorem{VariationalForm}[theorem]{Variational Formulation}
% =============================================


%% natbib.sty is loaded by default. However, natbib options can be
%% provided with \biboptions{...} command. Following options are
%% valid:

%%   round  -  round parentheses are used (default)
%%   square -  square brackets are used   [option]
%%   curly  -  curly braces are used      {option}
%%   angle  -  angle brackets are used    <option>
%%   semicolon  -  multiple citations separated by semi-colon
%%   colon  - same as semicolon, an earlier confusion
%%   comma  -  separated by comma
%%   numbers-  selects numerical citations
%%   super  -  numerical citations as superscripts
%%   sort   -  sorts multiple citations according to order in ref. list
%%   sort&compress   -  like sort, but also compresses numerical citations
%%   compress - compresses without sorting
%%
%% \biboptions{comma,round}

% \biboptions{}

% \journal{}
% \makeatletter
% \def\ps@pprintTitle{%
%  \let\@oddhead\@empty
%  \let\@evenhead\@empty
%  \def\@oddfoot{}%
%  \let\@evenfoot\@oddfoot}
% \makeatother

\pagestyle{plain} %\pagestyle{myheadings}

\textheight 8.4in
\textwidth  6.5in
\oddsidemargin 0pt \evensidemargin 0pt

\title{ \textbf{MFMC toy problem write up}}
\date{}

\begin{document}

\maketitle
\vspace{-2.3cm}

\section{Toy problem}
In this model we consider the diffusion problem
\begin{equation}
\label{eq: Model2_1}
\begin{array}{lll}
-\nabla \cdot (a_1(x,\xi_1) \nabla u(x)) = f_1(x), & x \in  (0,1),\\
-\nabla \cdot (a_2(x,\xi_2)\nabla v(x)) = f_2(x), & x \in  (1,2),\\
  u(x) = g(x), & x = 0,\\
  v(x) = g(x), & x = 2,\\
  u(x) = v(x) & x = 1,\\
a_1(x)\partial_\nu u(x) + a_2(x) \partial_\nu v(x) = 0, & x = 1.
\end{array}
\end{equation}

The diffusion coefficients and source terms for the toy problem are
\[
a_1=1+\xi_1,\quad a_2 = 10(1+\xi_2),\quad f_1 = -2-3x^2,\quad f_2 = 1-6x.
\]




\section{Code implementation}
This code employs a high-fidelity model with 5,121 grid points and eight low-fidelity models generated from a sparse grid, aligning with the same setup as the plasma problem.
\begin{enumerate}
    \item\texttt{script\textunderscore BuildSurrogforToy.m}: Builds surrogate models, which are later used for dynamic sampling.
    
    \item \texttt{script\textunderscore test\textunderscore covar.m}: Handles dynamic sampling.

    \begin{itemize}
        \item modify line 22 \texttt{tol}  to adjust the stopping criterion threshold.
        \item The model selection process (lines 195–238) computes parameters for the most recent and second most recent updates.
        \item The stopping criterion is located around line 240—this is where adjustments may be needed.
        \item All results are stored in \texttt{Result\textunderscore test\textunderscore covar}.
    \end{itemize}
    \item \texttt{FEM\textunderscore solver.m}:  Implements the finite element method to solve a 1D Poisson equation, with two parameters in the diffusion coefficient. (parameter dimension $d=2$).
    \item \texttt{Surrog\textunderscore Eval.m}: Evaluates the surrogate models, serving as low-fidelity models.
    \item \texttt{load\textunderscore hfm\textunderscore mesh\textunderscore n\textunderscore com\textunderscore mesh\textunderscore toy.m}: Loads the high-fidelity model mesh and a common mesh for interpolating solutions from low-fidelity models.
    \item \texttt{load\textunderscore lfm\textunderscore toy.m}: Loads surrogates for low-fidelity models.
    \item \texttt{MFMC\textunderscore model\textunderscore selection\textunderscore exhausted.m}: Handles model selection using the exhausted method.
    \item \texttt{MFMC\textunderscore model\textunderscore selection\textunderscore backtrack.m}: Handles model selection using the backtrack method.
    \item \texttt{L2\textunderscore inprod\textunderscore toy.m}: Computes the L2 inner product of two functions.
    \item \texttt{interp2grid\textunderscore toy.m}:  Interpolates a function from a coarse mesh to a fine mesh.
    \item \texttt{plot\textunderscore n\textunderscore print.m}: Generates and plots tables from the dynamic sampling results.
\end{enumerate}







\end{document}


