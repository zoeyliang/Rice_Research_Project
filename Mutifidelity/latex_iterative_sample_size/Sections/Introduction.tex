% ====================================================
\section{Introduction}\label{sec:Intro}
% ====================================================

Monte Carlo (MC) methods are widely used for estimating statistical quantities in scientific and engineering applications. However, their computational cost can become prohibitive when each high-fidelity (HF) model evaluation is expensive. Multi-fidelity Monte Carlo (MFMC) methods \cite{PeWiGu:2016} address this challenge by exploiting correlations between models of varying accuracy and cost. By combining high-fidelity and low-fidelity (LF) evaluations within a single estimator, MFMC reduces the variance under a fixed computational budget while preserving the accuracy of the HF model.


A crucial aspect of MFMC is determining the optimal sample allocation across different model fidelities. In the standard formulation, the sample sizes are obtained by solving a continuous optimization problem that minimizes the estimator variance under a prescribed cost constraint. This yields real-valued sample sizes, which are then rounded to integers for practical implementation. Although rounding is computationally convenient, it introduces a discrepancy between the theoretical and implemented allocations, leading to potential suboptimality. This issue is particularly pronounced when the computational budget is limited or when model costs vary significantly.


This work introduces a novel iterative scheme that computes integer-valued sample allocations to solve the budget under-utilization problem. The proposed approach generates admissible integer allocations through sequentially generating a sequence of proxy variables at each fidelity level, where each component is updated based on residual budget after flooring previous allocations. This recursive procedure implicitly enforces integer feasibility,  ensuring stricter adherence to total cost constraints, and preserving the structural properties of the optimal continuous solution.



We provide theoretical analysis of the proposed scheme establishing bounds on variance and guarantees budget admissibility of the estimator using integer-valued sample estimation. Numerical experiments confirm that our iterative allocation consistently outperforms conventional rounding strategies, particularly in regimes with tight computational budgets or highly heterogeneous model costs.

The paper is organized as follows: Section~\ref{sec:MFMC} reviews the MFMC formulation and continuous allocation theory; Section~\ref{sec:Iterative_IntegerValued_Sample_Size} develops the iterative integer-valued allocation scheme and its theoretical foundations; Section~\ref{sec:Modified_IntegerValued_Sample_Size} addresses modifications for ill-conditioned scenarios; Section~\ref{sec:Num_Result} presents numerical results illustrating the performance of the method.









