% ====================================================
\section{Iterative sample size estimation for MFMC}\label{sec:Iterative_IntegerValued_Sample_Size}
% ====================================================
To address this question, we propose an iterative scheme for estimating sample sizes that preserves the total computational budget while naturally extending the real-valued MFMC formulation.  
Starting from the standard MFMC sample allocation \cite{PeGuWi:2018}, we define the iterative real-valued sample size sequence as
%
\begin{equation}
    \label{eq:MFMC_New_RealValued_Sample_Size}
    H_1^* = \sqrt{\frac{\Delta_1}{C_1}}\frac{p}{\sum_{j=1}^K\sqrt{C_j\Delta_j}}, 
    \qquad 
    H_k^* = \sqrt{\frac{\Delta_k}{C_k}}\frac{p-\sum_{j=1}^{k-1}C_jH_j^*}{\sum_{j=k}^K\sqrt{C_j\Delta_j}}, 
    \quad k = 2,\ldots, K.
\end{equation}
%
The next theorem establishes that this iterative construction yields exactly the same real-valued solution as the standard MFMC sample size \eqref{eq:MFMC_RealValued_Sample_Size}.  
Consequently, the total cost and normalized variance remain unchanged.


\begin{theorem}[Iterative real-valued sample size for MFMC]\label{thm:MFMC_Iteravie_RealValued_Sample_Size}

For the iterative real-valued sample size defined in \eqref{eq:MFMC_New_RealValued_Sample_Size}, 
the resulting values coincide with the standard real-valued MFMC sample sizes in 
\eqref{eq:MFMC_RealValued_Sample_Size}, i.e.,
%
\[
H_k^* = N_k^*
    = \sqrt{\frac{\Delta_k}{C_k}}\,
      \frac{p}{\sum_{j=1}^K \sqrt{C_j\Delta_j}}.
\]
%
Consequently,
%
\[
\sum_{k=1}^K C_k H_k^* = p, 
\qquad  
f(H_k^*) = \sum_{k=1}^K \frac{\Delta_k}{H_k^*} 
= \frac{1}{p} \left(\sum_{k=1}^K \sqrt{C_k\Delta_k}\right)^2.
\]
%
\end{theorem}




\begin{proof}
Define the partial sum and remainder
\[
    T_k = \sum_{j=1}^k C_j H_j^*, 
    \qquad 
    R_k = p - T_k,
\]
For $k=0$ and $k=1$, we have
\[
    T_0 = 0, 
    \qquad 
    T_1 = C_1H_1^* 
    = p\frac{\sqrt{C_1\Delta_1}}{\sum_{j=1}^K \sqrt{C_j\Delta_j}},
    \qquad
    R_0 = p,
    \qquad
    R_1 = p-T_1
    = p\frac{\sum_{j=2}^K \sqrt{C_j\Delta_j}}{\sum_{j=1}^K \sqrt{C_j\Delta_j}}.
\]
For general $k\ge 1$, the iterative definition \eqref{eq:MFMC_New_RealValued_Sample_Size} gives
\[
    H_k^*
    = \sqrt{\frac{\Delta_k}{C_k}}\,
      \frac{R_{k-1}}{\sum_{j=k}^K \sqrt{C_j \Delta_j}},
\]
%
and therefore
%
\[
    R_k 
    = R_{k-1} - C_k H_k^*
    = \frac{\sum_{j=k+1}^K \sqrt{C_j \Delta_j}}
           {\sum_{j=k}^K \sqrt{C_j \Delta_j}} \, R_{k-1}.
\]
%
Hence $R_k$ forms a geometric sequence, from which it follows that
%
\[
R_k=\frac{p}{S}\sum_{j=k+1}^K\sqrt{C_j\Delta_j}.
\]
%
where the aggregate cost–variance weight $S$ serves as a normalization factor that balances the contributions of model cost and variance reduction across all fidelity levels and is defined as
%
\begin{equation}\label{eq:aggregate_cost–variance_weight_S}
    S = \sum_{j=1}^K \sqrt{C_j \Delta_j}.
\end{equation}
%
Substituting this relation into the expression for $H_k^*$ yields
%
\[
    H_k^*
    = \frac{p}{S}\sqrt{\frac{\Delta_k}{C_k}}.
\]
In particular, for $k = K$ we obtain $R_K = 0$, implying
\[
    T_K = p - R_K = p,
\]
which verifies that $\sum_{k=1}^K C_k H_k^* = p$. Finally,
\[
    f(H_k^*)
    = \sum_{k=1}^K \frac{\Delta_k}{H_k^*}
    = \frac{1}{p} \left(\sum_{k=1}^K \sqrt{C_k\Delta_k}\right)^2.
\]
\end{proof}
%

Now we formulate an iterative scheme for computing the \textit{integer-valued} sample sizes in MFMC. We introduce the following real-valued proxy sequence, unlike the remaining budget in \eqref{eq:MFMC_New_RealValued_Sample_Size} where the remaining budget $p$ is subtract from the total budget with real-valued sample size, here the remaining budget $p$ is subtract from the total budget with integer-valued sample size
%
\begin{equation}
    \label{eq:MFMC_New_IntegerValued_Sample_Size}
    M_1^* = \sqrt{\frac{\Delta_1}{C_1}}\frac{p}{\sum_{j=1}^K\sqrt{C_j\Delta_j}}, 
    \qquad 
    M_k^* = \sqrt{\frac{\Delta_k}{C_k}}\frac{p-\sum_{j=1}^{k-1}C_j\left\lfloor M_j^* \right\rfloor}{\sum_{j=k}^K\sqrt{C_j\Delta_j}}, 
    \quad k = 2,\ldots, K.
\end{equation}
%
The integer-valued sample sizes are then given by $\lfloor M_k^* \rfloor$ for $k = 1,\ldots, K$.  
This procedure begins with the standard real-valued MFMC solution $M_1^*$ in \eqref{eq:MFMC_RealValued_Sample_Size}, from which the integer allocation $\lfloor M_1^* \rfloor$ is obtained.  
The cost associated with $\lfloor M_1^* \rfloor$ samples is subtracted from the total budget, and the same allocation principle \eqref{eq:MFMC_RealValued_Sample_Size} is applied to the remaining budget to compute $\lfloor M_2^* \rfloor$, and so on.  
In this way, integer-valued sample sizes are determined iteratively while respecting the total budget constraint.

In particular, Theorem \ref{thm:MFMC_New_IntegerValued_Cost} shows that the iterative allocation scheme ensures that the accumulated cost does not exceed the prescribed budget, while improving upon the direct flooring approach in terms of budget utilization.


\begin{theorem}[Cost bound for the iterative integer-valued sample allocation]
\label{thm:MFMC_New_IntegerValued_Cost} 

Let $\lfloor M_k^* \rfloor$ denote the integer-valued sample sizes obtained from the iterative scheme \eqref{eq:MFMC_New_IntegerValued_Sample_Size}, and let $\lfloor N_k^* \rfloor$ denote those obtained by directly flooring the standard real-valued MFMC sample allocation \eqref{eq:MFMC_RealValued_Sample_Size}. 
Assume that the prescribed computational budget satisfies 
%
\begin{equation}\label{eq:p_bound}
     p \ge \sum_{k=1}^K C_k.
\end{equation}
%
Then, the total cost associated with the iterative integer-valued allocation is bounded by
\begin{equation}\label{eq:Iterative_integer_sample_size_cost_bound}
    \sum_{k=1}^K C_k \left\lfloor N_k^* \right\rfloor
    \;\le\;
    \sum_{k=1}^K C_k \left\lfloor M_k^* \right\rfloor
    \;\le\;
    p.
\end{equation}
\end{theorem}


\begin{proof}
We first establish that the total cost of the iterative scheme does not exceed the prescribed budget, i.e.,
\[
\sum_{k=1}^K C_k \left\lfloor M_k^* \right\rfloor \le p.
\]
Define the cumulative integer cost up to level $k$ as
\[
T_k = \sum_{j=1}^k C_j\left\lfloor M_j^* \right\rfloor.
\]
Since $\lfloor M_j^* \rfloor \le M_j^*$, we claim, and prove by induction, that for each $k = 1, \ldots, K$,
\begin{equation}\label{eq:Tk_bound}
T_k \le \frac{p}{S}\sum_{j=1}^k \sqrt{C_j \Delta_j}.
\end{equation}
where $S$ is defined in \eqref{eq:aggregate_cost–variance_weight_S}. Inequality \eqref{eq:Tk_bound} shows that the cumulative integer cost up to level $k$ is bounded by a proportional share of the total budget, scaled by $S$.




The base case $k=1$ follows immediately,
\[
T_1=C_1 \left\lfloor M_1^* \right\rfloor \le C_1M_1^* = \frac{p}{S}\sqrt{C_1\Delta_1},
\]
so \eqref{eq:Tk_bound} holds for \(k=1\). Assume \eqref{eq:Tk_bound} holds for \(k-1\). By definition of \(M_k^*\),
%
\[
M_k^* = \sqrt{\frac{\Delta_k}{C_k}}\frac{p - T_{k-1}}{\sum_{j=k}^K \sqrt{C_j\Delta_j}},
\]
%
and hence
%
\[
C_k \left\lfloor M_k^* \right\rfloor \le C_k M_k^*  = \sqrt{C_k\Delta_k}\frac{p-T_{k-1}}{\sum_{j=k}^K\sqrt{C_j\Delta_j}}.
\]
%

Using the inductive hypothesis and simplifying the resulting algebraic expression yields
\begin{align*}
    T_k &= T_{k-1}+C_k\left\lfloor M_k^* \right\rfloor \\
    &\le T_{k-1} + \sqrt{C_k\Delta_k}\frac{p-T_{k-1}}{\sum_{j=k}^K\sqrt{C_j\Delta_j}}
    =T_{k-1}\left(1-\frac{\sqrt{C_k\Delta_k}}{\sum_{j=k}^K\sqrt{C_j\Delta_j}}\right) + p\frac{\sqrt{C_k\Delta_k}}{\sum_{j=k}^K\sqrt{C_j\Delta_j}}\\
    &\le \frac{p}{S}\sum_{j=1}^{k-1} \sqrt{C_j\Delta_j}\frac{\sum_{j=k+1}^K\sqrt{C_j\Delta_j}}{\sum_{j=k}^K\sqrt{C_j\Delta_j}}+p\frac{\sqrt{C_k\Delta_k}}{\sum_{j=k}^K\sqrt{C_j\Delta_j}}=\frac{p}{S}\cdot \frac{\sum_{j=1}^{k-1} \sqrt{C_j\Delta_j}\sum_{j=k+1}^K\sqrt{C_j\Delta_j}+S\sqrt{C_k\Delta_k}}{\sum_{j=k}^K\sqrt{C_j\Delta_j}}\\
    &=\frac{p}{S}\cdot \frac{\sum_{j=1}^{k-1} \sqrt{C_j\Delta_j}\sum_{j=k+1}^K\sqrt{C_j\Delta_j}+\sqrt{C_k\Delta_k}\left(\sum_{j=1}^{k-1}\sqrt{C_j\Delta_j}+\sum_{j=k}^K\sqrt{C_j\Delta_j}\right)}{\sum_{j=k}^K\sqrt{C_j\Delta_j}}
    %=\frac{p}{S}\frac{\sum_{j=k}^K\sqrt{C_j\Delta_j}\sum_{j=1}^k\sqrt{C_j\Delta_j}}{\sum_{j=k}^K\sqrt{C_j\Delta_j}}
    =\frac{p}{S}\sum_{j=1}^k\sqrt{C_j\Delta_j},
\end{align*}
%
which completes the inductive step. Thus, inequality \eqref{eq:Tk_bound} holds for all $k$. 
In particular, when $k=K$, we obtain
\begin{equation}\label{eq:MFMC_iterative_total_cost}
T_K = \sum_{j=1}^K C_j\left\lfloor M_j^*\right\rfloor \le p,
\end{equation}
confirming that the iterative scheme never exceeds the prescribed computational budget. To establish the lower bound in \eqref{eq:Iterative_integer_sample_size_cost_bound}, we compare the auxiliary sequences $M_k^*$ and $N_k^*$. Using \eqref{eq:Tk_bound}, we have
%
\[
M_k^* = \sqrt{\frac{\Delta_k}{C_k}}\frac{p - T_{k-1}}{\sum_{j=k}^K\sqrt{C_j\Delta_j}} \ge \sqrt{\frac{\Delta_k}{C_k}}\frac{p-\frac{p}{S}\sum_{j=1}^{k-1}\sqrt{C_j\Delta_j}}{\sum_{j=k}^K\sqrt{C_j\Delta_j}} = \sqrt{\frac{\Delta_k}{C_k}}\frac{p}{S}=N_k^*, \qquad k \ge 1.
\]
% 
Monotonicity of the floor function yields \(\lfloor M_k^*\rfloor\ge\lfloor N_k^*\rfloor\) for every \(k\), and summing after multiplying by \(C_k\) gives the desired lower bound in \eqref{eq:Iterative_integer_sample_size_cost_bound}.  Combining this result with \eqref{eq:MFMC_iterative_total_cost} completes the proof.

\vspace{4mm}
\noindent{{\it When equality holds.}}
The upper bound in \eqref{eq:Iterative_integer_sample_size_cost_bound} is attained if and only if no budget remains unused, i.e.,
\[
\sum_{k=1}^K C_k \left(M_k^* - \left\lfloor M_k^*\right\rfloor\right) = 0.
\]
Because each fractional part satisfies $M_k^* - \lfloor M_k^* \rfloor \in [0,1)$, this condition holds precisely when every $M_k^*$ is an integer. Hence, equality occurs when all $M_k^* \in \mathbb{Z}$.

The lower bound is attained if and only if $\lfloor M_k^* \rfloor = \lfloor N_k^* \rfloor$ for every $k$, or equivalently, when both real-valued allocations $M_k^*$ and $N_k^*$ lie in the same integer interval:
\[
\left\lfloor N_k^*\right\rfloor\le M_k^* < \left\lfloor N_k^*\right\rfloor + 1.
\]
Since $M_k^* \ge N_k^*$, this occurs precisely when the two share the same integer part for all $k$.
\end{proof}



Theorem \ref{thm:MFMC_New_IntegerValued_Variance} shows that the iterative integer-valued allocation achieves a variance no smaller than the continuous optimum but no greater than that obtained by direct flooring, providing a better variance–cost tradeoff under a fixed budget.

\begin{theorem}[Normalized variance bound for the iterative integer-valued sample allocation]
\label{thm:MFMC_New_IntegerValued_Variance}

Let $\lfloor M_k^* \rfloor$ denote the integer-valued sample sizes obtained from the iterative allocation scheme \eqref{eq:MFMC_New_IntegerValued_Sample_Size}, and let $\lfloor N_k^* \rfloor$ denote those obtained by directly flooring the real-valued optimal allocation \eqref{eq:MFMC_RealValued_Sample_Size}. 
Assume that the computational budget $p$ satisfies \eqref{eq:p_bound}, and that the variance-related quantities $\Delta_k$ are defined as in Theorem~\ref{thm:Sample_size_est} and meet the same conditions. 
Then, the normalized variance associated with the iterative integer-valued allocation satisfies the following bounds
%
\begin{equation}\label{eq:Iterative_Integer_Variance_Bound}
\frac{1}{p}\left(\sum_{k=1}^K \sqrt{C_k \Delta_k}\right)^2
= \sum_{k=1}^K \frac{\Delta_k}{N_k^*}
\;\le\;
\sum_{k=1}^K \frac{\Delta_k}{\left\lfloor M_k^* \right\rfloor}
\;\le\;
\sum_{k=1}^K \frac{\Delta_k}{\left\lfloor N_k^* \right\rfloor}.
\end{equation}
%
\end{theorem}



\begin{proof}
We first establish the right-hand inequality in \eqref{eq:Iterative_Integer_Variance_Bound}.  
From Theorem~\ref{thm:MFMC_New_IntegerValued_Cost}, it follows that $\lfloor M_k^* \rfloor \ge \lfloor N_k^* \rfloor$ for all $k$.  
Since $x \mapsto \Delta_k/x$ is strictly decreasing for $x > 0$, we obtain
\[
\sum_{k=1}^K \frac{\Delta_k}{\left\lfloor M_k^* \right\rfloor}
\le 
\sum_{k=1}^K \frac{\Delta_k}{\left\lfloor N_k^* \right\rfloor},
\]
establishing the upper bound. To show the lower bound, we apply the Cauchy--Schwarz inequality:
\[
\left(\sum_{k=1}^K \sqrt{C_k \Delta_k}\right)^2
\le
\left(\sum_{k=1}^K C_k \left\lfloor M_k^* \right\rfloor\right)
\left(\sum_{k=1}^K \frac{\Delta_k}{\left\lfloor M_k^* \right\rfloor}\right).
\]
By the cost bound proved in \eqref{eq:MFMC_iterative_total_cost}, we have 
$\sum_{k=1}^K C_k \lfloor M_k^* \rfloor \le p$, hence
%
\[
\frac{1}{p}\left(\sum_{k=1}^K \sqrt{C_k \Delta_k}\right)^2
\le
\sum_{k=1}^K \frac{\Delta_k}{\left\lfloor M_k^* \right\rfloor}.
\]
%
This establishes the lower bound in \eqref{eq:Iterative_Integer_Variance_Bound}.

\vspace{4mm}
\noindent{{\it When equality holds.}}
Equality in the Cauchy--Schwarz step holds if and only if there exists a constant $\lambda>0$ such that 
\[
\left\lfloor M_k^* \right\rfloor = \lambda \sqrt{\frac{\Delta_k}{C_k}},
\]
which corresponds to the continuous optimal allocation $M_k^* = N_k^*$. Therefore, equality in \eqref{eq:Iterative_Integer_Variance_Bound} holds if and only if the iterative scheme reproduces the continuous solution exactly, i.e., when all $\lfloor M_k^* \rfloor = N_k^*$ and the total cost equals $p$.

\end{proof}
% ------------------
% Next consider the variance
% \begin{align*}
%     f_{\text{act}}(\overline{N_k})&=\sum_{i=1}^{K}\frac{\Delta_k}{\overline{N_k}}=\sum_{i=1}^{k-1}\frac{\Delta_i}{\overline{N_i}}+\frac{\Delta_k}{\overline{N_k}}+\sum_{i=k+1}^{K}\frac{\Delta_i}{\overline{N_i}}\\
%     &\in \left[\sum_{i=1}^{k-1}\frac{\Delta_i}{\overline{N_i}}+\frac{\Delta_k}{\overline{N_k}}+\sum_{i=k+1}^K\frac{\Delta_{i}}{N_i^*},\; \sum_{i=1}^{k-1}\frac{\Delta_i}{\overline{N_i}}+\frac{\Delta_k}{\overline{N_k}}+\sum_{i=k+1}^K\frac{\Delta_{i}}{N_i^*-1}\right)=[f_1,f_2)\\
%     f_1&=\sum_{i=1}^{k-1}\frac{\Delta_i}{\overline{N_i}}+\frac{\Delta_k}{\overline{N_k}}+\sum_{i=k+1}^K\frac{\Delta_{i}}{N_i^*}=\sum_{i=1}^{k-1}\frac{\Delta_i}{\overline{N_i}}+\frac{\Delta_k}{\overline{N_k}}+\sum_{i=k+1}^K\sqrt{C_i\Delta_i}\frac{\sum_{j=i}^K\sqrt{C_j\Delta_j}}{p-\sum_{j=1}^{i-1}C_j\overline{N_j}}\\
%     f_2&=\sum_{i=1}^{k-1}\frac{\Delta_i}{\overline{N_i}}+\frac{\Delta_k}{\overline{N_k}}+\sum_{i=k+1}^K \ldots\\
% \end{align*}
% \begin{align*}
%     \frac{d f_1}{d \overline{N_k}}&=-\frac{\Delta_k}{\overline{N_k}^2}+C_k\sum_{i=k+1}^K\sqrt{C_i\Delta_i}\frac{\sum_{j=i}^K\sqrt{C_j\Delta_j}}{\left(p-\sum_{j=1}^{i-1}C_j\overline{N_j}\right)^2}\\
%     \frac{d^2 f_1}{d^2 \overline{N_k}}&=\frac{2\Delta_k}{\overline{N_k}^3}+2C_k^2\sum_{i=k+1}^K\sqrt{C_i\Delta_i}\frac{\sum_{j=i}^K\sqrt{C_j\Delta_j}}{\left(p-\sum_{j=1}^{i-1}C_j\overline{N_j}\right)^3}
% \end{align*}
% Note that $\frac{d^2 f_1}{d^2 \overline{N_k}}>0$ whenever $p>\sum_{j=1}^{i-1}C_j\overline{N_j}$. this means $f_1$ is convex in $\overline{N_k}$.