% ============================================================
\section{Sparse grid stochastic collocation}\label{sec:SC}
% ============================================================
This section provides a brief overview of the sparse grid stochastic collocation approach \cite{BaNoRi:2000, KlBa:2005, MaNi:2009, Sm:1963}, demonstrated using a generic solution $u$. 

The method starts by defining a univariate set of $m_i$ collocation nodes $X^i = \left\{x_1^i,\ldots, x_{m_i}^i\right\}$ over the interval $[-1,1]$. Using these nodes, a univariate interpolation operator is built as $I_{X^{i}}[u]:=\sum_{j=1}^{m_{i}} u(\cdot, x_j^i)\phi_j$, where the basis function $\phi_k(x_j^i)$ is the Kronecker delta, evaluating to 1 when $k=j$ and $0$ otherwise. To extend this approach to high-dimensional parameter space, the method combines univariate nodes across multiple dimensions using a tensor product. Instead of using a full tensor grid -- which grows exponentially with the number of dimensions $d$ -- the sparse grid framework selects fewer nodes $m_i$ per dimension, drastically reducing computational costs while maintaining accuracy. The {\it sparse grid nodes} for a domain of dimension $d$ and {\it level} $q\; (\text{where }q\ge d)$ are defined as
%
\begin{equation*}
H(q,d) = \bigcup_{q-d+1\le|\boldsymbol{i}|\le q} \left(X^{i_1}\times \cdots\times X^{i_d}\right)\in [-1,1]^d, 
\end{equation*}
where $|\boldsymbol{i}| = i_1+\ldots+i_d$ specifies the refinement rule. These nodes form a sparse representation of the domain, capturing the essential features of $u$ with fewer evaluations than a full grid. 


For our problem, the collocation nodes are selected as the extrema of the Chebyshev polynomials \cite{BaNoRi:2000, ClCu:1960}, with the $j$-th node calculated as $x_j^i=-\cos (\pi(j-1)/(m_i-1))$ for $j=1, \ldots, m_i$. The number of nodes is determined so that $m_1 =1$ and $m_i = 2^{i-1}+1$ for $i\ge 2$. This specific choice ensures that the univariate nodal sets $X^i$ exhibit a {\it nested} structure, satisfying $X^i\subset X^{i+1}$. Consequently, the multidimensional sparse grid nodes inherit this nesting property,  resulting in
%
\begin{equation}
\label{eq:NestedColPts}
H(q,d)\subset H(q+1,d),\quad \text{and}\quad H(q,d) = \bigcup_{|\boldsymbol{i}|=q} \left(X^{i_1}\times \cdots\times X^{i_d}\right).
\end{equation}
%
This hierarchical nesting allows for the reuse of function evaluations at coarser levels when constructing finer levels, further enhancing computational efficiency compared to non-nested grids. Interpolation over the hierarchical sparse grid nodes $H(q,d)$ is performed using the {\it Smolyak quadrature formula}, which combines univariate interpolation operators across dimensions as
%
\begin{equation}
\label{eq: Smolyak_Quad_formula}
\mathcal{S}_{q, d}[u] = \sum_{q+1\le |\boldsymbol{i}|\le q+d} (-1)^{q+d-|\boldsymbol{i}|} \binom{d-1}{q+d-|\boldsymbol{i}|}\cdot \left(\mathrm I_{X^{i_1}}\otimes\cdots\otimes \mathrm I_{X^{i_d}}\right) [u].
\end{equation} 
%
This method efficiently computes high-dimensional interpolations by exploiting the sparsity and hierarchical structure of the grid, providing an effectively balances between computational efficiency and precision, making it particularly suited for solving deterministic parametrized problems involving high-dimensional uncertainties.



% Let $N$ denote the number of sparse grid nodes. The sparse grid stochastic collocation method is equivalent to solving $N$ deterministic parametrized problems \eqref{eq:FreeBoundary} at each nodal point in $H(q,d)$.

% For our model problem, the sparse grid stochastic collocation method constructs the surrogate function $ \mathcal{S}_{q,d}(u)$ as per \eqref{eq: Smolyak_Quad_formula} by computing the direct solution of the discrete version of \eqref{eq:FreeBoundary} at isotropic sparse grid nodes \eqref{eq:NestedColPts} with the Clenshaw-Curtis quadrature abscissa  \cite{BaNoRi:2000,ClCu:1960}. 

As discussed in \cite{NoTeWe:2008,TeJaWe:2015}, consider the function $u \in C^0(W,Z)$, where the parameter space $W$ and the solution space $Z$ are defined in \eqref{eq:ParameterSpace} and \eqref{eq:Soln_space} respectively. Let the interval in the $k$-th dimension be defined as $W_k = \left[\widetilde{\omega}_k-\tau \left\vert \widetilde{\omega}_k\right\vert, \widetilde{\omega}_k+\tau \left\vert \widetilde{\omega}_k\right\vert\right]$. The complementary multi-dimensional parameter space that excludes the $k$-th dimension is 
%
\[
W_k^c = \prod_{i=1, i\neq k}^d W_i.
\]
%
Now, for any fixed element $\omega_k^c \in W_k^c$, and for each $\omega_k\in W_k$, we assume the function $u(\cdot,\omega_k,\omega_k^c): W_k \rightarrow C^0(W_k^c;Z)$ admits an analytic extension  $u(\cdot, z,\omega_k^c)$ in the complex plane, specifically in the region 
%
\[
W_k^{*}:=\{z\in \mathbb{C}: \text{dist} (z,W_k)\le \iota_k \;\text{ for some } \iota_k>0\},
\]
%
where $\iota_k$ denotes the proximity of the analytic extension to the real interval $W_k$. Under these assumptions, the interpolation error associated with the sparse grid method demonstrates an algebraic convergence rate
%
\begin{equation} \label{eq:coll-error-bound_2}
  \big\|u-\mathcal{S}_{q, d} (u)\big\|_\infty = C P^{-\mu},
\end{equation}
%
where $P$ denotes the sparse grid node count, $C$ is a constant dependent on dimension $d$ and analytic extension proximity to the interval $W_k$, and $\displaystyle \mu$ is related to the dimension of parameter space and function's analytic extension in the complex plane.

% Compared to the regularity assumption of $u$ in \cite{ElLiSa:2022}, the assumption for \eqref{eq:coll-error-bound_2} is stronger in the sense that the solution $u$ with respect to the random variable $\boldsymbol{\omega}$ can be analytically extended into the complex plane region by varying with one dimension of the random variable while keeping the other dimensions fixed.  This enhancement allows for a tighter interpolation error bound compared to the regularity assumption in \cite{ElLiSa:2022}.
